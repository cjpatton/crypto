\label{sec:pkead}
\cptodo{Define sources.}
\cptodo{The \fwdpke and \bwdpke notions do not guarantee privacy of associated
data, which is needed for the application to AKE. (See Section~\ref{sec:ake}.
How about passing $\ad_0$ and $\ad_1$ to the oracle in the game?}

\begin{definition}[PKEAD]\rm
  A \emph{public-key encryption scheme with associated data} is a 4-tuple of
  algorithms $\pkead = (\Kg, \Enc.\init, \Enc, \Dec)$ defined as follows:
  \begin{itemize}
    \item $\Kg$ is the randomized key generation algorithm that outputs a
      public/private key pair $(\pk, \sk)$.
      %
      Its execution is denoted $(\pk, \sk) \getsr \Kg$.

    \item $\Enc.\init$ is the deterministic encryption initialization
      algorithm. It takes as input an initializer~$\stsel$ and outputs the state.
      %
      Its execution is denoted $\st \gets \Enc.\init(\stsel)$.

    \item $\Enc$ is the deterministic encryption algorithm that takes as input the
      public key~$\pk$, a triple of strings $(\nonce, \ad, \msg)$, called the
      nonce, associated data, and plaintext respectively, and the state~$\st$, and
      outputs~$\cipher$, either a string or~$\bot$, and the updated state.
      %
      This is denoted $(\cipher, \st) \gets \Enc_\pk^{\nonce,\ad}(\msg, \st)$.

    \item $\Dec$ is the determinstic decryption that takes as input the secret
      key~$\sk$ and $(\nonce, \ad, \cipher)$ and outputs~$\msg$, either the
      message or $\bot$.
      %
      This is denoted $\msg \gets \Dec_\sk^{\nonce,\ad}(\cipher)$.
      \dqed
  \end{itemize}
\end{definition}
%
\cpnote{How does this syntax compare to stateful encryption as already defined
in the literature? Tom says that typically decryption is stateful, too.}

We define two notions of security for PKEAD schemes in
Figure~\ref{fig:pkead-sec}.
%
The first, \fwdpke, demands indistinguishibility and forward security.
The second, \bwdpke, demands indistinguishibility and backward security. In the
latter, since the state is completely exposed prior to encryption, the
$\ENCO$-oracle takes a \emph{distribution} on the nonce, associated data, and
messages as input. This way any entropy in this distribution can be leveraged
for security.

\begin{figure}[t]
  \newcommand{\rdy}{\flagfont{rdy}}
  \twoCols{0.48}
  {
    \experimentv{$\Exp{\fwdpkeX{b}}_{\pkead}(\advA)$}\\[2pt]
      $\rdy \gets \false$; $\setC \gets \emptyset$\\
      $(\pk, \sk) \getsr \pkead.\Kg$\\
      $b' \getsr \advA^{\ENCO,\DECO,\INITRO,\EXPO}(\pk)$\\
      return $b'$
    \\[9pt]
    \oraclev{$\ENCO(\nonce, \ad, \msg_0, \msg_1)$}\\[2pt]
      if $\rdy=\false \OR |\msg_0|\ne|\msg_1|$ then return $\bot$\\
      $(\cipher, \st) \gets \pkead.\Enc_\pk^{\nonce,\ad}(\msg_b, \st)$\\
      $\setC \gets \setC \union \{(\nonce, \ad, \cipher)\}$\\
      return $\cipher$
    \\[6pt]
    \oraclev{$\DECO(\nonce, \ad, \cipher)$}\\[2pt]
      if $(\nonce, \ad, \cipher) \in \setC$ then return $\bot$\\
      return $\pkead.\Dec_\sk^{\nonce,\ad}(\cipher)$
    \\[6pt]
    \oraclev{$\INITRO(\dist)$}\\[2pt]
      $\rdy \gets \true$;
      $\stsel \getsr \dist$;
      $\st \gets \Enc.\init(\stsel)$
    \\[6pt]
    \oraclev{$\EXPO(\,)$}\\[2pt]
      $\rdy \gets \false$; return $\st$
  }
  {
    \experimentv{$\Exp{\bwdpkeX{b}}_{\pkead}(\advA_1, \advA_2)$}\\[2pt]
      $\rdy \gets \false$;
      $\setC \gets \emptyset$\\
      $(\pk, \sk) \getsr \pkead.\Kg$\\
      $\st \getsr \advA_1^{\ENCO,\DECO,\INITO}$\\
      $b' \getsr \advA_2(\pk,\st)$\\
      return $b'$
    \\[6pt]
    \oraclev{$\ENCO(\srcM)$}\\[2pt]
      if $\rdy = \false$ then return $\bot$\\
      $(\vnonce, \vad, \vmsg_0, \vmsg_1) \getsr \srcM$\\
      $(\vcipher, \st) \gets \pkead.\Enc_\pk^{\vnonce,\vad}(\vmsg_b, \st)$\\
      for each $i \gets 1$ to $|\vcipher|$ do\\
        \tab $\setC \gets \setC \union \{(\vnonce_i, \vad_i, \vcipher_i\}$\\
      return $\vcipher$
    \\[6pt]
    \oraclev{$\DECO(\nonce, \ad, \cipher)$}\\[2pt]
      if $(\nonce, \ad, \cipher) \in \setC$ then return $\bot$\\
      return $\pkead.\Dec_\sk^{\nonce,\ad}(\cipher)$
    \\[6pt]
    \oraclev{$\INITO(\stsel)$}\\[2pt]
      $\rdy \gets \true$;
      $\st \gets \Enc.\init(\stsel)$
  }
  \caption{Security notion for PKEAD schemes.}
  \label{fig:pkead-sec}
  \vspace{6pt}\hrule
\end{figure}

\ignore{
  \heading{Constructions.}
  %
  \newcommand{\kem}{\schemefont{kem}}
  \newcommand{\calE}{\mathcal{E}}
  \newcommand{\calD}{\mathcal{D}}
  \cpnote{KEM (key encapsulation mechanism) + symmetric AEAD + PRNG:
    \begin{itemize}
      \item $\Kg$: $(\pk, \sk) \getsr \kem.\Kg$; return $(\pk, \sk)$
      \item $\Enc.\init$: $\st \getsr \prng.\init$; return $\st$
      \item $\Enc_\pk^{\nonce,\ad}(\msg, \st)$:
        $(K, \st) \gets \prng(\str(\nonce, \ad, \msg), k, \st)$;
        $\cipher \gets \calE_K^{\nonce,\ad}(\msg)$;
        $X \gets \kem.\Enc_\pk(K)$;
        return $\str(X, \cipher)$
      \item $\Dec_\sk^{\nonce,\ad}(\str(X, \cipher))$:
        $K \gets \kem.\Dec_\sk(X)$;
        $\msg \gets \calD_K^{\nonce,\ad}(\cipher)$;
        return $\msg$
    \end{itemize}
    (I'm not sure if I got the KEM syntax quite right.) We should aim to prove
    that $\pkead$ is exposure-resilient if and only if $\prng$ is
    exposure-resilient.
  }
  %
  \cpnote{I'd also llke to lo look into an OAEP-like construction.}
}
