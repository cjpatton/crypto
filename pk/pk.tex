%
\documentclass{build/llncs}
\usepackage{color,soul}
\usepackage[dvipsnames]{xcolor}
\usepackage{enumitem}
% header.tex
%
% Formatting and common macros for crypto papers. Include this first.
\usepackage[utf8]{inputenc}
\usepackage[margin=4cm]{geometry}
\usepackage{graphics}
\usepackage[toc,page]{appendix}
\usepackage[font={small}]{caption}
\usepackage{hyperref}
\usepackage{amsmath}
\usepackage{amsthm}
\usepackage{amsfonts}
\usepackage{parskip}
\usepackage{framed}
\usepackage{multicol}

\hypersetup{
    colorlinks,%
    citecolor=black,%
    filecolor=black,%
    linkcolor=black,%
    urlcolor=black
}

\def\dashuline{\bgroup
  \ifdim\ULdepth=\maxdimen  % Set depth based on font, if not set already
	  \settodepth\ULdepth{(j}\advance\ULdepth.4pt\fi
  \markoverwith{\kern.15em
	\vtop{\kern\ULdepth \hrule width .3em}%
	\kern.15em}\ULon}

\newcounter{foot}
\setcounter{foot}{1}
\setlength\parindent{2em}

% Editorial
\newcommand{\heading}[1]{\noindent \textsc{#1}}
\newtheorem{theorem}{Theorem}[section]

% Fonts for various types
\newcommand{\notionfont}[1]{\textup{#1}}
\newcommand{\varfont}[1]{\textit{#1}}
\newcommand{\flagfont}[1]{\mathsf{#1}}
\newcommand{\vectorfont}[1]{\vec{#1}}
\newcommand{\oraclefont}[1]{\textsc{#1}}
\newcommand{\schemefont}[1]{\mathsf{#1}}
\newcommand{\procfont}[1]{\schemefont{#1}}
\newcommand{\adversaryfont}[1]{\mathcal{#1}}
\newcommand{\cryptofont}[1]{\mathbf{#1}\hspace{0.5pt}}
\newcommand{\prinfont}[1]{\mathcal{#1}}

% Math
\DeclareMathAlphabet\mathbfcal{OMS}{cmsy}{b}{n}

% - Sets
\newcommand{\Z}{\mathbb{Z}}
\newcommand{\N}{\mathbb{N}}
\newcommand{\bits}{\{0,1\}}
\newcommand*\bigunion{\bigcup}
\newcommand*\bigintersection{\bigcap}
\newcommand*\union{\cup}
\newcommand*\intersection{\cap}
\newcommand*\cross{\times}
\newcommand*\by{\cross}
%\newcommand{\getsr}{\mathrel{\leftarrow\mkern-14mu\leftarrow}}
%\newcommand{\getsr}{\xleftarrow{\text{\tiny{\$}}}}
\newcommand{\getsr}{{\:{\leftarrow{\hspace*{-3pt}\raisebox{.75pt}{$\scriptscriptstyle\$$}}}\:}}

% - String operations
\newcommand{\emptystring}{\varepsilon}
\newcommand{\cat}{\, \| \,}
\newcommand{\str}[1]{\langle #1 \rangle}

% - Boolean operators
\newcommand*\AND{\wedge}
\newcommand*\OR{\vee}
\newcommand*\NOT{\neg}
\newcommand*\IMPLIES{\implies}
\newcommand*\XOR{\mathbin{\oplus}}
\newcommand*\xor{\XOR}

% - Crypto functions
\newcommand{\game}[1]{\cryptofont{Exp}^{\textnormal{\tiny \MakeLowercase{#1}}}}
\newcommand{\adv}[1]{\cryptofont{Adv}^{\textnormal{\tiny \MakeLowercase{#1}}}}

% - Asymptotics
\newcommand{\negl}{\procfont{negl}}
\newcommand{\poly}{\procfont{poly}}

% - Probablity
\newcommand{\E}{\mathrm{E}}
\newcommand{\Prob}[1]{\Pr\hspace{-2pt}\Big[\,#1\,\Big]}

% Games
\newcommand{\gamesfontsize}{\footnotesize}
\newcommand{\foreach}[3]{$\text{for }#1 \gets #2\text{ to }#3\text{ do}$}
\newcommand{\ind}{\hspace*{10pt}}
\newcommand{\outputs}{=}
\newcommand{\true}{1}
\newcommand{\false}{0}

% - Inline comment
\definecolor{CommentColor}{RGB}{125,175,230}
\newcommand{\comment}[1]{\textcolor{CommentColor}{\,\textbf{\#}\,#1}}

% - One game
\newcommand{\oneCol}[2]{
\begin{center}
        \framebox{
        \begin{tabular}{c@{\hspace*{.4em}}}
        \begin{minipage}[t]{#1\textwidth}\gamesfontsize #2 \end{minipage}
        \end{tabular}
        }
\end{center}
}

% - One game, two columns
\newcommand{\twoColsNoDivide}[3]{
  \providecommand{\pad}{6pt}
  \providecommand{\colwidth}{#1\textwidth}
  \makebox[\textwidth][c]{
  \begin{tabular}{|@{\hskip \pad}l@{}@{}@{\hskip \pad}l|}
    \hline
    \rule{0pt}{1\normalbaselineskip}
    \begin{minipage}[t]{\colwidth}\gamesfontsize
      #2 \vspace{\pad}
    \end{minipage} &
    \begin{minipage}[t]{\colwidth}\gamesfontsize
      #3 \vspace{\pad}
    \end{minipage} \\
    \hline
  \end{tabular}
  }
}

% - Two games, one per column
\newcommand{\twoCols}[3]{
  \providecommand{\pad}{6pt}
  \providecommand{\colwidth}{#1\textwidth}
  \makebox[\textwidth][c]{
  \begin{tabular}{|@{\hskip \pad}l@{}|@{}@{\hskip \pad}l|}
    \hline
    \rule{0pt}{1\normalbaselineskip}
    \begin{minipage}[t]{\colwidth}\gamesfontsize
      #2 \vspace{\pad}
    \end{minipage} &
    \begin{minipage}[t]{\colwidth}\gamesfontsize
      #3 \vspace{\pad}
    \end{minipage} \\
    \hline
  \end{tabular}
  }
}

% Notes
\newcounter{notectr}[section]
\newcommand{\getnotectr}{\stepcounter{notectr}\thesection.\thenotectr}
\newcommand{\basenote}[4]{{
  \textcolor{#1}{(#3#4---#2~\getnotectr)}
}}

\newcommand{\note}[3]{\basenote{#1}{#2}{}{#3}}
\newcommand{\todo}[3]{\basenote{#1}{#2}{TODO~}{#3}}

% - Chris' notes
\definecolor{darkgreen}{RGB}{50,127,0}
\newcommand{\cpnote}[1]{\note{darkgreen}{Chris}{#1}}
\newcommand{\cptodo}[1]{\todo{darkgreen}{Chris}{#1}}

% macros.tex
%
% Macros for this paper. Include headers.tex, then this file.

% Math
\newcommand{\perm}{\procfont{Perm}}

% Point function
\def\pt[#1,#2,#3]{{#1}_{#2,#3}}

% Function families
\newcommand{\fns}{\setfont{F}}
\newcommand{\dom}{\setfont{X}}
\newcommand{\rng}{\setfont{Y}}
\newcommand{\Fn}{\schemefont{Fn}}

%
\newcommand{\kk}{\vectorfont{K}}
\renewcommand{\k}{K}
\newcommand{\pub}{\procfont{pub}}

% TODO Automate \def\*x.
\newcommand{\Kg}{\schemefont{Kg}}
\newcommand{\Rep}{\schemefont{Rep}}
\newcommand{\Comb}{\schemefont{Comb}}
\newcommand{\Qry}{\schemefont{Qry}}
\def\Repx(#1,#2){\Rep_{#1}\!\left(#2\right)}
\def\Combx(#1,#2){\Comb_{#1}\!\left(#2\right)}
\def\Qryx(#1,#2){\Qry_{#1}\!\left(#2\right)}

% Adversaries
\def\advA{\adversaryfont{A}}

% Sets
%
% TODO Automate \def\set*.
\def\setS{\setfont{S}}
\def\setX{\setfont{X}}
\def\setY{\setfont{Y}}

% Vectors
\def\vv{\vectorfont{v}}

% Misc.
\def\coins{r}


\date{\today}
\title{Real-world{\color{gray}(?)} public-key cryptography\\
\emph{\color{gray}or}\\
Robust public-key cryptography with applications to multi-factor authentication
}
\author{Christopher Patton and Thomas Shrimpton and FSM}
\institute{}

\setcounter{tocdepth}{2}

\pagestyle{plain}

\begin{document}

\maketitle

\begin{abstract}
  \cite{bellare2009hedged} shows that public-key encryption can be hardened
  against RNG failures by manipulating the coins used to encrypt the message.
  %
  \cite{bellare2016nonce} changes the model for PKE so that only minimal use of
  the RNG is required. They show that encryption in this setting can be hardened
  against (partial) exposure of the encryption state.
  %
  \cite{boldyreva2017real} points out that the techniques of
  \cite{bellare2009hedged,bellare2016nonce} are impractical in the sense that
  directly manipulating the coins breaks the ``intended use'' of many crypto
  APIs, including OpenSSL.
  %
  This work adopts the perspective that good API design and sound cryptographic
  theory are inextricable.
  %
  Bearing this in mind, we consider the security of public-key cryptography
  when instantiated with real-world (P)RNGS.
\end{abstract}

\section{Introduction}
%
%
%
\renewcommand{\ng}{\schemefont{ng}}
\newcommand{\he}{\schemefont{he}}
\newcommand{\foobar}{\textsc{foobar}}
In their treatment of nonce-based, public-key
cryptography~\cite{bellare2016nonce}, Bellare and Tackmann introduce two notions
of secure encryption. \foobar!
%
The first, NBP1, extends the usual IND-CCA notion to a new setting in which
encryption is determinsitc and takes as input the public key~$\pk$ and
message~$m$, as well as a nonce~$n$ and an input~$\xk$, called the \emph{seed},
known only to the sender. Decryption has the usual syntax; it requires only the
ciphertext and secret key. The nonce~$n$ is output by a stateful (and possibly
randomized) procedure~$\ng$, called the \emph{nonce generator}. This algorithm
takes as input a string~$\sel$, called the \emph{nonce selector}. In the NBP1
setting we assume that, at a minimum, the pair $(m, \sel)$ does not repeat.
%
The second notion, NBP2, models the setting where the sender's state is
(partially) exposed to the adversary. The primary distinction between NBP1 and
NBP2 is that, in the latter, the adversary is given~$\xk$ as input. As a result,
ensuring that $(m, \sel)$ not repeat is not enough for security; the output
of~$\ng$ must also be \emph{unpredictable} to the adversary.

Their main construction involves a novel primitive~$\he$, called a \emph{hedged
extractor}. It takes as input the seed, the message~$m$, and the nonce~$n$, and
outputs a string~$r$. This string is used as the coins for encryption of~$m$
using a standard PKE scheme. This composition achieves NBP1 if~$\he$ is a PRF,
and it achieves NBP2 if~$\he$ is secure in a new sense
that~\cite{bellare2016nonce} defines.
%
Roughly speaking, this RoR (``real-or-random'') notion ensures that, if the
output of~$\ng$ is unpredictable, then the output of~$\he$ is indistinguishable
from a random string, even to an adversary in possession of the seed.

Interestingly, the composition of~$\he$ and~$\ng$ is remarkably similar to how
pseudo random number generation works in real systems. PRNGs \emph{with input},
first formalized by Barak and Halevi~\cite{barak2005model}, typically have two
interfaces: one that fetches (any number of) pseudorandom bits, and another with
which the programmer can provide additional randomness to the PRNG state. A number
of notions of security have been considered for this primitive, including some
that model exposure of the state to the adversary. The motivation for these
notions is not dissimilar to that of the RoR game described above.

\cptodo{Segway}
%
In the NBP2 game, the adversary is given the seed as input, but the state
associated with nonce generation remains hidden. In my opinion, this setting is
not clearly motivated; an adversary that is able to penetrate the sender's
system and exfiltrate the seed ought to be able to recover the~$\ng$ state.
%
Of course, secure encryption \emph{after} this state is exposed is impossible, a
limitation that~\cite{bellare2016nonce} recognizes: in the RoR game for hedged
extractors, the adversary is given access to the~$\ng$ state, but only
\emph{after} it makes its queries. Said another way, RoR security for hedged
extractors is only guaranteed for coins generated prior to state exposure.

In this work, we reconsider the adversarial model described by NBP2, opting for
a simpler notion in which the encryption state is exposed to the adversary after
it makes its encryption queries.
%
We consolidate the stateless, deterministic encryption algorithm
and the stateful nonce generator into one stateful, deterministic encryption
procedure. We do away with the seed and instead define stateful, pseudorandom
number generation as a primitive for constructing such a scheme.

Finally, our syntax and security notions are geared towards the application of
password-based authenticated key exchange. In this direction, our syntax admits
a nonce and associated data, which may be used as the session number and
password respectively.

\cptodo{Talk about why decryption should take the nonce as input. (In
\cite{bellare2016nonce}, the nonce is used only to ensure coins freshness.)}


\section{Pseudorandom number generators}
%
%
%
\label{sec:prng}
This syntax provides syntax and security notions for pseudorandom number
generators \emph{with inputs}, as first formalized by~\cite{barak2005model}. It
also borrows ideas from~\cite{dodis2013security,shrimpton2015provable}.

\begin{definition}[Entropy source]\rm
  An \emph{entropy source} is a randomized algorithm~$\dist$ with no inputs and
  that outputs a string.
  %
  \cpnote{This syntax follows \cite{barak2005model}, but diverges
  from~\cite{dodis2013security,shrimpton2015provable}, where the entropy source
  may be stateful.}
  %
  We say that~$\dist$ has min-entropy~$\mu$ if for every
  $x \in \bits^*$, it holds that
  $
  \Prob{y \getsr \dist\colon x = y} \leq 2^{-\mu}
  $. \dqed
\end{definition}
%
\cpnote{We could extend this notion along the lines of
\cite{bellare2009hedged,boldyreva2017real} so that $\dist$ outputs a vector of
strings.}
%
\cpnote{Define the maximum output length?}
%
\cpnote{Could model \emph{side information} about the output.}

\begin{definition}[PRNG]\rm
  A \emph{pseudorandom number generation} scheme, $\prng$, is a triple of
  deterministic algorithms $(\init, \add, \get)$ defined as follows:
  %
  \begin{itemize}
    \item $\init$ is the initialization algorithm. It takes as input a
      string~$\stsel$, called the \emph{initializer}, and outputs a string~$\st$
      called the \emph{state}.
      %
      This is denoted $\st \gets \init(\stsel)$.

    \item $\add$ takes a string~$\sel$ called the \emph{selector},
      %
      \cpnote{Nomenclature is borrowed from~\cite{bellare2016nonce}. There might
      be a better name.}
      %
      and the state and outputs the updated state.
      %
      This is denoted $\st \gets \add(\sel, \st)$.

    \item $\get$ takes as input an integer $\rho \geq 0$
      %
      \cpnote{This parameter does not appear in earlier work.}
      %
      and the state and
      outputs a string $\coins \in \bits^\rho$ and the updated state.
      %
      This is denoted $(\coins, \st) \gets \get(\coinslen, \st)$.
      \dqed
  \end{itemize}
\end{definition}

We define four notions of security PRNG schemes in Figure~\ref{fig:prng-sec}.
%
\indfwd and \indbwd capture \emph{indistingushibility} of the output of
$\prng.\get(\cdot)$ from random, the first in the \emph{forward-secure} sense,
and the second in the \emph{backward-secure} sense.
%
Roughly speaking, \emph{forward security} demands that if the state is exposed
to the adversary, then all prior uses of the PRNG remain secure, and
\emph{backward security} demands that if the state is exposed, then future uses
of the PRNG are secure as long as the state is refreshed.
%
\upfwd and \upbwd capture only \emph{unpredictability} of the output of
$\prng.\get(\cdot)$.
%
Something we'll need to figure out is what restrictions we need to make
on the source(s) of entropy for these notions to be achievable. For example, in
the \indfwd game, if we make no restrictions on the $\INITRO$-queries, then
there is an easy distinguishing attack.

\begin{figure}[t]
  \newcommand{\coll}{\flagfont{coll}}
  \twoColsTwoRows{0.48}
  {
    \experimentv{$\Exp{\indfwdX{b}}_\prng(\advA)$}\\[2pt]
      $\stout \gets \true$\\
      $b' \getsr \advA^{\INITRO,\EXPO,\ADDO,\GETINDO}$\\
      return $b'$
  }
  {
    \experimentv{$\Exp{\indbwdX{b}}_\prng(\advA)$}\\[2pt]
      $\stout \gets \true$\\
      $b' \getsr \advA^{\INITO,\ADDRO,\GETINDO}$\\
      return $b'$
  }
  {
    \experimentv{$\Exp{\upfwd}_\prng(\advA)$}\\[2pt]
      $\stout \gets \true$; $\coll \gets \false$;
      $\setX \gets \emptyset$\\
      $\coins \getsr \advA^{\INITRO,\EXPO,\ADDO,\GETUPO}$\\
      return $\coll \OR (\coins \in \setX)$
  }
  {
    \experimentv{$\Exp{\upbwd}_\prng(\advA)$}\\[2pt]
      $\stout \gets \true$; $\coll \gets \false$;
      $\setX \gets \emptyset$\\
      $\coins \getsr \advA^{\INITO,\ADDRO,\GETUPO}$\\
      return $\coll \OR (\coins \in \setX)$
  }
  \twoColsNoDivide{0.48}
  {
    \oraclev{$\INITRO(\dist)$}\\[2pt]
      $\stout \gets \false$;
      $\stsel \getsr \dist$;
      $\st \gets \prng.\init(\stsel)$
    \\[6pt]
    \oraclev{$\ADDRO(\dist)$}\\[2pt]
      $\stout \gets \false$;
      $\sel \getsr \dist$;
      $\st \gets \prng.\add(\st, \sel)$
    \\[6pt]
    \oraclev{$\INITO(\stsel)$}\\[2pt]
      $\stout \gets \true$;
      $\st \gets \prng.\init(\stsel)$
    \\[6pt]
    \oraclev{$\ADDO(\sel)$}\\[2pt]
      $\st \gets \prng.\add(\st, \sel)$
    \\[6pt]
    \oraclev{$\EXPO(\,)$}\\[2pt]
      $\stout \gets \true$;
      return $\st$
  }
  {
    \oraclev{$\GETINDO(\coinslen)$}\\[2pt]
      if $\stout = \true$ then return $\bot$\\
      $(\coins_1, \st) \gets \prng.\get(\coinslen, \st)$\\
      $\coins_0 \getsr \bits^\coinslen$\\
      return $\coins_b$
    \\[6pt]
    \oraclev{$\GETUPO(\coinslen)$}\\[2pt]
      if $\stout = \true$ then return $\bot$\\
      $(\coins, \st) \gets \prng.\get(\coinslen, \st)$\\
      if $\coins \in \setX$ then $\coll \gets \true$\\
      $\setX \gets \setX \union \{ \coins \}$\\
      return~$1$
  }
  \caption{\textbf{Top:} Security notions for PRNG schemes.
  %
  \textbf{Bottom:} Oracles for the security notions.}
  \vspace{6pt}\hrule
  \label{fig:prng-sec}
\end{figure}


\section{Public-key encryption with associated data}
\label{sec:pkead}
\cptodo{Define sources.}
\cptodo{The \fwdpke and \bwdpke notions do not guarantee privacy of associated
data, which is needed for the application to AKE. (See Section~\ref{sec:ake}.
How about passing $\ad_0$ and $\ad_1$ to the oracle in the game?}

\begin{definition}[PKEAD]\rm
  A \emph{public-key encryption scheme with associated data} is a 4-tuple of
  algorithms $\pkead = (\Kg, \Enc.\init, \Enc, \Dec)$ defined as follows:
  \begin{itemize}
    \item $\Kg$ is the randomized key generation algorithm that outputs a
      public/private key pair $(\pk, \sk)$.
      %
      Its execution is denoted $(\pk, \sk) \getsr \Kg$.

    \item $\Enc.\init$ is the deterministic encryption initialization
      algorithm. It takes as input an initializer~$\stsel$ and outputs the state.
      %
      Its execution is denoted $\st \gets \Enc.\init(\stsel)$.

    \item $\Enc$ is the deterministic encryption algorithm that takes as input the
      public key~$\pk$, a triple of strings $(\nonce, \ad, \msg)$, called the
      nonce, associated data, and plaintext respectively, and the state~$\st$, and
      outputs~$\cipher$, either a string or~$\bot$, and the updated state.
      %
      This is denoted $(\cipher, \st) \gets \Enc_\pk^{\nonce,\ad}(\msg, \st)$.

    \item $\Dec$ is the determinstic decryption that takes as input the secret
      key~$\sk$ and $(\nonce, \ad, \cipher)$ and outputs~$\msg$, either the
      message or $\bot$.
      %
      This is denoted $\msg \gets \Dec_\sk^{\nonce,\ad}(\cipher)$.
      \dqed
  \end{itemize}
\end{definition}
%
\cpnote{How does this syntax compare to stateful encryption as already defined
in the literature? Tom says that typically decryption is stateful, too.}

We define two notions of security for PKEAD schemes in
Figure~\ref{fig:pkead-sec}.
%
The first, \fwdpke, demands indistinguishibility and forward security.
The second, \bwdpke, demands indistinguishibility and backward security. In the
latter, since the state is completely exposed prior to encryption, the
$\ENCO$-oracle takes a \emph{distribution} on the nonce, associated data, and
messages as input. This way any entropy in this distribution can be leveraged
for security.

\begin{figure}[t]
  \newcommand{\rdy}{\flagfont{rdy}}
  \twoCols{0.48}
  {
    \experimentv{$\Exp{\fwdpkeX{b}}_{\pkead}(\advA)$}\\[2pt]
      $\rdy \gets \false$; $\setC \gets \emptyset$\\
      $(\pk, \sk) \getsr \pkead.\Kg$\\
      $b' \getsr \advA^{\ENCO,\DECO,\INITRO,\EXPO}(\pk)$\\
      return $b'$
    \\[9pt]
    \oraclev{$\ENCO(\nonce, \ad, \msg_0, \msg_1)$}\\[2pt]
      if $\rdy=\false \OR |\msg_0|\ne|\msg_1|$ then return $\bot$\\
      $(\cipher, \st) \gets \pkead.\Enc_\pk^{\nonce,\ad}(\msg_b, \st)$\\
      $\setC \gets \setC \union \{(\nonce, \ad, \cipher)\}$\\
      return $\cipher$
    \\[6pt]
    \oraclev{$\DECO(\nonce, \ad, \cipher)$}\\[2pt]
      if $(\nonce, \ad, \cipher) \in \setC$ then return $\bot$\\
      return $\pkead.\Dec_\sk^{\nonce,\ad}(\cipher)$
    \\[6pt]
    \oraclev{$\INITRO(\dist)$}\\[2pt]
      $\rdy \gets \true$;
      $\stsel \getsr \dist$;
      $\st \gets \Enc.\init(\stsel)$
    \\[6pt]
    \oraclev{$\EXPO(\,)$}\\[2pt]
      $\rdy \gets \false$; return $\st$
  }
  {
    \experimentv{$\Exp{\bwdpkeX{b}}_{\pkead}(\advA_1, \advA_2)$}\\[2pt]
      $\rdy \gets \false$;
      $\setC \gets \emptyset$\\
      $(\pk, \sk) \getsr \pkead.\Kg$\\
      $\st \getsr \advA_1^{\ENCO,\DECO,\INITO}$\\
      $b' \getsr \advA_2(\pk,\st)$\\
      return $b'$
    \\[6pt]
    \oraclev{$\ENCO(\srcM)$}\\[2pt]
      if $\rdy = \false$ then return $\bot$\\
      $(\vnonce, \vad, \vmsg_0, \vmsg_1) \getsr \srcM$\\
      $(\vcipher, \st) \gets \pkead.\Enc_\pk^{\vnonce,\vad}(\vmsg_b, \st)$\\
      for each $i \gets 1$ to $|\vcipher|$ do\\
        \tab $\setC \gets \setC \union \{(\vnonce_i, \vad_i, \vcipher_i\}$\\
      return $\vcipher$
    \\[6pt]
    \oraclev{$\DECO(\nonce, \ad, \cipher)$}\\[2pt]
      if $(\nonce, \ad, \cipher) \in \setC$ then return $\bot$\\
      return $\pkead.\Dec_\sk^{\nonce,\ad}(\cipher)$
    \\[6pt]
    \oraclev{$\INITO(\stsel)$}\\[2pt]
      $\rdy \gets \true$;
      $\st \gets \Enc.\init(\stsel)$
  }
  \caption{Security notion for PKEAD schemes.}
  \label{fig:pkead-sec}
  \vspace{6pt}\hrule
\end{figure}

\ignore{
  \heading{Constructions.}
  %
  \newcommand{\kem}{\schemefont{kem}}
  \newcommand{\calE}{\mathcal{E}}
  \newcommand{\calD}{\mathcal{D}}
  \cpnote{KEM (key encapsulation mechanism) + symmetric AEAD + PRNG:
    \begin{itemize}
      \item $\Kg$: $(\pk, \sk) \getsr \kem.\Kg$; return $(\pk, \sk)$
      \item $\Enc.\init$: $\st \getsr \prng.\init$; return $\st$
      \item $\Enc_\pk^{\nonce,\ad}(\msg, \st)$:
        $(K, \st) \gets \prng(\str(\nonce, \ad, \msg), k, \st)$;
        $\cipher \gets \calE_K^{\nonce,\ad}(\msg)$;
        $X \gets \kem.\Enc_\pk(K)$;
        return $\str(X, \cipher)$
      \item $\Dec_\sk^{\nonce,\ad}(\str(X, \cipher))$:
        $K \gets \kem.\Dec_\sk(X)$;
        $\msg \gets \calD_K^{\nonce,\ad}(\cipher)$;
        return $\msg$
    \end{itemize}
    (I'm not sure if I got the KEM syntax quite right.) We should aim to prove
    that $\pkead$ is exposure-resilient if and only if $\prng$ is
    exposure-resilient.
  }
  %
  \cpnote{I'd also llke to lo look into an OAEP-like construction.}
}


\section{The killer app: 0-RTT, password-based AKE}
%
%
%
%
\label{sec:ake}
A client has an identity~$I$ (say, her email address) and a password~$W$ and
would like to authenticate herself to---and exchange a key with---a server in
possession of a public/private key pair $(\pk,\sk)$.
%
Let $h\geq0$ be an integer and $H \colon \bits^* \to \bits^h$ be a cryptographic
hash function. Suppose that, out-of-band, the client is furnished with~$\pk$ and
the server with~$I$ and $\ad= H(I \cat W)$.
%
Consider the following key exchange protocol for session number $\nonce$:
\begin{itemize}
  \item $\cli^{\,\pk,I,W,\nonce}$:
    Run $K \getsr \setK$;
    $\ad \gets H(I \cat W)$;
    $(\cipher, \st) \gets \Enc_\pk^{\nonce,\ad}(K, \st)$; and
    send $(I, \nonce, \cipher)$ to \srv.

  \item $\srv^{\,\sk,U}$ on input $(I, \nonce', \cipher)$:
    Look up~$(\ad, \nonce) \gets U[I]$;
    if $\nonce \geq \nonce'$ then reject;
    run $K \gets \Dec_\sk^{\nonce',\ad}(\cipher)$;
    if $K \ne \bot$, then let $U[I] \gets (\ad, \nonce')$ and accept; otherwise reject.
\end{itemize}
%
An adversary in possession of~$\ad$ can easily impersonate the client. Hence
this simple protocol works only if the server is never compromised.
%
\cpnote{Yet, in the setting where the server has a public/private key pair and
the client just has just a password, is there any solution that does any better
when the server is compromised? As far as I can tell, the protocol
of~\cite{bellare2000authenticated} doesn't fair much better if the server is
compromised. (Although this paper might be a bit out-dated ... I don't know this
literature very well.) I think the only way to do better is if the client also
has a public/private key pair.}
%
Rotating the password is similarly easy as pie:
%
\begin{itemize}
  \item $\cli^{\,\pk,I,W,W',\nonce}$:
    Run $\ad \gets H(I \cat W)$; $\ad' \gets H(I \cat W')$;
    $(\cipher, \st) \gets \Enc_\pk^{\nonce,\ad}(\ad', \st)$; and
    send $(I, \nonce, \cipher)$ to \srv.

  \item $\srv^{\,\sk,U}$ on input $(I, \nonce', \cipher)$:
    Look up~$(\ad, \nonce) \gets U[I]$;
    if $\nonce \geq \nonce'$ then reject;
    run $\ad' \gets \Dec_\sk^{\nonce',\ad}(\cipher)$; and
    if $\ad' \ne \bot$, then let $U[I] \gets (\ad', \nonce')$ and accept;
    otherwise reject.
\end{itemize}


\section{Digital signatures}
%
%
%
Following~\cite{bellare2016nonce} we extend our framework to digital signatures.
\begin{definition}[DS]\rm
  A digital signature scheme~$\ds$ is a 4-tuple of algorithms $(\Kg,\\ \Sgn.\init, \Sgn,
  \Vfy)$ defined as follows:
  %
  \begin{itemize}
    \item $\Kg$ is the randomized key generation algorithm.

    \item $\Sgn.\init$ is the deterministic signing state initialization
      algorithm. It takes as input an initializer~$\stsel$ and returns the
      state. This is denoted $\st \gets \Sgn.\init(\stsel)$.

    \item $\Sgn$ is the deterministic signing algorithm, taking as input the
      secret key~$\sk$, the message~$\msg$, and the current state, and
      outputting~$\sig$, either the signature or~$\bot$, and the updated state.
      This is denoted $(\sig, \st) \gets \Sgn_\sk(\msg, \st)$.

    \item $\Vfy$ is the deterministic verification algorithm, taking as input the
      public key~$\pk$, the message~$\msg$, and the signature~$\sig$ and
      outputting $v$, either a bit indicating whether the signature is valid,
      or~$\bot$. This is denoted $v \gets \Vfy_\pk(\msg, \sig)$.
  \end{itemize}
  %
  \cpnote{$\Sgn$ and $\Vfy$ could use a nonce, but I don't see what the practical
  benefit would be. Though, doesn't DJB's Poly1305 MAC use nonce? What's the
  reasonf or this?}
  \dqed
\end{definition}

\heading{\ufsig.}
%
Unforgeability of signature schemes is defined in Figure~\ref{fig:ufsig}. As
usual, we consider a setting in which the signing state is eventually exposed to
the adversary.
%
A reasonable criticism of this notion is that if the adversary is able to
exfiltrate the signing state, then it ought to be able to exfiltrate the signing
key. To this point, \cite{bellare2016nonce} argues that the signer might take
extra care in storing the key, but choose to store the signing state (i.e., the
seed and nonce generation state in their setting) in a less secure part of the
signer's system. For example, the long-term signing key might be stored in an
HSM (``hardware security module'') and the short-term signing state in main
memory.
%
\cptodo{Look at attestation/signing APIs for SGX. Are the signing algorithms
randomized, and if so, where do the coins come from?}

\begin{figure}[t]
  \twoColsNoDivide{0.48}
  {
    \experimentv{$\Exp{\ufsig}_{\ds}(\advA,\dist)$}\\[2pt]
      $\setQ \gets \emptyset$\\
      $\stsel \getsr \dist$;
      $\st \gets \Sgn.\init(\stsel)$\\
      $(\pk, \sk) \getsr \Kg$\\
      $(\msg, \sig) \getsr \advA^{\OO}(\pk)$\\
      return $\Vfy_\pk(\msg, \sig)$ and $\msg \not\in \setQ$
  }
  {
    \oraclev{$\SGNO(\msg)$}\\[2pt]
      $\setQ \gets \setQ \union \{\msg\}$\\
      $(\sig, \st) \gets \Sgn_\sk(\msg, \st)$\\
      return $\sig$
    \\[6pt]
    \oraclev{$\INITO(\stsel)$}\\[2pt]
      $\st \gets \pkead.\Enc.\init(\stsel)$
    \\[6pt]
    \oraclev{$\EXPO$}\\[2pt]
      return $\st$
  }
  \caption{Security notion for digital signature schemes. Let $\OO =
  (\INITO,\EXPO,\SGNO)$.}
  \label{fig:ufsig}
  \vspace{6pt}\hrule
\end{figure}

\heading{Constructions.}
%
\cpnote{The solution that~\cite{bellare2016nonce} had in mind will work here.
That is, use a PRNG to generate coins for a standard, randomized digital
signature scheme.}


\section{0-RTT 2F-AKE}
%
%
%
\newcommand{\ek}{\varfont{ek}}
\newcommand{\dk}{\varfont{dk}}
\newcommand{\vk}{\varfont{vk}}
\label{sec:ake2f}
Let $H\colon\bits^* \to \bits^h$ be a hash function.
%
The client has a signing key~$\sk$, say, stored on a
YubiKey,\footnote{\url{https://yubico.com}} and the server has the
corresponding verifying key~$\vk$.
%
The client also has an identity~$I$ and a password~$W$, and the server has~$I$
and $\ad = H(I \cat W)$.
%
The server has a decrypting key~$\dk$ and the client has the corresponding
encrypting key~$\ek$.
%
Consider the following key exchange protocol for session number~$\nonce$:
\begin{itemize}
  \item $\cli^{\,\sk,\ek,I,W,\nonce}$:
    Run $K \getsr \setK$;
    $\ad \gets H(I \cat W)$;
    $(\cipher, \st_E) \gets \Enc_\ek^{\nonce,\ad}(K, \st_E)$;
    $(\sig, \st_S) \gets \Sgn_\sk(\cipher, \st_S)$; and
    send $(I, \nonce, \cipher, \sig)$ to \srv.

  \item $\srv^{,\vk,\dk,U}$ on input $(I, \nonce', \cipher, \sig)$:
    Look up $(\ad, \nonce) \gets U[I]$;
    if $\nonce \geq \nonce'$ then reject;
    if $\Vfy_\vk(C, T)=\false$ then reject;
    run $K \gets \Dec_\dk^{\nonce',\ad}(\cipher)$;
    if $K \ne \bot$ then let $U[I] \gets (\ad, \nonce')$ and accept; otherwise
    reject.
\end{itemize}
%
This is an improvement on the protocol in Section~\ref{sec:ake} in that if the
server is compromised, then adversary also needs to exfiltrate the client's
signing key in order to impersonate her.
%
\cpnote{Alternatively we could express this protocol in terms of a dedicated
\emph{authenticated encryption} scheme, where both sender and receiver has a
secret key.}


\section{Notes}
\cpnote{Mihir pointed out that the MM attack setting is unnecessarily weak, in
that the randomness source could be stateful. We will \emph{not} address this in
the full version of~\cite{boldyreva2017real}, but will mention it here in
related work.}

\heading{Mihir's feedback on \cite{boldyreva2017real}.}
%
\newenvironment{displayquote}
{ \indent
  \footnotesize\color{gray}
  \begin{tabular}{|@{\hspace{4pt}}p{10cm}}
}
{
  \end{tabular}\\[6pt]
}

\begin{displayquote}
  Your motivation was the way crypto libraries treat encryption and RNGs. Why not
  address that directly? This means the model allows a stateful algorithm~$R$
  that via $(r,s) \getsr R(s)$ produces coins~$r$ while updating its state
  to~$s$. We could have oracles that allow the caller to change the state s, or
  mix something into it, reflecting what you say happens in the libraries. A
  simple definition in this model is a game just like IND-CPA/CCA (messages, not
  message vectors, and no entropy requirement on these) except that coins are
  created, for each message, via $(r,s) \getsr R(s)$. This is stronger than
  MM-CPA/CCA because coins can be related even across adaptive queries. I would
  guess/hope that OAEP continues to be secure. Then one can also consider how
  message entropy can be factored in.
\end{displayquote}
%
\cpnote{}
This is addressed in Sections~\ref{sec:prng} and~\ref{sec:pkead}, but with a few
differences.
%
One, $R$ takes as input a selector~$\sel$. This idea is inspired from the
\emph{nonce selector} of~\cite{bellare2016nonce}.
%
Two, $R$ is deterministic instead of randomized. (It has a randomized
initialization algorithm.)
%
Three, encryption is stateful; I'm thinking of a stateful PRNG as away to
instantiate the encryption scheme. This abstraction is in keeping with the theme
of API driven cryptography.

\begin{displayquote}
  DSA/Schnorr/PSS are randomized signature schemes but in practice people like
  to implement the first two, at least, by deterministically deriving coins by
  hashing the secret key and message. This is analyzed
  in~\cite{bellare2016nonce}. Some schemes like Ed25519 directly implement it.
  But what about signature interfaces provided by the libraries you surveyed? Do
  they let the signer pick the coins?  If not, what do you do? Could we look at
  the libraries and see?
\end{displayquote}
%
\cptodo{Look at digital signatures offered by OpenSSL, PyCrypto, golang/crypto,
etc. Also, what are the interfaces like for HSMs, e.g. YubiKey and SGX? Note
that ECDSA requires a random nonce.}

\begin{displayquote}
  I'm dubious about what MM for OAEP buys, for the following reason. My sense is
  that RNGs $R$ would usually output both~$r$ and~$s$ to be results of applying some
  hash function to some stuff that includes~$s$. This means that either (1) $r$ is
  (indistinguishable from) random, or (2) $r$ is predictable. If (1), MM is not
  needed. If (2), it does not help. In other words my worry is that MM addresses
  the case that~$r$ is unpredictable but not random, and this case does not arise,
  because of the way RNGs work. To assess stuff like this it would be good to
  know more about how the RNGs actually work.
\end{displayquote}
%

\begin{displayquote}
  Another question is, what about using nonce-based PKE as per
  \cite{bellare2016nonce}? One issue is that the sender must maintain state. Is
  that a problem? I'd imagine not, since there is so much static stuff it needs
  to maintain anyway, like its secret key or other people's public keys, but I'd
  be interested to know how this works for implementation. The other issue is
  that the solutions of \cite{bellare2016nonce} again fail to conform to crypto
  library interfaces the way you want them to, so one might ask if there are
  definitions or schemes for nonce-based PKE that are crypto-library friendly.
\end{displayquote}
%
\cpnote{}
See the introduction and Section~\ref{sec:pkead}.

\begin{displayquote}
  A more ambitious question is the following. All definitions so far give no
  security unless there is enough entropy in something (messages, randomness,
  both). In practice, the way RNGs work, one might at some point have low
  entropy (and encryption is insecure) but then entropy returns because the
  entropy pool is refurbished. This would mean encryption security returns. Can
  we give definitions that capture this type of self-healing property, where,
  for some messages, encryption is not secure, but then it becomes secure again?
  It seems to be what really happens.
\end{displayquote}
%

\begin{displayquote}
  No skin off my back, but I can see Shoup or others in the community unhappy
  about your rebranding of labels as AD. If you must change the name, I'd first
  clearly say that the common term is labels, as introduced by Shoup, and then
  say you are changing the name. After all, even the standards use the term
  labels.  Right now it looks like you claim to introduce this concept and then
  as an afterthought say that it existed under another name.
\end{displayquote}


\bibliography{pk}
\bibliographystyle{alpha}

\end{document}
