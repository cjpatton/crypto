\cpnote{Mihir pointed out that the MM attack setting is unnecessarily weak, in
that the randomness source could be stateful. We will \emph{not} address this in
the full version of~\cite{boldyreva2017real}, but will mention it here in
related work.}

\heading{Mihir's feedback on \cite{boldyreva2017real}.}
%
\newenvironment{displayquote}
{ \indent
  \footnotesize\color{gray}
  \begin{tabular}{|@{\hspace{4pt}}p{10cm}}
}
{
  \end{tabular}\\[6pt]
}

\begin{displayquote}
  Your motivation was the way crypto libraries treat encryption and RNGs. Why not
  address that directly? This means the model allows a stateful algorithm~$R$
  that via $(r,s) \getsr R(s)$ produces coins~$r$ while updating its state
  to~$s$. We could have oracles that allow the caller to change the state s, or
  mix something into it, reflecting what you say happens in the libraries. A
  simple definition in this model is a game just like IND-CPA/CCA (messages, not
  message vectors, and no entropy requirement on these) except that coins are
  created, for each message, via $(r,s) \getsr R(s)$. This is stronger than
  MM-CPA/CCA because coins can be related even across adaptive queries. I would
  guess/hope that OAEP continues to be secure. Then one can also consider how
  message entropy can be factored in.
\end{displayquote}
%
\cpnote{}
This is addressed in Sections~\ref{sec:prng} and~\ref{sec:pkead}, but with a few
differences.
%
One, $R$ takes as input a selector~$\sel$. This idea is inspired from the
\emph{nonce selector} of~\cite{bellare2016nonce}.
%
Two, $R$ is deterministic instead of randomized. (It has a randomized
initialization algorithm.)
%
Three, encryption is stateful; I'm thinking of a stateful PRNG as away to
instantiate the encryption scheme. This abstraction is in keeping with the theme
of API driven cryptography.

\begin{displayquote}
  DSA/Schnorr/PSS are randomized signature schemes but in practice people like
  to implement the first two, at least, by deterministically deriving coins by
  hashing the secret key and message. This is analyzed
  in~\cite{bellare2016nonce}. Some schemes like Ed25519 directly implement it.
  But what about signature interfaces provided by the libraries you surveyed? Do
  they let the signer pick the coins?  If not, what do you do? Could we look at
  the libraries and see?
\end{displayquote}
%
\cptodo{Look at digital signatures offered by OpenSSL, PyCrypto, golang/crypto,
etc. Also, what are the interfaces like for HSMs, e.g. YubiKey and SGX? Note
that ECDSA requires a random nonce.}

\begin{displayquote}
  I'm dubious about what MM for OAEP buys, for the following reason. My sense is
  that RNGs $R$ would usually output both~$r$ and~$s$ to be results of applying some
  hash function to some stuff that includes~$s$. This means that either (1) $r$ is
  (indistinguishable from) random, or (2) $r$ is predictable. If (1), MM is not
  needed. If (2), it does not help. In other words my worry is that MM addresses
  the case that~$r$ is unpredictable but not random, and this case does not arise,
  because of the way RNGs work. To assess stuff like this it would be good to
  know more about how the RNGs actually work.
\end{displayquote}
%

\begin{displayquote}
  Another question is, what about using nonce-based PKE as per
  \cite{bellare2016nonce}? One issue is that the sender must maintain state. Is
  that a problem? I'd imagine not, since there is so much static stuff it needs
  to maintain anyway, like its secret key or other people's public keys, but I'd
  be interested to know how this works for implementation. The other issue is
  that the solutions of \cite{bellare2016nonce} again fail to conform to crypto
  library interfaces the way you want them to, so one might ask if there are
  definitions or schemes for nonce-based PKE that are crypto-library friendly.
\end{displayquote}
%
\cpnote{}
See the introduction and Section~\ref{sec:pkead}.

\begin{displayquote}
  A more ambitious question is the following. All definitions so far give no
  security unless there is enough entropy in something (messages, randomness,
  both). In practice, the way RNGs work, one might at some point have low
  entropy (and encryption is insecure) but then entropy returns because the
  entropy pool is refurbished. This would mean encryption security returns. Can
  we give definitions that capture this type of self-healing property, where,
  for some messages, encryption is not secure, but then it becomes secure again?
  It seems to be what really happens.
\end{displayquote}
%

\begin{displayquote}
  No skin off my back, but I can see Shoup or others in the community unhappy
  about your rebranding of labels as AD. If you must change the name, I'd first
  clearly say that the common term is labels, as introduced by Shoup, and then
  say you are changing the name. After all, even the standards use the term
  labels.  Right now it looks like you claim to introduce this concept and then
  as an afterthought say that it existed under another name.
\end{displayquote}
