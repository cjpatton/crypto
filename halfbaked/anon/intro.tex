% intro.tex
%
%
\label{sec:intro}
\newcommand{\pta}{\pt[P,X,Y]} A point function is a function $\pta$ such that
$\pta(X') = Y$ if $X' = X$ and $\pta(X') = 0$ otherwise.
%
Gilboa and Ishai introduced the notion of distributed point functions (DPFs) at
Eurocrypt'14~\cite{gilboa2014distributed} and demonstrated its applicability to
private information retrieval and
storage~\cite{chor1998private,ostrovsky1997private}.
%
The idea is to split the function into a sequence of shares so that~$\pta$ can
be computed given all of the shares, but cannot be deduced from a subset of
them.
%
At Oakland'15, Corrigan-Gibbs, Boneh, and Mazi\`{e}res presented an application
of DPFs to secure messaging (i.e., anonymous
communication~\cite{chaum1981untraceable,dingledine2004tor}) called
Riposte~\cite{corrigangibbs2015riposte}.
%
They introduced a novel security property of DPFs, called \emph{disruption
resistance}, later formalized by Boyle, Gibloa, and
Ishai~\cite{boyle2016function} as \emph{verifiability} in the more general
setting of function secret sharing (FSS)~\cite{boyle2015function}.

Using verifiable FSS to construct secure messaging systems has two limitations:
%
(1) Security is predicated on an assumption of \emph{non-collusion} among the
servers; and (2) Verifying the secret shares requires a multi-party computation
executed by the servers.
%
I present a new approach in which trust is distributed among the clients instead
of the servers. Detecting disruption is simpler in this setting (requiring no
MPC), but identifying the offending client appears to be impossible.

\noindent
\cpnote{Before diving too deep, I need to do some reading.
  \begin{itemize}
    \item I'll need to survey recent work on secure messaging. First, look at
      \emph{Riffle} and figure out what that dude is up to these days.
    \item Look at children of~\cite{boyle2015function,boyle2016function}. What are
      those authors up to?
    \item Learn about homomorphic encryption.
    \item Look at key-homomorphic PRFs.
  \end{itemize}
  Some basic questions to address:
  \begin{itemize}
    \item How to do key distribution?
    \item What does ``disruption'' mean?
  \end{itemize}
}
