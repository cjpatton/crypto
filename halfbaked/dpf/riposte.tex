% riposte.tex
%
% Specification of Riposte protocols.
\label{sec-riposte}

\subsection{The $2$-server+auditor protocol}
\begin{figure}
\twoColsNoDivide{0.45}
{
  \underline{algorithm $\gen(\idx, \msg)$}\\[2pt]
  if $\msg \not\in \msgsp \OR \idx \not\in [\dblen]$ then return $\bot$\\
  $(i, j) \gets (\lceil \idx/y \rceil, \idx - \lfloor \idx/y \rfloor)$\\
  $r_A \getsr \bits^x$; $r_B \gets r_A \xor e_i$\\
  $\vecs_A \getsr (\bits^k)^x$ \\
  $\vecs_B \gets \vecs_A$; $\vecs_B[i] \getsr \bits^x$ \\
  $\vecx_A \gets \prg(\vecs_A[i])$; $\vecx_B \gets \prg(\vecs_B[i])$ \\
  $\vecv \gets \msg\cdot \vece_j + \vecx_A + \vecx_B$ \\
  $X_A \gets \str{r_A, \vecs_A, \vecv}$; $X_B \gets \str(r_B, \vecs_B, \vecv)$\\
  return $(X_A, X_B)$
}
{
  \underline{algorithm $\eval(X, \idx)$}\\[2pt]
  if $\idx \not\in [\dblen]$ then return $\bot$\\
  $(i, j) \gets (\lceil \idx/y \rceil, \idx - \lfloor \idx/y \rfloor)$\\
  $\str(r, \vecs, \vecv) \gets X$\\
  $\vecx \gets \prg(\vecs[i])$\\
  if $r[i] = 1$ then return $\vecv[j] + \vecx[j]$\\
  else return $\vecx[j]$
}
\caption{The 2-share DPF scheme of \cite{riposte} constructed from a PRG
  $\prg$.}
\label{fig-riposte-2share}
\end{figure}
We specify the construction of \cite{riposte} of a 2-share DPF scheme from a
PRG. Let $\dpf = (\gen, \eval)$ be the DPF scheme defined in
figure~\ref{fig-riposte-2share} with parameters $(\ring, \sharect, \dblen, x,
y)$ where $x$ and $y$ are positive powers of 2 and $xy \ge \dblen$.
We fix some encoding of each abstract point in $\ring$ as an $n$-bit string for
$n \in \N$.
Let $\vece_i$ denote the
$y$-vector over $\ring$ with $0$ in each position except for the $i$-th, which is 1.
Let $e_i$ denote the $x$-bit string with 0's everyone except for the $i$-th bit.
Let $\prg : \bits^x \to \bits^{yn}$ be a function.
When we write $\vecx \gets w$ where $w \in \bits^{yn}$, we mean divide $w$ into
a sequence of $n$-bit strings and map each string to its corresponding point in
$\ring$.

\heading{Optimal choice for $x$ and $y$.}
According to \cite{riposte}, the length of the keys is $x(s+1) + yn$. The
optimal values for $x$ and $y$ are $x = c\sqrt{\dblen}$ and $y =
c^{-1}\sqrt{\dblen}$ where $c = \sqrt{n/(1+s)}$. Hence, the share size is
$O(\sqrt{\dblen})$.

\heading{Verifying the write shares.}
\begin{figure}
  \oneCol{0.90}{

  \underline{proto $\verifykey(X_A, X_B)$}\\
  \vspace{-2pt}
  \begin{enumerate}[leftmargin=*]
    \item Server $A$: $\str(r_A, \vecs_A, \vecv) \gets X_A$; for each $i \in [x]$,
      do $\vect_A[i] \gets r_A[i] \cat \vecs_A[i]$. Server $B$ does the same
      with its input. Execute $\almosteq(\vect_A, \vect_B)$. If the result is
      $\reject$, then $A$ and $B$ output $\reject$.

    \item Server $A$: $\vecu_A \gets \sum_{i=1}^x \prg(\vecs_A[i])$.
      Server $B$: $\vecu_B \gets \vecv + \sum_{i=1}^x \prg(\vecs_B[i])$.
      Execute $\almosteq(\vecu_A, \vecu_B)$. Both $A$ and $B$ output the result.
  \end{enumerate}

  \vspace{2pt}
  \underline{proto $\almosteq(\vecv_A, \vecv_B)$}\\
  \vspace{-2pt}
  \begin{enumerate}[leftmargin=*]
    \item Let $m = |\vecv_A| = |\vecv_B|$. Servers $A$ and $B$ engage in a
      coin-flipping protocol \cite{telephone} in order to establish a shared
      $(mk + \lg m)$-bit string $R$. Let $(K_1, \ldots, K_m)$ denote the first
      $m$ $k$-bit chunks of $R$ and let $F$ denote the last $(\lg m)$-bit chunk.

    \item Server $A$: for each $i \in [m]$, let $m_i = \hash(K_i, \vecv_A[i])$.
      Let $f \in [m]$ denote the positive integer encoded by $F$. Send
      $(m_f, m_{f+1}, \ldots,$ $m_1, \ldots m_{f-1})$ to $C$. Server $B$ does
      the same.

    \item Auditor $C$ checks that the sequence of messages received from $A$ and $B$
      differ by exactly one element. Output $\accept$ if this holds and
      $\reject$ otherwise.
  \end{enumerate}
  \vspace{1pt}
  }
  \caption{A 3-party protocol for verifying the write shares are well-formed. The
  function $\hash$ is an $\epsilon$-almost-universal hash function with key
  space $\bits^k$.}
  \label{fig-riposte-2server+auditor}
\end{figure}
The request verification protocol for $\dpf$ suggested by \cite{riposte} is
specified in figure~\ref{fig-riposte-2server+auditor}. The principals are $A$,
$B$, and $C$ where the $A$ is the first write server, $B$ the second, and $C$ is
the auditor.
Our presentation differs from \cite{riposte} in two respects.
One, a step of the protocol requires that the write servers perform a
coin-flipping protocol in order sample from a family of pairwise-independent
universal hash functions.  However, in their implementation, they instantiate
this process by sharing keys for the Poly1305 almost-universal hash function
designed by Dan Bernstein \cite{poly1305}. We make this explicit in the
presentation.  Let $k,t \in \N$, $\epsilon \in (0,1]$, and $\hash : \bits^k \by
\bits^* \to \bits^t$ be an $\epsilon$-almost universal hash function.
Two, we require that $x$ and $y$ be powers of 2. This ensures that we always
perform the coin flipping protocol \cite{telephone} a finite number of times.

\heading{A complexity improvement.}
Implementing the coin-flipping protocol would be prohibitively expensive. In
practice, one would replace this protocol with a PRG, whose seed is shared by
$A$ and $B$,\cpnote{This was suggested to me by Henry.} but this makes the
analysis much more complex. Instead, we modify the protocol to use a PRF whose
key is shared by $A$ and $B$.

\noindent\hl{Things to do here:}
\begin{itemize}
  \item Specify the modified protocol $\reqvfy$. Show that it is complete.

  \item Show that $\reqvfy$ is disruption resistant if the underlying PRF
    satisfies its security notion.

  \item Show that the composition of $\dpf$ and $\reqvfy$ is PRIV2 secure for
    $t=1$ if the underlying primitives (a PRG and a PRF) satisfy their
    respective security notions.
\end{itemize}

\subsection{The $s$-server protocol}
\noindent\hl{Things to do here:}
\begin{itemize}
  \item Specify their $s$-share DPF scheme and $s$-server request verification
    protocol. This involves something they call a seed-homomorphic PRG.
\end{itemize}
