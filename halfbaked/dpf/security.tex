% security.tex
%
% Definition of PRIV1 and PRIV2.
\label{sec-security}

In this section, we model a coalition of write servers attempting to learn
something about the message or index chosen by the client. We give two notions
in figure~\ref{fig-priv}.
The first applies to DPF schemes and the second to verifiable DPF schemes.
\begin{figure}[t]
  \newcommand{\ctr}{\flagfont{ctr}}
  \twoCols{0.45}
  {
    \underline{$\game{\privone}_{\dpf,t}(\advA,\advS)$}\\[2pt]
    $b \getsr \bits$\\
    $b^\prime \getsr \advA^{\GENO,\CORO}$\\
    return $b=b^\prime$
    \\[6pt]
    \underline{$\GENO(\idx,\msg)$}\\[2pt]
    if $C = \bot$ then return $\bot$\\
    if $\msg \not\in \msgsp \OR \idx \not\in [\dblen]$ then return $\bot$\\
    if $b=1$ then $(X_1, \ldots, X_\sharect) \getsr \gen(\idx,\msg)$\\
    else $(X_1, \ldots, X_\sharect) \getsr \advS(C)$\\
    return $(X_i)_{i\in C}$
    \\[6pt]
    \underline{$\CORO(C^\prime)$}\\[2pt]
    if $C \ne \bot \OR C^\prime \not\subseteq [\sharect] \OR |C^\prime| > t$
      then\\\tab return $\bot$\\
    $C \gets C^\prime$\\
  }
  {
    \underline{$\game{\privtwo}_{\dpf,\reqvfy,t}(\advA, \advS)$}\\[2pt]
    $(K_u)_{u\in P} \getsr \kg$\\
    $b \getsr \bits$\\
    $b^\prime \getsr \advA^{\GENO,\ENQO,\DEQO,\CORO}$\\
    return $b=b^\prime$
    \\[6pt]
    \underline{$\GENO^i(\idx,\msg)$}\\[2pt]
    if $C = \bot$ then return $\bot$\\
    if $\msg \not\in \msgsp \OR \idx \not\in [\dblen]$ then return $\bot$\\
    if $b=1$ then $(X_1, \ldots, X_\sharect) \getsr \gen(\idx,\msg)$\\
    else $(X_1, \ldots, X_\sharect) \getsr \advS(C)$\\
    \foreach{j}{1}{S}\\
    \tab $Q_{\writer_j}^i \gets Q_{\writer_j}^i.\enqueue(X_j, \client)$\\
    return $(X_u)_{u\in C}$
    \\[6pt]
    \underline{$\ENQO_{y,x}^i(X)$}\\[2pt]
      $Q_y^i.\enqueue(X, x)$
    \\[6pt]
    \underline{$\DEQO_{y}^i()$}\\[2pt]
      $(X, x) \gets Q_y^i.\dequeue()$\\
      $(Y, z, \verdict, \st_y^i) \getsr \vfy_{y,x}^i(K_y, X, \st_y^i)$\\
      if $C \ne \bot \AND v \in C$ then return $(Y, z, \verdict)$\\
      else if $z \ne \bot$ then $Q_z^i.\enqueue(Y, y)$\\
      return $(\bot, z, \verdict)$
    \\[6pt]
    \underline{$\CORO(C^\prime)$}\\[2pt]
    if $C \ne \bot \OR \client \in C^\prime$ then return $\bot$\\
    if $|C^\prime \intersection \{\writer_j\}_{j\in[\sharect]}| > t$ then return $\bot$\\
    $C \gets C^\prime$\\
    return $(K_u)_{u\in C}$
  }
  \caption{Security notions for \textbf{(left)} DPF scheme $\dpf = (\gen,
  \eval)$ with parameters $(\ring, \sharect, \dblen)$ where $\msgsp = \ring
  \setminus \{0\}$, and
  \textbf{(right)} verifiable DPF scheme $\dpf$ with write share verification
  protocol $\reqvfy = (\kg, \vfy)$ executed with principals $P \supseteq \{\client, \writer_1,
  \ldots, \writer_\sharect\}$.}
  \vspace{6pt}\hrule
  \label{fig-priv}
\end{figure}

\heading{\privone.}
We describe the simulation-based notion of \cite{riposte,dpf}, which models the
ability of a coalition of malicious write servers to distinguish their subset of
the key shares from the output of a simulator. Let $\dpf = (\gen, \eval)$ be a
standard DPF scheme with parameters $(\ring, \sharect, \dblen)$. Let $\msgsp = R
\setminus \{0\}$. Refer to the experiment in the left-hand side of
Figure~\ref{fig-priv} associated to~$\dpf$, adversary~$\advA$,
simulator~$\advS$, and coalition threshold $t \in \N$ where $1 < t < \sharect$.
We define the advantage of~$\advA$ in attacking~$\dpf$ in the game instantiated
with simulator $\advS$ as
\[
  \adv{\privone}_{\dpf,t}(\advA,\advS) = 2 \cdot
  \Prob{ \game{\privone}_{\dpf,t}(\advA,\advS) \outputs \true } - 1.
\]
One might informally we say that~$\dpf$ is $(\sharect,t)$-\privone~secure if for
every ``reasonable'' adversary~$\advA$, there exists an ``efficient''
simulator~$\advS$ such that $\adv{\privone}_{\dpf,t}(\advA,\advS)$ is ``small''.
We will forego a rigorous definition and instead give concrete security results.

Achieving security in this sense does not ensure privacy if the write servers
engage in a write share verification protocol.  We put forward a unified
approach based on the communication model described in section~\ref{sec-com}.

\heading{\privtwo.}
Consider the experiment on the right hand side of Figure~\ref{fig-priv}
associated to DPF scheme $\dpf$, write share verification protocol $\reqvfy =
(\kg, \vfy)$ with principals $P \supseteq \{\client, \writer_1, \ldots
\writer_s\}$, coalition threshold~$t$, adversary~$\advA$, and simulator~$\advS$.
%
Just as before, the goal of the adversary is to distinguish its subset of the
write shares from the output of the simulator.
%
At the beginning of the game,
the initialization algorithm~$\kg$ is executed and a random bit~$b$ is chosen.
%
When $\advA$ asks $(i, \idx, \msg)$ of~$\GENO$, if $b=1$ then the write share
generation algorithm is run; otherwise the simulator is run on. Let $(\shares)$
denote the output. Next, the output $X_j$ is added to the top of the queue of
write server $\writer_j$ in session $i$ for each $j\in[\sharect]$. Finally, the
adversary is given the shares corresponding to servers it corrupts.
%
Corruptions are made by querying~$\CORO$ with a subset of~$P$. The coalition is
non-adaptive, meaning the adversary chooses which servers it controls before it may
query~$\GENO$. It may corrupt any set of principals that does not include the
client~$\client$ or more than~$t$ write servers. It is given the long-term input
of each principal it corrupts.
%
The adversary is given a~$\ENQO$ oracle and a~$\DEQO$ oracle, which have the
same semantics as in the \protosec~game defined in figure~\ref{fig-proto}. It is
on path to any principal it corrupts.
%
After interacting with its oracles, the adversary outputs a bit~$b^\prime$. The
output of the game is the predicate $b=b^\prime$. We define the advantage of
$\advA$ against $\dpf$ and $\reqvfy$ in the game instantiated with simulator
$\advS$ as
\[
  \adv{\privtwo}_{\dpf,\reqvfy,t}(\advA,\advS) =
  2 \cdot \Prob{\game{\privtwo}_{\dpf,\reqvfy,t}(\advA,\advS) \outputs \true} - 1.
\]

\heading{Non-colluding auditor.}
The 2-server DPF scheme of \cite{riposte} is accompanied by a request
verification protocol with principals $\{\client, \auditor, \writer_1,
\writer_2\}$, where $\auditor$ is called the \emph{auditor}. The auditor is not
trusted more than $\writer_1$ or $\writer_2$, but the non-collusion assumption
is stronger than usual: namely, that no two servers among $\{\auditor, \writer_1,
\writer_2\}$ collude. To capture this special case, we let $t=1$. Note,
however, that our model cannot be used to capture the general case where any set
of $t$ of $\sharect$ write serves may collude, but no write server colludes with
$\auditor$. For $\sharect>2$ and $t>1$, we assume the auditor may collude.
