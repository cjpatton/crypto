% dpf.tex
%
% Syntax, definitions, and applications of verifiable DPF schemes.

In this section, we give syntax and security notions for distributed point
functions. We diverge somewhat from the syntax of \cite{dpf} as it will be
convenient in our setting to assume the domain have algebraic
structure.\cpnote{Is this actually more convenient? Since our constructions are
mostly based on pseudorandom generators, it might make more sense to think of
the domain as strings. Yet there's the $\sharect$-server Riposte variant to
consider.}
Let~$\ring$ be a finite ring and let $\msgsp = \ring \setminus \{ 0 \}$
where~$0$ denotes the additive identity. Let $\dblen \in \N$, $\idx \in
[\dblen]$, and $\msg \in \msgsp$.
%
The \emph{point function} of $(\idx, \msg)$ is the map $P_{\idx,\msg} : [\dblen]
\to \ring$ where $P_{\idx,\msg}(\idx) = \msg$ and for every $\idx^\prime \ne
\idx$, it holds that $P_{\idx,\msg}(\idx^\prime) = 0$.
%
The notation~$\msgsp$ signifies that the set of non-zero elements of~$\ring$
constitutes a message space and that~$\msg$ encodes the message to be written
into the database. The value~$\idx$ indicates the row of the database to
which~$\msg$ is to be written.

We remark that the message space in the formalization of~\cite{riposte} is~$F
\setminus \{0\}$ where~$F$ is a finite field. Each element having a
multiplicative inverse is necessary in order to implement their error correcting
code for dealing with collisions in the table, but this extra structure is not
essential for formalizing the security properties of the scheme.\cpnote{Does the
seed-homomorphic PRG need field structure? I don't think so.}

Concretely, we may define a ring on $n$-bit strings as follows.  Addition of~$X$
and~$Y$ is defined by $X \xor Y$. The all-zero string~$0^n$ is then the additive
identity. To compute $Z = X\cdot Y$, we regard the $i$-th bit of~$X$ (resp.~$Y$)
as the $i$-th coefficient of a polynomial~$p_X(x)$ (resp.~$p_Y(x)$), compute the
polynomial~$p_Z(x)$ congruent to $p_X(x)\cdot p_Y(x) \mod x^{n+1}$, and
let~$Z[i]$ be the $i$-th coefficient of~$p_Z(x)$ for each $i \in
[n]$.

\subsection{Syntax}
% syntax.tex
%
% Syntax of DPF schemes and verifiable DPF schemes, definitions of correctness,
% completeness, and soundness.
\label{sec-syntax}

A \emph{distributed point function} (DPF) scheme is a pair of probabilistic
algorithms $\dpf = (\gen, \eval)$ with associated parameters $(\ring, \sharect,
\dblen)$ where~$\ring$ is a finite ring and $\sharect, \dblen \in \N$.
%
We call $\msgsp = \ring \setminus \{0\}$ the \emph{message space},
$\sharect$ the \emph{share count},and $\dblen$ the \emph{database length}.
%
On input of integer $\idx \in [\dblen]$, called the \emph{database index}, and
message $\msg \in \msgsp$, the share generation algorithm~$\gen$ outputs a
sequence of strings $(X_1, \ldots, X_\sharect)$ called \emph{write shares}.
%
On input of
write share~$X$ and index~$\idx$, algorithm~$\eval$ deterministically
outputs an element of~$\ring$.
%
A DPF scheme is \emph{correct} if for every $\idx, \idx^\prime \in [\dblen]$ and
$\msg \in \msgsp$, it holds that
\[
  \Prob{ (X_1, \ldots, X_s) \getsr \gen(\idx,\msg):
       \sum_{i=1}^S \eval(X_i, \idx^\prime) = P_{\idx,\msg}(\idx^\prime)} = 1.
\]
where $P_{\idx,\msg}$ denotes the point function of $(\idx, \msg)$.\cpnote{One
might consider endowing the write share generator with a key. This would admit
deterministic constructions, which would be more efficient. They write share
generator of Riposte uses \emph{a lot} of random bits.}

To initiate a write request, the client runs $(X_1, \ldots X_\sharect) \getsr
\gen(\idx,\msg)$ and transmits each share to one of the write servers. Upon
receiving their shares, the write servers engage in a write share verification
protocol in order to ensure the request is well-formed.  We say that $\dpf$ is
\emph{verifiable} if there exists a protocol $\reqvfy$ executed by principals
$P$ satisfying the following properties.
%
First, it holds that $\{\client, \writer_1, \ldots, \writer_{\sharect}\}
\subseteq P$ where $\client$ denotes the client making the write request and
$\writer_i$ denotes the $i$-th write server.
%
Second, let $\lang_\dpf$ be the language comprised of strings $\str{\shares}$
where $(\shares) \in [\gen(\idx, \msg)]$ for some $\msg \in \msgsp$ and $\idx
\in [L]$. There exists an adversary $\advB$ called the \textit{benign adversary}
such that the following conditions hold:
\begin{itemize}
  \item \textit{Completeness.}
    For every $\str{\shares} \in \lang_\dpf$, it holds that
    $
      \Pr[\game{proto}_{\reqvfy,I,P}(\advB) \outputs \accept] = 1
    $
  where $I(\writer_i) = (X_i, \client)$ for each $i\in[\sharect]$.
  \item \textit{Disruption resistance} \cite[def. 3]{riposte}.
    For every probabilistic, polynomial-time (in the implicit security
    parameter) adversary $\advA$, it holds that
    \[
      \Prob{ (\shares) \getsr \advA:
             \str{X_1, \ldots, X_\sharect} \not\in \lang_\dpf \AND
             \game{proto}_{\reqvfy,I,P}(\advB) \outputs \accept } \le \epsilon
    \]
    where $\epsilon$ is negligible (in the implicit security parameter) and
    $I(\writer_i) = (X_i, \client)$ for each $i\in[\sharect]$. (We leave the
    security parameter implicit because we will give concrete security results.)
    \cpnote{This is actually a security property.}
\end{itemize}

\heading{Discussion.}
The disruption resistance property resembles the collision resistance property
of cryptographiic hash functions~\cite{collision-resistance}. We will show that
our schemes achieve this under standard assumptions.
%
We could pull out a soundness property for~$\reqvfy$ similar to
that of interactive proof systems \cite[def. 4.2.10]{oded}: for every
$\str{\shares} \not\in \lang_\dpf$ and every PPT adversary~$\advA$, it
holds that $\Pr[\game{proto}_{\reqvfy,I,P}(\advA) = 1] \le \epsilon$.
%
However, this is stronger than needed in our setting, since we only expect the
system to detect mal-formed requests when all the servers are honest.  Recall
that a malicious server can always corrupt its own state, thereby disrupting the
system; nevertheless, we require that doing so does not violate privacy of
honest clients. This is captured by our privacy notions below.

\noindent\hl{Disruption resistance is a pretty weak notion}, but this is indeed
the property that the authors intend. (Actually, the adversary can try to $n$
times to forge a bad request, but a simple hybrid argument shows that our notion
implies theirs.) The following is a stronger, if not uglier target:
\begin{figure}[h]
  \oneCol{0.50}{
    \underline{$\game{\disres}_\reqvfy(\advA, \advB)$}\\[2pt]
      $(K_u)_{u\in P} \getsr \protocol.\init$;
      $O \gets P$\\
      $(X_1, \ldots, X_\sharect) \getsr \advA^{\ENQO,\DEQO}$\\
      \foreach{j}{1}{S} $Q_{\writer_j}^\sess \gets Q_{\writer_j}^\sess.\enqueue(X_j, \client)$\\
      $\advB^{\ENQO,\DEQO}$\\
      if $\sess$ accepts and $\str{X_1, \ldots, X_\sharect} \not\in \lang_\dpf$
      then return $\true$\\
      else return $\false$
    \vspace{4pt}
  }
  \caption{$\advB$ is the benign adversary.}
\end{figure}


\subsection{Security}
% security.tex
%
% Definition of PRIV1 and PRIV2.
\label{sec-security}

In this section, we model a coalition of write servers attempting to learn
something about the message or index chosen by the client. We give two notions
in figure~\ref{fig-priv}.
The first applies to DPF schemes and the second to verifiable DPF schemes.
\begin{figure}[t]
  \newcommand{\ctr}{\flagfont{ctr}}
  \twoCols{0.45}
  {
    \underline{$\game{\privone}_{\dpf,t}(\advA,\advS)$}\\[2pt]
    $b \getsr \bits$\\
    $b^\prime \getsr \advA^{\GENO,\CORO}$\\
    return $b=b^\prime$
    \\[6pt]
    \underline{$\GENO(\idx,\msg)$}\\[2pt]
    if $C = \bot$ then return $\bot$\\
    if $\msg \not\in \msgsp \OR \idx \not\in [\dblen]$ then return $\bot$\\
    if $b=1$ then $(X_1, \ldots, X_\sharect) \getsr \gen(\idx,\msg)$\\
    else $(X_1, \ldots, X_\sharect) \getsr \advS(C)$\\
    return $(X_i)_{i\in C}$
    \\[6pt]
    \underline{$\CORO(C^\prime)$}\\[2pt]
    if $C \ne \bot \OR C^\prime \not\subseteq [\sharect] \OR |C^\prime| > t$
      then\\\ind return $\bot$\\
    $C \gets C^\prime$\\
  }
  {
    \underline{$\game{\privtwo}_{\dpf,\reqvfy,t}(\advA, \advS)$}\\[2pt]
    $(K_u)_{u\in P} \getsr \kg$\\
    $b \getsr \bits$\\
    $b^\prime \getsr \advA^{\GENO,\ENQO,\DEQO,\CORO}$\\
    return $b=b^\prime$
    \\[6pt]
    \underline{$\GENO^i(\idx,\msg)$}\\[2pt]
    if $C = \bot$ then return $\bot$\\
    if $\msg \not\in \msgsp \OR \idx \not\in [\dblen]$ then return $\bot$\\
    if $b=1$ then $(X_1, \ldots, X_\sharect) \getsr \gen(\idx,\msg)$\\
    else $(X_1, \ldots, X_\sharect) \getsr \advS(C)$\\
    \foreach{j}{1}{S}\\
    \ind $Q_{\writer_j}^i \gets Q_{\writer_j}^i.\enqueue(X_j, \client)$\\
    return $(X_u)_{u\in C}$
    \\[6pt]
    \underline{$\ENQO_{y,x}^i(X)$}\\[2pt]
      $Q_y^i.\enqueue(X, x)$
    \\[6pt]
    \underline{$\DEQO_{y}^i()$}\\[2pt]
      $(X, x) \gets Q_y^i.\dequeue()$\\
      $(Y, z, \verdict, \st_y^i) \getsr \vfy_{y,x}^i(K_y, X, \st_y^i)$\\
      if $C \ne \bot \AND v \in C$ then return $(Y, z, \verdict)$\\
      else if $z \ne \bot$ then $Q_z^i.\enqueue(Y, y)$\\
      return $(\bot, z, \verdict)$
    \\[6pt]
    \underline{$\CORO(C^\prime)$}\\[2pt]
    if $C \ne \bot \OR \client \in C^\prime$ then return $\bot$\\
    if $|C^\prime \intersection \{\writer_j\}_{j\in[\sharect]}| > t$ then return $\bot$\\
    $C \gets C^\prime$\\
    return $(K_u)_{u\in C}$
  }
  \caption{Security notions for \textbf{(left)} DPF scheme $\dpf = (\gen,
  \eval)$ with parameters $(\ring, \sharect, \dblen)$ where $\msgsp = \ring
  \setminus \{0\}$, and
  \textbf{(right)} verifiable DPF scheme $\dpf$ with write share verification
  protocol $\reqvfy = (\kg, \vfy)$ executed with principals $P \supseteq \{\client, \writer_1,
  \ldots, \writer_\sharect\}$.}
  \vspace{6pt}\hrule
  \label{fig-priv}
\end{figure}

\heading{\privone.}
We describe the simulation-based notion of \cite{riposte,dpf}, which models the
ability of a coalition of malicious write servers to distinguish their subset of
the key shares from the output of a simulator. Let $\dpf = (\gen, \eval)$ be a
standard DPF scheme with parameters $(\ring, \sharect, \dblen)$. Let $\msgsp = R
\setminus \{0\}$. Refer to the experiment in the left-hand side of
Figure~\ref{fig-priv} associated to~$\dpf$, adversary~$\advA$,
simulator~$\advS$, and coalition threshold $t \in \N$ where $1 < t < \sharect$.
We define the advantage of~$\advA$ in attacking~$\dpf$ in the game instantiated
with simulator $\advS$ as
\[
  \adv{\privone}_{\dpf,t}(\advA,\advS) = 2 \cdot
  \Prob{ \game{\privone}_{\dpf,t}(\advA,\advS) \outputs \true } - 1.
\]
One might informally we say that~$\dpf$ is $(\sharect,t)$-\privone~secure if for
every ``reasonable'' adversary~$\advA$, there exists an ``efficient''
simulator~$\advS$ such that $\adv{\privone}_{\dpf,t}(\advA,\advS)$ is ``small''.
We will forego a rigorous definition and instead give concrete security results.

Achieving security in this sense does not ensure privacy if the write servers
engage in a write share verification protocol.  We put forward a unified
approach based on the communication model described in section~\ref{sec-com}.

\heading{\privtwo.}
Consider the experiment on the right hand side of Figure~\ref{fig-priv}
associated to DPF scheme $\dpf$, write share verification protocol $\reqvfy =
(\kg, \vfy)$ with principals $P \supseteq \{\client, \writer_1, \ldots
\writer_s\}$, coalition threshold~$t$, adversary~$\advA$, and simulator~$\advS$.
%
Just as before, the goal of the adversary is to distinguish its subset of the
write shares from the output of the simulator.
%
At the beginning of the game,
the initialization algorithm~$\kg$ is executed and a random bit~$b$ is chosen.
%
When $\advA$ asks $(i, \idx, \msg)$ of~$\GENO$, if $b=1$ then the write share
generation algorithm is run; otherwise the simulator is run on. Let $(\shares)$
denote the output. Next, the output $X_j$ is added to the top of the queue of
write server $\writer_j$ in session $i$ for each $j\in[\sharect]$. Finally, the
adversary is given the shares corresponding to servers it corrupts.
%
Corruptions are made by querying~$\CORO$ with a subset of~$P$. The coalition is
non-adaptive, meaning the adversary chooses which servers it controls before it may
query~$\GENO$. It may corrupt any set of principals that does not include the
client~$\client$ or more than~$t$ write servers. It is given the long-term input
of each principal it corrupts.
%
The adversary is given a~$\ENQO$ oracle and a~$\DEQO$ oracle, which have the
same semantics as in the \protosec~game defined in figure~\ref{fig-proto}. It is
on path to any principal it corrupts.
%
After interacting with its oracles, the adversary outputs a bit~$b^\prime$. The
output of the game is the predicate $b=b^\prime$. We define the advantage of
$\advA$ against $\dpf$ and $\reqvfy$ in the game instantiated with simulator
$\advS$ as
\[
  \adv{\privtwo}_{\dpf,\reqvfy,t}(\advA,\advS) =
  2 \cdot \Prob{\game{\privtwo}_{\dpf,\reqvfy,t}(\advA,\advS) \outputs \true} - 1.
\]

\heading{Non-colluding auditor.}
The 2-server DPF scheme of \cite{riposte} is accompanied by a request
verification protocol with principals $\{\client, \auditor, \writer_1,
\writer_2\}$, where $\auditor$ is called the \emph{auditor}. The auditor is not
trusted more than $\writer_1$ or $\writer_2$, but the non-collusion assumption
is stronger than usual: namely, that no two servers among $\{\auditor, \writer_1,
\writer_2\}$ collude. To capture this special case, we let $t=1$. Note,
however, that our model cannot be used to capture the general case where any set
of $t$ of $\sharect$ write serves may collude, but no write server colludes with
$\auditor$. For $\sharect>2$ and $t>1$, we assume the auditor may collude.


\subsection{Application to anonymous reporting}
% anon.tex
%
% Application of verifiable DPF schemes to anonymous communication.
\label{sec-anonymity}

We describe the application of verifiable DPF schemes to anonymous communication
as specified by the Riposte system~\cite{riposte}. We parameterize the protocol
by a positive integer~$\cohortct$, called the \emph{cohort size}, which
determines the number of valid write requests the servers process before
outputting their state. The time interval in which each of the requests are
processed is referred to as an \emph{epoch}. Let~$\dpf$ be a DPF scheme with
parameters~$(\ring, \sharect, \dblen)$ and~$\reqvfy$ be a write request
verification protocol for~$\dpf$ with benign adversary~$\advB$ and principals $P
\supseteq \{\client, \writer_1, \ldots \writer_{\sharect}\}$ and let $\msgsp =
\ring \setminus \{0\}$.

\begin{itemize}
  \item To initiate a write request for $\msg \in \msgsp$, the client samples
    $\idx \getsr [\dblen]$, executes the share generation algorithm $(X_1,
    \ldots, X_{\sharect}) \getsr \gen(\idx, \msg)$, and sends~$X_i$
    to~$\writer_i$ for each $i \in [\sharect]$.

  \item Next, the request verification protocol is executed. This means that the
    \protosec~game is run with~$\reqvfy$, $I$, and~$\advB$ where for every $i
    \in [\sharect]$, we have that $I(\writer_i) = (X_i, \client)$.

  \item When server~$\writer_i$ accepts with private input~$X_i$, it updates its
    local state as follows. Let~$\dbst_i$ be a $\dblen$-vector over
    $\ring$ where each $\dbst_i[\idx]$ is initially equal to~$0$. Let
    $\dbst_i[\idx^\prime] = \dbst_i[\idx^\prime] + \eval(X_i, \idx^\prime)$
    for every~$\idx^\prime$.

  \item Finally, once~$\cohortct$ valid write requests have been processed, each
    write server its final state~$\dbst_i$ to the data consumer. The data
    consumer recovers the database state (and thus the set of messages) by
    computing $\dbst[\ell^\prime] = \sum_{i=1}^{\sharect}
    \dbst_i[\ell^\prime]$ for every~$\ell^\prime$.
\end{itemize}

\heading{Dealing with collisions.} Because the clients choose their index into
the database table randomly, there is a reasonable chance that two or more
inadvertently choose the same index. However, it can be shown that if the
database length is at least 20 times the cohort size, then average collision
rate will not exceed $5\%$ \cite{riposte}. Moreover, if~$\ring$ is a field, then
techniques based on error correcting codes can be used to further reduce the
collision rate \cite{riposte}.

\noindent\hl{Things we might do here:}
\begin{itemize}
  \item Formalize the security notion intended by this system (ANON). This was
    already done in \cite{riposte}.
  \item Show that PRIV1 implies ANON.  This was not done in \cite{riposte} as
    far as I can tell.
\end{itemize}

