% main.tex
%
% Root of the document tree. Includes the abstract and notes.
\documentclass{llncs}
\usepackage{color,soul}
\usepackage[dvipsnames]{xcolor}
\usepackage{enumitem}
% header.tex
%
% Formatting and common macros for crypto papers. Include this first.
\usepackage[utf8]{inputenc}
\usepackage[margin=4cm]{geometry}
\usepackage{graphics}
\usepackage[toc,page]{appendix}
\usepackage[font={small}]{caption}
\usepackage{hyperref}
\usepackage{amsmath}
\usepackage{amsthm}
\usepackage{amsfonts}
\usepackage{parskip}
\usepackage{framed}
\usepackage{multicol}

\hypersetup{
    colorlinks,%
    citecolor=black,%
    filecolor=black,%
    linkcolor=black,%
    urlcolor=black
}

\def\dashuline{\bgroup
  \ifdim\ULdepth=\maxdimen  % Set depth based on font, if not set already
	  \settodepth\ULdepth{(j}\advance\ULdepth.4pt\fi
  \markoverwith{\kern.15em
	\vtop{\kern\ULdepth \hrule width .3em}%
	\kern.15em}\ULon}

\newcounter{foot}
\setcounter{foot}{1}
\setlength\parindent{2em}

% Editorial
\newcommand{\heading}[1]{\noindent \textsc{#1}}
\newtheorem{theorem}{Theorem}[section]

% Fonts for various types
\newcommand{\notionfont}[1]{\textup{#1}}
\newcommand{\varfont}[1]{\textit{#1}}
\newcommand{\flagfont}[1]{\mathsf{#1}}
\newcommand{\vectorfont}[1]{\vec{#1}}
\newcommand{\oraclefont}[1]{\textsc{#1}}
\newcommand{\schemefont}[1]{\mathsf{#1}}
\newcommand{\procfont}[1]{\schemefont{#1}}
\newcommand{\adversaryfont}[1]{\mathcal{#1}}
\newcommand{\cryptofont}[1]{\mathbf{#1}\hspace{0.5pt}}
\newcommand{\prinfont}[1]{\mathcal{#1}}

% Math
\DeclareMathAlphabet\mathbfcal{OMS}{cmsy}{b}{n}

% - Sets
\newcommand{\Z}{\mathbb{Z}}
\newcommand{\N}{\mathbb{N}}
\newcommand{\bits}{\{0,1\}}
\newcommand*\bigunion{\bigcup}
\newcommand*\bigintersection{\bigcap}
\newcommand*\union{\cup}
\newcommand*\intersection{\cap}
\newcommand*\cross{\times}
\newcommand*\by{\cross}
%\newcommand{\getsr}{\mathrel{\leftarrow\mkern-14mu\leftarrow}}
%\newcommand{\getsr}{\xleftarrow{\text{\tiny{\$}}}}
\newcommand{\getsr}{{\:{\leftarrow{\hspace*{-3pt}\raisebox{.75pt}{$\scriptscriptstyle\$$}}}\:}}

% - String operations
\newcommand{\emptystring}{\varepsilon}
\newcommand{\cat}{\, \| \,}
\newcommand{\str}[1]{\langle #1 \rangle}

% - Boolean operators
\newcommand*\AND{\wedge}
\newcommand*\OR{\vee}
\newcommand*\NOT{\neg}
\newcommand*\IMPLIES{\implies}
\newcommand*\XOR{\mathbin{\oplus}}
\newcommand*\xor{\XOR}

% - Crypto functions
\newcommand{\game}[1]{\cryptofont{Exp}^{\textnormal{\tiny \MakeLowercase{#1}}}}
\newcommand{\adv}[1]{\cryptofont{Adv}^{\textnormal{\tiny \MakeLowercase{#1}}}}

% - Asymptotics
\newcommand{\negl}{\procfont{negl}}
\newcommand{\poly}{\procfont{poly}}

% - Probablity
\newcommand{\E}{\mathrm{E}}
\newcommand{\Prob}[1]{\Pr\hspace{-2pt}\Big[\,#1\,\Big]}

% Games
\newcommand{\gamesfontsize}{\footnotesize}
\newcommand{\foreach}[3]{$\text{for }#1 \gets #2\text{ to }#3\text{ do}$}
\newcommand{\ind}{\hspace*{10pt}}
\newcommand{\outputs}{=}
\newcommand{\true}{1}
\newcommand{\false}{0}

% - Inline comment
\definecolor{CommentColor}{RGB}{125,175,230}
\newcommand{\comment}[1]{\textcolor{CommentColor}{\,\textbf{\#}\,#1}}

% - One game
\newcommand{\oneCol}[2]{
\begin{center}
        \framebox{
        \begin{tabular}{c@{\hspace*{.4em}}}
        \begin{minipage}[t]{#1\textwidth}\gamesfontsize #2 \end{minipage}
        \end{tabular}
        }
\end{center}
}

% - One game, two columns
\newcommand{\twoColsNoDivide}[3]{
  \providecommand{\pad}{6pt}
  \providecommand{\colwidth}{#1\textwidth}
  \makebox[\textwidth][c]{
  \begin{tabular}{|@{\hskip \pad}l@{}@{}@{\hskip \pad}l|}
    \hline
    \rule{0pt}{1\normalbaselineskip}
    \begin{minipage}[t]{\colwidth}\gamesfontsize
      #2 \vspace{\pad}
    \end{minipage} &
    \begin{minipage}[t]{\colwidth}\gamesfontsize
      #3 \vspace{\pad}
    \end{minipage} \\
    \hline
  \end{tabular}
  }
}

% - Two games, one per column
\newcommand{\twoCols}[3]{
  \providecommand{\pad}{6pt}
  \providecommand{\colwidth}{#1\textwidth}
  \makebox[\textwidth][c]{
  \begin{tabular}{|@{\hskip \pad}l@{}|@{}@{\hskip \pad}l|}
    \hline
    \rule{0pt}{1\normalbaselineskip}
    \begin{minipage}[t]{\colwidth}\gamesfontsize
      #2 \vspace{\pad}
    \end{minipage} &
    \begin{minipage}[t]{\colwidth}\gamesfontsize
      #3 \vspace{\pad}
    \end{minipage} \\
    \hline
  \end{tabular}
  }
}

% Notes
\newcounter{notectr}[section]
\newcommand{\getnotectr}{\stepcounter{notectr}\thesection.\thenotectr}
\newcommand{\basenote}[4]{{
  \textcolor{#1}{(#3#4---#2~\getnotectr)}
}}

\newcommand{\note}[3]{\basenote{#1}{#2}{}{#3}}
\newcommand{\todo}[3]{\basenote{#1}{#2}{TODO~}{#3}}

% - Chris' notes
\definecolor{darkgreen}{RGB}{50,127,0}
\newcommand{\cpnote}[1]{\note{darkgreen}{Chris}{#1}}
\newcommand{\cptodo}[1]{\todo{darkgreen}{Chris}{#1}}

% macros.tex
%
% Macros for this paper. Include headers.tex, then this file.

% Math
\newcommand{\perm}{\procfont{Perm}}

% Point function
\def\pt[#1,#2,#3]{{#1}_{#2,#3}}

% Function families
\newcommand{\fns}{\setfont{F}}
\newcommand{\dom}{\setfont{X}}
\newcommand{\rng}{\setfont{Y}}
\newcommand{\Fn}{\schemefont{Fn}}

%
\newcommand{\kk}{\vectorfont{K}}
\renewcommand{\k}{K}
\newcommand{\pub}{\procfont{pub}}

% TODO Automate \def\*x.
\newcommand{\Kg}{\schemefont{Kg}}
\newcommand{\Rep}{\schemefont{Rep}}
\newcommand{\Comb}{\schemefont{Comb}}
\newcommand{\Qry}{\schemefont{Qry}}
\def\Repx(#1,#2){\Rep_{#1}\!\left(#2\right)}
\def\Combx(#1,#2){\Comb_{#1}\!\left(#2\right)}
\def\Qryx(#1,#2){\Qry_{#1}\!\left(#2\right)}

% Adversaries
\def\advA{\adversaryfont{A}}

% Sets
%
% TODO Automate \def\set*.
\def\setS{\setfont{S}}
\def\setX{\setfont{X}}
\def\setY{\setfont{Y}}

% Vectors
\def\vv{\vectorfont{v}}

% Misc.
\def\coins{r}


\date{\today}
\title{\textbf{Security notions for DPF schemes}}
\author{Christopher Patton}
\institute{University of Florida}

\setcounter{tocdepth}{2}

\pagestyle{plain}

\begin{document}

\maketitle

\begin{abstract}
  The point function of binary strings~$X$ and~ $Y$ is defined by $P_{X,Y}(X) =
  Y$ and $P_{X,Y}(X^\prime) = 0^{|Y|}$ for all $X^\prime \ne X$. A distributed
  point function (DPF) scheme maps~$P_{X,Y}$ to a sequence of shares which, when
  combined, yield the point function, but from a subset of which neither~$X$
  nor~$Y$ can be recovered. This primitive was first suggested in the context of
  private information retrieval where the goal is to allow clients to \emph{read
  from} a database distributed among a set of servers without revealing which
  data were read. It was recently proposed as a way to address traffic analysis
  attacks in the context of anonymous communication.
  %
  The goal of the Riposte cryptosystem \cite{riposte} is to allow clients to
  \emph{write to} a distributed database without revealing to the servers which
  data were written. However, by sending the servers mal-formed shares, a
  malicious client or network adversary can easily corrupt the database state.
  As pointed out by the designers, it is crucial in this setting that the
  servers be able to verify their shares are correct.
  %
  \cpnote{What's our answer?}
\end{abstract}

\section{Introduction}
%
%
%
\renewcommand{\ng}{\schemefont{ng}}
\newcommand{\he}{\schemefont{he}}
\newcommand{\foobar}{\textsc{foobar}}
In their treatment of nonce-based, public-key
cryptography~\cite{bellare2016nonce}, Bellare and Tackmann introduce two notions
of secure encryption. \foobar!
%
The first, NBP1, extends the usual IND-CCA notion to a new setting in which
encryption is determinsitc and takes as input the public key~$\pk$ and
message~$m$, as well as a nonce~$n$ and an input~$\xk$, called the \emph{seed},
known only to the sender. Decryption has the usual syntax; it requires only the
ciphertext and secret key. The nonce~$n$ is output by a stateful (and possibly
randomized) procedure~$\ng$, called the \emph{nonce generator}. This algorithm
takes as input a string~$\sel$, called the \emph{nonce selector}. In the NBP1
setting we assume that, at a minimum, the pair $(m, \sel)$ does not repeat.
%
The second notion, NBP2, models the setting where the sender's state is
(partially) exposed to the adversary. The primary distinction between NBP1 and
NBP2 is that, in the latter, the adversary is given~$\xk$ as input. As a result,
ensuring that $(m, \sel)$ not repeat is not enough for security; the output
of~$\ng$ must also be \emph{unpredictable} to the adversary.

Their main construction involves a novel primitive~$\he$, called a \emph{hedged
extractor}. It takes as input the seed, the message~$m$, and the nonce~$n$, and
outputs a string~$r$. This string is used as the coins for encryption of~$m$
using a standard PKE scheme. This composition achieves NBP1 if~$\he$ is a PRF,
and it achieves NBP2 if~$\he$ is secure in a new sense
that~\cite{bellare2016nonce} defines.
%
Roughly speaking, this RoR (``real-or-random'') notion ensures that, if the
output of~$\ng$ is unpredictable, then the output of~$\he$ is indistinguishable
from a random string, even to an adversary in possession of the seed.

Interestingly, the composition of~$\he$ and~$\ng$ is remarkably similar to how
pseudo random number generation works in real systems. PRNGs \emph{with input},
first formalized by Barak and Halevi~\cite{barak2005model}, typically have two
interfaces: one that fetches (any number of) pseudorandom bits, and another with
which the programmer can provide additional randomness to the PRNG state. A number
of notions of security have been considered for this primitive, including some
that model exposure of the state to the adversary. The motivation for these
notions is not dissimilar to that of the RoR game described above.

\cptodo{Segway}
%
In the NBP2 game, the adversary is given the seed as input, but the state
associated with nonce generation remains hidden. In my opinion, this setting is
not clearly motivated; an adversary that is able to penetrate the sender's
system and exfiltrate the seed ought to be able to recover the~$\ng$ state.
%
Of course, secure encryption \emph{after} this state is exposed is impossible, a
limitation that~\cite{bellare2016nonce} recognizes: in the RoR game for hedged
extractors, the adversary is given access to the~$\ng$ state, but only
\emph{after} it makes its queries. Said another way, RoR security for hedged
extractors is only guaranteed for coins generated prior to state exposure.

In this work, we reconsider the adversarial model described by NBP2, opting for
a simpler notion in which the encryption state is exposed to the adversary after
it makes its encryption queries.
%
We consolidate the stateless, deterministic encryption algorithm
and the stateful nonce generator into one stateful, deterministic encryption
procedure. We do away with the seed and instead define stateful, pseudorandom
number generation as a primitive for constructing such a scheme.

Finally, our syntax and security notions are geared towards the application of
password-based authenticated key exchange. In this direction, our syntax admits
a nonce and associated data, which may be used as the session number and
password respectively.

\cptodo{Talk about why decryption should take the nonce as input. (In
\cite{bellare2016nonce}, the nonce is used only to ensure coins freshness.)}


\section{Related work}
% related.tex
%
% Related work.
\label{sec:related}

\cpnote{Here's an outline:
\begin{itemize}
  % PIR
  \item Remark that single-server PIR systems are possible
    \cite{pir-single-server}, but at a higher computational cost. (OWFs are
    essential for security: http://dl.acm.org/citation.cfm?id=301277.) According
    to \cite{dpf}, PIS cannot be achieved with a single server \cite{dpf}.
  \item There's an interesting paper (http://dl.acm.org/citation.cfm?id=276723)
    that introduces symmetric PIR, designed to ensure that the reader learns
    nothing more than the bit they query. Do DPF schemes suffice in this
    setting?
  \item Although \cite{dpf} focus on 2-server protocols, methods described in
    \cite{dpf-multi-server} can be used to obtain $(2^t, t)$-private PIR from
    any $(2,1)$-private PIR scheme.
  % PIS
  \item A side-effect of our work is that PIS is more useful, since it can now
    support multiple writers while detecting malicious clients.
  \item \cite{pis} does consider a stronger ``active adversary'', but not one
    who tries to corrupt the database. This adversary may collude with users in
    order to try to learn the data being read/written to the database by the
    victim.
  \item Oblivious RAM is a related problem.
  % Riposte
  \item Under the ``Disruption resistance'' heading in the intro of
    \cite{riposte}, they list a bunch of AC systems in which malicious clients
    must be dealt with.
  \item Describe anonymous communication and the traffic analysis problem.
    Compare Riposte to mix- and DC-nets.
\end{itemize}
}

\ignore{ % This about anonymous communication generally.
  \heading{Information leakage.} Information is said to be identifiable if it can
  be used to link a client to their message. There are two things to remember
  about personally identifiable information: one, any identity information
  conveyed by the client's message is necessarily leaked in the output of the
  system; and two, any information is potentially identifiable given auxiliary
  information about the clients. For example, the number of messages sent by the
  client might identify which messages they sent. Anonymous communication systems
  can only address the problem of linking sensitive data to the identity of the
  client.
}


\section{Preliminaries}
% prelims.tex
%
% Notation, games, queues, syntax of protocols, and communication model.
% Definition of PROTO.
\label{sec-prelims}

\heading{Notation.}
Let $x \getsr S$ denote sampling an element~$x$ uniformly from a set~$S$.
%
If~$A$ is an algorithm, let $y \gets A(x_1, \ldots)$ denote running~$A$ on input
$x_1, \ldots$ and halting with output $y$.
%
Let $y \gets A(x_1, \ldots; \coins)$ denote running~$A$ with coins~$\coins$.
Let $y \getsr A(x_1, \ldots)$ denote sampling coins~$\coins$ uniformly and
letting $y = A(x_1, \ldots; \coins)$.
%
Let $[A(x_1, ...)]$ denote the set of possible outputs of $A$ run on input
$(x_1, \ldots \,; \coins)$ and with randomly sampled~$\coins$.
%
Let $[i..j] = \{ \ell \in \Z : i \le \ell \le j \}$ and $[n] = [1..n]$.
%
If~$\vecv$ is a vector, let~$\vecv[i]$ denote its~$i$-th element. Similarly,
if~$X$ is a string, let~$X[i]$ denote the~$i$-th bit of string~$X$.
%
Let~$\emptystring$ denote the empty string.
%
Let~$X \cat Y$ denote the concatenation of strings~$X$ and~$Y$.
%
If~$X$ and~$Y$ are equal length strings, let~$X \xor Y$ denote their
bitwise-xor.
Let~$\str{x}$ denote the invertible encoding of the arbitrary quantity~$x$ as a
string.
%
We say a sequence~$\vecv$ is indexed by a totally-ordered set~$S$ if $\vecv =
(v_{x_1}, \ldots, v_{x_n})$ where $(x_1, \ldots x_n)$ is the total ordering of
the elements of~$S$. This is denoted $(v_x)_{x\in S}$. Every set in this paper
has a total ordering.

\heading{Games.}
We adopt the game-playing framework of~\cite{games} with one exception: unless
otherwise stated, if a variable is undefined, then it is implicitly equal to the
formal symbol~$\bot$.~\cptodo{Double check if this is actually an exception.}

\heading{Queues.}
It will be convenient to use queues in our treatment of verifiable DPF schemes.
A queue~$Q$ is a first-in, first-out data structure with two methods.
%
By $Q.\enqueue(X)$ we mean add~$X$ to the top of~$Q$. If $Q=\bot$, then
$Q.\enqueue(X)$ means to instantiate a queue~$Q$ with~$X$ as its only element.
%
By $Q.\dequeue()$ we mean remove the element~$X$ from the bottom of~$Q$ and
return~$X$. If $Q = \bot$, then $Q.\dequeue()$ returns~$\bot$.

\subsection{Protocols}
We formalize protocols as a pair of probabilistic algorithms $\protocol =
(\init, \proto)$ with an associated set of principals~$P$.
%
Algorithm~$\init$ outputs a sequence of strings $(K_u)_{u\in P}$ called the
\emph{long-term inputs}.
%
Algorithm~$\proto$ dictates the actions of the principals at each step
of the protocol. It has the following inputs (all values are strings unless
otherwise noted):
\begin{itemize}
  \item[$u$] -- the identity of the receiver of~$X$, one of~$P$;
  \item[$w$] -- the identity of the alleged sender of~$X$, one of~$P$;
  \item[$i$] -- an identifier for the \emph{session}, an instance of the
    protocol. Either a string or the formal symbol~$\sess$;
  \item[$K$] -- the long-term input of~$u$ generated by $\init$;
  \item[$X$] -- the public input allegedly sent by~$w$; and
  \item[$\st$] -- the private state of the sender.
\end{itemize}
It has the following outputs:
\begin{itemize}
  \item[$Y$] -- the public output, the message to be sent out.
  \item[$v$] -- the identity of the intended recipient of $Y$ or $\bot$ if no
    message is sent.
  \item[$\verdict$] -- the sender's verdict, one of $\accept$, $\reject$, or
    $\bot$; and
  \item[$\st^\prime$] -- private state of~$u$ after processing the
    input.
\end{itemize}
By $(Y, v, \verdict, \st^\prime) \getsr \proto(u, w, i, K, X, \st)$ or
$(Y, v, \verdict, \st^\prime) \getsr \proto_{u,w}^i(K, X, \st)$ we denote the
execution of principal~$u$ on receipt of message~$X$, allegedly sent from~$w$,
in session~$i$. This causes the state of principal~$u$ in session~$i$ to
transition from~$\st$ to~$\st^\prime$.
%
Let~$\verdict_u^{i}$ denote the verdict of the last output of ~$u$ in
session~$i$. We say this session \emph{accepts} if for every $u \in P$, it holds
that $\verdict_u^{i} = \accept$.  Similarly, we say this session \emph{rejects}
if for every $u \in P$, it holds that $\verdict_u^{i} = \reject$.

\subsection{Communication model}\label{sec-com}
Protocols have two phases. In the \emph{initialization} phase, algorithm
$(K_u)_{u\in P} \getsr \init$ is run and the long-term inputs are
distributed to the principals. We assume the adversary is oblivious to this
process.
%
In the \emph{execution} phase, an instance of the protocol is carried out in the
presence of the adversary.
%
Bellare and Rogaway \cite{protos} model an adversary that controls
the network infrastructure by giving the adversary an oracle defined as follows:
%
on input $(u, w, i, X)$, first run $(Y, v, \verdict, \st_{u}^i) \getsr
\proto_{u,w}^i(K_u, X, \st_{u}^i)$ where $K_u$ is the long-term input of $u$
generated in the initialization phase and $\st_{u}^i$ is initially
$\emptystring$. Then return $(Y, v, \verdict)$ to the adversary.
%
Implicit in the formalization of any protocol is the existence of a so-called
\emph{benign adversary}, which faithfully relays the messages according to
specification, causing each session to accept.

In our treatment of verifiable DPF schemes, it will be useful to consider a more
robust communication model.  First, we would like to be able to model a weaker
adversary who does not observe all network flows, but only those along paths
under its control. Second, it will be useful for formalizing certain properties
of our protocols to allow each principal to be given a \emph{private input} in
addition to the long-term input generated by $\init$.
%
Essentially we need a technical means to allow the adversary to induce a
principal to process a message it did not explicitly send. Thus, we introduce a
\emph{message queue} for each principal and session.
%
The adversary is given a pair of oracles, one for enqueueing messages and
another for dequeueing them. Messages on the queue are accompanied by an alleged
sender. (The adversary need not control the network in order to send a
principal a message and fake the sender; IP address spoofing makes this
trivial.) When a message is dequeued, the corresponding principal is executed on
the incoming message. If the adversary is on path to the intended recipient of
the output message, then the adversary is given the principal's output message
and the identity of the recipient; otherwise the message and sender are added to
the top of the recipient's queue.

\heading{\protosec.}
\begin{figure}[t]
  \twoColsNoDivide{0.45}
  {
    \underline{$\game{\protosec}_{\protocol,I,O}(\advA)$}\\[2pt]
      $(K_u)_{u\in P} \getsr \protocol.\init$\\
      for each $u \in P$ do\\
      \ind if $I(u) \ne \bot$ then $Q_u^\sess \gets Q_u^\sess.\enqueue(I(u))$\\
      $\advA^{\ENQO,\DEQO}$\\
      if $\sess$ accepts then return $\accept$
  }
  {
    \underline{$\ENQO_{u,w}^i(X)$}\\[2pt]
      $Q_u^i.\enqueue(X, w)$
    \\[6pt]
    \underline{$\DEQO_u^i()$}\\[2pt]
      $(X, w) \gets Q_u^i.\dequeue()$\\
      $(Y, v, \verdict, \st_u^i) \getsr \protocol.\proto_{u,w}^i(K_u, X, \st_u^i)$\\
      if $v \in O$ then return $(Y, v, \verdict)$\\
      else if $v \ne \bot$ then $Q_v^i.\enqueue(Y, u)$\\
      return $(\bot, v, \verdict)$
  }
  \caption{A game modeling the execution of a protocol $\protocol$ with
  principals $P$ where adversary $\advA$ controls the network. Each $\st_u^i$ is
  initially $\emptystring$.}
  \vspace{6pt}\hrule
  \label{fig-proto}
\end{figure}
We formalize this communication model in Figure~\ref{fig-proto} in a game
associated to protocol~$\protocol$ with principal set~$P$,
adversary~$\advA$,
map $I : P \to \bits^* \by P$, and
set $O \subseteq P$.
% I
The quantity $I(u)$ is a pair $(X, w)$ where~$X$ is called the \emph{private
input} of $u\in P$ and~$w$ is the alleged sender of~$X$. Note that $I(u)$ need
not be defined for every $u \in P$.
% O
The set~$O$ denotes the set of principals who's incoming messages can be
intercepted by the adversary. If $v \in O$, we say that~$\advA$ is \emph{on
path} to~$v$.
%
We maintain a queue $Q_u^i$ for every principal $u \in P$ and every session $i$
instantiated by the adversary, as well as a special session $\sess$ instantiated
by the game.
%
The game begins by running the initialization algorithm and furnishing each
principal with its private input. For every $u \in P$ for which~$I(u)$ is
defined, the pair~$I(u)$ is added to the top of the queue~$Q_u^\sess$.
%
Next, the adversary is given access to two oracles. On input $(u, w, i, X)$,
oracle~$\ENQO$ adds $(X, w)$ to the top of~$Q_u^i$. On input $(u, i)$,
oracle~$\DEQO$ removes $(X, w)$ from the bottom of~$Q_u^i$ and executes~$u$ on
input~$X$, allegedly sent from~$w$, and updates the state~$\st_u^i$ (initially
~$\emptystring$). Let $(Y, v, \verdict, \st_u^i)$ denote the output.
If the adversary is on path to~$v$, then $(Y, v, \verdict)$ is given to the
adversary. Otherwise, the adversary only gets~$v$ and~$\verdict$.  Finally, the
game outputs~$\accept$ if session~$\sess$ accepts.

Following \cite{protos}, we implicitly assume, for a particular protocol
$\protocol$,  the existence of a benign adversary who always causes the game to
output $\accept$ for ``valid'' initial inputs $I$.  We remark that ours is a
generalization of their model.  In particular, if $I(u) = \bot$ for every $u \in
P$ and $O = P$, then this game is equivalent to the communication model of
Bellare and Rogaway.


\section{Syntax}
% syntax.tex
%
% Syntax of DPF schemes and verifiable DPF schemes, definitions of correctness,
% completeness, and soundness.
\label{sec-syntax}

A \emph{distributed point function} (DPF) scheme is a pair of probabilistic
algorithms $\dpf = (\gen, \eval)$ with associated parameters $(\ring, \sharect,
\dblen)$ where~$\ring$ is a finite ring and $\sharect, \dblen \in \N$.
%
We call $\msgsp = \ring \setminus \{0\}$ the \emph{message space},
$\sharect$ the \emph{share count},and $\dblen$ the \emph{database length}.
%
On input of integer $\idx \in [\dblen]$, called the \emph{database index}, and
message $\msg \in \msgsp$, the share generation algorithm~$\gen$ outputs a
sequence of strings $(X_1, \ldots, X_\sharect)$ called \emph{write shares}.
%
On input of
write share~$X$ and index~$\idx$, algorithm~$\eval$ deterministically
outputs an element of~$\ring$.
%
A DPF scheme is \emph{correct} if for every $\idx, \idx^\prime \in [\dblen]$ and
$\msg \in \msgsp$, it holds that
\[
  \Prob{ (X_1, \ldots, X_s) \getsr \gen(\idx,\msg):
       \sum_{i=1}^S \eval(X_i, \idx^\prime) = P_{\idx,\msg}(\idx^\prime)} = 1.
\]
where $P_{\idx,\msg}$ denotes the point function of $(\idx, \msg)$.\cpnote{One
might consider endowing the write share generator with a key. This would admit
deterministic constructions, which would be more efficient. They write share
generator of Riposte uses \emph{a lot} of random bits.}

To initiate a write request, the client runs $(X_1, \ldots X_\sharect) \getsr
\gen(\idx,\msg)$ and transmits each share to one of the write servers. Upon
receiving their shares, the write servers engage in a write share verification
protocol in order to ensure the request is well-formed.  We say that $\dpf$ is
\emph{verifiable} if there exists a protocol $\reqvfy$ executed by principals
$P$ satisfying the following properties.
%
First, it holds that $\{\client, \writer_1, \ldots, \writer_{\sharect}\}
\subseteq P$ where $\client$ denotes the client making the write request and
$\writer_i$ denotes the $i$-th write server.
%
Second, let $\lang_\dpf$ be the language comprised of strings $\str{\shares}$
where $(\shares) \in [\gen(\idx, \msg)]$ for some $\msg \in \msgsp$ and $\idx
\in [L]$. There exists an adversary $\advB$ called the \textit{benign adversary}
such that the following conditions hold:
\begin{itemize}
  \item \textit{Completeness.}
    For every $\str{\shares} \in \lang_\dpf$, it holds that
    $
      \Pr[\game{proto}_{\reqvfy,I,P}(\advB) \outputs \accept] = 1
    $
  where $I(\writer_i) = (X_i, \client)$ for each $i\in[\sharect]$.
  \item \textit{Disruption resistance} \cite[def. 3]{riposte}.
    For every probabilistic, polynomial-time (in the implicit security
    parameter) adversary $\advA$, it holds that
    \[
      \Prob{ (\shares) \getsr \advA:
             \str{X_1, \ldots, X_\sharect} \not\in \lang_\dpf \AND
             \game{proto}_{\reqvfy,I,P}(\advB) \outputs \accept } \le \epsilon
    \]
    where $\epsilon$ is negligible (in the implicit security parameter) and
    $I(\writer_i) = (X_i, \client)$ for each $i\in[\sharect]$. (We leave the
    security parameter implicit because we will give concrete security results.)
    \cpnote{This is actually a security property.}
\end{itemize}

\heading{Discussion.}
The disruption resistance property resembles the collision resistance property
of cryptographiic hash functions~\cite{collision-resistance}. We will show that
our schemes achieve this under standard assumptions.
%
We could pull out a soundness property for~$\reqvfy$ similar to
that of interactive proof systems \cite[def. 4.2.10]{oded}: for every
$\str{\shares} \not\in \lang_\dpf$ and every PPT adversary~$\advA$, it
holds that $\Pr[\game{proto}_{\reqvfy,I,P}(\advA) = 1] \le \epsilon$.
%
However, this is stronger than needed in our setting, since we only expect the
system to detect mal-formed requests when all the servers are honest.  Recall
that a malicious server can always corrupt its own state, thereby disrupting the
system; nevertheless, we require that doing so does not violate privacy of
honest clients. This is captured by our privacy notions below.

\noindent\hl{Disruption resistance is a pretty weak notion}, but this is indeed
the property that the authors intend. (Actually, the adversary can try to $n$
times to forge a bad request, but a simple hybrid argument shows that our notion
implies theirs.) The following is a stronger, if not uglier target:
\begin{figure}[h]
  \oneCol{0.50}{
    \underline{$\game{\disres}_\reqvfy(\advA, \advB)$}\\[2pt]
      $(K_u)_{u\in P} \getsr \protocol.\init$;
      $O \gets P$\\
      $(X_1, \ldots, X_\sharect) \getsr \advA^{\ENQO,\DEQO}$\\
      \foreach{j}{1}{S} $Q_{\writer_j}^\sess \gets Q_{\writer_j}^\sess.\enqueue(X_j, \client)$\\
      $\advB^{\ENQO,\DEQO}$\\
      if $\sess$ accepts and $\str{X_1, \ldots, X_\sharect} \not\in \lang_\dpf$
      then return $\true$\\
      else return $\false$
    \vspace{4pt}
  }
  \caption{$\advB$ is the benign adversary.}
\end{figure}


\section{Security}
% security.tex
%
% Definition of PRIV1 and PRIV2.
\label{sec-security}

In this section, we model a coalition of write servers attempting to learn
something about the message or index chosen by the client. We give two notions
in figure~\ref{fig-priv}.
The first applies to DPF schemes and the second to verifiable DPF schemes.
\begin{figure}[t]
  \newcommand{\ctr}{\flagfont{ctr}}
  \twoCols{0.45}
  {
    \underline{$\game{\privone}_{\dpf,t}(\advA,\advS)$}\\[2pt]
    $b \getsr \bits$\\
    $b^\prime \getsr \advA^{\GENO,\CORO}$\\
    return $b=b^\prime$
    \\[6pt]
    \underline{$\GENO(\idx,\msg)$}\\[2pt]
    if $C = \bot$ then return $\bot$\\
    if $\msg \not\in \msgsp \OR \idx \not\in [\dblen]$ then return $\bot$\\
    if $b=1$ then $(X_1, \ldots, X_\sharect) \getsr \gen(\idx,\msg)$\\
    else $(X_1, \ldots, X_\sharect) \getsr \advS(C)$\\
    return $(X_i)_{i\in C}$
    \\[6pt]
    \underline{$\CORO(C^\prime)$}\\[2pt]
    if $C \ne \bot \OR C^\prime \not\subseteq [\sharect] \OR |C^\prime| > t$
      then\\\tab return $\bot$\\
    $C \gets C^\prime$\\
  }
  {
    \underline{$\game{\privtwo}_{\dpf,\reqvfy,t}(\advA, \advS)$}\\[2pt]
    $(K_u)_{u\in P} \getsr \kg$\\
    $b \getsr \bits$\\
    $b^\prime \getsr \advA^{\GENO,\ENQO,\DEQO,\CORO}$\\
    return $b=b^\prime$
    \\[6pt]
    \underline{$\GENO^i(\idx,\msg)$}\\[2pt]
    if $C = \bot$ then return $\bot$\\
    if $\msg \not\in \msgsp \OR \idx \not\in [\dblen]$ then return $\bot$\\
    if $b=1$ then $(X_1, \ldots, X_\sharect) \getsr \gen(\idx,\msg)$\\
    else $(X_1, \ldots, X_\sharect) \getsr \advS(C)$\\
    \foreach{j}{1}{S}\\
    \tab $Q_{\writer_j}^i \gets Q_{\writer_j}^i.\enqueue(X_j, \client)$\\
    return $(X_u)_{u\in C}$
    \\[6pt]
    \underline{$\ENQO_{y,x}^i(X)$}\\[2pt]
      $Q_y^i.\enqueue(X, x)$
    \\[6pt]
    \underline{$\DEQO_{y}^i()$}\\[2pt]
      $(X, x) \gets Q_y^i.\dequeue()$\\
      $(Y, z, \verdict, \st_y^i) \getsr \vfy_{y,x}^i(K_y, X, \st_y^i)$\\
      if $C \ne \bot \AND v \in C$ then return $(Y, z, \verdict)$\\
      else if $z \ne \bot$ then $Q_z^i.\enqueue(Y, y)$\\
      return $(\bot, z, \verdict)$
    \\[6pt]
    \underline{$\CORO(C^\prime)$}\\[2pt]
    if $C \ne \bot \OR \client \in C^\prime$ then return $\bot$\\
    if $|C^\prime \intersection \{\writer_j\}_{j\in[\sharect]}| > t$ then return $\bot$\\
    $C \gets C^\prime$\\
    return $(K_u)_{u\in C}$
  }
  \caption{Security notions for \textbf{(left)} DPF scheme $\dpf = (\gen,
  \eval)$ with parameters $(\ring, \sharect, \dblen)$ where $\msgsp = \ring
  \setminus \{0\}$, and
  \textbf{(right)} verifiable DPF scheme $\dpf$ with write share verification
  protocol $\reqvfy = (\kg, \vfy)$ executed with principals $P \supseteq \{\client, \writer_1,
  \ldots, \writer_\sharect\}$.}
  \vspace{6pt}\hrule
  \label{fig-priv}
\end{figure}

\heading{\privone.}
We describe the simulation-based notion of \cite{riposte,dpf}, which models the
ability of a coalition of malicious write servers to distinguish their subset of
the key shares from the output of a simulator. Let $\dpf = (\gen, \eval)$ be a
standard DPF scheme with parameters $(\ring, \sharect, \dblen)$. Let $\msgsp = R
\setminus \{0\}$. Refer to the experiment in the left-hand side of
Figure~\ref{fig-priv} associated to~$\dpf$, adversary~$\advA$,
simulator~$\advS$, and coalition threshold $t \in \N$ where $1 < t < \sharect$.
We define the advantage of~$\advA$ in attacking~$\dpf$ in the game instantiated
with simulator $\advS$ as
\[
  \adv{\privone}_{\dpf,t}(\advA,\advS) = 2 \cdot
  \Prob{ \game{\privone}_{\dpf,t}(\advA,\advS) \outputs \true } - 1.
\]
One might informally we say that~$\dpf$ is $(\sharect,t)$-\privone~secure if for
every ``reasonable'' adversary~$\advA$, there exists an ``efficient''
simulator~$\advS$ such that $\adv{\privone}_{\dpf,t}(\advA,\advS)$ is ``small''.
We will forego a rigorous definition and instead give concrete security results.

Achieving security in this sense does not ensure privacy if the write servers
engage in a write share verification protocol.  We put forward a unified
approach based on the communication model described in section~\ref{sec-com}.

\heading{\privtwo.}
Consider the experiment on the right hand side of Figure~\ref{fig-priv}
associated to DPF scheme $\dpf$, write share verification protocol $\reqvfy =
(\kg, \vfy)$ with principals $P \supseteq \{\client, \writer_1, \ldots
\writer_s\}$, coalition threshold~$t$, adversary~$\advA$, and simulator~$\advS$.
%
Just as before, the goal of the adversary is to distinguish its subset of the
write shares from the output of the simulator.
%
At the beginning of the game,
the initialization algorithm~$\kg$ is executed and a random bit~$b$ is chosen.
%
When $\advA$ asks $(i, \idx, \msg)$ of~$\GENO$, if $b=1$ then the write share
generation algorithm is run; otherwise the simulator is run on. Let $(\shares)$
denote the output. Next, the output $X_j$ is added to the top of the queue of
write server $\writer_j$ in session $i$ for each $j\in[\sharect]$. Finally, the
adversary is given the shares corresponding to servers it corrupts.
%
Corruptions are made by querying~$\CORO$ with a subset of~$P$. The coalition is
non-adaptive, meaning the adversary chooses which servers it controls before it may
query~$\GENO$. It may corrupt any set of principals that does not include the
client~$\client$ or more than~$t$ write servers. It is given the long-term input
of each principal it corrupts.
%
The adversary is given a~$\ENQO$ oracle and a~$\DEQO$ oracle, which have the
same semantics as in the \protosec~game defined in figure~\ref{fig-proto}. It is
on path to any principal it corrupts.
%
After interacting with its oracles, the adversary outputs a bit~$b^\prime$. The
output of the game is the predicate $b=b^\prime$. We define the advantage of
$\advA$ against $\dpf$ and $\reqvfy$ in the game instantiated with simulator
$\advS$ as
\[
  \adv{\privtwo}_{\dpf,\reqvfy,t}(\advA,\advS) =
  2 \cdot \Prob{\game{\privtwo}_{\dpf,\reqvfy,t}(\advA,\advS) \outputs \true} - 1.
\]

\heading{Non-colluding auditor.}
The 2-server DPF scheme of \cite{riposte} is accompanied by a request
verification protocol with principals $\{\client, \auditor, \writer_1,
\writer_2\}$, where $\auditor$ is called the \emph{auditor}. The auditor is not
trusted more than $\writer_1$ or $\writer_2$, but the non-collusion assumption
is stronger than usual: namely, that no two servers among $\{\auditor, \writer_1,
\writer_2\}$ collude. To capture this special case, we let $t=1$. Note,
however, that our model cannot be used to capture the general case where any set
of $t$ of $\sharect$ write serves may collude, but no write server colludes with
$\auditor$. For $\sharect>2$ and $t>1$, we assume the auditor may collude.


\section{A $2$-server+auditor protocol}
% twowriter.tex

\label{sec-twowriter}
\hl{A 2-server+auditor protocol.} The goal here is the strongest-possible notion
of verifiability using a generic 2-share DPF scheme.

\heading{Notation.}
Let $X, Y \in \bits^*$.
%
Let~$X[i..j]$ denote substring of~$X$ from the $i$-th bit to the $j$-th bit
inclusively.
%
If $|X| = |Y|$, then let $X \xor Y$ denote the bitwise XOR of~$X$ and~$Y$.  If
$|X| \ne |Y|$, then $X \xor Y$ means to truncate the longer string to the length
of the shorter string and compute the bitwise XOR of the resulting pair of
strings.
%
Suppose there are positive integers~$m$ and~$n$ such that $|X| = mn$. We write
$\block{n}{X}{i} = X[ni+1..n(i+1)]$ to denote the $i$-th, $n$-bit block of~$X$.
Let $\block{n}{X}{i..j} = \block{n}{X}{i} \cat \cdots \cat \block{n}{X}{j}$.

The 2-share DPF scheme of~\cite{dpf} is specified in Figure~\ref{fig-two-dpf}.
The function~$\prg$ is instantiated with a pseudorandom generator who's
signature is determined by the lengths of the inputs to the write share
generation algorithm.
%
Let~$\alpha$ and~$\beta$ denote the maximum lengths of~$X$ and~$Y$ respectively.
%
By~\cite[Proposition 1]{dpf}, if
the seed length~$\kappa$ of~$\prg$ is chosen such that $\beta \le \kappa + 1$
and the number of iterations $r$ is $\lceil \log \alpha \rceil$ as specified in
Figure~\ref{fig-two-dpf}, then the length of the shares is at most
$8(\kappa+1)\alpha^{\log 3}\beta^{-\alpha}$ bits. This also upper bounds
the output length of~$\prg$. \cpnote{We could get a tighter bound on the output
length using an inductive argument similar to Proposition 1 of~\cite{dpf}.}


\begin{figure}
  \twoColsNoDivide{0.45}
  {
    \underline{$\gen(X,Y)$}\\[2pt]
      $r \gets \lceil \log |X| \rceil$\\
      $(K_0, K_1) \getsr \gen_r(X,Y)$\\
      $L \gets \varphi^{-1}(|Y|)$\\
      return $\str(\str(K_0, L), \str(K_1, L))$
    \\[6pt]
    \underline{$\gen_{r}(X, Y)$}\\[2pt]
      if $r=0$ then\\
      \tab $K_0 \getsr \bits^{|Y|\cdot2^{|X|}}$; $K_1 \gets K_0$\\
      \tab $x \gets \varphi(X)$; $\block{|Y|}{K_1}{x} \gets \block{|Y|}{K_1}{x} \xor Y$\\
      \tab return $(K_0, K_1)$\\
      $m \gets \psi(|X|, |Y|)$; $n \gets |X| - m$\\
      $I \gets X[1..m]$; $J \gets X[m+1..m+n]$\\
      $X^* \getsr \bits^\kappa$;
      $(S_0, S_1) \getsr \gen_{r-1}(I, X^*\cat 1)$\\
      $(P_0, P_1) \getsr \gen_{r-1}(J, Y)$\\
      for each $b\in \bits$ do\\
      \tab $W \gets \eval_{r-1}^{\kappa+1}(S_b, I)$\\
      \tab $Z \gets W[1..\kappa]$; $t \gets W[\kappa+1]$\\
      \tab $R_t \gets \prg(Z) \xor P_b$\\
      \tab $K_b \gets S_b \cat R_0 \cat R_1$\\
      return $(K_0, K_1)$
  }
  {
    \underline{$\eval(\str(K,L), X)$}\\[2pt]
      $r \gets \lceil \log |X| \rceil$\\
      $\ell \gets \varphi(L)$\\
      $Y \gets \eval^\ell_r(K, X)$\\
      return $Y$
    \\[6pt]
    \underline{$\eval_{r}^\ell(K, X)$}\\[2pt]
      if $r=0$ then $x \gets \varphi(X)$; return $\block{\ell}{K}{x}$\\
      $m \gets \psi(|X|, \ell)$; $n \gets |X| - m$\\
      $I \gets X[1..m]$; $J \gets X[m+1..m+n]$\\
      \comment{Parse $K$ into sub-strings.}\\
      $S \gets K[1..|K|-2n\ell]$;
      $R \gets K[|K|-2n\ell+1..|K|]$\\
      $R_0 \gets \block{\ell}{R}{1..n}$;
      $R_1 \gets \block{\ell}{R}{n+1..2n}$\\
      \comment{Expand row.}\\
      $W \gets \eval_{r-1}^{\kappa+1}(S, I)$\\
      $Z \gets W[1..\kappa]$; $t \gets W[\kappa+1]$\\
      $P \gets \prg(Z) \xor R_t$\\
      \comment{Evaluate cell.}\\
      $Y \gets \eval_{r-1}^\ell(P, J)$\\
      return $Y$
  }
  \caption{The 2-share DPF scheme of \cite{dpf}.
  Let $\prg : \bits^\kappa \to \bits^\infty$ be a function,
  let $\varphi : \bits^* \to \Z^+$ be the bijection between a string and the
  positive integer it represents (according to some fixed encoding), and let
  $\psi(\alpha,\beta) = \lceil 1/2 \cdot
  \log({\beta\cdot2^{\alpha}}/{(\kappa+1)}) \rceil$.
  }
  \label{fig-two-dpf}
\end{figure}


\if{0}
\section{Riposte}
% riposte.tex
%
% Specification of Riposte protocols.
\label{sec-riposte}

\subsection{The $2$-server+auditor protocol}
\begin{figure}
\twoColsNoDivide{0.45}
{
  \underline{algorithm $\gen(\idx, \msg)$}\\[2pt]
  if $\msg \not\in \msgsp \OR \idx \not\in [\dblen]$ then return $\bot$\\
  $(i, j) \gets (\lceil \idx/y \rceil, \idx - \lfloor \idx/y \rfloor)$\\
  $r_A \getsr \bits^x$; $r_B \gets r_A \xor e_i$\\
  $\vecs_A \getsr (\bits^k)^x$ \\
  $\vecs_B \gets \vecs_A$; $\vecs_B[i] \getsr \bits^x$ \\
  $\vecx_A \gets \prg(\vecs_A[i])$; $\vecx_B \gets \prg(\vecs_B[i])$ \\
  $\vecv \gets \msg\cdot \vece_j + \vecx_A + \vecx_B$ \\
  $X_A \gets \str{r_A, \vecs_A, \vecv}$; $X_B \gets \str{r_B, \vecs_B, \vecv}$\\
  return $(X_A, X_B)$
}
{
  \underline{algorithm $\eval(X, \idx)$}\\[2pt]
  if $\idx \not\in [\dblen]$ then return $\bot$\\
  $(i, j) \gets (\lceil \idx/y \rceil, \idx - \lfloor \idx/y \rfloor)$\\
  $\str{r, \vecs, \vecv} \gets X$\\
  $\vecx \gets \prg(\vecs[i])$\\
  if $r[i] = 1$ then return $\vecv[j] + \vecx[j]$\\
  else return $\vecx[j]$
}
\caption{The 2-share DPF scheme of \cite{riposte} constructed from a PRG
  $\prg$.}
\label{fig-riposte-2share}
\end{figure}
We specify the construction of \cite{riposte} of a 2-share DPF scheme from a
PRG. Let $\dpf = (\gen, \eval)$ be the DPF scheme defined in
figure~\ref{fig-riposte-2share} with parameters $(\ring, \sharect, \dblen, x,
y)$ where $x$ and $y$ are positive powers of 2 and $xy \ge \dblen$.
We fix some encoding of each abstract point in $\ring$ as an $n$-bit string for
$n \in \N$.
Let $\vece_i$ denote the
$y$-vector over $\ring$ with $0$ in each position except for the $i$-th, which is 1.
Let $e_i$ denote the $x$-bit string with 0's everyone except for the $i$-th bit.
Let $\prg : \bits^x \to \bits^{yn}$ be a function.
When we write $\vecx \gets w$ where $w \in \bits^{yn}$, we mean divide $w$ into
a sequence of $n$-bit strings and map each string to its corresponding point in
$\ring$.

\heading{Optimal choice for $x$ and $y$.}
According to \cite{riposte}, the length of the keys is $x(s+1) + yn$. The
optimal values for $x$ and $y$ are $x = c\sqrt{\dblen}$ and $y =
c^{-1}\sqrt{\dblen}$ where $c = \sqrt{n/(1+s)}$. Hence, the share size is
$O(\sqrt{\dblen})$.

\heading{Verifying the write shares.}
\begin{figure}
  \oneCol{0.90}{

  \underline{proto $\verifykey(X_A, X_B)$}\\
  \vspace{-2pt}
  \begin{enumerate}[leftmargin=*]
    \item Server $A$: $\str{r_A, \vecs_A, \vecv} \gets X_A$; for each $i \in [x]$,
      do $\vect_A[i] \gets r_A[i] \cat \vecs_A[i]$. Server $B$ does the same
      with its input. Execute $\almosteq(\vect_A, \vect_B)$. If the result is
      $\reject$, then $A$ and $B$ output $\reject$.

    \item Server $A$: $\vecu_A \gets \sum_{i=1}^x \prg(\vecs_A[i])$.
      Server $B$: $\vecu_B \gets \vecv + \sum_{i=1}^x \prg(\vecs_B[i])$.
      Execute $\almosteq(\vecu_A, \vecu_B)$. Both $A$ and $B$ output the result.
  \end{enumerate}

  \vspace{2pt}
  \underline{proto $\almosteq(\vecv_A, \vecv_B)$}\\
  \vspace{-2pt}
  \begin{enumerate}[leftmargin=*]
    \item Let $m = |\vecv_A| = |\vecv_B|$. Servers $A$ and $B$ engage in a
      coin-flipping protocol \cite{telephone} in order to establish a shared
      $(mk + \lg m)$-bit string $R$. Let $(K_1, \ldots, K_m)$ denote the first
      $m$ $k$-bit chunks of $R$ and let $F$ denote the last $(\lg m)$-bit chunk.

    \item Server $A$: for each $i \in [m]$, let $m_i = \hash(K_i, \vecv_A[i])$.
      Let $f \in [m]$ denote the positive integer encoded by $F$. Send
      $(m_f, m_{f+1}, \ldots,$ $m_1, \ldots m_{f-1})$ to $C$. Server $B$ does
      the same.

    \item Auditor $C$ checks that the sequence of messages received from $A$ and $B$
      differ by exactly one element. Output $\accept$ if this holds and
      $\reject$ otherwise.
  \end{enumerate}
  \vspace{1pt}
  }
  \caption{A 3-party protocol for verifying the write shares are well-formed. The
  function $\hash$ is an $\epsilon$-almost-universal hash function with key
  space $\bits^k$.}
  \label{fig-riposte-2server+auditor}
\end{figure}
The request verification protocol for $\dpf$ suggested by \cite{riposte} is
specified in figure~\ref{fig-riposte-2server+auditor}. The principals are $A$,
$B$, and $C$ where the $A$ is the first write server, $B$ the second, and $C$ is
the auditor.
Our presentation differs from \cite{riposte} in two respects.
One, a step of the protocol requires that the write servers perform a
coin-flipping protocol in order sample from a family of pairwise-independent
universal hash functions.  However, in their implementation, they instantiate
this process by sharing keys for the Poly1305 almost-universal hash function
designed by Dan Bernstein \cite{poly1305}. We make this explicit in the
presentation.  Let $k,t \in \N$, $\epsilon \in (0,1]$, and $\hash : \bits^k \by
\bits^* \to \bits^t$ be an $\epsilon$-almost universal hash function.
Two, we require that $x$ and $y$ be powers of 2. This ensures that we always
perform the coin flipping protocol \cite{telephone} a finite number of times.

\heading{A complexity improvement.}
Implementing the coin-flipping protocol would be prohibitively expensive. In
practice, one would replace this protocol with a PRG, whose seed is shared by
$A$ and $B$,\cpnote{This was suggested to me by Henry.} but this makes the
analysis much more complex. Instead, we modify the protocol to use a PRF whose
key is shared by $A$ and $B$.

\noindent\hl{Things to do here:}
\begin{itemize}
  \item Specify the modified protocol $\reqvfy$. Show that it is complete.

  \item Show that $\reqvfy$ is disruption resistant if the underlying PRF
    satisfies its security notion.

  \item Show that the composition of $\dpf$ and $\reqvfy$ is PRIV2 secure for
    $t=1$ if the underlying primitives (a PRG and a PRF) satisfy their
    respective security notions.
\end{itemize}

\subsection{The $s$-server protocol}
\noindent\hl{Things to do here:}
\begin{itemize}
  \item Specify their $s$-share DPF scheme and $s$-server request verification
    protocol. This involves something they call a seed-homomorphic PRG.
\end{itemize}

\fi

\bibliography{main}
\bibliographystyle{alpha}

\end{document}
