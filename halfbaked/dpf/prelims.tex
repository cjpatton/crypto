% prelims.tex
%
% Notation, games.
\ignore{ %FIXME This makes syntax highlighting work.
  \begin{figure}
    Dumb figure
  \end{figure}
}
\label{sec-prelims}

\heading{Notation.}
Let $x \getsr \setS$ denote sampling an element~$x$ uniformly from a set~$S$.
%
If~$\advA$ is an algorithm, let $y \gets \advA(x_1, \ldots)$ denote
running~$\advA$ on input $x_1, \ldots$ and halting with output $y$.
%
Let $y \gets \advA(x_1, \ldots; \coins)$ denote running~$A$ with coins~$\coins$.
Let $y \getsr \advA(x_1, \ldots)$ denote sampling coins~$\coins$ uniformly and
letting $y = \advA(x_1, \ldots; \coins)$.
%
Let $[\advA(x_1, ...)]$ denote the set of possible outputs of $\advA$ run on input
$(x_1, \ldots \,; \coins)$ and with randomly sampled~$\coins$.
%
Any algorithm may output the distinguished symbol~$\bot$. In other words, the
symbol~$\bot$ is implicitly in the range of every algorithm.

Let $[i..j] = \{ \ell \in \Z : i \le \ell \le j \}$ and $[n] = [1..n]$.
%
If~$\vv$ is a vector, let~$\vv[i]$ and $\vv_i$ denote its~$i$-th element.

Let~$\emptystr$ denote the empty string.
%
Let~$\str(x_1, \ldots)$ denote an invertible encoding of the arbitrary
quantities~$x_1, \ldots$ as a string.
%
Let~$X \cat Y$ denote the concatenation of strings~$X$ and~$Y$.
%
If~$X$ is a string, let~$X[i]$ and $X_i$ denote the~$i$-th bit of string~$X$ and
let $\substr(X,i,j)$ denote the substring $X_i \cat \cdots \cat X_j$ for every $1
\leq i \leq j \leq |X|$.
%
If~$X$ and~$Y$ are equal-length strings, let~$X \xor Y$ denote their
bitwise-XOR.

Let~$\setS$ be a set of totally-ordered values. We say a sequence~$\vv$ is
indexed by~$\setS$ if $\vv = (\vv[x_1], \ldots, \vv[x_n])$ where $(x_1, \ldots,
x_n)$ is the total ordering of the elements of~$\setS$. This is denoted
$(\vv_x)_{x\in \setS}$.

\heading{Games.}
We adopt the game-playing framework of~\cite{games} with one exception: unless
otherwise stated, if a variable is undefined, then it is implicitly equal to the
formal symbol~$\bot$.
