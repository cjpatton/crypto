% intro.tex
%
% Introduction.
\label{sec-intro}

\hl{So here's where I wanna go with this.}
I've written up the scheme of Gilboa+Ishai in Section~\ref{sec-twowriter}.
Consider making this scheme ``verifiable''. Currently the idea is to add a
$0^t$ tag to the message. The write servers MAC the last~$t$ bits of each row of
their evaluated share; the auditor then checks to make sure the MACs are the same.
\begin{itemize}
  \item Give up on proofs for Riposte. What is the simplest, strongest notion
    that captures the security of such a mechanism?

  \item How to modify the Gilboa+Ishai construction? Does this require a
    stronger assumption about PRGs? If so, could I add more crypto in key share
    generation so that I don't need stronger assumptions?

  \item Gilboa+Ishai scheme is more naturally expressed in terms of strings~$X$
    and~$Y$ rather than table index~$\idx$ and message $\msg \in \ring
    \setminus\{0\}$, where $\ring$ is a ring. The reason to use this is for
    capturing the $(\sharect,\sharect-1)$-private Riposte scheme. Maybe just use
    this notation for this special case?

  \item Change \emph{write shares} to \emph{key shares}. Share verification
    makes sense both in the read-from and write-to settings.

  \item Frame the Riposte schemes as extensions of Gilboa+Ishai's level-1
    scheme.
\end{itemize}
\hl{And now back to our featured presentation.}

% PIR is the motivation for formalizing DPF schemes.
Distributed point functions were first proposed by Niv Gilboa and Yuval Ishai
\cite{dpf} as a building block for private information retrieval (PIR) systems,
whose study was initiated by Benny Chor et al \cite{pir}. The goal of such
systems is to allow a client to query a database without divulging to the
service provider the query or the result of the query.
%
Traditionally the query is a predicate of the database modeled as a binary
string~$D$. For example, the client might like to learn the~$\idx$-th bit of~$D$
without revealing~$\idx$.
%
Keyword search~\cite{pir-kws} is another application whereby the database
encodes a set of strings and the query is whether a particular string is in the
set.
%
Closely related is the problem of private information storage (PIS) put forward
by Rafail Ostrovsky and Victor Shoup~\cite{pis} where the goal is to write a bit
to~$\idx$-th row of~$D$ without revealing to the service provider which bit was
modified.

% Communication and threat model of PIR systems.
The most efficient PIR systems distribute the database among a set of servers.
%
The client maps its query to a sequence of shares and sends one to each
of~$\sharect$ servers. Each evaluates its share on its local state and returns
the result to the client. Combining the results of each of the servers yields
the result of the query.
%
Privacy of the user's query is achieved under the assumption of non-collusion,
meaning that honest servers do not communicate outside of the protocol.
Correspondingly, we consider security with respect to a \emph{coalition} of
malicious servers who may act arbitrarily to violate the privacy of the query.
%
The adversary is only given a partial view of the network. In particular, it
sees only the messages sent to a server in the coalition.
%
Informally, we say a PIR system is $(\sharect,t)$-private if no coalition of at
most~$t$ servers can deduce the query from their shares alone.

% Syntax of DPF schemes.
\heading{DPF schemes.}
Each of the queries described above can be written as a point function.
%
The \emph{point function} of $(X,Y) \in (\bits^*)^2$ is defined by $P_{X,Y}(X) =
Y$ and $P_{X,Y}(X^\prime) = 0^{|Y|}$ for every $X^\prime \ne X$.
%
Gilboa and Ishai define a \emph{distributed point function} (DPF) scheme as a
pair of algorithms $(\gen, \eval)$ where $\gen$ probabilistically maps $(X,Y)$
to a sequence of shares $(K_1, \ldots, K_\sharect)$ and $\eval$
deterministically maps $(K_i, X^\prime)$ to a $|Y|$-bit string such that
$P_{X,Y}(X^\prime) = \xor_{i=1}^s \eval(K_i, X^\prime)$ for every $X^\prime$.

% Bandwidth is the main concern.
A simple way to construct a DPF is to let $\eval(K_0, \cdot)$ be a random
function and let $\eval(K_1, \cdot) = P_{X,Y}(\cdot) \xor \eval(K_0, \cdot)$.
%
This yields a 2-server PIR protocol that is information theoretically
$(2,1)$-private, but the length of the shares is exponential in $|X|$. Indeed,
one of the main goals of PIR systems is to minimize the communication bandwidth.
%
It is well known that polynomial-length encodings are possible. Even shorter
encodings are possible in the computational setting; Gilboa and Ishai \cite{dpf}
give the most bandwidth efficient, 2-share DPF scheme known, which achieves
polylogarithmic bandwidth and is secure against polynomial-time adversaries.

% Overview of Riposte.
\heading{Riposte.}
Distributed point function schemes are central to the design of Riposte
\cite{riposte}, a cryptosystem recently proposed by Henry Corrigan-Gibbs, Dan
Boneh, and David Mazi\`{e}res,
which allows clients to anonymously write messages to a database. It can be used
by a service provider to collect usage data, crash reports, and other metrics
without revealing to the data collector who sent the report.\footnote{ This
application assumes, crucially, that the report itself does not contain
personally identifiable information about the sender.} It can also be used to
facilitate anonymous communication by making the contents of the database
public, with the advantage of being much more scalable than mix- or DC-nets
\cite{mix-nets,dc-nets} while providing stronger defense against traffic analysis
than onion routing networks \cite{tor}.

A client initiates a \emph{write request} by mapping its message to a sequence
of~$\sharect$ \emph{write shares} and sends each to one of~$\sharect$ distinct
\emph{write servers}. When a write server receives a share, it updates its local
state.  After processing the requests of~$\cohortct$ different clients, it
outputs its state to the data consumer. Finally, the data consumer recovers the
set of messages written to the database by combining the states of each of the
write servers.

% Overview of the security properties of Riposte.
The designers specify two DPF schemes: one that is $(2,1)$-private and another
that is $(\sharect, \sharect-1)$-private for any $\sharect$. Their use in the
Riposte system is similar to PIS. Each server $i$ maintains a share of the
database modeled as an $\dblen$-vector $\dbst_i$ of $n$-bit strings, each
initially equal to $0^n$. To write a message $\msg \in \bits^n$ into the
database, the client executes $(X_1, \ldots X_\sharect) \getsr \gen(\str{\idx},
\msg)$, where $\str{\idx}$ is the encoding of a randomly chosen
$\idx\in[\dblen]$. When it receives its share, server $i$ updates its state by
letting $\dbst_i[\idx^\prime] = \dbst_i[\idx^\prime] \xor \eval(X_i,
\str{\idx^\prime})$ for every $\idx^\prime\in[\dblen]$. (Recall that
$P_{\str{\idx},\msg}(\str{\idx^\prime}) = \xor_{i=1}^\sharect \eval(X_i,
\str{\idx^\prime})$.) Suppose client 1 writes $\msg_1$ into row $\idx_1$ and
then client 2 writes $\msg_2$ into a different row $\idx_2$.  Given their shares
and the final states of every server, no coalition of at most $\sharect-1$
servers can link either message to its sender given only its key shares and the
outputs of the servers. Intuitively, this is because the view of the adversary
is identically distributed no matter what order the clients send their write
requests.\cpnote{I'm being hand-wavy about this claim.  \cite{riposte} give a
security notion for anonymity, but I don't think they show that security of the
DPF implies anonymity. This could be interesting to show rigurously in our
paper.}

% The need for write share verification.
\heading{Disruption resistance.}
The fact that many users write to a single database poses a problem not
considered in the PIS setting. By sending one or more of the write servers a
mal-formed write share, a malicious client, or network adversary who intercepts
the client's messages, can corrupt the database state.
%
For example, suppose we let $\eval(X_1, \cdot)$ be a random function and
$\eval(X_2, \cdot) = P_{X,Y}(\cdot) \xor \eval(X_1, \cdot)$ as in the simple
$(2,1)$-private scheme described above.
%
A malicious client could instead let $\eval(X_2, \cdot)$ be a random function
independent of $\eval(X_2, \cdot)$.
As a result, the combined state $\dbst[\idx^\prime] = \dbst_1[\idx^\prime] \xor
\dbst_2[\idx^\prime]$ will be indistiguishable from a random string for every
$\idx^\prime \in[\dblen]$, rendering the entire database unrecoverable.

It is therefore crucial in this setting that the write servers be able to verify
their shares are well-formed, meaning that they yield the point function of some
index~$\idx$ and message~$\msg$. Roughly speaking, the system is
\emph{disruption resistant} \cite{riposte} if the write servers engage in a
protocol (possibly with a third party \emph{auditing server}), which allows them
to detect malicious clients, but leaks no information about the inputs (even to
the auditor). It is assumed that each server faithfully executes the protocol.
Disrupting the protocol amounts to a denial-of-service attack \cite{riposte},
which a malicious server can always do anyway by corrupting its own state.
Since we cannot hope to defend against disrupting servers, we require only that
disrupting the protocol does not violate privacy.
The designers give protocols for verifying write requests for both the 2-server
and $\sharect$-server variants of Riposte. The former involves a non-colluding
auditor, thus achieving greater efficiency than the latter, which requires an
expensive multiparty computation.

% I'm not sure if this is what I want the paper to say.
\if{0}
% Security of their composed notions is unclear.
Their proofs are modular in the sense that the privacy property of the DPF
scheme is treated separately from the privacy property of the write share
verification protocol. However, it is not clear that the respective security
notions compose in a way that ensures end-to-end security of the client's write
request. The DPF adversary is active in the sense that a coalition of
malicious servers may act arbitrarily to violate security. On the other hand,
the privacy adversary in the verification protocol is semi-honest in the sense
that the servers execute the protocol faithfully. (In particular, they do not
collude.) Since the latter adversary is weaker than the former, there might be
share verification protocols that are private with respect to semi-honest
adversaries, but not active, colluding ones.

% Thesis.
It is not our contention that the protocols of \cite{riposte} are not end-to-end
secure; in fact, we find that they are. Rather, our thesis is that a unified
analytical framework is needed in order to avoid proposing protocols that do not
compose securely. In addition, we find that this approach yields simpler, more
efficient designs.
\fi

\heading{Our contributions.}
We endow DPF schemes with a property called \emph{verifiability}, which demands
that the composition of the DPF scheme with a write request verification protocol
be secure. We define the simulation-based notion of \cite{dpf,riposte} for
standard DPF schemes, which models an active coalition of malicious write
servers (PRIV1). We give a new notion, which extends this model to include
the execution of the verification protocol (PRIV2). To accomplish this, we give
the adversary access to oracles, which provide it with its view of the
protocol's execution. In particular, we assume the adversary only has access to
messages sent to colluding servers.

\noindent\hl{Here's a list of things we can consider doing:}
\begin{enumerate}
  \item Prove the protocols of \cite{riposte} suffice for PRIV2, but show that
    more efficient protocols are possible in our new framework. In particular,
    we extend the execution model by allowing the write servers to have private
    keys. This immediately yields a more efficient variant of their
    2-server+auditor protocol by using a PRF with a shared key instead of
    telephone coin-flipping to establish a shared set of pairwise independent
    hash functions.

  \item 1-round, 2-serer+auditor protocol in a weaker trust model. (An auditor
    may collude with one write server.) What I'm thinking is that clients append
    $0^t$ to their message before applying $\gen$. The write server computes the
    ``tag'' from the write share by applying a PRF to the string resulting from
    successively concatenating the last $t$ bits of $Y_\idx = \eval(X_i, \idx)$
    for each $\idx$ from 1 to $\dblen$. The auditor just makes sure that the tags
    match. I'm not sure what properties are required of the PRG. Using the
    scheme of \cite{dpf}, this would yield a scheme that is much more efficient
    than Riposte in terms of communication complexity and bandwidth, yet
    functions in a weaker trust model. (Note that this answers two of their open
    questions in the affirmative.)

  \item 1-round, $2^t$-server+auditor protocol where the auditor is colluding,
    using the methods of \cite{dpf-multi-server}? This yields $(2^t,
    t)$-privacy.

  \item Reduce the round complexity of their $\sharect$-server protocol? Theirs
    is $(\sharect, \sharect-1)$-private. (This addresses one of their open
    questions.)

  \item The protocol in (1) actually achieves security against a stronger
    adversary, one that intercepts all messages sent between servers in the
    protocol. Consider the application where Google wants to collect crash
    reports. It operates each of the servers, but is also likely on path between
    each of the servers. (It probably operates the network infrastructure!)
    The protocols of \cite{riposte} DO NOT achieve this stronger notion,
    however.
\end{enumerate}
