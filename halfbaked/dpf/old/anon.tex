% anon.tex
%
% Application of verifiable DPF schemes to anonymous communication.
\label{sec-anonymity}

We describe the application of verifiable DPF schemes to anonymous communication
as specified by the Riposte system~\cite{riposte}. We parameterize the protocol
by a positive integer~$\cohortct$, called the \emph{cohort size}, which
determines the number of valid write requests the servers process before
outputting their state. The time interval in which each of the requests are
processed is referred to as an \emph{epoch}. Let~$\dpf$ be a DPF scheme with
parameters~$(\ring, \sharect, \dblen)$ and~$\reqvfy$ be a write request
verification protocol for~$\dpf$ with benign adversary~$\advB$ and principals $P
\supseteq \{\client, \writer_1, \ldots \writer_{\sharect}\}$ and let $\msgsp =
\ring \setminus \{0\}$.

\begin{itemize}
  \item To initiate a write request for $\msg \in \msgsp$, the client samples
    $\idx \getsr [\dblen]$, executes the share generation algorithm $(X_1,
    \ldots, X_{\sharect}) \getsr \gen(\idx, \msg)$, and sends~$X_i$
    to~$\writer_i$ for each $i \in [\sharect]$.

  \item Next, the request verification protocol is executed. This means that the
    \protosec~game is run with~$\reqvfy$, $I$, and~$\advB$ where for every $i
    \in [\sharect]$, we have that $I(\writer_i) = (X_i, \client)$.

  \item When server~$\writer_i$ accepts with private input~$X_i$, it updates its
    local state as follows. Let~$\dbst_i$ be a $\dblen$-vector over
    $\ring$ where each $\dbst_i[\idx]$ is initially equal to~$0$. Let
    $\dbst_i[\idx^\prime] = \dbst_i[\idx^\prime] + \eval(X_i, \idx^\prime)$
    for every~$\idx^\prime$.

  \item Finally, once~$\cohortct$ valid write requests have been processed, each
    write server its final state~$\dbst_i$ to the data consumer. The data
    consumer recovers the database state (and thus the set of messages) by
    computing $\dbst[\ell^\prime] = \sum_{i=1}^{\sharect}
    \dbst_i[\ell^\prime]$ for every~$\ell^\prime$.
\end{itemize}

\heading{Dealing with collisions.} Because the clients choose their index into
the database table randomly, there is a reasonable chance that two or more
inadvertently choose the same index. However, it can be shown that if the
database length is at least 20 times the cohort size, then average collision
rate will not exceed $5\%$ \cite{riposte}. Moreover, if~$\ring$ is a field, then
techniques based on error correcting codes can be used to further reduce the
collision rate \cite{riposte}.

\noindent\hl{Things we might do here:}
\begin{itemize}
  \item Formalize the security notion intended by this system (ANON). This was
    already done in \cite{riposte}.
  \item Show that PRIV1 implies ANON.  This was not done in \cite{riposte} as
    far as I can tell.
\end{itemize}
