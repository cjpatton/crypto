% intro.tex
%
% Introduction.
\label{sec-related}

\hl{Outline:}
\begin{itemize}

  % PIR
  \item Remark that single-server PIR systems are possible
    \cite{pir-single-server}, but at a higher computational cost. (OWFs are
    essential for security: http://dl.acm.org/citation.cfm?id=301277.) According
    to \cite{dpf}, PIS cannot be achieved with a single server \cite{dpf}.

  \item There's an interesting paper (http://dl.acm.org/citation.cfm?id=276723)
    that introduces symmetric PIR, designed to ensure that the reader learns
    nothing more than the bit they query. Do DPF schemes suffice in this
    setting?

  \item Although \cite{dpf} focus on 2-server protocols, methods described in
    \cite{dpf-multi-server} can be used to obtain $(2^t, t)$-private PIR from
    any $(2,1)$-private PIR scheme.

  % PIS
  \item A side-effect of our work is that PIS is more useful, since it can now
    support multiple writers while detecting malicious clients.

  \item \cite{pis} does consider a stronger ``active adversary'', but not one
    who tries to corrupt the database. This adversary may collude with users in
    order to try to learn the data being read/written to the database by the
    victim.

  \item Oblivious RAM is a related problem.

  % Riposte
  \item Under the ``Disruption resistance'' heading in the intro of
    \cite{riposte}, they list a bunch of AC systems in which malicious clients
    must be dealt with.

  \item Describe anonymous communication and the traffic analysis problem.
    Compare Riposte to mix- and DC-nets.
\end{itemize}

\iffalse % This about anonymous communication generally.
\heading{Information leakage.} Information is said to be identifiable if it can
be used to link a client to their message. There are two things to remember
about personally identifiable information: one, any identity information
conveyed by the client's message is necessarily leaked in the output of the
system; and two, any information is potentially identifiable given auxiliary
information about the clients. For example, the number of messages sent by the
client might identify which messages they sent. Anonymous communication systems
can only address the problem of linking sensitive data to the identity of the
client.
\fi
