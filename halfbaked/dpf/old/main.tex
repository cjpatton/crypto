% main.tex
%
% Root of the document tree. Includes the abstract and notes.
\documentclass[letter]{article}
\usepackage{color,soul}
\usepackage[dvipsnames]{xcolor}
\usepackage{enumitem}
% header.tex
%
% Formatting and common macros for crypto papers. Include this first.
\usepackage[dvipsnames]{xcolor}
\usepackage{color,soul}
\usepackage{enumitem}
\usepackage{xspace}
\usepackage{amsmath}
\usepackage{amssymb}
\usepackage{amsfonts}
\usepackage{xcolor}
\usepackage{xstring}
\usepackage{enumitem}


\usepackage{tikz}
\tikzset{>=latex}

\usepackage{listings}
\lstset{
  basicstyle=\small\ttfamily,
  columns=flexible,
  breaklines=true
}

\usepackage{hyperref}
\hypersetup{
  colorlinks,%
  citecolor=black,%
  filecolor=black,%
  linkcolor=black,%
  urlcolor=black
}

% Colors
\definecolor{theblue}{RGB}{85,135,170}
\definecolor{thedarkgray}{RGB}{95,95,95}
\definecolor{thedarkgreen}{RGB}{50,127,0}
\definecolor{thegray}{RGB}{165,165,165}
\definecolor{thelightblue}{RGB}{185,220,250}
\definecolor{thelightgray}{RGB}{230,230,230}
\definecolor{thelightred}{RGB}{250,185,165}
\definecolor{thered}{RGB}{210,145,125}

% Editorial
\newcommand{\ala}{\text{a la}\xspace}
\newcommand{\etal}{et al.\xspace}
\newcommand{\apriori}{\textit{a priori}\xspace}
\newcommand{\viceversa}{\textit{vice versa}\xspace}

% Fonts
\newcommand{\algorithmfont}[1]{\mathcal{#1}}
\newcommand{\codefont}[1]{\lstinline|#1|}
\newcommand{\cryptofont}[1]{\textup{\textbf{#1}}\hspace{0.5pt}}
\newcommand{\expfont}[1]{{{\tiny\MakeLowercase{\textnormal{#1}}}}}
\newcommand{\lifont}[1]{\textsf{\footnotesize\color{thegray}#1}}
\newcommand{\notionfont}[1]{{#1}\xspace}
\newcommand{\oraclefont}[1]{\cryptofont{#1}}
\newcommand{\procfont}[1]{\textnormal{#1}\hspace{0.33pt}}
\newcommand{\schemefont}[1]{\mathcal{#1}}
\newcommand{\setfont}[1]{\mathcal{#1}}
\newcommand{\rwordfont}[1]{\cryptofont{#1}\xspace}
\newcommand{\strfont}[1]{\textup{\textsf{{\color{theblue}#1}}}}
\newcommand{\varfont}[1]{\mathit{#1}}
\newcommand{\symbolfont}[1]{\textsf{\footnotesize\color{thegray}#1}}

\newcommand{\stringer}[1]{\strfont{#1}}
\newcommand{\rword}[1]{\rwordfont{#1}}

% Crypto functions
\newcommand{\Hybrid}{\cryptofont{H}}
\newcommand{\Gam}{\cryptofont{G}}
\newcommand{\Exp}[1]{\cryptofont{Exp}^{\expfont{#1}}}
\newcommand{\Adv}[1]{\cryptofont{Adv}^{\expfont{#1}}}

% Types
\newcommand{\typ}[1]{\typs\rword{#1}}
\newcommand{\typs}{\,}
\newcommand{\us}{\!} % previously \unskip

\newcommand{\rbool}{\typ{bool}}
\newcommand{\relem}[1]{\typ{elem}(#1)}
\newcommand{\rint}{\typ{int}}
\newcommand{\rset}{\typ{set}}
\newcommand{\rstr}{\typ{str}}
\newcommand{\rstruct}{\typ{struct}}
\newcommand{\rtuple}{\typ{tup}}

\newcommand{\rdet}{\rword{det}\typs}
\newcommand{\rtype}{\rword{type}}
\newcommand{\rrand}{\rword{rand}\typs}

%% Fancy syntax stuff
\newcommand{\nil}{\symbolfont{nil}}
\newcommand{\ptr}{\typs\symbolfont{*}\us}
\newcommand{\rfr}{\symbolfont{\&}}
\newcommand{\any}{\symbolfont{\_}}

%% Reserved words
\newcommand{\rand}{\rwordfont{and}}
\newcommand{\rdeclare}{\rwordfont{dec}}
\newcommand{\relse}{\rwordfont{else}}
\newcommand{\rif}{\rwordfont{if}}
\newcommand{\rwhile}{\rwordfont{while}}
\newcommand{\runtil}{\rwordfont{until}}
\newcommand{\rdo}{\rwordfont{do}}
\newcommand{\ror}{\rwordfont{or}}
\newcommand{\rfor}{\rwordfont{for}}
\newcommand{\rforeach}{\rwordfont{for each}}
\newcommand{\rreturn}{\rwordfont{ret}}
\newcommand{\rbreak}{\rwordfont{break}}
\newcommand{\rthen}{\rwordfont{then}}
\newcommand{\rto}{\rwordfont{to}}
\newcommand{\rswitch}{\rwordfont{switch}}
\newcommand{\rcase}{\rwordfont{case}}
\newcommand{\rdefault}{\rwordfont{default}}

% Math
\DeclareMathAlphabet\mathbfcal{OMS}{cmsy}{b}{n}

\def\v.#1{\boldsymbol{#1}}
\newcommand{\dqed}{\hfill$\Diamond$}
\newcommand{\GF}{\mathit{GF}}

%% Sets
\DeclareMathOperator*{\argmin}{arg\,min}
\DeclareMathOperator*{\dom}{Dom}
\DeclareMathOperator*{\rng}{Rng}
\def\bydef{\stackrel{\rm def}{=}}
\newcommand*\by{\times}
\newcommand{\getsr}{\xleftarrow{\text{\tiny{\$}}}}
%\newcommand{\getsr}{{\:{\leftarrow{\hspace*{-3pt}\raisebox{.75pt}{$\scriptscriptstyle\$$}}}\:}}
\newcommand*\intersection{\cap}
\newcommand{\multiunion}{\uplus}
\newcommand*\union{\cup}

%% String/tuple operators
\def\slicerange#1,#2{#1\mbox{\hspace{0.5pt}:\hspace{0.5pt}}#2}
\newcommand{\bits}{\{0,1\}}
\newcommand{\cat}{\, \| \,}
\newcommand{\emptystr}{\varepsilon}
\newcommand{\emptystring}{\emptystr}
\newcommand{\hassubstr}{\sim}
\newcommand{\prefixes}{\preceq}
\newcommand{\slice}[2]{#1[\slicerange#2]}
\newcommand{\toint}[2]{\underline{\smash{#1}}_{\hspace{0.5pt}#2}}
\newcommand{\tostr}[1]{\underline{\smash{#1}}}
\newcommand{\trunc}{\%}

%% Boolean operators
\newcommand*\band{\wedge}
\newcommand*\bor{\vee}
\newcommand*\bnot{\neg}
\newcommand*\bimplies{\implies}
\newcommand*\bxor{\mathbin{\oplus}}

%% Probability
\newcommand{\E}{\mathrm{E}}
\newcommand{\Prob}[1]{\Pr\hspace{-1pt}\big[\,#1\,\big]}
\newcommand{\Probs}[2]{\Pr_{#1}\hspace{-1pt}{\big[\,#2\,\big]}}
\newcommand{\given}{\mid}

% References
\newcommand{\fig}[1]{Fig.~\ref{fig:#1}}
\newcommand{\lem}[1]{Lemma~\ref{lemma:#1}}

% Dev
\def\coloroverride{0}
\newenvironment{cleanup}{
  \def\coloroverride{1}
  \color{gray}
}{
}

% Notes
\newcounter{notectr}[section]
\newcommand{\getnotectr}{\stepcounter{notectr}\thesection.\thenotectr}

\newcommand{\basenoteshape}[3]{(\getnotectr: #1\xspace#2: #3)}

\newcommand{\basenote}[4]{%
  \textrm{%
    \IfEqCase{\coloroverride}{%
        {1}{\basenoteshape{#2}{#3}{#4}}%
        {0}{\textcolor{#1}{\basenoteshape{#2}{#3}{#4}}}%
    }[\PackageError{basenote}{Undefined option for basenote: \coloroverride}{}]%
  }%
}

%% Uncomment to mute notes.
%\renewcommand{\basenote}[4]{\ignorespaces}

\newcommand{\todo}[3]{\basenote{#1}{#2}{todo}{#3}}
\newcommand{\fixme}[1]{\basenote{red}{}{fixme}{{\bf#1}}}

\newcommand{\ignore}[1]{\if{0}#1\fi}

% Games
\renewcommand{\#}[1]{\text{\textcolor{theblue}{\,\text{//}\,{\small #1}}}}
\newcommand{\g}[1]{\cryptofont{G}_{#1}}
\newcommand{\hybrid}[1]{\cryptofont{H}_{#1}}
\newcommand{\gouts}{=}
\newcommand{\gsets}{\,\cryptofont{sets}\,}
\newcommand{\tab}{\hspace*{10pt}}

%% Diffs
\newcommand{\diffplus}[1]{\fbox{#1}}
\newcommand{\diffplusbox}[1]{\fbox{\parbox{\dimexpr\textwidth-2\fboxsep-2\fboxrule\relax}{#1}}\vspace{1pt}}
\newcommand{\diffminus}[1]{\colorbox{lightgray}{#1}}
\newcommand{\diffminusbox}[1]{\colorbox{lightgray}{\parbox{\dimexpr\textwidth-2\fboxsep-2\fboxrule\relax}{#1}}}
\newcommand{\diff}[2]{\diffminus{#1}\diffplus{#2}}

%% Line numbers
\newcommand{\liref}[2]{\fig{#1}:#2}
\newcounter{FunctionCounter}
\newcounter{LineCounter}[FunctionCounter]

%%% Reset numbering (new game)
\newcommand{\ga}{\setcounter{FunctionCounter}{0}\setcounter{LineCounter}{0}}

%%% Increment function number
\newcommand{\fu}{
%  \stepcounter{FunctionCounter}
}

%%% Increment line number
\usepackage{calc}
\newcommand{\li}{
%  \stepcounter{LineCounter}\lifont{\scriptsize\theFunctionCounter\ifnum\value{LineCounter}<10\ignorespaces0\fi\theLineCounter}
  \stepcounter{LineCounter}\makebox[8pt][r]{\lifont{\scriptsize\theLineCounter}}\hspace{2pt}
}

%% Boxes
\newcommand{\gamesfontsize}{\small}
\newcommand{\gamespadleft}{\hskip 1pt}
\newcommand{\gamespad}{\hskip 4pt}
\newcommand{\gamespacer}{\\[-6.5pt]}

\newcommand{\oneCol}[2]{
  \begin{center}
    \makebox[\textwidth][l]{
      \begin{tabular}{|@{\gamespadleft}l@{\gamespad}@{}|}
      \hline
      \rule{0pt}{1\normalbaselineskip}
      \begin{minipage}[t]{#1\textwidth}\gamesfontsize
        #2 \vspace{6pt}
      \end{minipage} \\
      \hline
    \end{tabular}
    }
  \end{center}
}

\newcommand{\twoCols}[3]{
  \makebox[\textwidth][c]{
    \begin{tabular}{|@{\gamespadleft}l@{\gamespad}|@{}@{\gamespad}l@{\gamespad}|}
    \hline
    \rule{0pt}{1\normalbaselineskip}
    \begin{minipage}[t]{#1\textwidth}\gamesfontsize
      #2 \vspace{6pt}
    \end{minipage} &
    \begin{minipage}[t]{#1\textwidth}\gamesfontsize
      #3 \vspace{6pt}
    \end{minipage} \\
    \hline
  \end{tabular}
  }
}

\newcommand{\twoColsNoBox}[3]{
  \makebox[\textwidth][c]{
    \begin{tabular}{@{\gamespadleft}l@{\gamespad}|@{}@{\gamespad}l@{\gamespad}}
    \rule{0pt}{1\normalbaselineskip}
    \begin{minipage}[t]{#1\textwidth}\gamesfontsize
      #2 \vspace{6pt}
    \end{minipage} &
    \begin{minipage}[t]{#1\textwidth}\gamesfontsize
      #3 \vspace{6pt}
    \end{minipage} \\
  \end{tabular}
  }
}

\newcommand{\twoColsUnbalanced}[4]{
  \makebox[\textwidth][c]{
    \begin{tabular}{|@{\gamespadleft}l@{\gamespad}|@{}@{\gamespad}l@{\gamespad}|}
    \hline
    \rule{0pt}{1\normalbaselineskip}
    \begin{minipage}[t]{#1\textwidth}\gamesfontsize
      #3 \vspace{6pt}
    \end{minipage} &
    \begin{minipage}[t]{#2\textwidth}\gamesfontsize
      #4 \vspace{6pt}
    \end{minipage} \\
    \hline
  \end{tabular}
  }
}

\newcommand{\twoColsNoLinesUnbalanced}[4]{
  \makebox[\textwidth][c]{
    \begin{tabular}{@{\gamespadleft}l@{\gamespad}@{}@{\gamespad}l@{\gamespad}}
    \rule{0pt}{1\normalbaselineskip}
    \begin{minipage}[t]{#1\textwidth}\gamesfontsize
      #3 \vspace{6pt}
    \end{minipage} &
    \begin{minipage}[t]{#2\textwidth}\gamesfontsize
      #4 \vspace{6pt}
    \end{minipage} \\
  \end{tabular}
  }
}

\newcommand{\twoColsNoDivide}[3]{
  \makebox[\textwidth][c]{
    \begin{tabular}{|@{\gamespadleft}l@{\gamespad}@{}@{\gamespad}l@{\gamespad}|}
    \hline
    \rule{0pt}{1\normalbaselineskip}
    \begin{minipage}[t]{#1\textwidth}\gamesfontsize
      #2 \vspace{6pt}
    \end{minipage} &
    \begin{minipage}[t]{#1\textwidth}\gamesfontsize
      #3 \vspace{6pt}
    \end{minipage} \\
    \hline
  \end{tabular}
  }
}

\newcommand{\twoColsNoDivideUnbalanced}[4]{
  \makebox[\textwidth][c]{
    \begin{tabular}{|@{\gamespadleft}l@{\gamespad}@{}@{\gamespad}l@{\gamespad}|}
    \hline
    \rule{0pt}{1\normalbaselineskip}
    \begin{minipage}[t]{#1\textwidth}\gamesfontsize
      #3 \vspace{6pt}
    \end{minipage} &
    \begin{minipage}[t]{#2\textwidth}\gamesfontsize
      #4 \vspace{6pt}
    \end{minipage} \\
    \hline
  \end{tabular}
  }
}

\newcommand{\twoColsTwoRows}[5]{
  \makebox[\textwidth][c]{
  \begin{tabular}{|@{\gamespadleft}l@{\gamespad}|@{}@{\gamespad}l@{\gamespad}|}
    \hline
    \rule{0pt}{1\normalbaselineskip}
    \begin{minipage}[t]{#1\textwidth}\gamesfontsize
      #2 \vspace{6pt}
    \end{minipage} &
    \begin{minipage}[t]{#1\textwidth}\gamesfontsize
      #3 \vspace{6pt}
    \end{minipage} \\
    \hline
    \rule{0pt}{1\normalbaselineskip}
    \begin{minipage}[t]{#1\textwidth}\gamesfontsize
      #4 \vspace{6pt}
    \end{minipage} &
    \begin{minipage}[t]{#1\textwidth}\gamesfontsize
      #5 \vspace{6pt}
    \end{minipage} \\
    \hline
  \end{tabular}
  }
}

\newcommand{\threeCols}[4]{
  \makebox[\textwidth][c]{
    \begin{tabular}{|@{\gamespadleft}l@{\gamespad}|@{}@{\gamespad}l@{\gamespad}|@{}@{\gamespad}l@{\gamespad}|}
    \hline
    \rule{0pt}{1\normalbaselineskip}
    \begin{minipage}[t]{#1\textwidth}\gamesfontsize
      #2 \vspace{6pt}
    \end{minipage} &
    \begin{minipage}[t]{#1\textwidth}\gamesfontsize
      #3 \vspace{6pt}
    \end{minipage} &
    \begin{minipage}[t]{#1\textwidth}\gamesfontsize
      #4 \vspace{6pt}
    \end{minipage} \\
    \hline
  \end{tabular}
  }
}

\newcommand{\threeColsUnbalanced}[6]{
  \makebox[\textwidth][c]{
    \begin{tabular}{|@{\gamespadleft}l@{\gamespad}|@{}@{\gamespad}l@{\gamespad}|@{}@{\gamespad}l@{\gamespad}|}
    \hline
    \rule{0pt}{1\normalbaselineskip}
    \begin{minipage}[t]{#1\textwidth}\gamesfontsize
      #4 \vspace{6pt}
    \end{minipage} &
    \begin{minipage}[t]{#2\textwidth}\gamesfontsize
      #5 \vspace{6pt}
    \end{minipage} &
    \begin{minipage}[t]{#3\textwidth}\gamesfontsize
      #6 \vspace{6pt}
    \end{minipage} \\
    \hline
  \end{tabular}
  }
}

\newcommand{\threeColsFirstDivideUnbalanced}[6]{
  \makebox[\textwidth][c]{
    \begin{tabular}{|@{\gamespadleft}l@{\gamespad}|@{}@{\gamespad}l@{\gamespad}@{}@{\gamespad}l@{\gamespad}|}
    \hline
    \rule{0pt}{1\normalbaselineskip}
    \begin{minipage}[t]{#1\textwidth}\gamesfontsize
      #4 \vspace{6pt}
    \end{minipage} &
    \begin{minipage}[t]{#2\textwidth}\gamesfontsize
      #5 \vspace{6pt}
    \end{minipage} &
    \begin{minipage}[t]{#3\textwidth}\gamesfontsize
      #6 \vspace{6pt}
    \end{minipage} \\
    \hline
  \end{tabular}
  }
}

\newcommand{\threeColsSecondDivideUnbalanced}[6]{
  \makebox[\textwidth][c]{
    \begin{tabular}{|@{\gamespadleft}l@{\gamespad}@{}@{\gamespad}l@{\gamespad}@{}|@{\gamespad}l@{\gamespad}|}
    \hline
    \rule{0pt}{1\normalbaselineskip}
    \begin{minipage}[t]{#1\textwidth}\gamesfontsize
      #4 \vspace{6pt}
    \end{minipage} &
    \begin{minipage}[t]{#2\textwidth}\gamesfontsize
      #5 \vspace{6pt}
    \end{minipage} &
    \begin{minipage}[t]{#3\textwidth}\gamesfontsize
      #6 \vspace{6pt}
    \end{minipage} \\
    \hline
  \end{tabular}
  }
}

\newcommand{\threeColsNoDivideUnbalanced}[6]{
  \makebox[\textwidth][c]{
    \begin{tabular}{|@{\gamespadleft}l@{\gamespad}@{}@{\gamespad}l@{\gamespad}@{}@{\gamespad}l@{\gamespad}|}
    \hline
    \rule{0pt}{1\normalbaselineskip}
    \begin{minipage}[t]{#1\textwidth}\gamesfontsize
      #4 \vspace{6pt}
    \end{minipage} &
    \begin{minipage}[t]{#2\textwidth}\gamesfontsize
      #5 \vspace{6pt}
    \end{minipage} &
    \begin{minipage}[t]{#3\textwidth}\gamesfontsize
      #6 \vspace{6pt}
    \end{minipage} \\
    \hline
  \end{tabular}
  }
}

\newcommand{\fourCols}[5]{
  \makebox[\textwidth][c]{
    \begin{tabular}{|@{\gamespadleft}l@{\gamespad}|@{}@{\gamespad}l@{\gamespad}|@{}@{\gamespad}l@{\gamespad}|@{}@{\gamespad}l@{\gamespad}|}
    \hline
    \rule{0pt}{1\normalbaselineskip}
    \begin{minipage}[t]{#1\textwidth}\gamesfontsize
      #2 \vspace{6pt}
    \end{minipage} &
    \begin{minipage}[t]{#1\textwidth}\gamesfontsize
      #3 \vspace{6pt}
    \end{minipage} &
    \begin{minipage}[t]{#1\textwidth}\gamesfontsize
      #4 \vspace{6pt}
    \end{minipage} &
    \begin{minipage}[t]{#1\textwidth}\gamesfontsize
      #5 \vspace{6pt}
    \end{minipage} \\
    \hline
  \end{tabular}
  }
}

\newcommand{\twoColsThreeRows}[7]{
  \makebox[\textwidth][c]{
  \begin{tabular}{|@{\gamespadleft}l@{\gamespad}|@{}@{\gamespad}l@{\gamespad}|}
    \hline
    \rule{0pt}{1\normalbaselineskip}
    \begin{minipage}[t]{#1\textwidth}\gamesfontsize
      #2 \vspace{6pt}
    \end{minipage} &
    \begin{minipage}[t]{#1\textwidth}\gamesfontsize
      #3 \vspace{6pt}
    \end{minipage} \\
    \hline
    \rule{0pt}{1\normalbaselineskip}
    \begin{minipage}[t]{#1\textwidth}\gamesfontsize
      #4 \vspace{6pt}
    \end{minipage} &
    \begin{minipage}[t]{#1\textwidth}\gamesfontsize
      #5 \vspace{6pt}
    \end{minipage} \\
    \hline
    \rule{0pt}{1\normalbaselineskip}
    \begin{minipage}[t]{#1\textwidth}\gamesfontsize
      #6 \vspace{6pt}
    \end{minipage} &
    \begin{minipage}[t]{#1\textwidth}\gamesfontsize
       #7 \vspace{6pt}
    \end{minipage} \\
    \hline
  \end{tabular}
  }
}

% macros.tex
%
% Macros for this paper. (Include build/headers.tex, then this.)
\usepackage[dvipdf]{graphicx}
\usepackage{pifont} % \ding{109}
\usepackage{afterpage} % \afterpage{ ... }

% Variables
\newcommand{\dist}{\delta}
\newcommand{\doride}{\varphi}
\newcommand{\De}{\procfont{De}}
\newcommand{\En}{\procfont{En}}
\newcommand{\Ev}{\procfont{Ev}}
\newcommand{\ev}{\procfont{ev}}
\newcommand{\garbler}{\schemefont{G}}
\newcommand{\Gb}{\procfont{Gb}}
\newcommand{\id}{\funcfont{id}}
\newcommand{\loc}{\funcfont{loc}}
\newcommand{\rep}{\funcfont{rep}}
\newcommand{\SpA}{\procfont{SpA}}
\newcommand{\SpB}{\procfont{SpB}}

% Authors' comments.
\definecolor{darkgreen}{RGB}{50,127,0}
\newcommand{\cpnote}[1]{\note{darkgreen}{Chris}{#1}}
\newcommand{\tsnote}[1]{\note{blue}{Tom}{#1}}
\newcommand{\cptodo}[1]{\todo{red}{Chris}{#1}}
\newcommand{\tstodo}[1]{\todo{red}{Tom}{#1}}
\newcommand{\anytodo}[1]{\todo{red}{Any}{#1}}


\date{\today}
\title{\textbf{Verifiable DPF Schemes}}
\author{Christopher Patton}

\setcounter{tocdepth}{2}

\begin{document}

\maketitle

\begin{abstract}
  The point function of binary strings~$X$ and~ $Y$ is defined by $P_{X,Y}(X) =
  Y$ and $P_{X,Y}(X^\prime) = 0^{|Y|}$ for all $X^\prime \ne X$. A distributed
  point function (DPF) scheme maps~$P_{X,Y}$ to a sequence of shares which, when
  combined, yield the point function, but from a subset of which neither~$X$
  nor~$Y$ can be recovered. This primitive was first suggested in the context of
  private information retrieval where the goal is to allow clients to \emph{read
  from} a database distributed among a set of servers without revealing which
  data were read. It was recently proposed as a way to address traffic analysis
  attacks in the context of anonymous communication.
  %
  The goal of the Riposte cryptosystem \cite{riposte} is to allow clients to
  \emph{write to} a distributed database without revealing to the servers which
  data were written. However, by sending the servers mal-formed shares, a
  malicious client or network adversary can easily corrupt the database state.
  As pointed out by the designers, it is crucial in this setting that the
  servers be able to verify their shares are correct. We endow DPF schemes with
  a property we call \emph{verifiability} and put forward a unified framework
  for their analysis, which yields more efficient verification protocols with
  stronger security properties.
\end{abstract}

\section{Introduction}
% intro.tex
%
% Introduction.
\label{sec-intro}

\hl{So here's where I wanna go with this.}
I've written up the scheme of Gilboa+Ishai in Section~\ref{sec-twowriter}.
Consider making this scheme ``verifiable''. Currently the idea is to add a
$0^t$ tag to the message. The write servers MAC the last~$t$ bits of each row of
their evaluated share; the auditor then checks to make sure the MACs are the same.
\begin{itemize}
  \item Give up on proofs for Riposte. What is the simplest, strongest notion
    that captures the security of such a mechanism?

  \item How to modify the Gilboa+Ishai construction? Does this require a
    stronger assumption about PRGs? If so, could I add more crypto in key share
    generation so that I don't need stronger assumptions?

  \item Gilboa+Ishai scheme is more naturally expressed in terms of strings~$X$
    and~$Y$ rather than table index~$\idx$ and message $\msg \in \ring
    \setminus\{0\}$, where $\ring$ is a ring. The reason to use this is for
    capturing the $(\sharect,\sharect-1)$-private Riposte scheme. Maybe just use
    this notation for this special case?

  \item Change \emph{write shares} to \emph{key shares}. Share verification
    makes sense both in the read-from and write-to settings.

  \item Frame the Riposte schemes as extensions of Gilboa+Ishai's level-1
    scheme.
\end{itemize}
\hl{And now back to our featured presentation.}

% PIR is the motivation for formalizing DPF schemes.
Distributed point functions were first proposed by Niv Gilboa and Yuval Ishai
\cite{dpf} as a building block for private information retrieval (PIR) systems,
whose study was initiated by Benny Chor et al \cite{pir}. The goal of such
systems is to allow a client to query a database without divulging to the
service provider the query or the result of the query.
%
Traditionally the query is a predicate of the database modeled as a binary
string~$D$. For example, the client might like to learn the~$\idx$-th bit of~$D$
without revealing~$\idx$.
%
Keyword search~\cite{pir-kws} is another application whereby the database
encodes a set of strings and the query is whether a particular string is in the
set.
%
Closely related is the problem of private information storage (PIS) put forward
by Rafail Ostrovsky and Victor Shoup~\cite{pis} where the goal is to write a bit
to~$\idx$-th row of~$D$ without revealing to the service provider which bit was
modified.

% Communication and threat model of PIR systems.
The most efficient PIR systems distribute the database among a set of servers.
%
The client maps its query to a sequence of shares and sends one to each
of~$\sharect$ servers. Each evaluates its share on its local state and returns
the result to the client. Combining the results of each of the servers yields
the result of the query.
%
Privacy of the user's query is achieved under the assumption of non-collusion,
meaning that honest servers do not communicate outside of the protocol.
Correspondingly, we consider security with respect to a \emph{coalition} of
malicious servers who may act arbitrarily to violate the privacy of the query.
%
The adversary is only given a partial view of the network. In particular, it
sees only the messages sent to a server in the coalition.
%
Informally, we say a PIR system is $(\sharect,t)$-private if no coalition of at
most~$t$ servers can deduce the query from their shares alone.

% Syntax of DPF schemes.
\heading{DPF schemes.}
Each of the queries described above can be written as a point function.
%
The \emph{point function} of $(X,Y) \in (\bits^*)^2$ is defined by $P_{X,Y}(X) =
Y$ and $P_{X,Y}(X^\prime) = 0^{|Y|}$ for every $X^\prime \ne X$.
%
Gilboa and Ishai define a \emph{distributed point function} (DPF) scheme as a
pair of algorithms $(\gen, \eval)$ where $\gen$ probabilistically maps $(X,Y)$
to a sequence of shares $(K_1, \ldots, K_\sharect)$ and $\eval$
deterministically maps $(K_i, X^\prime)$ to a $|Y|$-bit string such that
$P_{X,Y}(X^\prime) = \xor_{i=1}^s \eval(K_i, X^\prime)$ for every $X^\prime$.

% Bandwidth is the main concern.
A simple way to construct a DPF is to let $\eval(K_0, \cdot)$ be a random
function and let $\eval(K_1, \cdot) = P_{X,Y}(\cdot) \xor \eval(K_0, \cdot)$.
%
This yields a 2-server PIR protocol that is information theoretically
$(2,1)$-private, but the length of the shares is exponential in $|X|$. Indeed,
one of the main goals of PIR systems is to minimize the communication bandwidth.
%
It is well known that polynomial-length encodings are possible. Even shorter
encodings are possible in the computational setting; Gilboa and Ishai \cite{dpf}
give the most bandwidth efficient, 2-share DPF scheme known, which achieves
polylogarithmic bandwidth and is secure against polynomial-time adversaries.

% Overview of Riposte.
\heading{Riposte.}
Distributed point function schemes are central to the design of Riposte
\cite{riposte}, a cryptosystem recently proposed by Henry Corrigan-Gibbs, Dan
Boneh, and David Mazi\`{e}res,
which allows clients to anonymously write messages to a database. It can be used
by a service provider to collect usage data, crash reports, and other metrics
without revealing to the data collector who sent the report.\footnote{ This
application assumes, crucially, that the report itself does not contain
personally identifiable information about the sender.} It can also be used to
facilitate anonymous communication by making the contents of the database
public, with the advantage of being much more scalable than mix- or DC-nets
\cite{mix-nets,dc-nets} while providing stronger defense against traffic analysis
than onion routing networks \cite{tor}.

A client initiates a \emph{write request} by mapping its message to a sequence
of~$\sharect$ \emph{write shares} and sends each to one of~$\sharect$ distinct
\emph{write servers}. When a write server receives a share, it updates its local
state.  After processing the requests of~$\cohortct$ different clients, it
outputs its state to the data consumer. Finally, the data consumer recovers the
set of messages written to the database by combining the states of each of the
write servers.

% Overview of the security properties of Riposte.
The designers specify two DPF schemes: one that is $(2,1)$-private and another
that is $(\sharect, \sharect-1)$-private for any $\sharect$. Their use in the
Riposte system is similar to PIS. Each server $i$ maintains a share of the
database modeled as an $\dblen$-vector $\dbst_i$ of $n$-bit strings, each
initially equal to $0^n$. To write a message $\msg \in \bits^n$ into the
database, the client executes $(X_1, \ldots X_\sharect) \getsr \gen(\str{\idx},
\msg)$, where $\str{\idx}$ is the encoding of a randomly chosen
$\idx\in[\dblen]$. When it receives its share, server $i$ updates its state by
letting $\dbst_i[\idx^\prime] = \dbst_i[\idx^\prime] \xor \eval(X_i,
\str{\idx^\prime})$ for every $\idx^\prime\in[\dblen]$. (Recall that
$P_{\str{\idx},\msg}(\str{\idx^\prime}) = \xor_{i=1}^\sharect \eval(X_i,
\str{\idx^\prime})$.) Suppose client 1 writes $\msg_1$ into row $\idx_1$ and
then client 2 writes $\msg_2$ into a different row $\idx_2$.  Given their shares
and the final states of every server, no coalition of at most $\sharect-1$
servers can link either message to its sender given only its key shares and the
outputs of the servers. Intuitively, this is because the view of the adversary
is identically distributed no matter what order the clients send their write
requests.\cpnote{I'm being hand-wavy about this claim.  \cite{riposte} give a
security notion for anonymity, but I don't think they show that security of the
DPF implies anonymity. This could be interesting to show rigurously in our
paper.}

% The need for write share verification.
\heading{Disruption resistance.}
The fact that many users write to a single database poses a problem not
considered in the PIS setting. By sending one or more of the write servers a
mal-formed write share, a malicious client, or network adversary who intercepts
the client's messages, can corrupt the database state.
%
For example, suppose we let $\eval(X_1, \cdot)$ be a random function and
$\eval(X_2, \cdot) = P_{X,Y}(\cdot) \xor \eval(X_1, \cdot)$ as in the simple
$(2,1)$-private scheme described above.
%
A malicious client could instead let $\eval(X_2, \cdot)$ be a random function
independent of $\eval(X_2, \cdot)$.
As a result, the combined state $\dbst[\idx^\prime] = \dbst_1[\idx^\prime] \xor
\dbst_2[\idx^\prime]$ will be indistiguishable from a random string for every
$\idx^\prime \in[\dblen]$, rendering the entire database unrecoverable.

It is therefore crucial in this setting that the write servers be able to verify
their shares are well-formed, meaning that they yield the point function of some
index~$\idx$ and message~$\msg$. Roughly speaking, the system is
\emph{disruption resistant} \cite{riposte} if the write servers engage in a
protocol (possibly with a third party \emph{auditing server}), which allows them
to detect malicious clients, but leaks no information about the inputs (even to
the auditor). It is assumed that each server faithfully executes the protocol.
Disrupting the protocol amounts to a denial-of-service attack \cite{riposte},
which a malicious server can always do anyway by corrupting its own state.
Since we cannot hope to defend against disrupting servers, we require only that
disrupting the protocol does not violate privacy.
The designers give protocols for verifying write requests for both the 2-server
and $\sharect$-server variants of Riposte. The former involves a non-colluding
auditor, thus achieving greater efficiency than the latter, which requires an
expensive multiparty computation.

% I'm not sure if this is what I want the paper to say.
\if{0}
% Security of their composed notions is unclear.
Their proofs are modular in the sense that the privacy property of the DPF
scheme is treated separately from the privacy property of the write share
verification protocol. However, it is not clear that the respective security
notions compose in a way that ensures end-to-end security of the client's write
request. The DPF adversary is active in the sense that a coalition of
malicious servers may act arbitrarily to violate security. On the other hand,
the privacy adversary in the verification protocol is semi-honest in the sense
that the servers execute the protocol faithfully. (In particular, they do not
collude.) Since the latter adversary is weaker than the former, there might be
share verification protocols that are private with respect to semi-honest
adversaries, but not active, colluding ones.

% Thesis.
It is not our contention that the protocols of \cite{riposte} are not end-to-end
secure; in fact, we find that they are. Rather, our thesis is that a unified
analytical framework is needed in order to avoid proposing protocols that do not
compose securely. In addition, we find that this approach yields simpler, more
efficient designs.
\fi

\heading{Our contributions.}
We endow DPF schemes with a property called \emph{verifiability}, which demands
that the composition of the DPF scheme with a write request verification protocol
be secure. We define the simulation-based notion of \cite{dpf,riposte} for
standard DPF schemes, which models an active coalition of malicious write
servers (PRIV1). We give a new notion, which extends this model to include
the execution of the verification protocol (PRIV2). To accomplish this, we give
the adversary access to oracles, which provide it with its view of the
protocol's execution. In particular, we assume the adversary only has access to
messages sent to colluding servers.

\noindent\hl{Here's a list of things we can consider doing:}
\begin{enumerate}
  \item Prove the protocols of \cite{riposte} suffice for PRIV2, but show that
    more efficient protocols are possible in our new framework. In particular,
    we extend the execution model by allowing the write servers to have private
    keys. This immediately yields a more efficient variant of their
    2-server+auditor protocol by using a PRF with a shared key instead of
    telephone coin-flipping to establish a shared set of pairwise independent
    hash functions.

  \item 1-round, 2-serer+auditor protocol in a weaker trust model. (An auditor
    may collude with one write server.) What I'm thinking is that clients append
    $0^t$ to their message before applying $\gen$. The write server computes the
    ``tag'' from the write share by applying a PRF to the string resulting from
    successively concatenating the last $t$ bits of $Y_\idx = \eval(X_i, \idx)$
    for each $\idx$ from 1 to $\dblen$. The auditor just makes sure that the tags
    match. I'm not sure what properties are required of the PRG. Using the
    scheme of \cite{dpf}, this would yield a scheme that is much more efficient
    than Riposte in terms of communication complexity and bandwidth, yet
    functions in a weaker trust model. (Note that this answers two of their open
    questions in the affirmative.)

  \item 1-round, $2^t$-server+auditor protocol where the auditor is colluding,
    using the methods of \cite{dpf-multi-server}? This yields $(2^t,
    t)$-privacy.

  \item Reduce the round complexity of their $\sharect$-server protocol? Theirs
    is $(\sharect, \sharect-1)$-private. (This addresses one of their open
    questions.)

  \item The protocol in (1) actually achieves security against a stronger
    adversary, one that intercepts all messages sent between servers in the
    protocol. Consider the application where Google wants to collect crash
    reports. It operates each of the servers, but is also likely on path between
    each of the servers. (It probably operates the network infrastructure!)
    The protocols of \cite{riposte} DO NOT achieve this stronger notion,
    however.
\end{enumerate}


\section{Related work}
% intro.tex
%
% Introduction.
\label{sec-related}

\hl{Outline:}
\begin{itemize}

  % PIR
  \item Remark that single-server PIR systems are possible
    \cite{pir-single-server}, but at a higher computational cost. (OWFs are
    essential for security: http://dl.acm.org/citation.cfm?id=301277.) According
    to \cite{dpf}, PIS cannot be achieved with a single server \cite{dpf}.

  \item There's an interesting paper (http://dl.acm.org/citation.cfm?id=276723)
    that introduces symmetric PIR, designed to ensure that the reader learns
    nothing more than the bit they query. Do DPF schemes suffice in this
    setting?

  \item Although \cite{dpf} focus on 2-server protocols, methods described in
    \cite{dpf-multi-server} can be used to obtain $(2^t, t)$-private PIR from
    any $(2,1)$-private PIR scheme.

  % PIS
  \item A side-effect of our work is that PIS is more useful, since it can now
    support multiple writers while detecting malicious clients.

  \item \cite{pis} does consider a stronger ``active adversary'', but not one
    who tries to corrupt the database. This adversary may collude with users in
    order to try to learn the data being read/written to the database by the
    victim.

  \item Oblivious RAM is a related problem.

  % Riposte
  \item Under the ``Disruption resistance'' heading in the intro of
    \cite{riposte}, they list a bunch of AC systems in which malicious clients
    must be dealt with.

  \item Describe anonymous communication and the traffic analysis problem.
    Compare Riposte to mix- and DC-nets.
\end{itemize}

\iffalse % This about anonymous communication generally.
\heading{Information leakage.} Information is said to be identifiable if it can
be used to link a client to their message. There are two things to remember
about personally identifiable information: one, any identity information
conveyed by the client's message is necessarily leaked in the output of the
system; and two, any information is potentially identifiable given auxiliary
information about the clients. For example, the number of messages sent by the
client might identify which messages they sent. Anonymous communication systems
can only address the problem of linking sensitive data to the identity of the
client.
\fi


\section{Preliminaries}
% prelims.tex
%
% Notation, games.
\ignore{ %FIXME This makes syntax highlighting work.
  \begin{figure}
    Dumb figure
  \end{figure}
}
\label{sec-prelims}

\heading{Notation.}
Let $x \getsr \setS$ denote sampling an element~$x$ uniformly from a set~$S$.
%
If~$\advA$ is an algorithm, let $y \gets \advA(x_1, \ldots)$ denote
running~$\advA$ on input $x_1, \ldots$ and halting with output $y$.
%
Let $y \gets \advA(x_1, \ldots; \coins)$ denote running~$A$ with coins~$\coins$.
Let $y \getsr \advA(x_1, \ldots)$ denote sampling coins~$\coins$ uniformly and
letting $y = \advA(x_1, \ldots; \coins)$.
%
Let $[\advA(x_1, ...)]$ denote the set of possible outputs of $\advA$ run on input
$(x_1, \ldots \,; \coins)$ and with randomly sampled~$\coins$.
%
Any algorithm may output the distinguished symbol~$\bot$. In other words, the
symbol~$\bot$ is implicitly in the range of every algorithm.

Let $[i..j] = \{ \ell \in \Z : i \le \ell \le j \}$ and $[n] = [1..n]$.
%
If~$\vv$ is a vector, let~$\vv[i]$ and $\vv_i$ denote its~$i$-th element.

Let~$\emptystr$ denote the empty string.
%
Let~$\str(x_1, \ldots)$ denote an invertible encoding of the arbitrary
quantities~$x_1, \ldots$ as a string.
%
Let~$X \cat Y$ denote the concatenation of strings~$X$ and~$Y$.
%
If~$X$ is a string, let~$X[i]$ and $X_i$ denote the~$i$-th bit of string~$X$ and
let $\substr(X,i,j)$ denote the substring $X_i \cat \cdots \cat X_j$ for every $1
\leq i \leq j \leq |X|$.
%
If~$X$ and~$Y$ are equal-length strings, let~$X \xor Y$ denote their
bitwise-XOR.

Let~$\setS$ be a set of totally-ordered values. We say a sequence~$\vv$ is
indexed by~$\setS$ if $\vv = (\vv[x_1], \ldots, \vv[x_n])$ where $(x_1, \ldots,
x_n)$ is the total ordering of the elements of~$\setS$. This is denoted
$(\vv_x)_{x\in \setS}$.

\heading{Games.}
We adopt the game-playing framework of~\cite{games} with one exception: unless
otherwise stated, if a variable is undefined, then it is implicitly equal to the
formal symbol~$\bot$.


\section{DPF schemes}
% dpf.tex
%
% Syntax, definitions, and applications of verifiable DPF schemes.

In this section, we give syntax and security notions for distributed point
functions. We diverge somewhat from the syntax of \cite{dpf} as it will be
convenient in our setting to assume the domain have algebraic
structure.\cpnote{Is this actually more convenient? Since our constructions are
mostly based on pseudorandom generators, it might make more sense to think of
the domain as strings. Yet there's the $\sharect$-server Riposte variant to
consider.}
Let~$\ring$ be a finite ring and let $\msgsp = \ring \setminus \{ 0 \}$
where~$0$ denotes the additive identity. Let $\dblen \in \N$, $\idx \in
[\dblen]$, and $\msg \in \msgsp$.
%
The \emph{point function} of $(\idx, \msg)$ is the map $P_{\idx,\msg} : [\dblen]
\to \ring$ where $P_{\idx,\msg}(\idx) = \msg$ and for every $\idx^\prime \ne
\idx$, it holds that $P_{\idx,\msg}(\idx^\prime) = 0$.
%
The notation~$\msgsp$ signifies that the set of non-zero elements of~$\ring$
constitutes a message space and that~$\msg$ encodes the message to be written
into the database. The value~$\idx$ indicates the row of the database to
which~$\msg$ is to be written.

We remark that the message space in the formalization of~\cite{riposte} is~$F
\setminus \{0\}$ where~$F$ is a finite field. Each element having a
multiplicative inverse is necessary in order to implement their error correcting
code for dealing with collisions in the table, but this extra structure is not
essential for formalizing the security properties of the scheme.\cpnote{Does the
seed-homomorphic PRG need field structure? I don't think so.}

Concretely, we may define a ring on $n$-bit strings as follows.  Addition of~$X$
and~$Y$ is defined by $X \xor Y$. The all-zero string~$0^n$ is then the additive
identity. To compute $Z = X\cdot Y$, we regard the $i$-th bit of~$X$ (resp.~$Y$)
as the $i$-th coefficient of a polynomial~$p_X(x)$ (resp.~$p_Y(x)$), compute the
polynomial~$p_Z(x)$ congruent to $p_X(x)\cdot p_Y(x) \mod x^{n+1}$, and
let~$Z[i]$ be the $i$-th coefficient of~$p_Z(x)$ for each $i \in
[n]$.

\subsection{Syntax}
% syntax.tex
%
% Syntax of DPF schemes and verifiable DPF schemes, definitions of correctness,
% completeness, and soundness.
\label{sec-syntax}

A \emph{distributed point function} (DPF) scheme is a pair of probabilistic
algorithms $\dpf = (\gen, \eval)$ with associated parameters $(\ring, \sharect,
\dblen)$ where~$\ring$ is a finite ring and $\sharect, \dblen \in \N$.
%
We call $\msgsp = \ring \setminus \{0\}$ the \emph{message space},
$\sharect$ the \emph{share count},and $\dblen$ the \emph{database length}.
%
On input of integer $\idx \in [\dblen]$, called the \emph{database index}, and
message $\msg \in \msgsp$, the share generation algorithm~$\gen$ outputs a
sequence of strings $(X_1, \ldots, X_\sharect)$ called \emph{write shares}.
%
On input of
write share~$X$ and index~$\idx$, algorithm~$\eval$ deterministically
outputs an element of~$\ring$.
%
A DPF scheme is \emph{correct} if for every $\idx, \idx^\prime \in [\dblen]$ and
$\msg \in \msgsp$, it holds that
\[
  \Prob{ (X_1, \ldots, X_s) \getsr \gen(\idx,\msg):
       \sum_{i=1}^S \eval(X_i, \idx^\prime) = P_{\idx,\msg}(\idx^\prime)} = 1.
\]
where $P_{\idx,\msg}$ denotes the point function of $(\idx, \msg)$.\cpnote{One
might consider endowing the write share generator with a key. This would admit
deterministic constructions, which would be more efficient. They write share
generator of Riposte uses \emph{a lot} of random bits.}

To initiate a write request, the client runs $(X_1, \ldots X_\sharect) \getsr
\gen(\idx,\msg)$ and transmits each share to one of the write servers. Upon
receiving their shares, the write servers engage in a write share verification
protocol in order to ensure the request is well-formed.  We say that $\dpf$ is
\emph{verifiable} if there exists a protocol $\reqvfy$ executed by principals
$P$ satisfying the following properties.
%
First, it holds that $\{\client, \writer_1, \ldots, \writer_{\sharect}\}
\subseteq P$ where $\client$ denotes the client making the write request and
$\writer_i$ denotes the $i$-th write server.
%
Second, let $\lang_\dpf$ be the language comprised of strings $\str{\shares}$
where $(\shares) \in [\gen(\idx, \msg)]$ for some $\msg \in \msgsp$ and $\idx
\in [L]$. There exists an adversary $\advB$ called the \textit{benign adversary}
such that the following conditions hold:
\begin{itemize}
  \item \textit{Completeness.}
    For every $\str{\shares} \in \lang_\dpf$, it holds that
    $
      \Pr[\game{proto}_{\reqvfy,I,P}(\advB) \outputs \accept] = 1
    $
  where $I(\writer_i) = (X_i, \client)$ for each $i\in[\sharect]$.
  \item \textit{Disruption resistance} \cite[def. 3]{riposte}.
    For every probabilistic, polynomial-time (in the implicit security
    parameter) adversary $\advA$, it holds that
    \[
      \Prob{ (\shares) \getsr \advA:
             \str{X_1, \ldots, X_\sharect} \not\in \lang_\dpf \AND
             \game{proto}_{\reqvfy,I,P}(\advB) \outputs \accept } \le \epsilon
    \]
    where $\epsilon$ is negligible (in the implicit security parameter) and
    $I(\writer_i) = (X_i, \client)$ for each $i\in[\sharect]$. (We leave the
    security parameter implicit because we will give concrete security results.)
    \cpnote{This is actually a security property.}
\end{itemize}

\heading{Discussion.}
The disruption resistance property resembles the collision resistance property
of cryptographiic hash functions~\cite{collision-resistance}. We will show that
our schemes achieve this under standard assumptions.
%
We could pull out a soundness property for~$\reqvfy$ similar to
that of interactive proof systems \cite[def. 4.2.10]{oded}: for every
$\str{\shares} \not\in \lang_\dpf$ and every PPT adversary~$\advA$, it
holds that $\Pr[\game{proto}_{\reqvfy,I,P}(\advA) = 1] \le \epsilon$.
%
However, this is stronger than needed in our setting, since we only expect the
system to detect mal-formed requests when all the servers are honest.  Recall
that a malicious server can always corrupt its own state, thereby disrupting the
system; nevertheless, we require that doing so does not violate privacy of
honest clients. This is captured by our privacy notions below.

\noindent\hl{Disruption resistance is a pretty weak notion}, but this is indeed
the property that the authors intend. (Actually, the adversary can try to $n$
times to forge a bad request, but a simple hybrid argument shows that our notion
implies theirs.) The following is a stronger, if not uglier target:
\begin{figure}[h]
  \oneCol{0.50}{
    \underline{$\game{\disres}_\reqvfy(\advA, \advB)$}\\[2pt]
      $(K_u)_{u\in P} \getsr \protocol.\init$;
      $O \gets P$\\
      $(X_1, \ldots, X_\sharect) \getsr \advA^{\ENQO,\DEQO}$\\
      \foreach{j}{1}{S} $Q_{\writer_j}^\sess \gets Q_{\writer_j}^\sess.\enqueue(X_j, \client)$\\
      $\advB^{\ENQO,\DEQO}$\\
      if $\sess$ accepts and $\str{X_1, \ldots, X_\sharect} \not\in \lang_\dpf$
      then return $\true$\\
      else return $\false$
    \vspace{4pt}
  }
  \caption{$\advB$ is the benign adversary.}
\end{figure}


\subsection{Security}
% security.tex
%
% Definition of PRIV1 and PRIV2.
\label{sec-security}

In this section, we model a coalition of write servers attempting to learn
something about the message or index chosen by the client. We give two notions
in figure~\ref{fig-priv}.
The first applies to DPF schemes and the second to verifiable DPF schemes.
\begin{figure}[t]
  \newcommand{\ctr}{\flagfont{ctr}}
  \twoCols{0.45}
  {
    \underline{$\game{\privone}_{\dpf,t}(\advA,\advS)$}\\[2pt]
    $b \getsr \bits$\\
    $b^\prime \getsr \advA^{\GENO,\CORO}$\\
    return $b=b^\prime$
    \\[6pt]
    \underline{$\GENO(\idx,\msg)$}\\[2pt]
    if $C = \bot$ then return $\bot$\\
    if $\msg \not\in \msgsp \OR \idx \not\in [\dblen]$ then return $\bot$\\
    if $b=1$ then $(X_1, \ldots, X_\sharect) \getsr \gen(\idx,\msg)$\\
    else $(X_1, \ldots, X_\sharect) \getsr \advS(C)$\\
    return $(X_i)_{i\in C}$
    \\[6pt]
    \underline{$\CORO(C^\prime)$}\\[2pt]
    if $C \ne \bot \OR C^\prime \not\subseteq [\sharect] \OR |C^\prime| > t$
      then\\\ind return $\bot$\\
    $C \gets C^\prime$\\
  }
  {
    \underline{$\game{\privtwo}_{\dpf,\reqvfy,t}(\advA, \advS)$}\\[2pt]
    $(K_u)_{u\in P} \getsr \kg$\\
    $b \getsr \bits$\\
    $b^\prime \getsr \advA^{\GENO,\ENQO,\DEQO,\CORO}$\\
    return $b=b^\prime$
    \\[6pt]
    \underline{$\GENO^i(\idx,\msg)$}\\[2pt]
    if $C = \bot$ then return $\bot$\\
    if $\msg \not\in \msgsp \OR \idx \not\in [\dblen]$ then return $\bot$\\
    if $b=1$ then $(X_1, \ldots, X_\sharect) \getsr \gen(\idx,\msg)$\\
    else $(X_1, \ldots, X_\sharect) \getsr \advS(C)$\\
    \foreach{j}{1}{S}\\
    \ind $Q_{\writer_j}^i \gets Q_{\writer_j}^i.\enqueue(X_j, \client)$\\
    return $(X_u)_{u\in C}$
    \\[6pt]
    \underline{$\ENQO_{y,x}^i(X)$}\\[2pt]
      $Q_y^i.\enqueue(X, x)$
    \\[6pt]
    \underline{$\DEQO_{y}^i()$}\\[2pt]
      $(X, x) \gets Q_y^i.\dequeue()$\\
      $(Y, z, \verdict, \st_y^i) \getsr \vfy_{y,x}^i(K_y, X, \st_y^i)$\\
      if $C \ne \bot \AND v \in C$ then return $(Y, z, \verdict)$\\
      else if $z \ne \bot$ then $Q_z^i.\enqueue(Y, y)$\\
      return $(\bot, z, \verdict)$
    \\[6pt]
    \underline{$\CORO(C^\prime)$}\\[2pt]
    if $C \ne \bot \OR \client \in C^\prime$ then return $\bot$\\
    if $|C^\prime \intersection \{\writer_j\}_{j\in[\sharect]}| > t$ then return $\bot$\\
    $C \gets C^\prime$\\
    return $(K_u)_{u\in C}$
  }
  \caption{Security notions for \textbf{(left)} DPF scheme $\dpf = (\gen,
  \eval)$ with parameters $(\ring, \sharect, \dblen)$ where $\msgsp = \ring
  \setminus \{0\}$, and
  \textbf{(right)} verifiable DPF scheme $\dpf$ with write share verification
  protocol $\reqvfy = (\kg, \vfy)$ executed with principals $P \supseteq \{\client, \writer_1,
  \ldots, \writer_\sharect\}$.}
  \vspace{6pt}\hrule
  \label{fig-priv}
\end{figure}

\heading{\privone.}
We describe the simulation-based notion of \cite{riposte,dpf}, which models the
ability of a coalition of malicious write servers to distinguish their subset of
the key shares from the output of a simulator. Let $\dpf = (\gen, \eval)$ be a
standard DPF scheme with parameters $(\ring, \sharect, \dblen)$. Let $\msgsp = R
\setminus \{0\}$. Refer to the experiment in the left-hand side of
Figure~\ref{fig-priv} associated to~$\dpf$, adversary~$\advA$,
simulator~$\advS$, and coalition threshold $t \in \N$ where $1 < t < \sharect$.
We define the advantage of~$\advA$ in attacking~$\dpf$ in the game instantiated
with simulator $\advS$ as
\[
  \adv{\privone}_{\dpf,t}(\advA,\advS) = 2 \cdot
  \Prob{ \game{\privone}_{\dpf,t}(\advA,\advS) \outputs \true } - 1.
\]
One might informally we say that~$\dpf$ is $(\sharect,t)$-\privone~secure if for
every ``reasonable'' adversary~$\advA$, there exists an ``efficient''
simulator~$\advS$ such that $\adv{\privone}_{\dpf,t}(\advA,\advS)$ is ``small''.
We will forego a rigorous definition and instead give concrete security results.

Achieving security in this sense does not ensure privacy if the write servers
engage in a write share verification protocol.  We put forward a unified
approach based on the communication model described in section~\ref{sec-com}.

\heading{\privtwo.}
Consider the experiment on the right hand side of Figure~\ref{fig-priv}
associated to DPF scheme $\dpf$, write share verification protocol $\reqvfy =
(\kg, \vfy)$ with principals $P \supseteq \{\client, \writer_1, \ldots
\writer_s\}$, coalition threshold~$t$, adversary~$\advA$, and simulator~$\advS$.
%
Just as before, the goal of the adversary is to distinguish its subset of the
write shares from the output of the simulator.
%
At the beginning of the game,
the initialization algorithm~$\kg$ is executed and a random bit~$b$ is chosen.
%
When $\advA$ asks $(i, \idx, \msg)$ of~$\GENO$, if $b=1$ then the write share
generation algorithm is run; otherwise the simulator is run on. Let $(\shares)$
denote the output. Next, the output $X_j$ is added to the top of the queue of
write server $\writer_j$ in session $i$ for each $j\in[\sharect]$. Finally, the
adversary is given the shares corresponding to servers it corrupts.
%
Corruptions are made by querying~$\CORO$ with a subset of~$P$. The coalition is
non-adaptive, meaning the adversary chooses which servers it controls before it may
query~$\GENO$. It may corrupt any set of principals that does not include the
client~$\client$ or more than~$t$ write servers. It is given the long-term input
of each principal it corrupts.
%
The adversary is given a~$\ENQO$ oracle and a~$\DEQO$ oracle, which have the
same semantics as in the \protosec~game defined in figure~\ref{fig-proto}. It is
on path to any principal it corrupts.
%
After interacting with its oracles, the adversary outputs a bit~$b^\prime$. The
output of the game is the predicate $b=b^\prime$. We define the advantage of
$\advA$ against $\dpf$ and $\reqvfy$ in the game instantiated with simulator
$\advS$ as
\[
  \adv{\privtwo}_{\dpf,\reqvfy,t}(\advA,\advS) =
  2 \cdot \Prob{\game{\privtwo}_{\dpf,\reqvfy,t}(\advA,\advS) \outputs \true} - 1.
\]

\heading{Non-colluding auditor.}
The 2-server DPF scheme of \cite{riposte} is accompanied by a request
verification protocol with principals $\{\client, \auditor, \writer_1,
\writer_2\}$, where $\auditor$ is called the \emph{auditor}. The auditor is not
trusted more than $\writer_1$ or $\writer_2$, but the non-collusion assumption
is stronger than usual: namely, that no two servers among $\{\auditor, \writer_1,
\writer_2\}$ collude. To capture this special case, we let $t=1$. Note,
however, that our model cannot be used to capture the general case where any set
of $t$ of $\sharect$ write serves may collude, but no write server colludes with
$\auditor$. For $\sharect>2$ and $t>1$, we assume the auditor may collude.


\subsection{Application to anonymous reporting}
% anon.tex
%
% Application of verifiable DPF schemes to anonymous communication.
\label{sec-anonymity}

We describe the application of verifiable DPF schemes to anonymous communication
as specified by the Riposte system~\cite{riposte}. We parameterize the protocol
by a positive integer~$\cohortct$, called the \emph{cohort size}, which
determines the number of valid write requests the servers process before
outputting their state. The time interval in which each of the requests are
processed is referred to as an \emph{epoch}. Let~$\dpf$ be a DPF scheme with
parameters~$(\ring, \sharect, \dblen)$ and~$\reqvfy$ be a write request
verification protocol for~$\dpf$ with benign adversary~$\advB$ and principals $P
\supseteq \{\client, \writer_1, \ldots \writer_{\sharect}\}$ and let $\msgsp =
\ring \setminus \{0\}$.

\begin{itemize}
  \item To initiate a write request for $\msg \in \msgsp$, the client samples
    $\idx \getsr [\dblen]$, executes the share generation algorithm $(X_1,
    \ldots, X_{\sharect}) \getsr \gen(\idx, \msg)$, and sends~$X_i$
    to~$\writer_i$ for each $i \in [\sharect]$.

  \item Next, the request verification protocol is executed. This means that the
    \protosec~game is run with~$\reqvfy$, $I$, and~$\advB$ where for every $i
    \in [\sharect]$, we have that $I(\writer_i) = (X_i, \client)$.

  \item When server~$\writer_i$ accepts with private input~$X_i$, it updates its
    local state as follows. Let~$\dbst_i$ be a $\dblen$-vector over
    $\ring$ where each $\dbst_i[\idx]$ is initially equal to~$0$. Let
    $\dbst_i[\idx^\prime] = \dbst_i[\idx^\prime] + \eval(X_i, \idx^\prime)$
    for every~$\idx^\prime$.

  \item Finally, once~$\cohortct$ valid write requests have been processed, each
    write server its final state~$\dbst_i$ to the data consumer. The data
    consumer recovers the database state (and thus the set of messages) by
    computing $\dbst[\ell^\prime] = \sum_{i=1}^{\sharect}
    \dbst_i[\ell^\prime]$ for every~$\ell^\prime$.
\end{itemize}

\heading{Dealing with collisions.} Because the clients choose their index into
the database table randomly, there is a reasonable chance that two or more
inadvertently choose the same index. However, it can be shown that if the
database length is at least 20 times the cohort size, then average collision
rate will not exceed $5\%$ \cite{riposte}. Moreover, if~$\ring$ is a field, then
techniques based on error correcting codes can be used to further reduce the
collision rate \cite{riposte}.

\noindent\hl{Things we might do here:}
\begin{itemize}
  \item Formalize the security notion intended by this system (ANON). This was
    already done in \cite{riposte}.
  \item Show that PRIV1 implies ANON.  This was not done in \cite{riposte} as
    far as I can tell.
\end{itemize}



\section{A $2$-server+auditor protocol}
% twowriter.tex

\label{sec-twowriter}
\hl{A 2-server+auditor protocol.} The goal here is the strongest-possible notion
of verifiability using a generic 2-share DPF scheme.

\heading{Notation.}
Let $X, Y \in \bits^*$.
%
Let~$X[i..j]$ denote substring of~$X$ from the $i$-th bit to the $j$-th bit
inclusively.
%
If $|X| = |Y|$, then let $X \xor Y$ denote the bitwise XOR of~$X$ and~$Y$.  If
$|X| \ne |Y|$, then $X \xor Y$ means to truncate the longer string to the length
of the shorter string and compute the bitwise XOR of the resulting pair of
strings.
%
Suppose there are positive integers~$m$ and~$n$ such that $|X| = mn$. We write
$\block{n}{X}{i} = X[ni+1..n(i+1)]$ to denote the $i$-th, $n$-bit block of~$X$.
Let $\block{n}{X}{i..j} = \block{n}{X}{i} \cat \cdots \cat \block{n}{X}{j}$.

The 2-share DPF scheme of~\cite{dpf} is specified in Figure~\ref{fig-two-dpf}.
The function~$\prg$ is instantiated with a pseudorandom generator who's
signature is determined by the lengths of the inputs to the write share
generation algorithm.
%
Let~$\alpha$ and~$\beta$ denote the maximum lengths of~$X$ and~$Y$ respectively.
%
By~\cite[Proposition 1]{dpf}, if
the seed length~$\kappa$ of~$\prg$ is chosen such that $\beta \le \kappa + 1$
and the number of iterations $r$ is $\lceil \log \alpha \rceil$ as specified in
Figure~\ref{fig-two-dpf}, then the length of the shares is at most
$8(\kappa+1)\alpha^{\log 3}\beta^{-\alpha}$ bits. This also upper bounds
the output length of~$\prg$. \cpnote{We could get a tighter bound on the output
length using an inductive argument similar to Proposition 1 of~\cite{dpf}.}


\begin{figure}
  \twoColsNoDivide{0.45}
  {
    \underline{$\gen(X,Y)$}\\[2pt]
      $r \gets \lceil \log |X| \rceil$\\
      $(K_0, K_1) \getsr \gen_r(X,Y)$\\
      $L \gets \varphi^{-1}(|Y|)$\\
      return $\str(\str(K_0, L), \str(K_1, L))$
    \\[6pt]
    \underline{$\gen_{r}(X, Y)$}\\[2pt]
      if $r=0$ then\\
      \tab $K_0 \getsr \bits^{|Y|\cdot2^{|X|}}$; $K_1 \gets K_0$\\
      \tab $x \gets \varphi(X)$; $\block{|Y|}{K_1}{x} \gets \block{|Y|}{K_1}{x} \xor Y$\\
      \tab return $(K_0, K_1)$\\
      $m \gets \psi(|X|, |Y|)$; $n \gets |X| - m$\\
      $I \gets X[1..m]$; $J \gets X[m+1..m+n]$\\
      $X^* \getsr \bits^\kappa$;
      $(S_0, S_1) \getsr \gen_{r-1}(I, X^*\cat 1)$\\
      $(P_0, P_1) \getsr \gen_{r-1}(J, Y)$\\
      for each $b\in \bits$ do\\
      \tab $W \gets \eval_{r-1}^{\kappa+1}(S_b, I)$\\
      \tab $Z \gets W[1..\kappa]$; $t \gets W[\kappa+1]$\\
      \tab $R_t \gets \prg(Z) \xor P_b$\\
      \tab $K_b \gets S_b \cat R_0 \cat R_1$\\
      return $(K_0, K_1)$
  }
  {
    \underline{$\eval(\str(K,L), X)$}\\[2pt]
      $r \gets \lceil \log |X| \rceil$\\
      $\ell \gets \varphi(L)$\\
      $Y \gets \eval^\ell_r(K, X)$\\
      return $Y$
    \\[6pt]
    \underline{$\eval_{r}^\ell(K, X)$}\\[2pt]
      if $r=0$ then $x \gets \varphi(X)$; return $\block{\ell}{K}{x}$\\
      $m \gets \psi(|X|, \ell)$; $n \gets |X| - m$\\
      $I \gets X[1..m]$; $J \gets X[m+1..m+n]$\\
      \comment{Parse $K$ into sub-strings.}\\
      $S \gets K[1..|K|-2n\ell]$;
      $R \gets K[|K|-2n\ell+1..|K|]$\\
      $R_0 \gets \block{\ell}{R}{1..n}$;
      $R_1 \gets \block{\ell}{R}{n+1..2n}$\\
      \comment{Expand row.}\\
      $W \gets \eval_{r-1}^{\kappa+1}(S, I)$\\
      $Z \gets W[1..\kappa]$; $t \gets W[\kappa+1]$\\
      $P \gets \prg(Z) \xor R_t$\\
      \comment{Evaluate cell.}\\
      $Y \gets \eval_{r-1}^\ell(P, J)$\\
      return $Y$
  }
  \caption{The 2-share DPF scheme of \cite{dpf}.
  Let $\prg : \bits^\kappa \to \bits^\infty$ be a function,
  let $\varphi : \bits^* \to \Z^+$ be the bijection between a string and the
  positive integer it represents (according to some fixed encoding), and let
  $\psi(\alpha,\beta) = \lceil 1/2 \cdot
  \log({\beta\cdot2^{\alpha}}/{(\kappa+1)}) \rceil$.
  }
  \label{fig-two-dpf}
\end{figure}


\if{0}
\section{Riposte}
% riposte.tex
%
% Specification of Riposte protocols.
\label{sec-riposte}

\subsection{The $2$-server+auditor protocol}
\begin{figure}
\twoColsNoDivide{0.45}
{
  \underline{algorithm $\gen(\idx, \msg)$}\\[2pt]
  if $\msg \not\in \msgsp \OR \idx \not\in [\dblen]$ then return $\bot$\\
  $(i, j) \gets (\lceil \idx/y \rceil, \idx - \lfloor \idx/y \rfloor)$\\
  $r_A \getsr \bits^x$; $r_B \gets r_A \xor e_i$\\
  $\vecs_A \getsr (\bits^k)^x$ \\
  $\vecs_B \gets \vecs_A$; $\vecs_B[i] \getsr \bits^x$ \\
  $\vecx_A \gets \prg(\vecs_A[i])$; $\vecx_B \gets \prg(\vecs_B[i])$ \\
  $\vecv \gets \msg\cdot \vece_j + \vecx_A + \vecx_B$ \\
  $X_A \gets \str{r_A, \vecs_A, \vecv}$; $X_B \gets \str{r_B, \vecs_B, \vecv}$\\
  return $(X_A, X_B)$
}
{
  \underline{algorithm $\eval(X, \idx)$}\\[2pt]
  if $\idx \not\in [\dblen]$ then return $\bot$\\
  $(i, j) \gets (\lceil \idx/y \rceil, \idx - \lfloor \idx/y \rfloor)$\\
  $\str{r, \vecs, \vecv} \gets X$\\
  $\vecx \gets \prg(\vecs[i])$\\
  if $r[i] = 1$ then return $\vecv[j] + \vecx[j]$\\
  else return $\vecx[j]$
}
\caption{The 2-share DPF scheme of \cite{riposte} constructed from a PRG
  $\prg$.}
\label{fig-riposte-2share}
\end{figure}
We specify the construction of \cite{riposte} of a 2-share DPF scheme from a
PRG. Let $\dpf = (\gen, \eval)$ be the DPF scheme defined in
figure~\ref{fig-riposte-2share} with parameters $(\ring, \sharect, \dblen, x,
y)$ where $x$ and $y$ are positive powers of 2 and $xy \ge \dblen$.
We fix some encoding of each abstract point in $\ring$ as an $n$-bit string for
$n \in \N$.
Let $\vece_i$ denote the
$y$-vector over $\ring$ with $0$ in each position except for the $i$-th, which is 1.
Let $e_i$ denote the $x$-bit string with 0's everyone except for the $i$-th bit.
Let $\prg : \bits^x \to \bits^{yn}$ be a function.
When we write $\vecx \gets w$ where $w \in \bits^{yn}$, we mean divide $w$ into
a sequence of $n$-bit strings and map each string to its corresponding point in
$\ring$.

\heading{Optimal choice for $x$ and $y$.}
According to \cite{riposte}, the length of the keys is $x(s+1) + yn$. The
optimal values for $x$ and $y$ are $x = c\sqrt{\dblen}$ and $y =
c^{-1}\sqrt{\dblen}$ where $c = \sqrt{n/(1+s)}$. Hence, the share size is
$O(\sqrt{\dblen})$.

\heading{Verifying the write shares.}
\begin{figure}
  \oneCol{0.90}{

  \underline{proto $\verifykey(X_A, X_B)$}\\
  \vspace{-2pt}
  \begin{enumerate}[leftmargin=*]
    \item Server $A$: $\str{r_A, \vecs_A, \vecv} \gets X_A$; for each $i \in [x]$,
      do $\vect_A[i] \gets r_A[i] \cat \vecs_A[i]$. Server $B$ does the same
      with its input. Execute $\almosteq(\vect_A, \vect_B)$. If the result is
      $\reject$, then $A$ and $B$ output $\reject$.

    \item Server $A$: $\vecu_A \gets \sum_{i=1}^x \prg(\vecs_A[i])$.
      Server $B$: $\vecu_B \gets \vecv + \sum_{i=1}^x \prg(\vecs_B[i])$.
      Execute $\almosteq(\vecu_A, \vecu_B)$. Both $A$ and $B$ output the result.
  \end{enumerate}

  \vspace{2pt}
  \underline{proto $\almosteq(\vecv_A, \vecv_B)$}\\
  \vspace{-2pt}
  \begin{enumerate}[leftmargin=*]
    \item Let $m = |\vecv_A| = |\vecv_B|$. Servers $A$ and $B$ engage in a
      coin-flipping protocol \cite{telephone} in order to establish a shared
      $(mk + \lg m)$-bit string $R$. Let $(K_1, \ldots, K_m)$ denote the first
      $m$ $k$-bit chunks of $R$ and let $F$ denote the last $(\lg m)$-bit chunk.

    \item Server $A$: for each $i \in [m]$, let $m_i = \hash(K_i, \vecv_A[i])$.
      Let $f \in [m]$ denote the positive integer encoded by $F$. Send
      $(m_f, m_{f+1}, \ldots,$ $m_1, \ldots m_{f-1})$ to $C$. Server $B$ does
      the same.

    \item Auditor $C$ checks that the sequence of messages received from $A$ and $B$
      differ by exactly one element. Output $\accept$ if this holds and
      $\reject$ otherwise.
  \end{enumerate}
  \vspace{1pt}
  }
  \caption{A 3-party protocol for verifying the write shares are well-formed. The
  function $\hash$ is an $\epsilon$-almost-universal hash function with key
  space $\bits^k$.}
  \label{fig-riposte-2server+auditor}
\end{figure}
The request verification protocol for $\dpf$ suggested by \cite{riposte} is
specified in figure~\ref{fig-riposte-2server+auditor}. The principals are $A$,
$B$, and $C$ where the $A$ is the first write server, $B$ the second, and $C$ is
the auditor.
Our presentation differs from \cite{riposte} in two respects.
One, a step of the protocol requires that the write servers perform a
coin-flipping protocol in order sample from a family of pairwise-independent
universal hash functions.  However, in their implementation, they instantiate
this process by sharing keys for the Poly1305 almost-universal hash function
designed by Dan Bernstein \cite{poly1305}. We make this explicit in the
presentation.  Let $k,t \in \N$, $\epsilon \in (0,1]$, and $\hash : \bits^k \by
\bits^* \to \bits^t$ be an $\epsilon$-almost universal hash function.
Two, we require that $x$ and $y$ be powers of 2. This ensures that we always
perform the coin flipping protocol \cite{telephone} a finite number of times.

\heading{A complexity improvement.}
Implementing the coin-flipping protocol would be prohibitively expensive. In
practice, one would replace this protocol with a PRG, whose seed is shared by
$A$ and $B$,\cpnote{This was suggested to me by Henry.} but this makes the
analysis much more complex. Instead, we modify the protocol to use a PRF whose
key is shared by $A$ and $B$.

\noindent\hl{Things to do here:}
\begin{itemize}
  \item Specify the modified protocol $\reqvfy$. Show that it is complete.

  \item Show that $\reqvfy$ is disruption resistant if the underlying PRF
    satisfies its security notion.

  \item Show that the composition of $\dpf$ and $\reqvfy$ is PRIV2 secure for
    $t=1$ if the underlying primitives (a PRG and a PRF) satisfy their
    respective security notions.
\end{itemize}

\subsection{The $s$-server protocol}
\noindent\hl{Things to do here:}
\begin{itemize}
  \item Specify their $s$-share DPF scheme and $s$-server request verification
    protocol. This involves something they call a seed-homomorphic PRG.
\end{itemize}

\fi

\section{Notes}
\begin{itemize}
  \item Ilya Mironov (mironov@) suggests that the error correction coding could
    be done more efficient. In the paper, they consider coding schemes for
    general finite fields, but there are faster ways when you're working with
    fields with binary coefficients (i.e. $GF(2^n)$). Ananth (pseudorandom@)
    says to focus on arguing for distributed trust model from a privacy
    perspective.  Ulfar Erlingsson (ulfar@) argues that distributing trust is a
    means of keeping the data collector honest; in order to violate your
    privacy, they'd have to do it explicitly. It makes an insider attack much
    harder, avoids inadvertently linking collected data to client identity, and
    makes it so an outside attacker has two targets instead of one.


  \item ``Silent'' versus ``explicit'' collusion. I'll need to be rigorous about
    what it means to be a valid protocol message.

  \item Compare to mixnets: an active adversary controlling the entry mix can
    learn the message sent by client it wishes (all-but-one attack). Mounting
    this attack Riposte would require network control.

  \item Ulfar's straw man scheme: use a ``stock'' secret sharing scheme to map
    the message to $s$ shares. Generate a random nonce and append it to each of
    the shares. Send each string to one of the write servers. Once a write
    server has received $N$ shares, it outputs them to the auditor in a random
    order.  Once the auditor has received $N$ shares from each of the $s$ write
    servers, it combines the shares and outputs them in a random order to the
    data consumer. This protocol fails in two ways. First, one of the write
    servers could mount the all-but-one attack. (This can be overcome by
    having the clients encrypt their shares under the public key of the auditor
    or by making the messages ``self-authenticating'' as suggested by Ilya.)
    Second, it's possible for one of the write servers to silently collude with
    the auditor. By not shuffling, a write server communicates the order of
    arrival to the auditor.
\end{itemize}

\section{Acknowledgements}
% ack.tex
%
% Acknowledgements
\label{sec-ac}

Thanks to Brad for teaching me about an attacker being on path and IP spoofing.


\bibliography{main}
\bibliographystyle{plain}

\end{document}
