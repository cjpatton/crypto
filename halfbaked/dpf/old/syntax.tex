% syntax.tex
%
% Syntax of DPF schemes and verifiable DPF schemes, definitions of correctness,
% completeness, and soundness.
\label{sec-syntax}

A \emph{distributed point function} (DPF) scheme is a pair of probabilistic
algorithms $\dpf = (\gen, \eval)$ with associated parameters $(\ring, \sharect,
\dblen)$ where~$\ring$ is a finite ring and $\sharect, \dblen \in \N$.
%
We call $\msgsp = \ring \setminus \{0\}$ the \emph{message space},
$\sharect$ the \emph{share count},and $\dblen$ the \emph{database length}.
%
On input of integer $\idx \in [\dblen]$, called the \emph{database index}, and
message $\msg \in \msgsp$, the share generation algorithm~$\gen$ outputs a
sequence of strings $(X_1, \ldots, X_\sharect)$ called \emph{write shares}.
%
On input of
write share~$X$ and index~$\idx$, algorithm~$\eval$ deterministically
outputs an element of~$\ring$.
%
A DPF scheme is \emph{correct} if for every $\idx, \idx^\prime \in [\dblen]$ and
$\msg \in \msgsp$, it holds that
\[
  \Prob{ (X_1, \ldots, X_s) \getsr \gen(\idx,\msg):
       \sum_{i=1}^S \eval(X_i, \idx^\prime) = P_{\idx,\msg}(\idx^\prime)} = 1.
\]
where $P_{\idx,\msg}$ denotes the point function of $(\idx, \msg)$.\cpnote{One
might consider endowing the write share generator with a key. This would admit
deterministic constructions, which would be more efficient. They write share
generator of Riposte uses \emph{a lot} of random bits.}

To initiate a write request, the client runs $(X_1, \ldots X_\sharect) \getsr
\gen(\idx,\msg)$ and transmits each share to one of the write servers. Upon
receiving their shares, the write servers engage in a write share verification
protocol in order to ensure the request is well-formed.  We say that $\dpf$ is
\emph{verifiable} if there exists a protocol $\reqvfy$ executed by principals
$P$ satisfying the following properties.
%
First, it holds that $\{\client, \writer_1, \ldots, \writer_{\sharect}\}
\subseteq P$ where $\client$ denotes the client making the write request and
$\writer_i$ denotes the $i$-th write server.
%
Second, let $\lang_\dpf$ be the language comprised of strings $\str{\shares}$
where $(\shares) \in [\gen(\idx, \msg)]$ for some $\msg \in \msgsp$ and $\idx
\in [L]$. There exists an adversary $\advB$ called the \textit{benign adversary}
such that the following conditions hold:
\begin{itemize}
  \item \textit{Completeness.}
    For every $\str{\shares} \in \lang_\dpf$, it holds that
    $
      \Pr[\game{proto}_{\reqvfy,I,P}(\advB) \outputs \accept] = 1
    $
  where $I(\writer_i) = (X_i, \client)$ for each $i\in[\sharect]$.
  \item \textit{Disruption resistance} \cite[def. 3]{riposte}.
    For every probabilistic, polynomial-time (in the implicit security
    parameter) adversary $\advA$, it holds that
    \[
      \Prob{ (\shares) \getsr \advA:
             \str{X_1, \ldots, X_\sharect} \not\in \lang_\dpf \AND
             \game{proto}_{\reqvfy,I,P}(\advB) \outputs \accept } \le \epsilon
    \]
    where $\epsilon$ is negligible (in the implicit security parameter) and
    $I(\writer_i) = (X_i, \client)$ for each $i\in[\sharect]$. (We leave the
    security parameter implicit because we will give concrete security results.)
    \cpnote{This is actually a security property.}
\end{itemize}

\heading{Discussion.}
The disruption resistance property resembles the collision resistance property
of cryptographiic hash functions~\cite{collision-resistance}. We will show that
our schemes achieve this under standard assumptions.
%
We could pull out a soundness property for~$\reqvfy$ similar to
that of interactive proof systems \cite[def. 4.2.10]{oded}: for every
$\str{\shares} \not\in \lang_\dpf$ and every PPT adversary~$\advA$, it
holds that $\Pr[\game{proto}_{\reqvfy,I,P}(\advA) = 1] \le \epsilon$.
%
However, this is stronger than needed in our setting, since we only expect the
system to detect mal-formed requests when all the servers are honest.  Recall
that a malicious server can always corrupt its own state, thereby disrupting the
system; nevertheless, we require that doing so does not violate privacy of
honest clients. This is captured by our privacy notions below.

\noindent\hl{Disruption resistance is a pretty weak notion}, but this is indeed
the property that the authors intend. (Actually, the adversary can try to $n$
times to forge a bad request, but a simple hybrid argument shows that our notion
implies theirs.) The following is a stronger, if not uglier target:
\begin{figure}[h]
  \oneCol{0.50}{
    \underline{$\game{\disres}_\reqvfy(\advA, \advB)$}\\[2pt]
      $(K_u)_{u\in P} \getsr \protocol.\init$;
      $O \gets P$\\
      $(X_1, \ldots, X_\sharect) \getsr \advA^{\ENQO,\DEQO}$\\
      \foreach{j}{1}{S} $Q_{\writer_j}^\sess \gets Q_{\writer_j}^\sess.\enqueue(X_j, \client)$\\
      $\advB^{\ENQO,\DEQO}$\\
      if $\sess$ accepts and $\str{X_1, \ldots, X_\sharect} \not\in \lang_\dpf$
      then return $\true$\\
      else return $\false$
    \vspace{4pt}
  }
  \caption{$\advB$ is the benign adversary.}
\end{figure}
