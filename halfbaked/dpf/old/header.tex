% header.tex
%
% Formatting and common macros for crypto papers. Include this first.
\usepackage[utf8]{inputenc}
\usepackage[margin=4cm]{geometry}
\usepackage{graphics}
\usepackage[toc,page]{appendix}
\usepackage[font={small}]{caption}
\usepackage{hyperref}
\usepackage{amsmath}
\usepackage{amsthm}
\usepackage{amsfonts}
\usepackage{parskip}
\usepackage{framed}
\usepackage{multicol}

\hypersetup{
    colorlinks,%
    citecolor=black,%
    filecolor=black,%
    linkcolor=black,%
    urlcolor=black
}

\def\dashuline{\bgroup
  \ifdim\ULdepth=\maxdimen  % Set depth based on font, if not set already
	  \settodepth\ULdepth{(j}\advance\ULdepth.4pt\fi
  \markoverwith{\kern.15em
	\vtop{\kern\ULdepth \hrule width .3em}%
	\kern.15em}\ULon}

\newcounter{foot}
\setcounter{foot}{1}
\setlength\parindent{2em}

% Editorial
\newcommand{\heading}[1]{\noindent \textsc{#1}}
\newtheorem{theorem}{Theorem}[section]

% Fonts for various types
\newcommand{\notionfont}[1]{\textup{#1}}
\newcommand{\varfont}[1]{\textit{#1}}
\newcommand{\flagfont}[1]{\mathsf{#1}}
\newcommand{\vectorfont}[1]{\vec{#1}}
\newcommand{\oraclefont}[1]{\textsc{#1}}
\newcommand{\schemefont}[1]{\mathsf{#1}}
\newcommand{\procfont}[1]{\schemefont{#1}}
\newcommand{\adversaryfont}[1]{\mathcal{#1}}
\newcommand{\cryptofont}[1]{\mathbf{#1}\hspace{0.5pt}}
\newcommand{\prinfont}[1]{\mathcal{#1}}

% Math
\DeclareMathAlphabet\mathbfcal{OMS}{cmsy}{b}{n}

% - Sets
\newcommand{\Z}{\mathbb{Z}}
\newcommand{\N}{\mathbb{N}}
\newcommand{\bits}{\{0,1\}}
\newcommand*\bigunion{\bigcup}
\newcommand*\bigintersection{\bigcap}
\newcommand*\union{\cup}
\newcommand*\intersection{\cap}
\newcommand*\cross{\times}
\newcommand*\by{\cross}
%\newcommand{\getsr}{\mathrel{\leftarrow\mkern-14mu\leftarrow}}
%\newcommand{\getsr}{\xleftarrow{\text{\tiny{\$}}}}
\newcommand{\getsr}{{\:{\leftarrow{\hspace*{-3pt}\raisebox{.75pt}{$\scriptscriptstyle\$$}}}\:}}

% - String operations
\newcommand{\emptystring}{\varepsilon}
\newcommand{\cat}{\, \| \,}
\newcommand{\str}[1]{\langle #1 \rangle}

% - Boolean operators
\newcommand*\AND{\wedge}
\newcommand*\OR{\vee}
\newcommand*\NOT{\neg}
\newcommand*\IMPLIES{\implies}
\newcommand*\XOR{\mathbin{\oplus}}
\newcommand*\xor{\XOR}

% - Crypto functions
\newcommand{\game}[1]{\cryptofont{Exp}^{\textnormal{\tiny \MakeLowercase{#1}}}}
\newcommand{\adv}[1]{\cryptofont{Adv}^{\textnormal{\tiny \MakeLowercase{#1}}}}

% - Asymptotics
\newcommand{\negl}{\procfont{negl}}
\newcommand{\poly}{\procfont{poly}}

% - Probablity
\newcommand{\E}{\mathrm{E}}
\newcommand{\Prob}[1]{\Pr\hspace{-2pt}\Big[\,#1\,\Big]}

% Games
\newcommand{\gamesfontsize}{\footnotesize}
\newcommand{\foreach}[3]{$\text{for }#1 \gets #2\text{ to }#3\text{ do}$}
\newcommand{\ind}{\hspace*{10pt}}
\newcommand{\outputs}{=}
\newcommand{\true}{1}
\newcommand{\false}{0}

% - Inline comment
\definecolor{CommentColor}{RGB}{125,175,230}
\newcommand{\comment}[1]{\textcolor{CommentColor}{\,\textbf{\#}\,#1}}

% - One game
\newcommand{\oneCol}[2]{
\begin{center}
        \framebox{
        \begin{tabular}{c@{\hspace*{.4em}}}
        \begin{minipage}[t]{#1\textwidth}\gamesfontsize #2 \end{minipage}
        \end{tabular}
        }
\end{center}
}

% - One game, two columns
\newcommand{\twoColsNoDivide}[3]{
  \providecommand{\pad}{6pt}
  \providecommand{\colwidth}{#1\textwidth}
  \makebox[\textwidth][c]{
  \begin{tabular}{|@{\hskip \pad}l@{}@{}@{\hskip \pad}l|}
    \hline
    \rule{0pt}{1\normalbaselineskip}
    \begin{minipage}[t]{\colwidth}\gamesfontsize
      #2 \vspace{\pad}
    \end{minipage} &
    \begin{minipage}[t]{\colwidth}\gamesfontsize
      #3 \vspace{\pad}
    \end{minipage} \\
    \hline
  \end{tabular}
  }
}

% - Two games, one per column
\newcommand{\twoCols}[3]{
  \providecommand{\pad}{6pt}
  \providecommand{\colwidth}{#1\textwidth}
  \makebox[\textwidth][c]{
  \begin{tabular}{|@{\hskip \pad}l@{}|@{}@{\hskip \pad}l|}
    \hline
    \rule{0pt}{1\normalbaselineskip}
    \begin{minipage}[t]{\colwidth}\gamesfontsize
      #2 \vspace{\pad}
    \end{minipage} &
    \begin{minipage}[t]{\colwidth}\gamesfontsize
      #3 \vspace{\pad}
    \end{minipage} \\
    \hline
  \end{tabular}
  }
}

% Notes
\newcounter{notectr}[section]
\newcommand{\getnotectr}{\stepcounter{notectr}\thesection.\thenotectr}
\newcommand{\basenote}[4]{{
  \textcolor{#1}{(#3#4---#2~\getnotectr)}
}}

\newcommand{\note}[3]{\basenote{#1}{#2}{}{#3}}
\newcommand{\todo}[3]{\basenote{#1}{#2}{TODO~}{#3}}

% - Chris' notes
\definecolor{darkgreen}{RGB}{50,127,0}
\newcommand{\cpnote}[1]{\note{darkgreen}{Chris}{#1}}
\newcommand{\cptodo}[1]{\todo{darkgreen}{Chris}{#1}}
