%
%
%
Following~\cite{bellare2016nonce} we extend our framework to digital signatures.
\begin{definition}[DS]\rm
  A digital signature scheme~$\ds$ is a 4-tuple of algorithms $(\Kg,\\ \Sgn.\init, \Sgn,
  \Vfy)$ defined as follows:
  %
  \begin{itemize}
    \item $\Kg$ is the randomized key generation algorithm.

    \item $\Sgn.\init$ is the deterministic signing state initialization
      algorithm. It takes as input an initializer~$\stsel$ and returns the
      state. This is denoted $\st \gets \Sgn.\init(\stsel)$.

    \item $\Sgn$ is the deterministic signing algorithm, taking as input the
      secret key~$\sk$, the message~$\msg$, and the current state, and
      outputting~$\sig$, either the signature or~$\bot$, and the updated state.
      This is denoted $(\sig, \st) \gets \Sgn_\sk(\msg, \st)$.

    \item $\Vfy$ is the deterministic verification algorithm, taking as input the
      public key~$\pk$, the message~$\msg$, and the signature~$\sig$ and
      outputting $v$, either a bit indicating whether the signature is valid,
      or~$\bot$. This is denoted $v \gets \Vfy_\pk(\msg, \sig)$.
  \end{itemize}
  %
  \cpnote{$\Sgn$ and $\Vfy$ could use a nonce, but I don't see what the practical
  benefit would be. Though, doesn't DJB's Poly1305 MAC use nonce? What's the
  reasonf or this?}
  \dqed
\end{definition}

\heading{\ufsig.}
%
Unforgeability of signature schemes is defined in Figure~\ref{fig:ufsig}. As
usual, we consider a setting in which the signing state is eventually exposed to
the adversary.
%
A reasonable criticism of this notion is that if the adversary is able to
exfiltrate the signing state, then it ought to be able to exfiltrate the signing
key. To this point, \cite{bellare2016nonce} argues that the signer might take
extra care in storing the key, but choose to store the signing state (i.e., the
seed and nonce generation state in their setting) in a less secure part of the
signer's system. For example, the long-term signing key might be stored in an
HSM (``hardware security module'') and the short-term signing state in main
memory.
%
\cptodo{Look at attestation/signing APIs for SGX. Are the signing algorithms
randomized, and if so, where do the coins come from?}

\begin{figure}[t]
  \twoColsNoDivide{0.48}
  {
    \experimentv{$\Exp{\ufsig}_{\ds}(\advA,\dist)$}\\[2pt]
      $\setQ \gets \emptyset$\\
      $\stsel \getsr \dist$;
      $\st \gets \Sgn.\init(\stsel)$\\
      $(\pk, \sk) \getsr \Kg$\\
      $(\msg, \sig) \getsr \advA^{\OO}(\pk)$\\
      return $\Vfy_\pk(\msg, \sig)$ and $\msg \not\in \setQ$
  }
  {
    \oraclev{$\SGNO(\msg)$}\\[2pt]
      $\setQ \gets \setQ \union \{\msg\}$\\
      $(\sig, \st) \gets \Sgn_\sk(\msg, \st)$\\
      return $\sig$
    \\[6pt]
    \oraclev{$\INITO(\stsel)$}\\[2pt]
      $\st \gets \pkead.\Enc.\init(\stsel)$
    \\[6pt]
    \oraclev{$\EXPO$}\\[2pt]
      return $\st$
  }
  \caption{Security notion for digital signature schemes. Let $\OO =
  (\INITO,\EXPO,\SGNO)$.}
  \label{fig:ufsig}
  \vspace{6pt}\hrule
\end{figure}

\heading{Constructions.}
%
\cpnote{The solution that~\cite{bellare2016nonce} had in mind will work here.
That is, use a PRNG to generate coins for a standard, randomized digital
signature scheme.}
