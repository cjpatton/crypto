%
%
%
\label{sec:prng}
This syntax provides syntax and security notions for pseudorandom number
generators \emph{with inputs}, as first formalized by~\cite{barak2005model}. It
also borrows ideas from~\cite{dodis2013security,shrimpton2015provable}.

\begin{definition}[Entropy source]\rm
  An \emph{entropy source} is a randomized algorithm~$\dist$ with no inputs and
  that outputs a string.
  %
  \cpnote{This syntax follows \cite{barak2005model}, but diverges
  from~\cite{dodis2013security,shrimpton2015provable}, where the entropy source
  may be stateful.}
  %
  We say that~$\dist$ has min-entropy~$\mu$ if for every
  $x \in \bits^*$, it holds that
  $
  \Prob{y \getsr \dist\colon x = y} \leq 2^{-\mu}
  $. \dqed
\end{definition}
%
\cpnote{We could extend this notion along the lines of
\cite{bellare2009hedged,boldyreva2017real} so that $\dist$ outputs a vector of
strings.}
%
\cpnote{Define the maximum output length?}
%
\cpnote{Could model \emph{side information} about the output.}

\begin{definition}[PRNG]\rm
  A \emph{pseudorandom number generation} scheme, $\prng$, is a triple of
  deterministic algorithms $(\init, \add, \get)$ defined as follows:
  %
  \begin{itemize}
    \item $\init$ is the initialization algorithm. It takes as input a
      string~$\stsel$, called the \emph{initializer}, and outputs a string~$\st$
      called the \emph{state}.
      %
      This is denoted $\st \gets \init(\stsel)$.

    \item $\add$ takes a string~$\sel$ called the \emph{selector},
      %
      \cpnote{Nomenclature is borrowed from~\cite{bellare2016nonce}. There might
      be a better name.}
      %
      and the state and outputs the updated state.
      %
      This is denoted $\st \gets \add(\sel, \st)$.

    \item $\get$ takes as input an integer $\rho \geq 0$
      %
      \cpnote{This parameter does not appear in earlier work.}
      %
      and the state and
      outputs a string $\coins \in \bits^\rho$ and the updated state.
      %
      This is denoted $(\coins, \st) \gets \get(\coinslen, \st)$.
      \dqed
  \end{itemize}
\end{definition}

We define four notions of security PRNG schemes in Figure~\ref{fig:prng-sec}.
%
\indfwd and \indbwd capture \emph{indistingushibility} of the output of
$\prng.\get(\cdot)$ from random, the first in the \emph{forward-secure} sense,
and the second in the \emph{backward-secure} sense.
%
Roughly speaking, \emph{forward security} demands that if the state is exposed
to the adversary, then all prior uses of the PRNG remain secure, and
\emph{backward security} demands that if the state is exposed, then future uses
of the PRNG are secure as long as the state is refreshed.
%
\upfwd and \upbwd capture only \emph{unpredictability} of the output of
$\prng.\get(\cdot)$.
%
Something we'll need to figure out is what restrictions we need to make
on the source(s) of entropy for these notions to be achievable. For example, in
the \indfwd game, if we make no restrictions on the $\INITRO$-queries, then
there is an easy distinguishing attack.

\begin{figure}[t]
  \newcommand{\coll}{\flagfont{coll}}
  \twoColsTwoRows{0.48}
  {
    \experimentv{$\Exp{\indfwdX{b}}_\prng(\advA)$}\\[2pt]
      $\stout \gets \true$\\
      $b' \getsr \advA^{\INITRO,\EXPO,\ADDO,\GETINDO}$\\
      return $b'$
  }
  {
    \experimentv{$\Exp{\indbwdX{b}}_\prng(\advA)$}\\[2pt]
      $\stout \gets \true$\\
      $b' \getsr \advA^{\INITO,\ADDRO,\GETINDO}$\\
      return $b'$
  }
  {
    \experimentv{$\Exp{\upfwd}_\prng(\advA)$}\\[2pt]
      $\stout \gets \true$; $\coll \gets \false$;
      $\setX \gets \emptyset$\\
      $\coins \getsr \advA^{\INITRO,\EXPO,\ADDO,\GETUPO}$\\
      return $\coll \OR (\coins \in \setX)$
  }
  {
    \experimentv{$\Exp{\upbwd}_\prng(\advA)$}\\[2pt]
      $\stout \gets \true$; $\coll \gets \false$;
      $\setX \gets \emptyset$\\
      $\coins \getsr \advA^{\INITO,\ADDRO,\GETUPO}$\\
      return $\coll \OR (\coins \in \setX)$
  }
  \twoColsNoDivide{0.48}
  {
    \oraclev{$\INITRO(\dist)$}\\[2pt]
      $\stout \gets \false$;
      $\stsel \getsr \dist$;
      $\st \gets \prng.\init(\stsel)$
    \\[6pt]
    \oraclev{$\ADDRO(\dist)$}\\[2pt]
      $\stout \gets \false$;
      $\sel \getsr \dist$;
      $\st \gets \prng.\add(\st, \sel)$
    \\[6pt]
    \oraclev{$\INITO(\stsel)$}\\[2pt]
      $\stout \gets \true$;
      $\st \gets \prng.\init(\stsel)$
    \\[6pt]
    \oraclev{$\ADDO(\sel)$}\\[2pt]
      $\st \gets \prng.\add(\st, \sel)$
    \\[6pt]
    \oraclev{$\EXPO(\,)$}\\[2pt]
      $\stout \gets \true$;
      return $\st$
  }
  {
    \oraclev{$\GETINDO(\coinslen)$}\\[2pt]
      if $\stout = \true$ then return $\bot$\\
      $(\coins_1, \st) \gets \prng.\get(\coinslen, \st)$\\
      $\coins_0 \getsr \bits^\coinslen$\\
      return $\coins_b$
    \\[6pt]
    \oraclev{$\GETUPO(\coinslen)$}\\[2pt]
      if $\stout = \true$ then return $\bot$\\
      $(\coins, \st) \gets \prng.\get(\coinslen, \st)$\\
      if $\coins \in \setX$ then $\coll \gets \true$\\
      $\setX \gets \setX \union \{ \coins \}$\\
      return~$1$
  }
  \caption{\textbf{Top:} Security notions for PRNG schemes.
  %
  \textbf{Bottom:} Oracles for the security notions.}
  \vspace{6pt}\hrule
  \label{fig:prng-sec}
\end{figure}
