%
\documentclass{build/llncs}
\usepackage{color,soul}
\usepackage[dvipsnames]{xcolor}
\usepackage{enumitem}
% header.tex
%
% Formatting and common macros for crypto papers. Include this first.
\usepackage[dvipsnames]{xcolor}
\usepackage{color,soul}
\usepackage{enumitem}
\usepackage{xspace}
\usepackage{amsmath}
\usepackage{amssymb}
\usepackage{amsfonts}
\usepackage{xcolor}
\usepackage{xstring}
\usepackage{enumitem}


\usepackage{tikz}
\tikzset{>=latex}

\usepackage{listings}
\lstset{
  basicstyle=\small\ttfamily,
  columns=flexible,
  breaklines=true
}

\usepackage{hyperref}
\hypersetup{
  colorlinks,%
  citecolor=black,%
  filecolor=black,%
  linkcolor=black,%
  urlcolor=black
}

% Colors
\definecolor{theblue}{RGB}{85,135,170}
\definecolor{thedarkgray}{RGB}{95,95,95}
\definecolor{thedarkgreen}{RGB}{50,127,0}
\definecolor{thegray}{RGB}{165,165,165}
\definecolor{thelightblue}{RGB}{185,220,250}
\definecolor{thelightgray}{RGB}{230,230,230}
\definecolor{thelightred}{RGB}{250,185,165}
\definecolor{thered}{RGB}{210,145,125}

% Editorial
\newcommand{\ala}{\text{a la}\xspace}
\newcommand{\etal}{et al.\xspace}
\newcommand{\apriori}{\textit{a priori}\xspace}
\newcommand{\viceversa}{\textit{vice versa}\xspace}

% Fonts
\newcommand{\algorithmfont}[1]{\mathcal{#1}}
\newcommand{\codefont}[1]{\lstinline|#1|}
\newcommand{\cryptofont}[1]{\textup{\textbf{#1}}\hspace{0.5pt}}
\newcommand{\expfont}[1]{{{\tiny\MakeLowercase{\textnormal{#1}}}}}
\newcommand{\lifont}[1]{\textsf{\footnotesize\color{thegray}#1}}
\newcommand{\notionfont}[1]{{#1}\xspace}
\newcommand{\oraclefont}[1]{\cryptofont{#1}}
\newcommand{\procfont}[1]{\textnormal{#1}\hspace{0.33pt}}
\newcommand{\schemefont}[1]{\mathcal{#1}}
\newcommand{\setfont}[1]{\mathcal{#1}}
\newcommand{\rwordfont}[1]{\cryptofont{#1}\xspace}
\newcommand{\strfont}[1]{\textup{\textsf{{\color{theblue}#1}}}}
\newcommand{\varfont}[1]{\mathit{#1}}
\newcommand{\symbolfont}[1]{\textsf{\footnotesize\color{thegray}#1}}

\newcommand{\stringer}[1]{\strfont{#1}}
\newcommand{\rword}[1]{\rwordfont{#1}}

% Crypto functions
\newcommand{\Hybrid}{\cryptofont{H}}
\newcommand{\Gam}{\cryptofont{G}}
\newcommand{\Exp}[1]{\cryptofont{Exp}^{\expfont{#1}}}
\newcommand{\Adv}[1]{\cryptofont{Adv}^{\expfont{#1}}}

% Types
\newcommand{\typ}[1]{\typs\rword{#1}}
\newcommand{\typs}{\,}
\newcommand{\us}{\!} % previously \unskip

\newcommand{\rbool}{\typ{bool}}
\newcommand{\relem}[1]{\typ{elem}(#1)}
\newcommand{\rint}{\typ{int}}
\newcommand{\rset}{\typ{set}}
\newcommand{\rstr}{\typ{str}}
\newcommand{\rstruct}{\typ{struct}}
\newcommand{\rtuple}{\typ{tup}}

\newcommand{\rdet}{\rword{det}\typs}
\newcommand{\rtype}{\rword{type}}
\newcommand{\rrand}{\rword{rand}\typs}

%% Fancy syntax stuff
\newcommand{\nil}{\symbolfont{nil}}
\newcommand{\ptr}{\typs\symbolfont{*}\us}
\newcommand{\rfr}{\symbolfont{\&}}
\newcommand{\any}{\symbolfont{\_}}

%% Reserved words
\newcommand{\rand}{\rwordfont{and}}
\newcommand{\rdeclare}{\rwordfont{dec}}
\newcommand{\relse}{\rwordfont{else}}
\newcommand{\rif}{\rwordfont{if}}
\newcommand{\rwhile}{\rwordfont{while}}
\newcommand{\runtil}{\rwordfont{until}}
\newcommand{\rdo}{\rwordfont{do}}
\newcommand{\ror}{\rwordfont{or}}
\newcommand{\rfor}{\rwordfont{for}}
\newcommand{\rforeach}{\rwordfont{for each}}
\newcommand{\rreturn}{\rwordfont{ret}}
\newcommand{\rbreak}{\rwordfont{break}}
\newcommand{\rthen}{\rwordfont{then}}
\newcommand{\rto}{\rwordfont{to}}
\newcommand{\rswitch}{\rwordfont{switch}}
\newcommand{\rcase}{\rwordfont{case}}
\newcommand{\rdefault}{\rwordfont{default}}

% Math
\DeclareMathAlphabet\mathbfcal{OMS}{cmsy}{b}{n}

\def\v.#1{\boldsymbol{#1}}
\newcommand{\dqed}{\hfill$\Diamond$}
\newcommand{\GF}{\mathit{GF}}

%% Sets
\DeclareMathOperator*{\argmin}{arg\,min}
\DeclareMathOperator*{\dom}{Dom}
\DeclareMathOperator*{\rng}{Rng}
\def\bydef{\stackrel{\rm def}{=}}
\newcommand*\by{\times}
\newcommand{\getsr}{\xleftarrow{\text{\tiny{\$}}}}
%\newcommand{\getsr}{{\:{\leftarrow{\hspace*{-3pt}\raisebox{.75pt}{$\scriptscriptstyle\$$}}}\:}}
\newcommand*\intersection{\cap}
\newcommand{\multiunion}{\uplus}
\newcommand*\union{\cup}

%% String/tuple operators
\def\slicerange#1,#2{#1\mbox{\hspace{0.5pt}:\hspace{0.5pt}}#2}
\newcommand{\bits}{\{0,1\}}
\newcommand{\cat}{\, \| \,}
\newcommand{\emptystr}{\varepsilon}
\newcommand{\emptystring}{\emptystr}
\newcommand{\hassubstr}{\sim}
\newcommand{\prefixes}{\preceq}
\newcommand{\slice}[2]{#1[\slicerange#2]}
\newcommand{\toint}[2]{\underline{\smash{#1}}_{\hspace{0.5pt}#2}}
\newcommand{\tostr}[1]{\underline{\smash{#1}}}
\newcommand{\trunc}{\%}

%% Boolean operators
\newcommand*\band{\wedge}
\newcommand*\bor{\vee}
\newcommand*\bnot{\neg}
\newcommand*\bimplies{\implies}
\newcommand*\bxor{\mathbin{\oplus}}

%% Probability
\newcommand{\E}{\mathrm{E}}
\newcommand{\Prob}[1]{\Pr\hspace{-1pt}\big[\,#1\,\big]}
\newcommand{\Probs}[2]{\Pr_{#1}\hspace{-1pt}{\big[\,#2\,\big]}}
\newcommand{\given}{\mid}

% References
\newcommand{\fig}[1]{Fig.~\ref{fig:#1}}
\newcommand{\lem}[1]{Lemma~\ref{lemma:#1}}

% Dev
\def\coloroverride{0}
\newenvironment{cleanup}{
  \def\coloroverride{1}
  \color{gray}
}{
}

% Notes
\newcounter{notectr}[section]
\newcommand{\getnotectr}{\stepcounter{notectr}\thesection.\thenotectr}

\newcommand{\basenoteshape}[3]{(\getnotectr: #1\xspace#2: #3)}

\newcommand{\basenote}[4]{%
  \textrm{%
    \IfEqCase{\coloroverride}{%
        {1}{\basenoteshape{#2}{#3}{#4}}%
        {0}{\textcolor{#1}{\basenoteshape{#2}{#3}{#4}}}%
    }[\PackageError{basenote}{Undefined option for basenote: \coloroverride}{}]%
  }%
}

%% Uncomment to mute notes.
%\renewcommand{\basenote}[4]{\ignorespaces}

\newcommand{\todo}[3]{\basenote{#1}{#2}{todo}{#3}}
\newcommand{\fixme}[1]{\basenote{red}{}{fixme}{{\bf#1}}}

\newcommand{\ignore}[1]{\if{0}#1\fi}

% Games
\renewcommand{\#}[1]{\text{\textcolor{theblue}{\,\text{//}\,{\small #1}}}}
\newcommand{\g}[1]{\cryptofont{G}_{#1}}
\newcommand{\hybrid}[1]{\cryptofont{H}_{#1}}
\newcommand{\gouts}{=}
\newcommand{\gsets}{\,\cryptofont{sets}\,}
\newcommand{\tab}{\hspace*{10pt}}

%% Diffs
\newcommand{\diffplus}[1]{\fbox{#1}}
\newcommand{\diffplusbox}[1]{\fbox{\parbox{\dimexpr\textwidth-2\fboxsep-2\fboxrule\relax}{#1}}\vspace{1pt}}
\newcommand{\diffminus}[1]{\colorbox{lightgray}{#1}}
\newcommand{\diffminusbox}[1]{\colorbox{lightgray}{\parbox{\dimexpr\textwidth-2\fboxsep-2\fboxrule\relax}{#1}}}
\newcommand{\diff}[2]{\diffminus{#1}\diffplus{#2}}

%% Line numbers
\newcommand{\liref}[2]{\fig{#1}:#2}
\newcounter{FunctionCounter}
\newcounter{LineCounter}[FunctionCounter]

%%% Reset numbering (new game)
\newcommand{\ga}{\setcounter{FunctionCounter}{0}\setcounter{LineCounter}{0}}

%%% Increment function number
\newcommand{\fu}{
%  \stepcounter{FunctionCounter}
}

%%% Increment line number
\usepackage{calc}
\newcommand{\li}{
%  \stepcounter{LineCounter}\lifont{\scriptsize\theFunctionCounter\ifnum\value{LineCounter}<10\ignorespaces0\fi\theLineCounter}
  \stepcounter{LineCounter}\makebox[8pt][r]{\lifont{\scriptsize\theLineCounter}}\hspace{2pt}
}

%% Boxes
\newcommand{\gamesfontsize}{\small}
\newcommand{\gamespadleft}{\hskip 1pt}
\newcommand{\gamespad}{\hskip 4pt}
\newcommand{\gamespacer}{\\[-6.5pt]}

\newcommand{\oneCol}[2]{
  \begin{center}
    \makebox[\textwidth][l]{
      \begin{tabular}{|@{\gamespadleft}l@{\gamespad}@{}|}
      \hline
      \rule{0pt}{1\normalbaselineskip}
      \begin{minipage}[t]{#1\textwidth}\gamesfontsize
        #2 \vspace{6pt}
      \end{minipage} \\
      \hline
    \end{tabular}
    }
  \end{center}
}

\newcommand{\twoCols}[3]{
  \makebox[\textwidth][c]{
    \begin{tabular}{|@{\gamespadleft}l@{\gamespad}|@{}@{\gamespad}l@{\gamespad}|}
    \hline
    \rule{0pt}{1\normalbaselineskip}
    \begin{minipage}[t]{#1\textwidth}\gamesfontsize
      #2 \vspace{6pt}
    \end{minipage} &
    \begin{minipage}[t]{#1\textwidth}\gamesfontsize
      #3 \vspace{6pt}
    \end{minipage} \\
    \hline
  \end{tabular}
  }
}

\newcommand{\twoColsNoBox}[3]{
  \makebox[\textwidth][c]{
    \begin{tabular}{@{\gamespadleft}l@{\gamespad}|@{}@{\gamespad}l@{\gamespad}}
    \rule{0pt}{1\normalbaselineskip}
    \begin{minipage}[t]{#1\textwidth}\gamesfontsize
      #2 \vspace{6pt}
    \end{minipage} &
    \begin{minipage}[t]{#1\textwidth}\gamesfontsize
      #3 \vspace{6pt}
    \end{minipage} \\
  \end{tabular}
  }
}

\newcommand{\twoColsUnbalanced}[4]{
  \makebox[\textwidth][c]{
    \begin{tabular}{|@{\gamespadleft}l@{\gamespad}|@{}@{\gamespad}l@{\gamespad}|}
    \hline
    \rule{0pt}{1\normalbaselineskip}
    \begin{minipage}[t]{#1\textwidth}\gamesfontsize
      #3 \vspace{6pt}
    \end{minipage} &
    \begin{minipage}[t]{#2\textwidth}\gamesfontsize
      #4 \vspace{6pt}
    \end{minipage} \\
    \hline
  \end{tabular}
  }
}

\newcommand{\twoColsNoLinesUnbalanced}[4]{
  \makebox[\textwidth][c]{
    \begin{tabular}{@{\gamespadleft}l@{\gamespad}@{}@{\gamespad}l@{\gamespad}}
    \rule{0pt}{1\normalbaselineskip}
    \begin{minipage}[t]{#1\textwidth}\gamesfontsize
      #3 \vspace{6pt}
    \end{minipage} &
    \begin{minipage}[t]{#2\textwidth}\gamesfontsize
      #4 \vspace{6pt}
    \end{minipage} \\
  \end{tabular}
  }
}

\newcommand{\twoColsNoDivide}[3]{
  \makebox[\textwidth][c]{
    \begin{tabular}{|@{\gamespadleft}l@{\gamespad}@{}@{\gamespad}l@{\gamespad}|}
    \hline
    \rule{0pt}{1\normalbaselineskip}
    \begin{minipage}[t]{#1\textwidth}\gamesfontsize
      #2 \vspace{6pt}
    \end{minipage} &
    \begin{minipage}[t]{#1\textwidth}\gamesfontsize
      #3 \vspace{6pt}
    \end{minipage} \\
    \hline
  \end{tabular}
  }
}

\newcommand{\twoColsNoDivideUnbalanced}[4]{
  \makebox[\textwidth][c]{
    \begin{tabular}{|@{\gamespadleft}l@{\gamespad}@{}@{\gamespad}l@{\gamespad}|}
    \hline
    \rule{0pt}{1\normalbaselineskip}
    \begin{minipage}[t]{#1\textwidth}\gamesfontsize
      #3 \vspace{6pt}
    \end{minipage} &
    \begin{minipage}[t]{#2\textwidth}\gamesfontsize
      #4 \vspace{6pt}
    \end{minipage} \\
    \hline
  \end{tabular}
  }
}

\newcommand{\twoColsTwoRows}[5]{
  \makebox[\textwidth][c]{
  \begin{tabular}{|@{\gamespadleft}l@{\gamespad}|@{}@{\gamespad}l@{\gamespad}|}
    \hline
    \rule{0pt}{1\normalbaselineskip}
    \begin{minipage}[t]{#1\textwidth}\gamesfontsize
      #2 \vspace{6pt}
    \end{minipage} &
    \begin{minipage}[t]{#1\textwidth}\gamesfontsize
      #3 \vspace{6pt}
    \end{minipage} \\
    \hline
    \rule{0pt}{1\normalbaselineskip}
    \begin{minipage}[t]{#1\textwidth}\gamesfontsize
      #4 \vspace{6pt}
    \end{minipage} &
    \begin{minipage}[t]{#1\textwidth}\gamesfontsize
      #5 \vspace{6pt}
    \end{minipage} \\
    \hline
  \end{tabular}
  }
}

\newcommand{\threeCols}[4]{
  \makebox[\textwidth][c]{
    \begin{tabular}{|@{\gamespadleft}l@{\gamespad}|@{}@{\gamespad}l@{\gamespad}|@{}@{\gamespad}l@{\gamespad}|}
    \hline
    \rule{0pt}{1\normalbaselineskip}
    \begin{minipage}[t]{#1\textwidth}\gamesfontsize
      #2 \vspace{6pt}
    \end{minipage} &
    \begin{minipage}[t]{#1\textwidth}\gamesfontsize
      #3 \vspace{6pt}
    \end{minipage} &
    \begin{minipage}[t]{#1\textwidth}\gamesfontsize
      #4 \vspace{6pt}
    \end{minipage} \\
    \hline
  \end{tabular}
  }
}

\newcommand{\threeColsUnbalanced}[6]{
  \makebox[\textwidth][c]{
    \begin{tabular}{|@{\gamespadleft}l@{\gamespad}|@{}@{\gamespad}l@{\gamespad}|@{}@{\gamespad}l@{\gamespad}|}
    \hline
    \rule{0pt}{1\normalbaselineskip}
    \begin{minipage}[t]{#1\textwidth}\gamesfontsize
      #4 \vspace{6pt}
    \end{minipage} &
    \begin{minipage}[t]{#2\textwidth}\gamesfontsize
      #5 \vspace{6pt}
    \end{minipage} &
    \begin{minipage}[t]{#3\textwidth}\gamesfontsize
      #6 \vspace{6pt}
    \end{minipage} \\
    \hline
  \end{tabular}
  }
}

\newcommand{\threeColsFirstDivideUnbalanced}[6]{
  \makebox[\textwidth][c]{
    \begin{tabular}{|@{\gamespadleft}l@{\gamespad}|@{}@{\gamespad}l@{\gamespad}@{}@{\gamespad}l@{\gamespad}|}
    \hline
    \rule{0pt}{1\normalbaselineskip}
    \begin{minipage}[t]{#1\textwidth}\gamesfontsize
      #4 \vspace{6pt}
    \end{minipage} &
    \begin{minipage}[t]{#2\textwidth}\gamesfontsize
      #5 \vspace{6pt}
    \end{minipage} &
    \begin{minipage}[t]{#3\textwidth}\gamesfontsize
      #6 \vspace{6pt}
    \end{minipage} \\
    \hline
  \end{tabular}
  }
}

\newcommand{\threeColsSecondDivideUnbalanced}[6]{
  \makebox[\textwidth][c]{
    \begin{tabular}{|@{\gamespadleft}l@{\gamespad}@{}@{\gamespad}l@{\gamespad}@{}|@{\gamespad}l@{\gamespad}|}
    \hline
    \rule{0pt}{1\normalbaselineskip}
    \begin{minipage}[t]{#1\textwidth}\gamesfontsize
      #4 \vspace{6pt}
    \end{minipage} &
    \begin{minipage}[t]{#2\textwidth}\gamesfontsize
      #5 \vspace{6pt}
    \end{minipage} &
    \begin{minipage}[t]{#3\textwidth}\gamesfontsize
      #6 \vspace{6pt}
    \end{minipage} \\
    \hline
  \end{tabular}
  }
}

\newcommand{\threeColsNoDivideUnbalanced}[6]{
  \makebox[\textwidth][c]{
    \begin{tabular}{|@{\gamespadleft}l@{\gamespad}@{}@{\gamespad}l@{\gamespad}@{}@{\gamespad}l@{\gamespad}|}
    \hline
    \rule{0pt}{1\normalbaselineskip}
    \begin{minipage}[t]{#1\textwidth}\gamesfontsize
      #4 \vspace{6pt}
    \end{minipage} &
    \begin{minipage}[t]{#2\textwidth}\gamesfontsize
      #5 \vspace{6pt}
    \end{minipage} &
    \begin{minipage}[t]{#3\textwidth}\gamesfontsize
      #6 \vspace{6pt}
    \end{minipage} \\
    \hline
  \end{tabular}
  }
}

\newcommand{\fourCols}[5]{
  \makebox[\textwidth][c]{
    \begin{tabular}{|@{\gamespadleft}l@{\gamespad}|@{}@{\gamespad}l@{\gamespad}|@{}@{\gamespad}l@{\gamespad}|@{}@{\gamespad}l@{\gamespad}|}
    \hline
    \rule{0pt}{1\normalbaselineskip}
    \begin{minipage}[t]{#1\textwidth}\gamesfontsize
      #2 \vspace{6pt}
    \end{minipage} &
    \begin{minipage}[t]{#1\textwidth}\gamesfontsize
      #3 \vspace{6pt}
    \end{minipage} &
    \begin{minipage}[t]{#1\textwidth}\gamesfontsize
      #4 \vspace{6pt}
    \end{minipage} &
    \begin{minipage}[t]{#1\textwidth}\gamesfontsize
      #5 \vspace{6pt}
    \end{minipage} \\
    \hline
  \end{tabular}
  }
}

\newcommand{\twoColsThreeRows}[7]{
  \makebox[\textwidth][c]{
  \begin{tabular}{|@{\gamespadleft}l@{\gamespad}|@{}@{\gamespad}l@{\gamespad}|}
    \hline
    \rule{0pt}{1\normalbaselineskip}
    \begin{minipage}[t]{#1\textwidth}\gamesfontsize
      #2 \vspace{6pt}
    \end{minipage} &
    \begin{minipage}[t]{#1\textwidth}\gamesfontsize
      #3 \vspace{6pt}
    \end{minipage} \\
    \hline
    \rule{0pt}{1\normalbaselineskip}
    \begin{minipage}[t]{#1\textwidth}\gamesfontsize
      #4 \vspace{6pt}
    \end{minipage} &
    \begin{minipage}[t]{#1\textwidth}\gamesfontsize
      #5 \vspace{6pt}
    \end{minipage} \\
    \hline
    \rule{0pt}{1\normalbaselineskip}
    \begin{minipage}[t]{#1\textwidth}\gamesfontsize
      #6 \vspace{6pt}
    \end{minipage} &
    \begin{minipage}[t]{#1\textwidth}\gamesfontsize
       #7 \vspace{6pt}
    \end{minipage} \\
    \hline
  \end{tabular}
  }
}

% macros.tex
%
% Macros for this paper. (Include build/headers.tex, then this.)
\usepackage[dvipdf]{graphicx}
\usepackage{pifont} % \ding{109}
\usepackage{afterpage} % \afterpage{ ... }

% Variables
\newcommand{\dist}{\delta}
\newcommand{\doride}{\varphi}
\newcommand{\De}{\procfont{De}}
\newcommand{\En}{\procfont{En}}
\newcommand{\Ev}{\procfont{Ev}}
\newcommand{\ev}{\procfont{ev}}
\newcommand{\garbler}{\schemefont{G}}
\newcommand{\Gb}{\procfont{Gb}}
\newcommand{\id}{\funcfont{id}}
\newcommand{\loc}{\funcfont{loc}}
\newcommand{\rep}{\funcfont{rep}}
\newcommand{\SpA}{\procfont{SpA}}
\newcommand{\SpB}{\procfont{SpB}}

% Authors' comments.
\definecolor{darkgreen}{RGB}{50,127,0}
\newcommand{\cpnote}[1]{\note{darkgreen}{Chris}{#1}}
\newcommand{\tsnote}[1]{\note{blue}{Tom}{#1}}
\newcommand{\cptodo}[1]{\todo{red}{Chris}{#1}}
\newcommand{\tstodo}[1]{\todo{red}{Tom}{#1}}
\newcommand{\anytodo}[1]{\todo{red}{Any}{#1}}


\date{\today}
\title{Real-world{\color{gray}(?)} public-key cryptography\\
\emph{\color{gray}or}\\
Robust public-key cryptography with applications to multi-factor authentication
}
\author{Christopher Patton and Thomas Shrimpton and FSM}
\institute{}

\setcounter{tocdepth}{2}

\pagestyle{plain}

\begin{document}

\maketitle

\begin{abstract}
  \cite{bellare2009hedged} shows that public-key encryption can be hardened
  against RNG failures by manipulating the coins used to encrypt the message.
  %
  \cite{bellare2016nonce} changes the model for PKE so that only minimal use of
  the RNG is required. They show that encryption in this setting can be hardened
  against (partial) exposure of the encryption state.
  %
  \cite{boldyreva2017real} points out that the techniques of
  \cite{bellare2009hedged,bellare2016nonce} are impractical in the sense that
  directly manipulating the coins breaks the ``intended use'' of many crypto
  APIs, including OpenSSL.
  %
  This work adopts the perspective that good API design and sound cryptographic
  theory are inextricable.
  %
  Bearing this in mind, we consider the security of public-key cryptography
  when instantiated with real-world (P)RNGS.
\end{abstract}

\section{Introduction}
% intro.tex
%
% Introduction.
\label{sec-intro}

\hl{So here's where I wanna go with this.}
I've written up the scheme of Gilboa+Ishai in Section~\ref{sec-twowriter}.
Consider making this scheme ``verifiable''. Currently the idea is to add a
$0^t$ tag to the message. The write servers MAC the last~$t$ bits of each row of
their evaluated share; the auditor then checks to make sure the MACs are the same.
\begin{itemize}
  \item Give up on proofs for Riposte. What is the simplest, strongest notion
    that captures the security of such a mechanism?

  \item How to modify the Gilboa+Ishai construction? Does this require a
    stronger assumption about PRGs? If so, could I add more crypto in key share
    generation so that I don't need stronger assumptions?

  \item Gilboa+Ishai scheme is more naturally expressed in terms of strings~$X$
    and~$Y$ rather than table index~$\idx$ and message $\msg \in \ring
    \setminus\{0\}$, where $\ring$ is a ring. The reason to use this is for
    capturing the $(\sharect,\sharect-1)$-private Riposte scheme. Maybe just use
    this notation for this special case?

  \item Change \emph{write shares} to \emph{key shares}. Share verification
    makes sense both in the read-from and write-to settings.

  \item Frame the Riposte schemes as extensions of Gilboa+Ishai's level-1
    scheme.
\end{itemize}
\hl{And now back to our featured presentation.}

% PIR is the motivation for formalizing DPF schemes.
Distributed point functions were first proposed by Niv Gilboa and Yuval Ishai
\cite{dpf} as a building block for private information retrieval (PIR) systems,
whose study was initiated by Benny Chor et al \cite{pir}. The goal of such
systems is to allow a client to query a database without divulging to the
service provider the query or the result of the query.
%
Traditionally the query is a predicate of the database modeled as a binary
string~$D$. For example, the client might like to learn the~$\idx$-th bit of~$D$
without revealing~$\idx$.
%
Keyword search~\cite{pir-kws} is another application whereby the database
encodes a set of strings and the query is whether a particular string is in the
set.
%
Closely related is the problem of private information storage (PIS) put forward
by Rafail Ostrovsky and Victor Shoup~\cite{pis} where the goal is to write a bit
to~$\idx$-th row of~$D$ without revealing to the service provider which bit was
modified.

% Communication and threat model of PIR systems.
The most efficient PIR systems distribute the database among a set of servers.
%
The client maps its query to a sequence of shares and sends one to each
of~$\sharect$ servers. Each evaluates its share on its local state and returns
the result to the client. Combining the results of each of the servers yields
the result of the query.
%
Privacy of the user's query is achieved under the assumption of non-collusion,
meaning that honest servers do not communicate outside of the protocol.
Correspondingly, we consider security with respect to a \emph{coalition} of
malicious servers who may act arbitrarily to violate the privacy of the query.
%
The adversary is only given a partial view of the network. In particular, it
sees only the messages sent to a server in the coalition.
%
Informally, we say a PIR system is $(\sharect,t)$-private if no coalition of at
most~$t$ servers can deduce the query from their shares alone.

% Syntax of DPF schemes.
\heading{DPF schemes.}
Each of the queries described above can be written as a point function.
%
The \emph{point function} of $(X,Y) \in (\bits^*)^2$ is defined by $P_{X,Y}(X) =
Y$ and $P_{X,Y}(X^\prime) = 0^{|Y|}$ for every $X^\prime \ne X$.
%
Gilboa and Ishai define a \emph{distributed point function} (DPF) scheme as a
pair of algorithms $(\gen, \eval)$ where $\gen$ probabilistically maps $(X,Y)$
to a sequence of shares $(K_1, \ldots, K_\sharect)$ and $\eval$
deterministically maps $(K_i, X^\prime)$ to a $|Y|$-bit string such that
$P_{X,Y}(X^\prime) = \xor_{i=1}^s \eval(K_i, X^\prime)$ for every $X^\prime$.

% Bandwidth is the main concern.
A simple way to construct a DPF is to let $\eval(K_0, \cdot)$ be a random
function and let $\eval(K_1, \cdot) = P_{X,Y}(\cdot) \xor \eval(K_0, \cdot)$.
%
This yields a 2-server PIR protocol that is information theoretically
$(2,1)$-private, but the length of the shares is exponential in $|X|$. Indeed,
one of the main goals of PIR systems is to minimize the communication bandwidth.
%
It is well known that polynomial-length encodings are possible. Even shorter
encodings are possible in the computational setting; Gilboa and Ishai \cite{dpf}
give the most bandwidth efficient, 2-share DPF scheme known, which achieves
polylogarithmic bandwidth and is secure against polynomial-time adversaries.

% Overview of Riposte.
\heading{Riposte.}
Distributed point function schemes are central to the design of Riposte
\cite{riposte}, a cryptosystem recently proposed by Henry Corrigan-Gibbs, Dan
Boneh, and David Mazi\`{e}res,
which allows clients to anonymously write messages to a database. It can be used
by a service provider to collect usage data, crash reports, and other metrics
without revealing to the data collector who sent the report.\footnote{ This
application assumes, crucially, that the report itself does not contain
personally identifiable information about the sender.} It can also be used to
facilitate anonymous communication by making the contents of the database
public, with the advantage of being much more scalable than mix- or DC-nets
\cite{mix-nets,dc-nets} while providing stronger defense against traffic analysis
than onion routing networks \cite{tor}.

A client initiates a \emph{write request} by mapping its message to a sequence
of~$\sharect$ \emph{write shares} and sends each to one of~$\sharect$ distinct
\emph{write servers}. When a write server receives a share, it updates its local
state.  After processing the requests of~$\cohortct$ different clients, it
outputs its state to the data consumer. Finally, the data consumer recovers the
set of messages written to the database by combining the states of each of the
write servers.

% Overview of the security properties of Riposte.
The designers specify two DPF schemes: one that is $(2,1)$-private and another
that is $(\sharect, \sharect-1)$-private for any $\sharect$. Their use in the
Riposte system is similar to PIS. Each server $i$ maintains a share of the
database modeled as an $\dblen$-vector $\dbst_i$ of $n$-bit strings, each
initially equal to $0^n$. To write a message $\msg \in \bits^n$ into the
database, the client executes $(X_1, \ldots X_\sharect) \getsr \gen(\str{\idx},
\msg)$, where $\str{\idx}$ is the encoding of a randomly chosen
$\idx\in[\dblen]$. When it receives its share, server $i$ updates its state by
letting $\dbst_i[\idx^\prime] = \dbst_i[\idx^\prime] \xor \eval(X_i,
\str{\idx^\prime})$ for every $\idx^\prime\in[\dblen]$. (Recall that
$P_{\str{\idx},\msg}(\str{\idx^\prime}) = \xor_{i=1}^\sharect \eval(X_i,
\str{\idx^\prime})$.) Suppose client 1 writes $\msg_1$ into row $\idx_1$ and
then client 2 writes $\msg_2$ into a different row $\idx_2$.  Given their shares
and the final states of every server, no coalition of at most $\sharect-1$
servers can link either message to its sender given only its key shares and the
outputs of the servers. Intuitively, this is because the view of the adversary
is identically distributed no matter what order the clients send their write
requests.\cpnote{I'm being hand-wavy about this claim.  \cite{riposte} give a
security notion for anonymity, but I don't think they show that security of the
DPF implies anonymity. This could be interesting to show rigurously in our
paper.}

% The need for write share verification.
\heading{Disruption resistance.}
The fact that many users write to a single database poses a problem not
considered in the PIS setting. By sending one or more of the write servers a
mal-formed write share, a malicious client, or network adversary who intercepts
the client's messages, can corrupt the database state.
%
For example, suppose we let $\eval(X_1, \cdot)$ be a random function and
$\eval(X_2, \cdot) = P_{X,Y}(\cdot) \xor \eval(X_1, \cdot)$ as in the simple
$(2,1)$-private scheme described above.
%
A malicious client could instead let $\eval(X_2, \cdot)$ be a random function
independent of $\eval(X_2, \cdot)$.
As a result, the combined state $\dbst[\idx^\prime] = \dbst_1[\idx^\prime] \xor
\dbst_2[\idx^\prime]$ will be indistiguishable from a random string for every
$\idx^\prime \in[\dblen]$, rendering the entire database unrecoverable.

It is therefore crucial in this setting that the write servers be able to verify
their shares are well-formed, meaning that they yield the point function of some
index~$\idx$ and message~$\msg$. Roughly speaking, the system is
\emph{disruption resistant} \cite{riposte} if the write servers engage in a
protocol (possibly with a third party \emph{auditing server}), which allows them
to detect malicious clients, but leaks no information about the inputs (even to
the auditor). It is assumed that each server faithfully executes the protocol.
Disrupting the protocol amounts to a denial-of-service attack \cite{riposte},
which a malicious server can always do anyway by corrupting its own state.
Since we cannot hope to defend against disrupting servers, we require only that
disrupting the protocol does not violate privacy.
The designers give protocols for verifying write requests for both the 2-server
and $\sharect$-server variants of Riposte. The former involves a non-colluding
auditor, thus achieving greater efficiency than the latter, which requires an
expensive multiparty computation.

% I'm not sure if this is what I want the paper to say.
\if{0}
% Security of their composed notions is unclear.
Their proofs are modular in the sense that the privacy property of the DPF
scheme is treated separately from the privacy property of the write share
verification protocol. However, it is not clear that the respective security
notions compose in a way that ensures end-to-end security of the client's write
request. The DPF adversary is active in the sense that a coalition of
malicious servers may act arbitrarily to violate security. On the other hand,
the privacy adversary in the verification protocol is semi-honest in the sense
that the servers execute the protocol faithfully. (In particular, they do not
collude.) Since the latter adversary is weaker than the former, there might be
share verification protocols that are private with respect to semi-honest
adversaries, but not active, colluding ones.

% Thesis.
It is not our contention that the protocols of \cite{riposte} are not end-to-end
secure; in fact, we find that they are. Rather, our thesis is that a unified
analytical framework is needed in order to avoid proposing protocols that do not
compose securely. In addition, we find that this approach yields simpler, more
efficient designs.
\fi

\heading{Our contributions.}
We endow DPF schemes with a property called \emph{verifiability}, which demands
that the composition of the DPF scheme with a write request verification protocol
be secure. We define the simulation-based notion of \cite{dpf,riposte} for
standard DPF schemes, which models an active coalition of malicious write
servers (PRIV1). We give a new notion, which extends this model to include
the execution of the verification protocol (PRIV2). To accomplish this, we give
the adversary access to oracles, which provide it with its view of the
protocol's execution. In particular, we assume the adversary only has access to
messages sent to colluding servers.

\noindent\hl{Here's a list of things we can consider doing:}
\begin{enumerate}
  \item Prove the protocols of \cite{riposte} suffice for PRIV2, but show that
    more efficient protocols are possible in our new framework. In particular,
    we extend the execution model by allowing the write servers to have private
    keys. This immediately yields a more efficient variant of their
    2-server+auditor protocol by using a PRF with a shared key instead of
    telephone coin-flipping to establish a shared set of pairwise independent
    hash functions.

  \item 1-round, 2-serer+auditor protocol in a weaker trust model. (An auditor
    may collude with one write server.) What I'm thinking is that clients append
    $0^t$ to their message before applying $\gen$. The write server computes the
    ``tag'' from the write share by applying a PRF to the string resulting from
    successively concatenating the last $t$ bits of $Y_\idx = \eval(X_i, \idx)$
    for each $\idx$ from 1 to $\dblen$. The auditor just makes sure that the tags
    match. I'm not sure what properties are required of the PRG. Using the
    scheme of \cite{dpf}, this would yield a scheme that is much more efficient
    than Riposte in terms of communication complexity and bandwidth, yet
    functions in a weaker trust model. (Note that this answers two of their open
    questions in the affirmative.)

  \item 1-round, $2^t$-server+auditor protocol where the auditor is colluding,
    using the methods of \cite{dpf-multi-server}? This yields $(2^t,
    t)$-privacy.

  \item Reduce the round complexity of their $\sharect$-server protocol? Theirs
    is $(\sharect, \sharect-1)$-private. (This addresses one of their open
    questions.)

  \item The protocol in (1) actually achieves security against a stronger
    adversary, one that intercepts all messages sent between servers in the
    protocol. Consider the application where Google wants to collect crash
    reports. It operates each of the servers, but is also likely on path between
    each of the servers. (It probably operates the network infrastructure!)
    The protocols of \cite{riposte} DO NOT achieve this stronger notion,
    however.
\end{enumerate}


\section{Pseudorandom number generators}
%
%
%
\label{sec:prng}
This syntax provides syntax and security notions for pseudorandom number
generators \emph{with inputs}, as first formalized by~\cite{barak2005model}. It
also borrows ideas from~\cite{dodis2013security,shrimpton2015provable}.

\begin{definition}[Entropy source]\rm
  An \emph{entropy source} is a randomized algorithm~$\dist$ with no inputs and
  that outputs a string.
  %
  \cpnote{This syntax follows \cite{barak2005model}, but diverges
  from~\cite{dodis2013security,shrimpton2015provable}, where the entropy source
  may be stateful.}
  %
  We say that~$\dist$ has min-entropy~$\mu$ if for every
  $x \in \bits^*$, it holds that
  $
  \Prob{y \getsr \dist\colon x = y} \leq 2^{-\mu}
  $. \dqed
\end{definition}
%
\cpnote{We could extend this notion along the lines of
\cite{bellare2009hedged,boldyreva2017real} so that $\dist$ outputs a vector of
strings.}
%
\cpnote{Define the maximum output length?}
%
\cpnote{Could model \emph{side information} about the output.}

\begin{definition}[PRNG]\rm
  A \emph{pseudorandom number generation} scheme, $\prng$, is a triple of
  deterministic algorithms $(\init, \add, \get)$ defined as follows:
  %
  \begin{itemize}
    \item $\init$ is the initialization algorithm. It takes as input a
      string~$\stsel$, called the \emph{initializer}, and outputs a string~$\st$
      called the \emph{state}.
      %
      This is denoted $\st \gets \init(\stsel)$.

    \item $\add$ takes a string~$\sel$ called the \emph{selector},
      %
      \cpnote{Nomenclature is borrowed from~\cite{bellare2016nonce}. There might
      be a better name.}
      %
      and the state and outputs the updated state.
      %
      This is denoted $\st \gets \add(\sel, \st)$.

    \item $\get$ takes as input an integer $\rho \geq 0$
      %
      \cpnote{This parameter does not appear in earlier work.}
      %
      and the state and
      outputs a string $\coins \in \bits^\rho$ and the updated state.
      %
      This is denoted $(\coins, \st) \gets \get(\coinslen, \st)$.
      \dqed
  \end{itemize}
\end{definition}

We define four notions of security PRNG schemes in Figure~\ref{fig:prng-sec}.
%
\indfwd and \indbwd capture \emph{indistingushibility} of the output of
$\prng.\get(\cdot)$ from random, the first in the \emph{forward-secure} sense,
and the second in the \emph{backward-secure} sense.
%
Roughly speaking, \emph{forward security} demands that if the state is exposed
to the adversary, then all prior uses of the PRNG remain secure, and
\emph{backward security} demands that if the state is exposed, then future uses
of the PRNG are secure as long as the state is refreshed.
%
\upfwd and \upbwd capture only \emph{unpredictability} of the output of
$\prng.\get(\cdot)$.
%
Something we'll need to figure out is what restrictions we need to make
on the source(s) of entropy for these notions to be achievable. For example, in
the \indfwd game, if we make no restrictions on the $\INITRO$-queries, then
there is an easy distinguishing attack.

\begin{figure}[t]
  \newcommand{\coll}{\flagfont{coll}}
  \twoColsTwoRows{0.48}
  {
    \experimentv{$\Exp{\indfwdX{b}}_\prng(\advA)$}\\[2pt]
      $\stout \gets \true$\\
      $b' \getsr \advA^{\INITRO,\EXPO,\ADDO,\GETINDO}$\\
      return $b'$
  }
  {
    \experimentv{$\Exp{\indbwdX{b}}_\prng(\advA)$}\\[2pt]
      $\stout \gets \true$\\
      $b' \getsr \advA^{\INITO,\ADDRO,\GETINDO}$\\
      return $b'$
  }
  {
    \experimentv{$\Exp{\upfwd}_\prng(\advA)$}\\[2pt]
      $\stout \gets \true$; $\coll \gets \false$;
      $\setX \gets \emptyset$\\
      $\coins \getsr \advA^{\INITRO,\EXPO,\ADDO,\GETUPO}$\\
      return $\coll \OR (\coins \in \setX)$
  }
  {
    \experimentv{$\Exp{\upbwd}_\prng(\advA)$}\\[2pt]
      $\stout \gets \true$; $\coll \gets \false$;
      $\setX \gets \emptyset$\\
      $\coins \getsr \advA^{\INITO,\ADDRO,\GETUPO}$\\
      return $\coll \OR (\coins \in \setX)$
  }
  \twoColsNoDivide{0.48}
  {
    \oraclev{$\INITRO(\dist)$}\\[2pt]
      $\stout \gets \false$;
      $\stsel \getsr \dist$;
      $\st \gets \prng.\init(\stsel)$
    \\[6pt]
    \oraclev{$\ADDRO(\dist)$}\\[2pt]
      $\stout \gets \false$;
      $\sel \getsr \dist$;
      $\st \gets \prng.\add(\st, \sel)$
    \\[6pt]
    \oraclev{$\INITO(\stsel)$}\\[2pt]
      $\stout \gets \true$;
      $\st \gets \prng.\init(\stsel)$
    \\[6pt]
    \oraclev{$\ADDO(\sel)$}\\[2pt]
      $\st \gets \prng.\add(\st, \sel)$
    \\[6pt]
    \oraclev{$\EXPO(\,)$}\\[2pt]
      $\stout \gets \true$;
      return $\st$
  }
  {
    \oraclev{$\GETINDO(\coinslen)$}\\[2pt]
      if $\stout = \true$ then return $\bot$\\
      $(\coins_1, \st) \gets \prng.\get(\coinslen, \st)$\\
      $\coins_0 \getsr \bits^\coinslen$\\
      return $\coins_b$
    \\[6pt]
    \oraclev{$\GETUPO(\coinslen)$}\\[2pt]
      if $\stout = \true$ then return $\bot$\\
      $(\coins, \st) \gets \prng.\get(\coinslen, \st)$\\
      if $\coins \in \setX$ then $\coll \gets \true$\\
      $\setX \gets \setX \union \{ \coins \}$\\
      return~$1$
  }
  \caption{\textbf{Top:} Security notions for PRNG schemes.
  %
  \textbf{Bottom:} Oracles for the security notions.}
  \vspace{6pt}\hrule
  \label{fig:prng-sec}
\end{figure}


\section{Public-key encryption with associated data}
\label{sec:pkead}
\cptodo{Define sources.}
\cptodo{The \fwdpke and \bwdpke notions do not guarantee privacy of associated
data, which is needed for the application to AKE. (See Section~\ref{sec:ake}.
How about passing $\ad_0$ and $\ad_1$ to the oracle in the game?}

\begin{definition}[PKEAD]\rm
  A \emph{public-key encryption scheme with associated data} is a 4-tuple of
  algorithms $\pkead = (\Kg, \Enc.\init, \Enc, \Dec)$ defined as follows:
  \begin{itemize}
    \item $\Kg$ is the randomized key generation algorithm that outputs a
      public/private key pair $(\pk, \sk)$.
      %
      Its execution is denoted $(\pk, \sk) \getsr \Kg$.

    \item $\Enc.\init$ is the deterministic encryption initialization
      algorithm. It takes as input an initializer~$\stsel$ and outputs the state.
      %
      Its execution is denoted $\st \gets \Enc.\init(\stsel)$.

    \item $\Enc$ is the deterministic encryption algorithm that takes as input the
      public key~$\pk$, a triple of strings $(\nonce, \ad, \msg)$, called the
      nonce, associated data, and plaintext respectively, and the state~$\st$, and
      outputs~$\cipher$, either a string or~$\bot$, and the updated state.
      %
      This is denoted $(\cipher, \st) \gets \Enc_\pk^{\nonce,\ad}(\msg, \st)$.

    \item $\Dec$ is the determinstic decryption that takes as input the secret
      key~$\sk$ and $(\nonce, \ad, \cipher)$ and outputs~$\msg$, either the
      message or $\bot$.
      %
      This is denoted $\msg \gets \Dec_\sk^{\nonce,\ad}(\cipher)$.
      \dqed
  \end{itemize}
\end{definition}
%
\cpnote{How does this syntax compare to stateful encryption as already defined
in the literature? Tom says that typically decryption is stateful, too.}

We define two notions of security for PKEAD schemes in
Figure~\ref{fig:pkead-sec}.
%
The first, \fwdpke, demands indistinguishibility and forward security.
The second, \bwdpke, demands indistinguishibility and backward security. In the
latter, since the state is completely exposed prior to encryption, the
$\ENCO$-oracle takes a \emph{distribution} on the nonce, associated data, and
messages as input. This way any entropy in this distribution can be leveraged
for security.

\begin{figure}[t]
  \newcommand{\rdy}{\flagfont{rdy}}
  \twoCols{0.48}
  {
    \experimentv{$\Exp{\fwdpkeX{b}}_{\pkead}(\advA)$}\\[2pt]
      $\rdy \gets \false$; $\setC \gets \emptyset$\\
      $(\pk, \sk) \getsr \pkead.\Kg$\\
      $b' \getsr \advA^{\ENCO,\DECO,\INITRO,\EXPO}(\pk)$\\
      return $b'$
    \\[9pt]
    \oraclev{$\ENCO(\nonce, \ad, \msg_0, \msg_1)$}\\[2pt]
      if $\rdy=\false \OR |\msg_0|\ne|\msg_1|$ then return $\bot$\\
      $(\cipher, \st) \gets \pkead.\Enc_\pk^{\nonce,\ad}(\msg_b, \st)$\\
      $\setC \gets \setC \union \{(\nonce, \ad, \cipher)\}$\\
      return $\cipher$
    \\[6pt]
    \oraclev{$\DECO(\nonce, \ad, \cipher)$}\\[2pt]
      if $(\nonce, \ad, \cipher) \in \setC$ then return $\bot$\\
      return $\pkead.\Dec_\sk^{\nonce,\ad}(\cipher)$
    \\[6pt]
    \oraclev{$\INITRO(\dist)$}\\[2pt]
      $\rdy \gets \true$;
      $\stsel \getsr \dist$;
      $\st \gets \Enc.\init(\stsel)$
    \\[6pt]
    \oraclev{$\EXPO(\,)$}\\[2pt]
      $\rdy \gets \false$; return $\st$
  }
  {
    \experimentv{$\Exp{\bwdpkeX{b}}_{\pkead}(\advA_1, \advA_2)$}\\[2pt]
      $\rdy \gets \false$;
      $\setC \gets \emptyset$\\
      $(\pk, \sk) \getsr \pkead.\Kg$\\
      $\st \getsr \advA_1^{\ENCO,\DECO,\INITO}$\\
      $b' \getsr \advA_2(\pk,\st)$\\
      return $b'$
    \\[6pt]
    \oraclev{$\ENCO(\srcM)$}\\[2pt]
      if $\rdy = \false$ then return $\bot$\\
      $(\vnonce, \vad, \vmsg_0, \vmsg_1) \getsr \srcM$\\
      $(\vcipher, \st) \gets \pkead.\Enc_\pk^{\vnonce,\vad}(\vmsg_b, \st)$\\
      for each $i \gets 1$ to $|\vcipher|$ do\\
        \tab $\setC \gets \setC \union \{(\vnonce_i, \vad_i, \vcipher_i\}$\\
      return $\vcipher$
    \\[6pt]
    \oraclev{$\DECO(\nonce, \ad, \cipher)$}\\[2pt]
      if $(\nonce, \ad, \cipher) \in \setC$ then return $\bot$\\
      return $\pkead.\Dec_\sk^{\nonce,\ad}(\cipher)$
    \\[6pt]
    \oraclev{$\INITO(\stsel)$}\\[2pt]
      $\rdy \gets \true$;
      $\st \gets \Enc.\init(\stsel)$
  }
  \caption{Security notion for PKEAD schemes.}
  \label{fig:pkead-sec}
  \vspace{6pt}\hrule
\end{figure}

\ignore{
  \heading{Constructions.}
  %
  \newcommand{\kem}{\schemefont{kem}}
  \newcommand{\calE}{\mathcal{E}}
  \newcommand{\calD}{\mathcal{D}}
  \cpnote{KEM (key encapsulation mechanism) + symmetric AEAD + PRNG:
    \begin{itemize}
      \item $\Kg$: $(\pk, \sk) \getsr \kem.\Kg$; return $(\pk, \sk)$
      \item $\Enc.\init$: $\st \getsr \prng.\init$; return $\st$
      \item $\Enc_\pk^{\nonce,\ad}(\msg, \st)$:
        $(K, \st) \gets \prng(\str(\nonce, \ad, \msg), k, \st)$;
        $\cipher \gets \calE_K^{\nonce,\ad}(\msg)$;
        $X \gets \kem.\Enc_\pk(K)$;
        return $\str(X, \cipher)$
      \item $\Dec_\sk^{\nonce,\ad}(\str(X, \cipher))$:
        $K \gets \kem.\Dec_\sk(X)$;
        $\msg \gets \calD_K^{\nonce,\ad}(\cipher)$;
        return $\msg$
    \end{itemize}
    (I'm not sure if I got the KEM syntax quite right.) We should aim to prove
    that $\pkead$ is exposure-resilient if and only if $\prng$ is
    exposure-resilient.
  }
  %
  \cpnote{I'd also llke to lo look into an OAEP-like construction.}
}


\section{The killer app: 0-RTT, password-based AKE}
%
%
%
%
\label{sec:ake}
A client has an identity~$I$ (say, her email address) and a password~$W$ and
would like to authenticate herself to---and exchange a key with---a server in
possession of a public/private key pair $(\pk,\sk)$.
%
Let $h\geq0$ be an integer and $H \colon \bits^* \to \bits^h$ be a cryptographic
hash function. Suppose that, out-of-band, the client is furnished with~$\pk$ and
the server with~$I$ and $\ad= H(I \cat W)$.
%
Consider the following key exchange protocol for session number $\nonce$:
\begin{itemize}
  \item $\cli^{\,\pk,I,W,\nonce}$:
    Run $K \getsr \setK$;
    $\ad \gets H(I \cat W)$;
    $(\cipher, \st) \gets \Enc_\pk^{\nonce,\ad}(K, \st)$; and
    send $(I, \nonce, \cipher)$ to \srv.

  \item $\srv^{\,\sk,U}$ on input $(I, \nonce', \cipher)$:
    Look up~$(\ad, \nonce) \gets U[I]$;
    if $\nonce \geq \nonce'$ then reject;
    run $K \gets \Dec_\sk^{\nonce',\ad}(\cipher)$;
    if $K \ne \bot$, then let $U[I] \gets (\ad, \nonce')$ and accept; otherwise reject.
\end{itemize}
%
An adversary in possession of~$\ad$ can easily impersonate the client. Hence
this simple protocol works only if the server is never compromised.
%
\cpnote{Yet, in the setting where the server has a public/private key pair and
the client just has just a password, is there any solution that does any better
when the server is compromised? As far as I can tell, the protocol
of~\cite{bellare2000authenticated} doesn't fair much better if the server is
compromised. (Although this paper might be a bit out-dated ... I don't know this
literature very well.) I think the only way to do better is if the client also
has a public/private key pair.}
%
Rotating the password is similarly easy as pie:
%
\begin{itemize}
  \item $\cli^{\,\pk,I,W,W',\nonce}$:
    Run $\ad \gets H(I \cat W)$; $\ad' \gets H(I \cat W')$;
    $(\cipher, \st) \gets \Enc_\pk^{\nonce,\ad}(\ad', \st)$; and
    send $(I, \nonce, \cipher)$ to \srv.

  \item $\srv^{\,\sk,U}$ on input $(I, \nonce', \cipher)$:
    Look up~$(\ad, \nonce) \gets U[I]$;
    if $\nonce \geq \nonce'$ then reject;
    run $\ad' \gets \Dec_\sk^{\nonce',\ad}(\cipher)$; and
    if $\ad' \ne \bot$, then let $U[I] \gets (\ad', \nonce')$ and accept;
    otherwise reject.
\end{itemize}


\section{Digital signatures}
%
%
%
Following~\cite{bellare2016nonce} we extend our framework to digital signatures.
\begin{definition}[DS]\rm
  A digital signature scheme~$\ds$ is a 4-tuple of algorithms $(\Kg,\\ \Sgn.\init, \Sgn,
  \Vfy)$ defined as follows:
  %
  \begin{itemize}
    \item $\Kg$ is the randomized key generation algorithm.

    \item $\Sgn.\init$ is the deterministic signing state initialization
      algorithm. It takes as input an initializer~$\stsel$ and returns the
      state. This is denoted $\st \gets \Sgn.\init(\stsel)$.

    \item $\Sgn$ is the deterministic signing algorithm, taking as input the
      secret key~$\sk$, the message~$\msg$, and the current state, and
      outputting~$\sig$, either the signature or~$\bot$, and the updated state.
      This is denoted $(\sig, \st) \gets \Sgn_\sk(\msg, \st)$.

    \item $\Vfy$ is the deterministic verification algorithm, taking as input the
      public key~$\pk$, the message~$\msg$, and the signature~$\sig$ and
      outputting $v$, either a bit indicating whether the signature is valid,
      or~$\bot$. This is denoted $v \gets \Vfy_\pk(\msg, \sig)$.
  \end{itemize}
  %
  \cpnote{$\Sgn$ and $\Vfy$ could use a nonce, but I don't see what the practical
  benefit would be. Though, doesn't DJB's Poly1305 MAC use nonce? What's the
  reasonf or this?}
  \dqed
\end{definition}

\heading{\ufsig.}
%
Unforgeability of signature schemes is defined in Figure~\ref{fig:ufsig}. As
usual, we consider a setting in which the signing state is eventually exposed to
the adversary.
%
A reasonable criticism of this notion is that if the adversary is able to
exfiltrate the signing state, then it ought to be able to exfiltrate the signing
key. To this point, \cite{bellare2016nonce} argues that the signer might take
extra care in storing the key, but choose to store the signing state (i.e., the
seed and nonce generation state in their setting) in a less secure part of the
signer's system. For example, the long-term signing key might be stored in an
HSM (``hardware security module'') and the short-term signing state in main
memory.
%
\cptodo{Look at attestation/signing APIs for SGX. Are the signing algorithms
randomized, and if so, where do the coins come from?}

\begin{figure}[t]
  \twoColsNoDivide{0.48}
  {
    \experimentv{$\Exp{\ufsig}_{\ds}(\advA,\dist)$}\\[2pt]
      $\setQ \gets \emptyset$\\
      $\stsel \getsr \dist$;
      $\st \gets \Sgn.\init(\stsel)$\\
      $(\pk, \sk) \getsr \Kg$\\
      $(\msg, \sig) \getsr \advA^{\OO}(\pk)$\\
      return $\Vfy_\pk(\msg, \sig)$ and $\msg \not\in \setQ$
  }
  {
    \oraclev{$\SGNO(\msg)$}\\[2pt]
      $\setQ \gets \setQ \union \{\msg\}$\\
      $(\sig, \st) \gets \Sgn_\sk(\msg, \st)$\\
      return $\sig$
    \\[6pt]
    \oraclev{$\INITO(\stsel)$}\\[2pt]
      $\st \gets \pkead.\Enc.\init(\stsel)$
    \\[6pt]
    \oraclev{$\EXPO$}\\[2pt]
      return $\st$
  }
  \caption{Security notion for digital signature schemes. Let $\OO =
  (\INITO,\EXPO,\SGNO)$.}
  \label{fig:ufsig}
  \vspace{6pt}\hrule
\end{figure}

\heading{Constructions.}
%
\cpnote{The solution that~\cite{bellare2016nonce} had in mind will work here.
That is, use a PRNG to generate coins for a standard, randomized digital
signature scheme.}


\section{0-RTT 2F-AKE}
%
%
%
\newcommand{\ek}{\varfont{ek}}
\newcommand{\dk}{\varfont{dk}}
\newcommand{\vk}{\varfont{vk}}
\label{sec:ake2f}
Let $H\colon\bits^* \to \bits^h$ be a hash function.
%
The client has a signing key~$\sk$, say, stored on a
YubiKey,\footnote{\url{https://yubico.com}} and the server has the
corresponding verifying key~$\vk$.
%
The client also has an identity~$I$ and a password~$W$, and the server has~$I$
and $\ad = H(I \cat W)$.
%
The server has a decrypting key~$\dk$ and the client has the corresponding
encrypting key~$\ek$.
%
Consider the following key exchange protocol for session number~$\nonce$:
\begin{itemize}
  \item $\cli^{\,\sk,\ek,I,W,\nonce}$:
    Run $K \getsr \setK$;
    $\ad \gets H(I \cat W)$;
    $(\cipher, \st_E) \gets \Enc_\ek^{\nonce,\ad}(K, \st_E)$;
    $(\sig, \st_S) \gets \Sgn_\sk(\cipher, \st_S)$; and
    send $(I, \nonce, \cipher, \sig)$ to \srv.

  \item $\srv^{,\vk,\dk,U}$ on input $(I, \nonce', \cipher, \sig)$:
    Look up $(\ad, \nonce) \gets U[I]$;
    if $\nonce \geq \nonce'$ then reject;
    if $\Vfy_\vk(C, T)=\false$ then reject;
    run $K \gets \Dec_\dk^{\nonce',\ad}(\cipher)$;
    if $K \ne \bot$ then let $U[I] \gets (\ad, \nonce')$ and accept; otherwise
    reject.
\end{itemize}
%
This is an improvement on the protocol in Section~\ref{sec:ake} in that if the
server is compromised, then adversary also needs to exfiltrate the client's
signing key in order to impersonate her.
%
\cpnote{Alternatively we could express this protocol in terms of a dedicated
\emph{authenticated encryption} scheme, where both sender and receiver has a
secret key.}


\section{Notes}
\cpnote{Mihir pointed out that the MM attack setting is unnecessarily weak, in
that the randomness source could be stateful. We will \emph{not} address this in
the full version of~\cite{boldyreva2017real}, but will mention it here in
related work.}

\heading{Mihir's feedback on \cite{boldyreva2017real}.}
%
\newenvironment{displayquote}
{ \indent
  \footnotesize\color{gray}
  \begin{tabular}{|@{\hspace{4pt}}p{10cm}}
}
{
  \end{tabular}\\[6pt]
}

\begin{displayquote}
  Your motivation was the way crypto libraries treat encryption and RNGs. Why not
  address that directly? This means the model allows a stateful algorithm~$R$
  that via $(r,s) \getsr R(s)$ produces coins~$r$ while updating its state
  to~$s$. We could have oracles that allow the caller to change the state s, or
  mix something into it, reflecting what you say happens in the libraries. A
  simple definition in this model is a game just like IND-CPA/CCA (messages, not
  message vectors, and no entropy requirement on these) except that coins are
  created, for each message, via $(r,s) \getsr R(s)$. This is stronger than
  MM-CPA/CCA because coins can be related even across adaptive queries. I would
  guess/hope that OAEP continues to be secure. Then one can also consider how
  message entropy can be factored in.
\end{displayquote}
%
\cpnote{}
This is addressed in Sections~\ref{sec:prng} and~\ref{sec:pkead}, but with a few
differences.
%
One, $R$ takes as input a selector~$\sel$. This idea is inspired from the
\emph{nonce selector} of~\cite{bellare2016nonce}.
%
Two, $R$ is deterministic instead of randomized. (It has a randomized
initialization algorithm.)
%
Three, encryption is stateful; I'm thinking of a stateful PRNG as away to
instantiate the encryption scheme. This abstraction is in keeping with the theme
of API driven cryptography.

\begin{displayquote}
  DSA/Schnorr/PSS are randomized signature schemes but in practice people like
  to implement the first two, at least, by deterministically deriving coins by
  hashing the secret key and message. This is analyzed
  in~\cite{bellare2016nonce}. Some schemes like Ed25519 directly implement it.
  But what about signature interfaces provided by the libraries you surveyed? Do
  they let the signer pick the coins?  If not, what do you do? Could we look at
  the libraries and see?
\end{displayquote}
%
\cptodo{Look at digital signatures offered by OpenSSL, PyCrypto, golang/crypto,
etc. Also, what are the interfaces like for HSMs, e.g. YubiKey and SGX? Note
that ECDSA requires a random nonce.}

\begin{displayquote}
  I'm dubious about what MM for OAEP buys, for the following reason. My sense is
  that RNGs $R$ would usually output both~$r$ and~$s$ to be results of applying some
  hash function to some stuff that includes~$s$. This means that either (1) $r$ is
  (indistinguishable from) random, or (2) $r$ is predictable. If (1), MM is not
  needed. If (2), it does not help. In other words my worry is that MM addresses
  the case that~$r$ is unpredictable but not random, and this case does not arise,
  because of the way RNGs work. To assess stuff like this it would be good to
  know more about how the RNGs actually work.
\end{displayquote}
%

\begin{displayquote}
  Another question is, what about using nonce-based PKE as per
  \cite{bellare2016nonce}? One issue is that the sender must maintain state. Is
  that a problem? I'd imagine not, since there is so much static stuff it needs
  to maintain anyway, like its secret key or other people's public keys, but I'd
  be interested to know how this works for implementation. The other issue is
  that the solutions of \cite{bellare2016nonce} again fail to conform to crypto
  library interfaces the way you want them to, so one might ask if there are
  definitions or schemes for nonce-based PKE that are crypto-library friendly.
\end{displayquote}
%
\cpnote{}
See the introduction and Section~\ref{sec:pkead}.

\begin{displayquote}
  A more ambitious question is the following. All definitions so far give no
  security unless there is enough entropy in something (messages, randomness,
  both). In practice, the way RNGs work, one might at some point have low
  entropy (and encryption is insecure) but then entropy returns because the
  entropy pool is refurbished. This would mean encryption security returns. Can
  we give definitions that capture this type of self-healing property, where,
  for some messages, encryption is not secure, but then it becomes secure again?
  It seems to be what really happens.
\end{displayquote}
%

\begin{displayquote}
  No skin off my back, but I can see Shoup or others in the community unhappy
  about your rebranding of labels as AD. If you must change the name, I'd first
  clearly say that the common term is labels, as introduced by Shoup, and then
  say you are changing the name. After all, even the standards use the term
  labels.  Right now it looks like you claim to introduce this concept and then
  as an afterthought say that it existed under another name.
\end{displayquote}


\bibliography{pk}
\bibliographystyle{alpha}

\end{document}
