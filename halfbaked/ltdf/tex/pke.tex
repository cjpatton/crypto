\label{sec-pke}

Here describe how to construct \indcca secure public-key encryption from lossy
trapdoor functions. We present the scheme suggested by Peikert and Waters, which
composes an LTDF scheme, an ABO TDF scheme, and a strongly-unforgeable, one-time
digital signature scheme~\cite[Section 4.3]{pw08}.
%
First, we give the syntax and security for public-key encryption.

A public-key encryption is a triple of algorithms $\pkescheme = (\kg, \enc, \dec)$,
the first and second being probabilistic and the third being deterministic.
%
The key generation algorithm~$\kg$ outputs a pair of strings $(\pk, \sk)$.
%
The encryption algorithm takes as input the public key~$\pk$, a message~$\ptxt
\in \bits^*$, and outputs a ciphertext~$\ctxt \in \bits^*$ or the distinguished
symbol~$\bot$.
%
The decryption algorithm takes as input the secret key~$\sk$, a
ciphertext~$\ctxt \in \bits^*$, and outputs the corresponding message~$\ptxt \in
\bits^*$ or~$\bot$.
%
Correctness demands that for every $(\pk, \sk) \in [\kg]$, every message~$\ptxt
\in\bits^*$, and every sequence of coins~$\coins\in\bits^*$, if $\enc_\pk(\ptxt
\,;\coins) \ne \bot$, then $\dec_\sk(\enc_\pk(\ptxt \,;\coins)) = \ptxt$.

The standard \indcca notion for public-key encryption is given in
Figure~\ref{fig-indcca}. Let
$
  \Adv{\indcca}_\pkescheme(\advA) = 2\cdot
  \prob{\Exp{\indcca}_\pkescheme(\advA)} - 1.
$
%
\begin{figure}[t]
  \threeColsNoDivides{0.28}{0.28}{0.28}
  {
    \underline{$\Exp{\indcca}_\pkescheme(\advA)$}\\[2pt]
      $b \getsr \bits$; $Q \gets \emptyset$\\
      $(\pk, \sk) \getsr \kg$\\
      $b^\prime \getsr \advA^{\lro,\deco}(\pk)$\\
      return $b=b^\prime$
    }
    {
    \underline{$\lro(\ptxt_0, \ptxt_1)$}\\[2pt]
      if $|\ptxt_0| \ne |\ptxt_1|$ then\\
      \tab return $\bot$\\
      $\ctxt \getsr \enc_\pk(\ptxt_b)$\\
      $Q \gets Q \union \{\ctxt\}$\\
      return $\ctxt$
    }
    {
    \underline{$\deco(\ctxt)$}\\[2pt]
      if $\ctxt \in Q$ then\\
      \tab return $\bot$\\
      return $\dec_\sk(\ctxt)$
  }
  \caption{Indistinguishibility under chosen-ciphertext attacks.}
  \label{fig-indcca}
  \vspace{6pt}
  \hrule
\end{figure}

\subsection{Hashing}
\label{sec-extractors}

A family of hash functions is given by a pair of algorithms $(\hashgen, \hashf)$
with associated domain~$X$ and range $Y$.
%
The first algorithm is probabilistic and the second is deterministic.
%
Algorithm $\hashgen$ outputs a string~$h$, and $\hashf(h, \cdot) : X \to
Y$ is a function for each such output.
%
We say that $(\hashgen, \hashf)$ induces a \emph{pairwise-independent} family of
hash functions if for every distinct $x_0,x_1 \in X$ and every distinct $y_0,
y_1 \in Y$, it holds that $ \prob{ h \getsr \hashgen : \hashf(h,x_0) =
y_0 \AND \hashf(h,x_1) = y_1 } \leq {1}/{|Y|^2}.$
%
The family is \emph{universal} if for every distinct $x_0,x_1 \in X$, it holds
that $ \prob{ h \getsr \hashgen : \hashf(h, x_0) = \hashf(h, x_1) }
\leq {1}/{|Y|}.  $

Note that being pairwise-independent implies that~$(\hashgen, \hashf)$ is
universal, but the converse is not true.
%
Also, Peikert and Waters remark that that $\hashf(h,\cdot)$ is a hardcore
function for an LTDF's injective mode~\cite{pw08}.

A \emph{strong randomness extractor}~\cite{shaltiel02} is used to derive uniform
random bits from a weak randomness source. For our purposes, a family of
pairwise-independent hash functions will suffice. We give a couple lemmas
from~\cite{drs04} that will be useful.
%
Let $X,Y$ be random variables over the same countable domain~$D$.
%
The \emph{statistical distance} between $X$ and $Y$, denoted $\statdist(X,Y)$,
is
\[
  \frac{1}{2} \sum_{v\in D} | \prob{X=v} - \prob{Y=v}|.
\]
%
The \emph{min-entropy} of $X$ is $\minent(X) = -\log(\max_{x\in D}
\prob{X=x})$, where $\log$ is base-2 here and below.
%
The \emph{average min-entropy} of $X$ given $Y$ is, denoted $\avgminent(X
\given Y)$, is
\[
  - \log \sum_{y\in D} \prob{Y = y} \cdot 2^{-\minent(X \given Y=y)}
\]
%
Intuitively, average min-entropy corresponds to the optimal probability of
guessing~$X$ given knowledge of $Y$.

\begin{lemma}\label{lemma1}
  Let $X,Y,Z$ be random variables where $Y$ has $2^r$ possible values.
  Then $\avgminent(X \given Y,Z) \ge \minent(X \given Z) - r$.
\end{lemma}

\begin{lemma}\label{lemma2}
  Let $X,Y$ be random variables where $X \in \bits^n$ and $\avgminent(X
  \given Y) \ge k$.
  %
  Let $U_\ell$ denote a uniform-random, $\ell$-bit string.
  %
  Let $(\hashgen, \hashf)$ be hashing scheme inducing a family of
  pairwise-independent hash functions with domain $\bits^n$ and range
  $\bits^\ell$. Let $h \getsr \hashgen$. Then
  \[
    \statdist((Y, h, \hashf(h, X)), (Y, h, U_\ell)) \le \epsilon
  \]
  if $\ell \leq k - 2 \log(1/\epsilon)$.
\end{lemma}


%%%%%%%%%%%%%%%%%%%%%%%%%%%%%%%%%%%%%%%%%%%%%%%%%%%%%%%%%%%%%%%%%%%%%%%%%%%%%%%
\subsection{Signing}
\begin{figure}[t]
  \twoColsNoDivide{0.40}
  {
    \underline{$\Exp{\auth}_\sigscheme(\advA)$}\\[2pt]
      $q \gets 0$; $(\vk, \xk) \getsr \siggen$\\
      $(\msg^\prime, \sig^\prime) \getsr \advA^{\sigo}(\vk)$\\
      $r \gets \verify_\vk(\msg^\prime, \sig^\prime)$\\
      return $r \AND (\msg, \sig) \ne (\msg^\prime, \sig^\prime)$
  }
  {
    \underline{$\sigo(\msg)$}\\[2pt]
      if $q = 1$ then return $\bot$\\
      $q \gets 1$;
      $\sig \getsr \sign_\xk(\msg)$\\
      return $\sig$
  }
  \caption{Strongly unforgeable, one-time signatures.}
  \label{fig-auth}
  \vspace{6pt}
  \hrule
\end{figure}
A signature scheme is a triple of algorithms $\sigscheme =
(\siggen,\sign,\verify)$, the first and second being probabilistic and the
third being deterministic.
%
The key generation algorithm $\siggen$ outputs a pair of strings $(\vk, \xk)$.
%
On input of signing key~$\xk$ and a string~$\msg$, the signing algorithm
$\sign$ outputs a string~$\sig$, called the signature, or~$\bot$.
%
On input of verifying key~$\vk$, string~$\msg$, and signature~$\sig$, the
verification algorithm outputs one of $\bits \union \{\bot\}$.
%
Correctness requires that for every $(\vk,\xk) \in [\siggen]$ and every $\msg
\in \bits^*$ it holds that
\[
  \prob{ \sig \getsr \sign_\xk(\msg) : \sig \ne \bot \implies \verify_\vk(
  \msg, \sig) =1 } = 1.
\]
The notion of a strongly unforgeable, one-time signature scheme is defined in
Figure~\ref{fig-auth}. We define the advantage of adversary~$\advA$ in forging
against~$\sigscheme$ as
$
  \Adv{\auth}_\sigscheme(\advA) = \prob{\Exp{\auth}_\sigscheme(\advA)}.
$


\subsection{The scheme}
In Figure~\ref{fig-proto} we exhibit a public-key encryption scheme constructed
from an LTDF scheme, an ABO TDF scheme, a pairwise-independent hashing scheme,
and a strongly unforgeable, one-time signature scheme.
%
\begin{figure}[t]
  \threeColsNoDivides{0.24}{0.32}{0.40}
  {
    \underline{$\kg$}\\[2pt]
      $h \getsr \hashgen$\\
      $(f, t) \getsr \losgen(1)$\\
      $(g, u) \getsr \abogen(0^v)$\\
      $\pk \gets \str{f, g, h}$\\
      $\sk \gets \str{t, u, \pk}$\\
      return $(\pk, \sk)$
  }
  {
    \underline{$\enc_\pk(\ptxt)$}\\[2pt]
      $\str{f, g, h} \gets \pk$\\
      $(\vk, \xk) \getsr \siggen$\\
      $\coins \getsr \bits^n$\\
      $x \gets \losf(f,\coins)$\\
      $y \gets \hashf(h,\coins) \xor \ptxt$\\
      $z \gets \abof(g, \vk, \coins)$\\
      $\sig \getsr \sign_\xk(\str{x,y,z})$\\
      return $\str{x,y,z,\vk,\sig}$
  }
  {
    \underline{$\dec_\sk(\str{x,y,z,\vk,\sig})$}\\[2pt]
      $\str{t, u, \str{f, g, h}} \gets \sk$\\
      if $\verify_\vk(\str{x,y,z},\sig) \ne 1$ then\\
      \tab return $\bot$\\
      $\coins^\prime \gets \losfinv(t, x)$\\
      $\ptxt^\prime \gets \hashf(h,\coins^\prime) \xor y$\\
      $z^\prime \gets \abof(g,\vk,\coins^\prime)$\\
      if $z^\prime \ne z$ then
        return $\bot$\\
      return $\ptxt^\prime$
  }
  \caption{Public-key encryption scheme $\pkescheme$ with message
  space~$\bits^\ell$ and constructed from:
  %
  $(n,k_\abo)$-ABO TDF scheme $\tdfscheme_\abo = (\abogen,\abof,\abofinv)$ with branch set $\bits^v$,
  %
  $(n,k_\los)$-LTDF scheme $\tdfscheme_\los = (\losgen,\losf,\losfinv)$,
  %
  hashing scheme $(\hashgen,\hashf)$ with domain~$\bits^n$ and range
  $\bits^\ell$, and
  %
  signature scheme $\sigscheme = (\siggen, \sign, \verify)$ such that each
  verifying key output by $\siggen$ is in $\bits^v$.
  }
  \label{fig-proto}
  \vspace{6pt}
  \hrule
\end{figure}
%
We show this composition is \indcca secure if each of its constituents meets its
security goal.

\begin{theorem}\label{thm-indcca}
  Let $\pkescheme = (\kg,\enc,\dec)$ as specified in Figure~\ref{fig-indcca}.
  %
  Suppose there exists a positive, real number $\epsilon$ such that $\ell \leq k
  - n - 2\log(1/\epsilon)$.
  %
  Let~$\advA$ be an \indcca adversary making one query to its~$\lro$ oracle.
  %
  There exist adversaries~$\advB$, $\advC$, and~$\advD$ such that for every
  \indcca adversary~$\advA$, it holds that
  \[
    \Adv{\indcca}_\pkescheme(\advA) \leq
      \Adv{\auth}_{\sigscheme}(\advB) +
      \Adv{\abo}_{\tdfscheme_\abo}(\advC) +
      \Adv{\los}_{\tdfscheme_\los}(\advD) +
      \epsilon \,,
  \]
  where each has about the same runtime as~$\advA$.
\end{theorem}

\begin{proof}
  We argue by game rewriting, beginning with the \indcca game instantiated
  with~$\pkescheme$ and played by~$\advA$ and ending in a game in which~$\advA$
  has no advantage. We bound the advantage of~$\advA$ in distinguishing
  neighboring games after each revision.

  % Game 0
  We begin with game~$\G_0$ defined in the top panel of
  Figure~\ref{fig-thm-indcca}. This is precisely the \indcca game instantiated
  with $\pkescheme$, and hence $\Adv{\indcca}_\pkescheme(\advA) =
  2\cdot\prob{\G_0(\advA)}-1$.
  %
  % Revision 0-1
  Game~$\G_1$, defined in the same panel, is the same as $\G_0$, except the the
  signing keys~$(\vk^*, \sk^*)$ corresponding to the~$\lro$ query are computed
  beforehand. This is without loss of generality, since~$\advA$ makes just one
  query to~$\advA$.
  %
  Then $\prob{\G_0(\advA)} = \prob{\G_1(\advA)}$.

  % Revision 1-2
  Next, game~$\G_2$ is defined by the revisions to the decryption oracle in the
  first-middle-left panel.
  %
  It is identical to~$\G_1$ unless~$\advA$ ever asks $\deco(\ctxt)$ where $\ctxt
  = \str{x,y,z,\vk,\sig}$ was not output by~$\lro$, but $y=y^*$. (In this case,
  the revised oracle outputs~$\bot$.)
  %
  Hence, the difference $\prob{\G_1(\advA)} - \prob{\G_2(\advA)}$ is at most the
  probability that~$\advA$ asks such a query.
  %
  Doing so implies a way to forge against the signature scheme $\sigscheme$.
  %
  On input $\vk^*$, \auth adversary~$\advB$ executes $(\pk, \sk) \getsr \kg$, chooses
  a random bit~$b$, and executes~$\advA$ on input of~$\pk$, simulating its
  oracle queries according to $\G_2$. When~$\advA$ queries its~$\lro$ oracle,
  adversary~$\advB$ uses its own $\sigo$ oracle to sign the out-going ciphertext.
  %
  If ever~$\advA$ asks $\deco(\ctxt)$ such that $c = \str{x,y,z,\sig,\vk^*}$
  where $c$ is not the output of $\lro$, but $\verify_{\vk^*}(\str{x,y,z},\sig)
  = 1$, then~$\advB$ halts and outputs $(\str{x,y,z}, \sig)$.
  %
  Then the advantage of~$\advB$ is exactly the probability that~$\advA$ makes
  such a query. It follows that
  %
  \begin{eqnarray}
    \prob{\G_1(\advA)} - \prob{\G_2(\advA)} \leq
    \Adv{\auth}_{\sigscheme}(\advB) \,.
  \end{eqnarray}

  % Revision 2-3
  Next, game~$\G_3$ is revised so that~$\vk^*$ is chosen as the lossy branch for
  generation of $(g,u)$ instead of~$0^v$. We construct an \abo adversary $\advC
  = (\advC_1, \advC_2)$, which simulates~$\G_{3-a}(\advA)$, where $a$ denotes
  the challenge bit in the \abo game.
  %
  The first stage, $\advC_1$, computes $(\vk^*,\xk^*) \getsr \siggen$ and outputs
  $(\vk^*, 0^v, \str{\vk^*,\xk^*})$. (Recall that the first stage \abo adversary
  outputs a pair of branches and some carry-over state.)
  %
  On input $(g, \str{\vk^*,\sk^*})$, the second stage, $\advC_2$, generates
  the public and private key according to~$\kg$, using its input~$g$ for the
  \abo function.
  %
  It then executes~$\advA$, simulating $\G_{3-a}$ in the natural way. (Note that
  the trapdoor~$u$ corresponding to~$g$ is unknown to~$\advC_2$, but this value
  is not used by any oracle.)
  %
  Finally, when~$\advA$ halts on $b^\prime$, adversary~$\advC_2$ outputs
  $b^\prime$.
  %
  Since the simulation is perfect, a simple conditioning argument yields that
  \begin{eqnarray}
    \Adv{\abo}_{\tdfscheme_\abo}(\advC)=
    \prob{\G_2(\advA)} - \prob{\G_3(\advA)} \,.
  \end{eqnarray}

  % Revision 3-4
  Game~$\G_4$ is defined by the modifications to the decryption oracle in the
  second-middle-left panel. Instead of recovering the coins~$r^\prime$ by inverting~$x$
  under the LTDF, the revised oracle does so by inverting $z$ at branch $\vk$
  under the ABO TDF. This branch is injective, since $\vk \ne \vk^*$ by definition
  and $\vk^*$ is the lossy branch for $(g,u)$. It follows that $\prob{\G_3(\advA)}
  = \prob{\G_4(\advA)}$.

  % Revision 4-5
  Game $\G_5$ revises $\G_4$ so that the LTDF is used in its lossy mode
  rather than in its injective mode. Similarly as before, we may construct a
  \los adversary~$\advD$ that simulates $\G_{5-b}(\advA)$ where $b$ denotes the
  challenge bit in the \los game, and for which
  \begin{eqnarray}
    \Adv{\los}_{\tdfscheme_\los}(\advD)=
    \prob{\G_4(\advA)} - \prob{\G_5(\advA)} \,.
  \end{eqnarray}

  % Revision 5-6
  Finally, game~$\G_6$ is defined by the revisions in the bottom panel. Instead
  of XOR-ing the hash of the coins with the message to get~$y$, the revised
  oracle XORs a uniform random, $\ell$-bit string $y^*$ with the message.
  %
  Suppose that~$\advA$ asks $\lro(\ptxt_0, \ptxt_1)$ and gets
  $\str{x,y,z,\vk^*,\sig}$ in response.
  %
  Recall that $k = k_\los + k_\abo$.
  %
  The string $\str{x,z}$ has at most $2^k$ values, since $x=\losf(f,r)$ and
  $f$ is lossy, and $z = \abof(g,\vk^*,r)$ and $\vk^*$ is the lossy branch
  of~$g$.
  %
  By Lemma~\ref{lemma1}, it follows that $\avgminent(r \given \str{x,z},h) \ge
  \minent(r \given h) - 2n+k$.
  %
  Since~$r$ is a uniform string and~$r$ and~$h$ are independent, it follows that
  $\avgminent(r \given \str{x,z},h) \ge n- 2n+k = k-n$.
  %
  By hypothesis, we have that $\ell \leq k - n - 2\log(1/\epsilon)$.
  %
  By Lemma~\ref{lemma2}, it follows that
  $\statdist( (\str{x,z}, h, \hashf(h,r)), (\str{x,z}, h, y^*) ) \leq \epsilon$.
  %
  We conclude that
  \begin{eqnarray}
    \prob{\G_5(\advA)} - \prob{\G_6(\advA)} \leq \epsilon \,.
  \end{eqnarray}
  %
  % Game 6
  Noting that $\prob{\G_6(\advA)} = 1/2$ yields the claim.\qed

  \begin{figure}
    \superGame % Revision 0-1
    {
      \underline{$\G_0(\advA)$}\\[2pt]
        $b \getsr \bits$; $Q \gets \emptyset$\\
        \diffplus{$(\vk^*, \xk^*) \getsr \siggen$}\\
        $h \getsr \hashgen$\\
        $(f, t) \getsr \losgen(1)$\\
        $(g, u) \getsr \abogen(0^v)$\\
        $\pk \gets \str{f, g, h}$\\
        $\sk \gets \str{t, u, \pk}$\\
        $b^\prime \getsr \advA^{\lro,\deco}(\pk)$\\
        return $b=b^\prime$
    }
    {
      \underline{$\lro(\ptxt_0, \ptxt_1)$}\\[2pt]
        if $|\ptxt_0| \ne |\ptxt_1|$ then\\
          \tab return $\bot$\\
        $\str{f, g, h} \gets \pk$\\
        \diffminus{$(\vk^*, \xk^*) \getsr \siggen$}\\
        $\coins \getsr \bits^n$;
        $x \gets \losf(f,\coins)$\\
        $y \gets \hashf(h,\coins) \xor \ptxt_b$\\
        $z \gets \abof(g, \vk^*, \coins)$\\
        $\sig \getsr \sign_{\xk^*}(\str{x,y,z})$\\
        $\ctxt \gets \str{x,y,z,\vk^*,\sig}$\\
        $Q \gets Q \union \{\ctxt\}$;
        return $\ctxt$
    }
    {
      \underline{$\dec_\sk(\str{x,y,z,\vk,\sig})$}
          \hfill \diff{$\G_1(\advA)$}\\[2pt]
        if $\ctxt \in Q$ then return $\bot$\\
        $\str{t, u, \str{f, g, h}} \gets \sk$\\
        if $\verify_\vk(\str{x,y,z},\sig) \ne 1$ then\\
        \tab return $\bot$\\
        $\coins^\prime \gets \losfinv(t, x)$\\
        $\ptxt^\prime \gets \hashf(h,\coins^\prime) \xor y$\\
        $z^\prime \gets \abof(g,\vk,\coins^\prime)$\\
        if $z^\prime \ne z$ then
          return $\bot$\\
        return $\ptxt^\prime$
    }
    {
      \underline{$\dec_\sk(\str{x,y,z,\vk,\sig})$}
          \hfill \diff{$\G_2(\advA)$}\\[2pt]
        if $\ctxt \in Q$ then return $\bot$\\
        $\str{t, u, \str{f, g, h}} \gets \sk$\\
        if $\verify_\vk(\str{x,y,z},\sig) \ne 1$ then\\
        \tab return $\bot$\\
        \diffplus{if $\vk = \vk^*$ then return $\bot$}\\
        $\coins^\prime \gets \losfinv(t, x)$\\
        $\ptxt^\prime \gets \hashf(h,\coins^\prime) \xor y$\\
        $z^\prime \gets \abof(g,\vk,\coins^\prime)$\\
        if $z^\prime \ne z$ then
          return $\bot$\\
        return $\ptxt^\prime$
    }
    {
      \underline{$\G_2(\advA)$}
          \hfill \diff{$\G_3(\advA)$}\\[2pt]
        $b \getsr \bits$; $Q \gets \emptyset$\\
        $(\vk^*, \xk^*) \getsr \siggen$\\
        $h \getsr \hashgen$\\
        $(f, t) \getsr \losgen(1)$\\
        \diffminus{$(g, u) \getsr \abogen(0^v)$}\diffplus{$(g, u) \getsr \abogen(vk^*)$}\\
        $\pk \gets \str{f, g, h}$\\
        $\sk \gets \str{t, u, \pk}$\\
        $b^\prime \getsr \advA^{\lro,\deco}(\pk)$\\
        return $b=b^\prime$
    }
    {
      \underline{$\dec_\sk(\str{x,y,z,\vk,\sig})$}
          \hfill \diff{$\G_4(\advA)$}\\[2pt]
        if $\ctxt \in Q$ then return $\bot$\\
        $\str{t, u, \str{f, g, h}} \gets \sk$\\
        if $\verify_\vk(\str{x,y,z},\sig) \ne 1$ then\\
        \tab return $\bot$\\
        if $\vk = \vk^*$ then return $\bot$\\
        \diffminus{$\coins^\prime \gets \losfinv(t, x)$}\diffplus{$\coins^\prime \gets \abofinv(u, z)$}\\
        $\ptxt^\prime \gets \hashf(h,\coins^\prime) \xor y$\\
        $z^\prime \gets \abof(g,\vk,\coins^\prime)$\\
        if $z^\prime \ne z$ then
          return $\bot$\\
        return $\ptxt^\prime$
    }
    {
      \underline{$\G_4(\advA)$}
          \hfill \diff{$\G_5(\advA)$}\\[2pt]
        $b \getsr \bits$; $Q \gets \emptyset$\\
        $(\vk^*, \xk^*) \getsr \siggen$\\
        $h \getsr \hashgen$\\
        \diffminus{$(f, t) \getsr \losgen(1)$}\diffplus{$(f, t) \getsr \losgen(0)$}\\
        $(g, u) \getsr \abogen(vk^*)$\\
        $\pk \gets \str{f, g, h}$\\
        $\sk \gets \str{t, u, \pk}$\\
        $b^\prime \getsr \advA^{\lro,\deco}(\pk)$\\
        return $b=b^\prime$
    }
    \threeColsNoDivides{0.24}{0.35}{0.37} % Revision 0-1
    {
      \underline{$\G_5(\advA)$}\\[2pt]
        $b \getsr \bits$; $Q \gets \emptyset$\\
        $(\vk^*, \xk^*) \getsr \siggen$\\
        $h \getsr \hashgen$\\
        $(f, t) \getsr \losgen(0)$\\
        $(g, u) \getsr \abogen(\vk^*)$\\
        $\pk \gets \str{f, g, h}$\\
        $\sk \gets \str{t, u, \pk}$\\
        $b^\prime \getsr \advA^{\lro,\deco}(\pk)$\\
        return $b=b^\prime$
    }
    {
      \underline{$\lro(\ptxt_0, \ptxt_1)$}\\[2pt]
        if $|\ptxt_0| \ne |\ptxt_1|$ then\\
          \tab return $\bot$\\
        $\str{f, g, h} \gets \pk$;
        \diffplus{$y^* \getsr \bits^\ell$}\\
        $\coins \getsr \bits^n$;
        $x \gets \losf(f,\coins)$\\
        $y \gets$\diffminus{$\hashf(h,\coins)$}\diffplus{$y^*$}$\xor \ptxt_b$\\
        $z \gets \abof(g, \vk^*, \coins)$\\
        $\sig \getsr \sign_{\xk^*}(\str{x,y,z})$\\
        $\ctxt \gets \str{x,y,z,\vk^*,\sig}$\\
        $Q \gets Q \union \{\ctxt\}$;
        return $\ctxt$
    }
    {
      \underline{$\dec_\sk(\str{x,y,z,\vk,\sig})$}
          \hfill \diff{$\G_6(\advA)$}\\[2pt]
        if $\ctxt \in Q$ then return $\bot$\\
        $\str{t, u, \str{f, g, h}} \gets \sk$\\
        if $\verify_\vk(\str{x,y,z},\sig) \ne 1$ then\\
        \tab return $\bot$\\
        if $\vk = \vk^*$ then return $\bot$\\
        $\coins^\prime \gets \abofinv(u, z)$\\
        $\ptxt^\prime \gets \hashf(h,\coins^\prime) \xor y$\\
        $z^\prime \gets \abof(g,\vk,\coins^\prime)$\\
        if $z^\prime \ne z$ then
          return $\bot$\\
        return $\ptxt^\prime$
    }
    \caption{Games for proof of Theorem~\ref{thm-indcca}. Adversary~$\advA$ makes
    just one~$\lro$ query.}
    \label{fig-thm-indcca}
  \end{figure}
\end{proof}

\if{0}
% Constructed based on target-collision hash instead of a signature. This don't
% work!!! There's a simple distinguishing attack that uses just one decryption
% query.
\begin{figure}[t]
  \threeColsNoDivides{0.32}{0.32}{0.32}
  {
    \underline{$\kg$}\\[2pt]
      $h \getsr \hashgen$;
      $i \getsr \hashgen_\tcr$\\
      $(f, t) \getsr \losgen(1)$\\
      $(g, u) \getsr \abogen(0^v)$\\
      $\pk \gets \str{f, g, h, i}$\\
      $\sk \gets \str{t, u, \pk}$\\
      return $(\pk, \sk)$
  }
  {
    \underline{$\enc_\pk(\ptxt)$}\\[2pt]
      $\str{f, g, h, i} \gets \pk$\\
      $\coins \getsr \bits^n$;
      $w \gets \losf(f,\coins)$\\
      $x \gets \hashf(h,\coins) \xor \ptxt$\\
      $y \gets \hashf_\tcr(i,\coins)$\\
      $z \gets \abof(g, y, \coins)$\\
      return $\str{w,x,y,z}$
  }
  {
    \underline{$\dec_\sk(\str{w,x,y,z})$}\\[2pt]
      $\str{t, u, \str{f, g, h, i}} \gets \sk$\\
      $\coins^\prime \gets \losfinv(t, w)$\\
      $\ptxt^\prime \gets \hashf(h,\coins^\prime) \xor x$\\
      $y^\prime \gets \hashf_\tcr(i,\coins^\prime)$\\
      $z^\prime \gets \abof(g,y^\prime,\coins^\prime)$\\
      if $z^\prime \ne z$ then
        return $\bot$\\
      return $\ptxt^\prime$
  }
  \caption{Public-key encryption scheme $\pkescheme$ constructed from:
  %
  $(n,k)$-ABO TDF scheme $\tdfscheme_\abo = (\abogen,\abof,\abofinv)$ with branch set $\bits^v$,
  %
  $(n,k^\prime)$-LTDF scheme $\tdfscheme_\los = (\losgen,\losf,\losfinv)$,
  %
  hashing scheme $\hashscheme = (\hashgen,\hashf)$ with domain~$\bits^n$ and range
  $\bits^\ell$, and
  %
  hashing scheme $\hashscheme_\tcr = (\hashgen_\tcr,\hashf_\tcr)$ with domain~$\bits^n$ and range
  $\bits^v$.
  }
  \label{fig-proto}
  \vspace{6pt}
  \hrule
\end{figure}
\fi

\if{0}
% TCR notion
A hashing scheme $\hashscheme = (\hashgen, \hashf)$ is \emph{target-collision
resistant} (also called \emph{second-preimage resistant} \cite{rs04}) if the
advantage of any reasonable adversary in the \tcr experiment defined in
Figure~\ref{fig-tcr} is small, where the advantage of~$\advA$ in
attacking~$\hashscheme$ is defined as
$
  \Adv{\tcr}_\hashscheme(\advA) = \prob{ \Exp{\tcr}_\hashscheme(\advA) }.
$
\begin{figure}
  \oneCol{0.70}
  {
    \underline{$\Exp{\tcr}_\hashscheme(\advA)$}\\[2pt]
      $(x_0, \st) \getsr \advA_1$;
      $h \getsr \hashgen$;
      $x_1 \getsr \advA_2(\st, h)$\\
      if $\{x_0, x_1\} \not\subseteq X \OR x_0 = x_1$ then return $\bot$\\
      return $\hashf(h, x_0) = \hashf(h, x_1)$
  }
  \caption{Target-collision resistance of hashing scheme $\hashscheme =
  (\hashgen, \hashf)$ with domain~$X$.}
  \label{fig-tcr}
\end{figure}
\fi
