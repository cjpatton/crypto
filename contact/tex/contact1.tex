%
\def\camready{1}
\ifnum\camready=0
  \makeatletter
  \let\@twosidetrue\@twosidefalse
  \let\@mparswitchtrue\@mparswitchfalse
  \makeatother
\fi

\documentclass{build/llncs}
\usepackage{color,soul}
\usepackage[dvipsnames]{xcolor}
\usepackage{enumitem}
% header.tex
%
% Formatting and common macros for crypto papers. Include this first.
\usepackage[dvipsnames]{xcolor}
\usepackage{color,soul}
\usepackage{enumitem}
\usepackage{xspace}
\usepackage{amsmath}
\usepackage{amssymb}
\usepackage{amsfonts}
\usepackage{xcolor}
\usepackage{xstring}
\usepackage{enumitem}


\usepackage{tikz}
\tikzset{>=latex}

\usepackage{listings}
\lstset{
  basicstyle=\small\ttfamily,
  columns=flexible,
  breaklines=true
}

\usepackage{hyperref}
\hypersetup{
  colorlinks,%
  citecolor=black,%
  filecolor=black,%
  linkcolor=black,%
  urlcolor=black
}

% Colors
\definecolor{theblue}{RGB}{85,135,170}
\definecolor{thedarkgray}{RGB}{95,95,95}
\definecolor{thedarkgreen}{RGB}{50,127,0}
\definecolor{thegray}{RGB}{165,165,165}
\definecolor{thelightblue}{RGB}{185,220,250}
\definecolor{thelightgray}{RGB}{230,230,230}
\definecolor{thelightred}{RGB}{250,185,165}
\definecolor{thered}{RGB}{210,145,125}

% Editorial
\newcommand{\ala}{\text{a la}\xspace}
\newcommand{\etal}{et al.\xspace}
\newcommand{\apriori}{\textit{a priori}\xspace}
\newcommand{\viceversa}{\textit{vice versa}\xspace}

% Fonts
\newcommand{\algorithmfont}[1]{\mathcal{#1}}
\newcommand{\codefont}[1]{\lstinline|#1|}
\newcommand{\cryptofont}[1]{\textup{\textbf{#1}}\hspace{0.5pt}}
\newcommand{\expfont}[1]{{{\tiny\MakeLowercase{\textnormal{#1}}}}}
\newcommand{\lifont}[1]{\textsf{\footnotesize\color{thegray}#1}}
\newcommand{\notionfont}[1]{{#1}\xspace}
\newcommand{\oraclefont}[1]{\cryptofont{#1}}
\newcommand{\procfont}[1]{\textnormal{#1}\hspace{0.33pt}}
\newcommand{\schemefont}[1]{\mathcal{#1}}
\newcommand{\setfont}[1]{\mathcal{#1}}
\newcommand{\rwordfont}[1]{\cryptofont{#1}\xspace}
\newcommand{\strfont}[1]{\textup{\textsf{{\color{theblue}#1}}}}
\newcommand{\varfont}[1]{\mathit{#1}}
\newcommand{\symbolfont}[1]{\textsf{\footnotesize\color{thegray}#1}}

\newcommand{\stringer}[1]{\strfont{#1}}
\newcommand{\rword}[1]{\rwordfont{#1}}

% Crypto functions
\newcommand{\Hybrid}{\cryptofont{H}}
\newcommand{\Gam}{\cryptofont{G}}
\newcommand{\Exp}[1]{\cryptofont{Exp}^{\expfont{#1}}}
\newcommand{\Adv}[1]{\cryptofont{Adv}^{\expfont{#1}}}

% Types
\newcommand{\typ}[1]{\typs\rword{#1}}
\newcommand{\typs}{\,}
\newcommand{\us}{\!} % previously \unskip

\newcommand{\rbool}{\typ{bool}}
\newcommand{\relem}[1]{\typ{elem}(#1)}
\newcommand{\rint}{\typ{int}}
\newcommand{\rset}{\typ{set}}
\newcommand{\rstr}{\typ{str}}
\newcommand{\rstruct}{\typ{struct}}
\newcommand{\rtuple}{\typ{tup}}

\newcommand{\rdet}{\rword{det}\typs}
\newcommand{\rtype}{\rword{type}}
\newcommand{\rrand}{\rword{rand}\typs}

%% Fancy syntax stuff
\newcommand{\nil}{\symbolfont{nil}}
\newcommand{\ptr}{\typs\symbolfont{*}\us}
\newcommand{\rfr}{\symbolfont{\&}}
\newcommand{\any}{\symbolfont{\_}}

%% Reserved words
\newcommand{\rand}{\rwordfont{and}}
\newcommand{\rdeclare}{\rwordfont{dec}}
\newcommand{\relse}{\rwordfont{else}}
\newcommand{\rif}{\rwordfont{if}}
\newcommand{\rwhile}{\rwordfont{while}}
\newcommand{\runtil}{\rwordfont{until}}
\newcommand{\rdo}{\rwordfont{do}}
\newcommand{\ror}{\rwordfont{or}}
\newcommand{\rfor}{\rwordfont{for}}
\newcommand{\rforeach}{\rwordfont{for each}}
\newcommand{\rreturn}{\rwordfont{ret}}
\newcommand{\rbreak}{\rwordfont{break}}
\newcommand{\rthen}{\rwordfont{then}}
\newcommand{\rto}{\rwordfont{to}}
\newcommand{\rswitch}{\rwordfont{switch}}
\newcommand{\rcase}{\rwordfont{case}}
\newcommand{\rdefault}{\rwordfont{default}}

% Math
\DeclareMathAlphabet\mathbfcal{OMS}{cmsy}{b}{n}

\def\v.#1{\boldsymbol{#1}}
\newcommand{\dqed}{\hfill$\Diamond$}
\newcommand{\GF}{\mathit{GF}}

%% Sets
\DeclareMathOperator*{\argmin}{arg\,min}
\DeclareMathOperator*{\dom}{Dom}
\DeclareMathOperator*{\rng}{Rng}
\def\bydef{\stackrel{\rm def}{=}}
\newcommand*\by{\times}
\newcommand{\getsr}{\xleftarrow{\text{\tiny{\$}}}}
%\newcommand{\getsr}{{\:{\leftarrow{\hspace*{-3pt}\raisebox{.75pt}{$\scriptscriptstyle\$$}}}\:}}
\newcommand*\intersection{\cap}
\newcommand{\multiunion}{\uplus}
\newcommand*\union{\cup}

%% String/tuple operators
\def\slicerange#1,#2{#1\mbox{\hspace{0.5pt}:\hspace{0.5pt}}#2}
\newcommand{\bits}{\{0,1\}}
\newcommand{\cat}{\, \| \,}
\newcommand{\emptystr}{\varepsilon}
\newcommand{\emptystring}{\emptystr}
\newcommand{\hassubstr}{\sim}
\newcommand{\prefixes}{\preceq}
\newcommand{\slice}[2]{#1[\slicerange#2]}
\newcommand{\toint}[2]{\underline{\smash{#1}}_{\hspace{0.5pt}#2}}
\newcommand{\tostr}[1]{\underline{\smash{#1}}}
\newcommand{\trunc}{\%}

%% Boolean operators
\newcommand*\band{\wedge}
\newcommand*\bor{\vee}
\newcommand*\bnot{\neg}
\newcommand*\bimplies{\implies}
\newcommand*\bxor{\mathbin{\oplus}}

%% Probability
\newcommand{\E}{\mathrm{E}}
\newcommand{\Prob}[1]{\Pr\hspace{-1pt}\big[\,#1\,\big]}
\newcommand{\Probs}[2]{\Pr_{#1}\hspace{-1pt}{\big[\,#2\,\big]}}
\newcommand{\given}{\mid}

% References
\newcommand{\fig}[1]{Fig.~\ref{fig:#1}}
\newcommand{\lem}[1]{Lemma~\ref{lemma:#1}}

% Dev
\def\coloroverride{0}
\newenvironment{cleanup}{
  \def\coloroverride{1}
  \color{gray}
}{
}

% Notes
\newcounter{notectr}[section]
\newcommand{\getnotectr}{\stepcounter{notectr}\thesection.\thenotectr}

\newcommand{\basenoteshape}[3]{(\getnotectr: #1\xspace#2: #3)}

\newcommand{\basenote}[4]{%
  \textrm{%
    \IfEqCase{\coloroverride}{%
        {1}{\basenoteshape{#2}{#3}{#4}}%
        {0}{\textcolor{#1}{\basenoteshape{#2}{#3}{#4}}}%
    }[\PackageError{basenote}{Undefined option for basenote: \coloroverride}{}]%
  }%
}

%% Uncomment to mute notes.
%\renewcommand{\basenote}[4]{\ignorespaces}

\newcommand{\todo}[3]{\basenote{#1}{#2}{todo}{#3}}
\newcommand{\fixme}[1]{\basenote{red}{}{fixme}{{\bf#1}}}

\newcommand{\ignore}[1]{\if{0}#1\fi}

% Games
\renewcommand{\#}[1]{\text{\textcolor{theblue}{\,\text{//}\,{\small #1}}}}
\newcommand{\g}[1]{\cryptofont{G}_{#1}}
\newcommand{\hybrid}[1]{\cryptofont{H}_{#1}}
\newcommand{\gouts}{=}
\newcommand{\gsets}{\,\cryptofont{sets}\,}
\newcommand{\tab}{\hspace*{10pt}}

%% Diffs
\newcommand{\diffplus}[1]{\fbox{#1}}
\newcommand{\diffplusbox}[1]{\fbox{\parbox{\dimexpr\textwidth-2\fboxsep-2\fboxrule\relax}{#1}}\vspace{1pt}}
\newcommand{\diffminus}[1]{\colorbox{lightgray}{#1}}
\newcommand{\diffminusbox}[1]{\colorbox{lightgray}{\parbox{\dimexpr\textwidth-2\fboxsep-2\fboxrule\relax}{#1}}}
\newcommand{\diff}[2]{\diffminus{#1}\diffplus{#2}}

%% Line numbers
\newcommand{\liref}[2]{\fig{#1}:#2}
\newcounter{FunctionCounter}
\newcounter{LineCounter}[FunctionCounter]

%%% Reset numbering (new game)
\newcommand{\ga}{\setcounter{FunctionCounter}{0}\setcounter{LineCounter}{0}}

%%% Increment function number
\newcommand{\fu}{
%  \stepcounter{FunctionCounter}
}

%%% Increment line number
\usepackage{calc}
\newcommand{\li}{
%  \stepcounter{LineCounter}\lifont{\scriptsize\theFunctionCounter\ifnum\value{LineCounter}<10\ignorespaces0\fi\theLineCounter}
  \stepcounter{LineCounter}\makebox[8pt][r]{\lifont{\scriptsize\theLineCounter}}\hspace{2pt}
}

%% Boxes
\newcommand{\gamesfontsize}{\small}
\newcommand{\gamespadleft}{\hskip 1pt}
\newcommand{\gamespad}{\hskip 4pt}
\newcommand{\gamespacer}{\\[-6.5pt]}

\newcommand{\oneCol}[2]{
  \begin{center}
    \makebox[\textwidth][l]{
      \begin{tabular}{|@{\gamespadleft}l@{\gamespad}@{}|}
      \hline
      \rule{0pt}{1\normalbaselineskip}
      \begin{minipage}[t]{#1\textwidth}\gamesfontsize
        #2 \vspace{6pt}
      \end{minipage} \\
      \hline
    \end{tabular}
    }
  \end{center}
}

\newcommand{\twoCols}[3]{
  \makebox[\textwidth][c]{
    \begin{tabular}{|@{\gamespadleft}l@{\gamespad}|@{}@{\gamespad}l@{\gamespad}|}
    \hline
    \rule{0pt}{1\normalbaselineskip}
    \begin{minipage}[t]{#1\textwidth}\gamesfontsize
      #2 \vspace{6pt}
    \end{minipage} &
    \begin{minipage}[t]{#1\textwidth}\gamesfontsize
      #3 \vspace{6pt}
    \end{minipage} \\
    \hline
  \end{tabular}
  }
}

\newcommand{\twoColsNoBox}[3]{
  \makebox[\textwidth][c]{
    \begin{tabular}{@{\gamespadleft}l@{\gamespad}|@{}@{\gamespad}l@{\gamespad}}
    \rule{0pt}{1\normalbaselineskip}
    \begin{minipage}[t]{#1\textwidth}\gamesfontsize
      #2 \vspace{6pt}
    \end{minipage} &
    \begin{minipage}[t]{#1\textwidth}\gamesfontsize
      #3 \vspace{6pt}
    \end{minipage} \\
  \end{tabular}
  }
}

\newcommand{\twoColsUnbalanced}[4]{
  \makebox[\textwidth][c]{
    \begin{tabular}{|@{\gamespadleft}l@{\gamespad}|@{}@{\gamespad}l@{\gamespad}|}
    \hline
    \rule{0pt}{1\normalbaselineskip}
    \begin{minipage}[t]{#1\textwidth}\gamesfontsize
      #3 \vspace{6pt}
    \end{minipage} &
    \begin{minipage}[t]{#2\textwidth}\gamesfontsize
      #4 \vspace{6pt}
    \end{minipage} \\
    \hline
  \end{tabular}
  }
}

\newcommand{\twoColsNoLinesUnbalanced}[4]{
  \makebox[\textwidth][c]{
    \begin{tabular}{@{\gamespadleft}l@{\gamespad}@{}@{\gamespad}l@{\gamespad}}
    \rule{0pt}{1\normalbaselineskip}
    \begin{minipage}[t]{#1\textwidth}\gamesfontsize
      #3 \vspace{6pt}
    \end{minipage} &
    \begin{minipage}[t]{#2\textwidth}\gamesfontsize
      #4 \vspace{6pt}
    \end{minipage} \\
  \end{tabular}
  }
}

\newcommand{\twoColsNoDivide}[3]{
  \makebox[\textwidth][c]{
    \begin{tabular}{|@{\gamespadleft}l@{\gamespad}@{}@{\gamespad}l@{\gamespad}|}
    \hline
    \rule{0pt}{1\normalbaselineskip}
    \begin{minipage}[t]{#1\textwidth}\gamesfontsize
      #2 \vspace{6pt}
    \end{minipage} &
    \begin{minipage}[t]{#1\textwidth}\gamesfontsize
      #3 \vspace{6pt}
    \end{minipage} \\
    \hline
  \end{tabular}
  }
}

\newcommand{\twoColsNoDivideUnbalanced}[4]{
  \makebox[\textwidth][c]{
    \begin{tabular}{|@{\gamespadleft}l@{\gamespad}@{}@{\gamespad}l@{\gamespad}|}
    \hline
    \rule{0pt}{1\normalbaselineskip}
    \begin{minipage}[t]{#1\textwidth}\gamesfontsize
      #3 \vspace{6pt}
    \end{minipage} &
    \begin{minipage}[t]{#2\textwidth}\gamesfontsize
      #4 \vspace{6pt}
    \end{minipage} \\
    \hline
  \end{tabular}
  }
}

\newcommand{\twoColsTwoRows}[5]{
  \makebox[\textwidth][c]{
  \begin{tabular}{|@{\gamespadleft}l@{\gamespad}|@{}@{\gamespad}l@{\gamespad}|}
    \hline
    \rule{0pt}{1\normalbaselineskip}
    \begin{minipage}[t]{#1\textwidth}\gamesfontsize
      #2 \vspace{6pt}
    \end{minipage} &
    \begin{minipage}[t]{#1\textwidth}\gamesfontsize
      #3 \vspace{6pt}
    \end{minipage} \\
    \hline
    \rule{0pt}{1\normalbaselineskip}
    \begin{minipage}[t]{#1\textwidth}\gamesfontsize
      #4 \vspace{6pt}
    \end{minipage} &
    \begin{minipage}[t]{#1\textwidth}\gamesfontsize
      #5 \vspace{6pt}
    \end{minipage} \\
    \hline
  \end{tabular}
  }
}

\newcommand{\threeCols}[4]{
  \makebox[\textwidth][c]{
    \begin{tabular}{|@{\gamespadleft}l@{\gamespad}|@{}@{\gamespad}l@{\gamespad}|@{}@{\gamespad}l@{\gamespad}|}
    \hline
    \rule{0pt}{1\normalbaselineskip}
    \begin{minipage}[t]{#1\textwidth}\gamesfontsize
      #2 \vspace{6pt}
    \end{minipage} &
    \begin{minipage}[t]{#1\textwidth}\gamesfontsize
      #3 \vspace{6pt}
    \end{minipage} &
    \begin{minipage}[t]{#1\textwidth}\gamesfontsize
      #4 \vspace{6pt}
    \end{minipage} \\
    \hline
  \end{tabular}
  }
}

\newcommand{\threeColsUnbalanced}[6]{
  \makebox[\textwidth][c]{
    \begin{tabular}{|@{\gamespadleft}l@{\gamespad}|@{}@{\gamespad}l@{\gamespad}|@{}@{\gamespad}l@{\gamespad}|}
    \hline
    \rule{0pt}{1\normalbaselineskip}
    \begin{minipage}[t]{#1\textwidth}\gamesfontsize
      #4 \vspace{6pt}
    \end{minipage} &
    \begin{minipage}[t]{#2\textwidth}\gamesfontsize
      #5 \vspace{6pt}
    \end{minipage} &
    \begin{minipage}[t]{#3\textwidth}\gamesfontsize
      #6 \vspace{6pt}
    \end{minipage} \\
    \hline
  \end{tabular}
  }
}

\newcommand{\threeColsFirstDivideUnbalanced}[6]{
  \makebox[\textwidth][c]{
    \begin{tabular}{|@{\gamespadleft}l@{\gamespad}|@{}@{\gamespad}l@{\gamespad}@{}@{\gamespad}l@{\gamespad}|}
    \hline
    \rule{0pt}{1\normalbaselineskip}
    \begin{minipage}[t]{#1\textwidth}\gamesfontsize
      #4 \vspace{6pt}
    \end{minipage} &
    \begin{minipage}[t]{#2\textwidth}\gamesfontsize
      #5 \vspace{6pt}
    \end{minipage} &
    \begin{minipage}[t]{#3\textwidth}\gamesfontsize
      #6 \vspace{6pt}
    \end{minipage} \\
    \hline
  \end{tabular}
  }
}

\newcommand{\threeColsSecondDivideUnbalanced}[6]{
  \makebox[\textwidth][c]{
    \begin{tabular}{|@{\gamespadleft}l@{\gamespad}@{}@{\gamespad}l@{\gamespad}@{}|@{\gamespad}l@{\gamespad}|}
    \hline
    \rule{0pt}{1\normalbaselineskip}
    \begin{minipage}[t]{#1\textwidth}\gamesfontsize
      #4 \vspace{6pt}
    \end{minipage} &
    \begin{minipage}[t]{#2\textwidth}\gamesfontsize
      #5 \vspace{6pt}
    \end{minipage} &
    \begin{minipage}[t]{#3\textwidth}\gamesfontsize
      #6 \vspace{6pt}
    \end{minipage} \\
    \hline
  \end{tabular}
  }
}

\newcommand{\threeColsNoDivideUnbalanced}[6]{
  \makebox[\textwidth][c]{
    \begin{tabular}{|@{\gamespadleft}l@{\gamespad}@{}@{\gamespad}l@{\gamespad}@{}@{\gamespad}l@{\gamespad}|}
    \hline
    \rule{0pt}{1\normalbaselineskip}
    \begin{minipage}[t]{#1\textwidth}\gamesfontsize
      #4 \vspace{6pt}
    \end{minipage} &
    \begin{minipage}[t]{#2\textwidth}\gamesfontsize
      #5 \vspace{6pt}
    \end{minipage} &
    \begin{minipage}[t]{#3\textwidth}\gamesfontsize
      #6 \vspace{6pt}
    \end{minipage} \\
    \hline
  \end{tabular}
  }
}

\newcommand{\fourCols}[5]{
  \makebox[\textwidth][c]{
    \begin{tabular}{|@{\gamespadleft}l@{\gamespad}|@{}@{\gamespad}l@{\gamespad}|@{}@{\gamespad}l@{\gamespad}|@{}@{\gamespad}l@{\gamespad}|}
    \hline
    \rule{0pt}{1\normalbaselineskip}
    \begin{minipage}[t]{#1\textwidth}\gamesfontsize
      #2 \vspace{6pt}
    \end{minipage} &
    \begin{minipage}[t]{#1\textwidth}\gamesfontsize
      #3 \vspace{6pt}
    \end{minipage} &
    \begin{minipage}[t]{#1\textwidth}\gamesfontsize
      #4 \vspace{6pt}
    \end{minipage} &
    \begin{minipage}[t]{#1\textwidth}\gamesfontsize
      #5 \vspace{6pt}
    \end{minipage} \\
    \hline
  \end{tabular}
  }
}

\newcommand{\twoColsThreeRows}[7]{
  \makebox[\textwidth][c]{
  \begin{tabular}{|@{\gamespadleft}l@{\gamespad}|@{}@{\gamespad}l@{\gamespad}|}
    \hline
    \rule{0pt}{1\normalbaselineskip}
    \begin{minipage}[t]{#1\textwidth}\gamesfontsize
      #2 \vspace{6pt}
    \end{minipage} &
    \begin{minipage}[t]{#1\textwidth}\gamesfontsize
      #3 \vspace{6pt}
    \end{minipage} \\
    \hline
    \rule{0pt}{1\normalbaselineskip}
    \begin{minipage}[t]{#1\textwidth}\gamesfontsize
      #4 \vspace{6pt}
    \end{minipage} &
    \begin{minipage}[t]{#1\textwidth}\gamesfontsize
      #5 \vspace{6pt}
    \end{minipage} \\
    \hline
    \rule{0pt}{1\normalbaselineskip}
    \begin{minipage}[t]{#1\textwidth}\gamesfontsize
      #6 \vspace{6pt}
    \end{minipage} &
    \begin{minipage}[t]{#1\textwidth}\gamesfontsize
       #7 \vspace{6pt}
    \end{minipage} \\
    \hline
  \end{tabular}
  }
}

% macros.tex
%
% Macros for this paper. (Include build/headers.tex, then this.)
\usepackage[dvipdf]{graphicx}
\usepackage{pifont} % \ding{109}
\usepackage{afterpage} % \afterpage{ ... }

% Variables
\newcommand{\dist}{\delta}
\newcommand{\doride}{\varphi}
\newcommand{\De}{\procfont{De}}
\newcommand{\En}{\procfont{En}}
\newcommand{\Ev}{\procfont{Ev}}
\newcommand{\ev}{\procfont{ev}}
\newcommand{\garbler}{\schemefont{G}}
\newcommand{\Gb}{\procfont{Gb}}
\newcommand{\id}{\funcfont{id}}
\newcommand{\loc}{\funcfont{loc}}
\newcommand{\rep}{\funcfont{rep}}
\newcommand{\SpA}{\procfont{SpA}}
\newcommand{\SpB}{\procfont{SpB}}

% Authors' comments.
\definecolor{darkgreen}{RGB}{50,127,0}
\newcommand{\cpnote}[1]{\note{darkgreen}{Chris}{#1}}
\newcommand{\tsnote}[1]{\note{blue}{Tom}{#1}}
\newcommand{\cptodo}[1]{\todo{red}{Chris}{#1}}
\newcommand{\tstodo}[1]{\todo{red}{Tom}{#1}}
\newcommand{\anytodo}[1]{\todo{red}{Any}{#1}}


\ifnum\camready=0
  %\usepackage{times}
  \usepackage[letterpaper,margin=0.75in]{geometry}
\fi

\ifnum\camready=0
  \usepackage{hyperref}
  \hypersetup{
    colorlinks=true,
    allcolors={black},
    citecolor={theblue},
    urlcolor={theblue},
    pdfborder={0 0 0},
  }
\fi


\date{\today}
\title{Private contact discovery ({\color{thegray}version \#1})}
\author{Chris}
\institute{}

\setcounter{tocdepth}{2}

\pagestyle{plain}

\begin{document}

\maketitle

\begin{abstract}
  A contact-discovery protocol allows a user of a messaging service to discover
  which contacts in his or her phone book use the same service. The attendant
  privacy goal is to ensure that the neither the client nor the server learns
  anything beyond the set of contacts that use the service.
  %
  This can be solved using a protocol for private set intersection, but the
  communication complexity of such protocols is prohibitive for this
  application.
  %
  I propose an alternative that is optimal in terms of network utilization,
  but requires additional storage capacity on the part of the server. It applies
  bilinear maps (trust me, it's not as crazy as it seems) and uses SGX in a
  minimal, yet critical way.
\end{abstract}

\section*{Problem statement}

Let $\U \subseteq \bits^*$ denote the set of users of a
service.\footnote{$\bits^*$ denotes the set of all strings.}
%
A new user of the service has a set~$\X \subseteq \bits^*$ of contacts, and he
or she wishes to learn which of these are also users of the service.
%
We want to devise a protocol by which the client and server compute $\U
\intersection \X$ without the server learning $\X \setminus \U$ or the client
learning $\U \setminus \X$.
%
This is an instance of private set intersection, for which there are myriad
solutions. But these are prohibitively expensive, particularly in the current
application.\footnote{See slide 11 of
\url{https://www.usenix.org/sites/default/files/conference/protected-files/sec15_slides_pinkas.pdf}.
Accessed 2018-04-12.}
%
The goal here is to mitigate this cost.

I'll begin by refining the problem. Each user ``registers'' with the
contact-discovery service, which responds to queries made by registered users.
%
Specifically, registrant~$I$ may ask if~$J\in\U$ for any~$J$; the result should
be $\true$ (``true'') if and only if $I, J \in \U$.
%
The server can easily ensure that the clients from learns nothing about $\U
\setminus \X$. In addition, the bandwidth consumption is $O(m)$, where $m=|\X|$.
%
To prevent leakage of~$\X \setminus \U$, the protocol also needs to ensure that
queries only reveal~$J$ to the server if~$J$ is registered. Here I propose one
way to provide this assurance. The solution involves some cool crytography, as
well as careful use of SGX.\footnote{This is why you're reading this in
\url{bitbucket.org/uf_sensei/hybrid} :)}

The crucial resources are the time complexity of registration, queries, and the
space utilization of the server.
%
My proposal costs $\Theta(n)$ time for registration, $O(1)$ time for each query,
and $O(n^2)$ space, where $n=|\U|$.
%
Registration is completely parallelizable, so distributing the computation can
help mitigate this cost.
%
Still, the space complexity means deployment might only be feasible for a few
hundred-thousand users. However, the schemes are pretty simplistic, and there
is likely room for improvement.

\section*{Straw man scheme: why Diffie-Hellman isn't enough}

The first scheme I considered uses Diffie-Hellman.
%
Let~$G$ be an additive group of prime order $q$, and let~$P$ be a generator
for~$G$.\footnote{$P$ being a generator for~$G$ means that $G = \{ 0P, 1P, 2P,
\ldots, (q-1)P \}$.}
%
Let $\hash:\bits^* \to [0..q-1]$ be a crytpographic hash
function.\footnote{$[0..q-1]$ denotes the set of integers $\{0, 1, 2, \ldots,
q-1\}$.}

\heading{Setup.}
The server initializes a map~$\U$, which stores a set of key/value pairs. The
keys are strings and the values are elements of~$G$. This maps the set of
\emph{registered users} to their \emph{public keys}.

\heading{Registration.}
%
To register user~$I$, first initialize a set $\M_I \gets \emptyset$.
%
Next, sample an integer~$a$ uniformly from the set
$[1..q-1]$. (This is denoted $a \getsr [1..q-1]$.) This is the user's
\emph{secret
key}.
%
Next, compute $A \gets aP$; this is the user's public key.
%
Let $x \gets \hash(I)$.
%
For each $(J, B)$ in the set of key/value pairs in~$\U$, do as follows:
\begin{enumerate}
  \item Compute values: $y \gets \hash(J)$, $X \gets ay P$, and $Y \gets x B$.
  \item Set $\M_I \gets \M_I \union \{X\}$ and $\M_J \gets \M_J \union \{Y\}$.
\end{enumerate}
Next, set $\U[I] \gets A$.
Finally, return the secret key~$a$ to the user.

\heading{Making queries.}
%
User~$I$ with secret key~$a$ wishes to see if~$J$ is a registered user. To do
so, first compute $y \gets \hash(J)$ and $X \gets ay P$.
%
Then send~$(I, X)$ to the server.

\heading{Answering queries.}
%
On input of~$(I, X)$, if $\U[I]$ is undefined, then return $\false$.
%
Otherwise, if $X \not\in \M_I$, then return $\false$.
%
Otherwise return $\true$.

\heading{Use of SGX.}
%
Trusted computation is used for provisioning the user with their secret key.
It's also used for registration, since this requires computations using the
secret key.

\noindent\textbf{\color{red}Attack!}
%
On input of $(I, X)$ the server can try a dictionary attack. Let $A \gets
\U[I]$. To check if $J$ was used to compute query~$X$, compute $y \gets \hash(J)$, then
$A' \gets y^{-1}X$. If $A' = A$, then bingo.

\newpage
\section*{A scheme based on pairings}

The second scheme is more sophisticated, making use of \emph{bilinear maps}.
%
Let~$G$ be an additive group of prime order~$q$ with generator~$P$.
%
Let $H$ be a multiplicative group of order~$q$.
%
Let $e : G \by G \to H$ be a function. This function is called a \emph{non-degenerate, bilinear map}
if:
\begin{enumerate}
  \item For every $P, Q \in G$ and $a, b \in \N$ it holds that $e(aP, bQ) = e(P,
    Q)^{ab}$.
  \item Not all pairs map to the identity~$1\in H$. (This implies that if $P$ is
    a generator of~$G$, then $e(P, P)$ is a generator of~$H$, since the
    groups have prime order.\footnote{See the Boneh-Franklin paper.})
\end{enumerate}
%
Let $h\in\N$ and let $\hash_1 : \bits^* \to G$ and $\hash_2 : H \to
\bits^h$ be functions.

\heading{Setup.}
%
The server samples a long term secret $x \getsr [1..q-1]$, then
initializes sets~$\U, \M \gets \emptyset$.

\heading{Registration.}
%
To register user~$I$, the server first computes $A \gets \hash_1(I)$ and $A^*
\gets xA$. The value~$A^*$ is the user's secret key.\footnote{This is how
decryption keys are generated in pairing-based IBE.}
%
For each $J \in \U$ do as follows:
%
\begin{enumerate}
  \item Compute $B \gets \hash_1(J)$. If $I < J$\footnote{This means that~$I$ is a
    string that appears before~$J$ in the lexicographic ordering of bit
    strings.}, then compute $X \gets e(A^*, B)$; otherwise compute $X \gets e(B,
    A^*)$.
  \item Let $\M \gets \M \union \{\hash_2(X)\}$.
\end{enumerate}
%
Let $\U \gets \U \union \{I\}$.
%
Finally, return~$A^*$ to the user.

\heading{Making a query.}
%
User~$I$ with secret key~$A^*$ wishes to see if~$J$ is a registered user. To do
so, first compute $B \gets \hash_1(P)$. If $I < J$, then compute $X \gets e(A^*,
B)$; otherwise compute $X \gets e(B, A^*)$. Finally, compute $Y \gets
\hash_2(X)$ and send~$Y$ to the server.

\heading{Answering queries.}
%
If~$Y \in \M$, then return $\true$; otherwise return~$\false$. (Note that
$e(A^*, B) = e(xA, B) = e(A, B)^{x} = e(A, xB) = e(A, B^*)$. Thus, as long as
the pair the order in which users were registered doesn't matter.)

\heading{Use of SGX.}
%
Trusted computation is used for provisioning the secret key, as well as
registering the user, since this requires the secret key.




\ifnum\camready=0
 \bibliographystyle{alpha}
 \else
\bibliographystyle{build/splncs_srt}
\fi

\end{document}
