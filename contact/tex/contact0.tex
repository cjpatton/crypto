%
\def\camready{1}
\ifnum\camready=0
  \makeatletter
  \let\@twosidetrue\@twosidefalse
  \let\@mparswitchtrue\@mparswitchfalse
  \makeatother
\fi

\documentclass{build/llncs}
\usepackage{color,soul}
\usepackage[dvipsnames]{xcolor}
\usepackage{enumitem}
% header.tex
%
% Formatting and common macros for crypto papers. Include this first.
\usepackage[utf8]{inputenc}
\usepackage[margin=4cm]{geometry}
\usepackage{graphics}
\usepackage[toc,page]{appendix}
\usepackage[font={small}]{caption}
\usepackage{hyperref}
\usepackage{amsmath}
\usepackage{amsthm}
\usepackage{amsfonts}
\usepackage{parskip}
\usepackage{framed}
\usepackage{multicol}

\hypersetup{
    colorlinks,%
    citecolor=black,%
    filecolor=black,%
    linkcolor=black,%
    urlcolor=black
}

\def\dashuline{\bgroup
  \ifdim\ULdepth=\maxdimen  % Set depth based on font, if not set already
	  \settodepth\ULdepth{(j}\advance\ULdepth.4pt\fi
  \markoverwith{\kern.15em
	\vtop{\kern\ULdepth \hrule width .3em}%
	\kern.15em}\ULon}

\newcounter{foot}
\setcounter{foot}{1}
\setlength\parindent{2em}

% Editorial
\newcommand{\heading}[1]{\noindent \textsc{#1}}
\newtheorem{theorem}{Theorem}[section]

% Fonts for various types
\newcommand{\notionfont}[1]{\textup{#1}}
\newcommand{\varfont}[1]{\textit{#1}}
\newcommand{\flagfont}[1]{\mathsf{#1}}
\newcommand{\vectorfont}[1]{\vec{#1}}
\newcommand{\oraclefont}[1]{\textsc{#1}}
\newcommand{\schemefont}[1]{\mathsf{#1}}
\newcommand{\procfont}[1]{\schemefont{#1}}
\newcommand{\adversaryfont}[1]{\mathcal{#1}}
\newcommand{\cryptofont}[1]{\mathbf{#1}\hspace{0.5pt}}
\newcommand{\prinfont}[1]{\mathcal{#1}}

% Math
\DeclareMathAlphabet\mathbfcal{OMS}{cmsy}{b}{n}

% - Sets
\newcommand{\Z}{\mathbb{Z}}
\newcommand{\N}{\mathbb{N}}
\newcommand{\bits}{\{0,1\}}
\newcommand*\bigunion{\bigcup}
\newcommand*\bigintersection{\bigcap}
\newcommand*\union{\cup}
\newcommand*\intersection{\cap}
\newcommand*\cross{\times}
\newcommand*\by{\cross}
%\newcommand{\getsr}{\mathrel{\leftarrow\mkern-14mu\leftarrow}}
%\newcommand{\getsr}{\xleftarrow{\text{\tiny{\$}}}}
\newcommand{\getsr}{{\:{\leftarrow{\hspace*{-3pt}\raisebox{.75pt}{$\scriptscriptstyle\$$}}}\:}}

% - String operations
\newcommand{\emptystring}{\varepsilon}
\newcommand{\cat}{\, \| \,}
\newcommand{\str}[1]{\langle #1 \rangle}

% - Boolean operators
\newcommand*\AND{\wedge}
\newcommand*\OR{\vee}
\newcommand*\NOT{\neg}
\newcommand*\IMPLIES{\implies}
\newcommand*\XOR{\mathbin{\oplus}}
\newcommand*\xor{\XOR}

% - Crypto functions
\newcommand{\game}[1]{\cryptofont{Exp}^{\textnormal{\tiny \MakeLowercase{#1}}}}
\newcommand{\adv}[1]{\cryptofont{Adv}^{\textnormal{\tiny \MakeLowercase{#1}}}}

% - Asymptotics
\newcommand{\negl}{\procfont{negl}}
\newcommand{\poly}{\procfont{poly}}

% - Probablity
\newcommand{\E}{\mathrm{E}}
\newcommand{\Prob}[1]{\Pr\hspace{-2pt}\Big[\,#1\,\Big]}

% Games
\newcommand{\gamesfontsize}{\footnotesize}
\newcommand{\foreach}[3]{$\text{for }#1 \gets #2\text{ to }#3\text{ do}$}
\newcommand{\ind}{\hspace*{10pt}}
\newcommand{\outputs}{=}
\newcommand{\true}{1}
\newcommand{\false}{0}

% - Inline comment
\definecolor{CommentColor}{RGB}{125,175,230}
\newcommand{\comment}[1]{\textcolor{CommentColor}{\,\textbf{\#}\,#1}}

% - One game
\newcommand{\oneCol}[2]{
\begin{center}
        \framebox{
        \begin{tabular}{c@{\hspace*{.4em}}}
        \begin{minipage}[t]{#1\textwidth}\gamesfontsize #2 \end{minipage}
        \end{tabular}
        }
\end{center}
}

% - One game, two columns
\newcommand{\twoColsNoDivide}[3]{
  \providecommand{\pad}{6pt}
  \providecommand{\colwidth}{#1\textwidth}
  \makebox[\textwidth][c]{
  \begin{tabular}{|@{\hskip \pad}l@{}@{}@{\hskip \pad}l|}
    \hline
    \rule{0pt}{1\normalbaselineskip}
    \begin{minipage}[t]{\colwidth}\gamesfontsize
      #2 \vspace{\pad}
    \end{minipage} &
    \begin{minipage}[t]{\colwidth}\gamesfontsize
      #3 \vspace{\pad}
    \end{minipage} \\
    \hline
  \end{tabular}
  }
}

% - Two games, one per column
\newcommand{\twoCols}[3]{
  \providecommand{\pad}{6pt}
  \providecommand{\colwidth}{#1\textwidth}
  \makebox[\textwidth][c]{
  \begin{tabular}{|@{\hskip \pad}l@{}|@{}@{\hskip \pad}l|}
    \hline
    \rule{0pt}{1\normalbaselineskip}
    \begin{minipage}[t]{\colwidth}\gamesfontsize
      #2 \vspace{\pad}
    \end{minipage} &
    \begin{minipage}[t]{\colwidth}\gamesfontsize
      #3 \vspace{\pad}
    \end{minipage} \\
    \hline
  \end{tabular}
  }
}

% Notes
\newcounter{notectr}[section]
\newcommand{\getnotectr}{\stepcounter{notectr}\thesection.\thenotectr}
\newcommand{\basenote}[4]{{
  \textcolor{#1}{(#3#4---#2~\getnotectr)}
}}

\newcommand{\note}[3]{\basenote{#1}{#2}{}{#3}}
\newcommand{\todo}[3]{\basenote{#1}{#2}{TODO~}{#3}}

% - Chris' notes
\definecolor{darkgreen}{RGB}{50,127,0}
\newcommand{\cpnote}[1]{\note{darkgreen}{Chris}{#1}}
\newcommand{\cptodo}[1]{\todo{darkgreen}{Chris}{#1}}

% macros.tex
%
% Macros for this paper. Include headers.tex, then this file.

% Math
\newcommand{\perm}{\procfont{Perm}}

% Point function
\def\pt[#1,#2,#3]{{#1}_{#2,#3}}

% Function families
\newcommand{\fns}{\setfont{F}}
\newcommand{\dom}{\setfont{X}}
\newcommand{\rng}{\setfont{Y}}
\newcommand{\Fn}{\schemefont{Fn}}

%
\newcommand{\kk}{\vectorfont{K}}
\renewcommand{\k}{K}
\newcommand{\pub}{\procfont{pub}}

% TODO Automate \def\*x.
\newcommand{\Kg}{\schemefont{Kg}}
\newcommand{\Rep}{\schemefont{Rep}}
\newcommand{\Comb}{\schemefont{Comb}}
\newcommand{\Qry}{\schemefont{Qry}}
\def\Repx(#1,#2){\Rep_{#1}\!\left(#2\right)}
\def\Combx(#1,#2){\Comb_{#1}\!\left(#2\right)}
\def\Qryx(#1,#2){\Qry_{#1}\!\left(#2\right)}

% Adversaries
\def\advA{\adversaryfont{A}}

% Sets
%
% TODO Automate \def\set*.
\def\setS{\setfont{S}}
\def\setX{\setfont{X}}
\def\setY{\setfont{Y}}

% Vectors
\def\vv{\vectorfont{v}}

% Misc.
\def\coins{r}


\ifnum\camready=0
  %\usepackage{times}
  \usepackage[letterpaper,margin=0.75in]{geometry}
\fi

\ifnum\camready=0
  \usepackage{hyperref}
  \hypersetup{
    colorlinks=true,
    allcolors={black},
    citecolor={theblue},
    urlcolor={theblue},
    pdfborder={0 0 0},
  }
\fi


\date{\today}
\title{A proposal for private contact discovery}
\author{Chris}
\institute{}

\setcounter{tocdepth}{2}

\pagestyle{plain}

\begin{document}

\maketitle

\begin{abstract}
  A contact-discovery protocol allows a user of a messaging service to discover
  which contacts in his or her phone book use the same service. The attendant
  privacy goal is to ensure that the neither the client nor the server learns
  anything beyond the set of contacts that use the service.
  %
  This can be solved using a protocol for private set intersection, but the
  communication complexity of such protocols is prohibitive for this
  application.
  %
  I propose an alternative that is optimal in terms of network utilization,
  but requires additional storage capacity on the part of the server. It applies
  bilinear maps (trust me, it's not as crazy as it seems) and uses SGX in a
  minimal, yet critical way.
\end{abstract}

\section*{Problem statement}

Let $\U \subseteq \bits^*$ denote the set of users of a
service.\footnote{$\bits^*$ denotes the set of all strings.}
%
A new user of the service has a set~$\X \subseteq \bits^*$ of contacts, and he
or she wishes to learn which of these are also users of the service.
%
We want to devise a protocol by which the client and server compute $\U
\intersection \X$ without the server learning $\X \setminus \U$ or the client
learning $\U \setminus \X$.
%
This is an instance of private set intersection, for which there are myriad
solutions. But these are prohibitively expensive, particularly in the current
application.\footnote{See slide 11 of
\url{https://www.usenix.org/sites/default/files/conference/protected-files/sec15_slides_pinkas.pdf}.
Accessed 2018-04-12.}
%
The goal here is to mitigate this cost.

I'll begin by refining the problem. Each user ``registers'' with the
contact-discovery service, which responds to queries made by registered users.
%
Specifically, registrant~$I$ may ask if~$J\in\U$ for any~$J$; the result should
be $\true$ (``true'') if and only if $I, J \in \U$.
%
The server can easily ensure that the clients from learns nothing about $\U
\setminus \X$. In addition, the bandwidth consumption is $O(m)$, where $m=|\X|$.
%
To prevent leakage of~$\X \setminus \U$, the protocol also needs to ensure that
queries only reveal~$J$ to the server if~$J$ is registered. Here I propose one
way to provide this assurance. The solution involves some cool crytography, as
well as careful use of SGX.\footnote{This is why you're reading this in
\url{bitbucket.org/uf_sensei/hybrid} :)}

The crucial resources are the time complexity of registration, queries, and the
space utilization of the server.
%
My proposal costs $\Theta(n)$ time for registration, $O(1)$ time for each query,
and $O(n^2)$ space, where $n=|\U|$.
%
Registration is completely parallelizable, so distributing the computation can
help mitigate this cost.
%
Still, the space complexity means deployment might only be feasible for a few
hundred-thousand users. However, the schemes are pretty simplistic, and there
is likely room for improvement.

The scheme involves bilinear maps, a fairly sophisticated, but relatively common tool
in cryptography. \textit{Let us not fear the unknown!} But to understand why I
think they're necessary, I'll begin with a simple straw-man scheme that doesn't
work. I'll then present the real scheme in detail.

\noindent
\textbf{\color{red}Disclaimer:} I make no claims about the security of this
proposal. I think it's an interesting scheme and worth analyzing, don't assume
that it works.

A link from Joseph with definitions and constructions in this space:
\url{https://link.springer.com/chapter/10.1007\%2F978-3-642-21554-4_9}.

\section*{Warm up: A simple scheme that doesn't work}

The first scheme I considered uses Diffie-Hellman.
%
Let~$G$ be an additive group of prime order $q$, and let~$P$ be a generator
for~$G$.\footnote{$P$ being a generator for~$G$ means that $G = \{ 0P, 1P, 2P,
\ldots, (q-1)P \}$.}
%
Let $\hash:\bits^* \to [0..q-1]$ be a crytpographic hash
function.\footnote{$[0..q-1]$ denotes the set of integers $\{0, 1, 2, \ldots,
q-1\}$.}

\heading{Setup.}
The server initializes a map~$\U$, which stores a set of key/value pairs. The
keys are strings and the values are elements of~$G$. This maps the set of
\emph{registered users} to their \emph{public keys}.

\heading{Registration.}
%
To register user~$I$, first initialize a set $\M_I \gets \emptyset$.
%
Next, sample an integer~$a$ uniformly from the set
$[1..q-1]$. (This is denoted $a \getsr [1..q-1]$.) This is the user's
\emph{secret
key}.
%
Next, compute $A \gets aP$; this is the user's public key.
%
Let $x \gets \hash(I)$.
%
For each $(J, B)$ in the set of key/value pairs in~$\U$, do as follows:
\begin{enumerate}
  \item Compute values: $y \gets \hash(J)$, $X \gets ay P$, and $Y \gets x B$.
  \item Set $\M_I \gets \M_I \union \{X\}$ and $\M_J \gets \M_J \union \{Y\}$.
\end{enumerate}
Next, set $\U[I] \gets A$.
Finally, return the secret key~$a$ to the user.

\heading{Making queries.}
%
User~$I$ with secret key~$a$ wishes to see if~$J$ is a registered user. To do
so, first compute $y \gets \hash(J)$ and $X \gets ay P$.
%
Then send~$(I, X)$ to the server.

\heading{Answering queries.}
%
On input of~$(I, X)$, if $\U[I]$ is undefined, then return $\false$.
%
Otherwise, if $X \not\in \M_I$, then return $\false$.
%
Otherwise return $\true$.

\heading{Use of SGX.}
%
Trusted computation is used for provisioning the user with their secret key.
It's also used for registration, since this requires computations using the
secret key.

\noindent\textbf{\color{red}Attack!}
%
On input of $(I, X)$ the server can try a dictionary attack. Let $A \gets
\U[I]$. To check if $J$ was used to compute query~$X$, compute $y \gets \hash(J)$, then
$A' \gets y^{-1}X$. If $A' = A$, then bingo.

\newpage
\section*{Leveling up}

The second scheme is more sophisticated, making use of \emph{bilinear maps}.
%
Let~$G$ be an additive group of prime order~$q$ with generator~$P$.
%
Let $H$ be a multiplicative group of order~$q$.
%
Let $e : G \by G \to H$ be a function. This function is called a \emph{non-degenerate, bilinear map}
if:
\begin{enumerate}
  \item For every $P, Q \in G$ and $a, b \in \N$ it holds that $e(aP, bQ) = e(P,
    Q)^{ab}$.
  \item Not all pairs map to the identity~$1\in H$. (This implies that if $P$ is
    a generator of~$G$, then $e(P, P)$ is a generator of~$H$, since the
    groups have prime order.\footnote{See the Boneh-Franklin paper.})
\end{enumerate}
%
Let $h\in\N$ and let $\hash_1 : \bits^* \to [0..q-1]$ and $\hash_2 : H \to
\bits^h$ be functions.

\heading{Setup.}
%
The server samples a \emph{long term secret} $x \getsr [1..q-1]$, then
initializes sets~$\U, \M \gets \emptyset$.

\heading{Registration.}
%
To register user~$I$, the server first computes $a \gets \hash_1(I)$, $A \gets aP$, and $A^*
\gets xA$. The value~$A^*$ is the user's \emph{secret key}.\footnote{It's
analogous to a decryption key in pairing-based IBE.}
%
For each $J \in \U$ do as follows:
%
\begin{enumerate}
  \item Compute $b \gets \hash_1(J)$, $B \gets bP$, and $X \gets e(B, A^*)$.
  \item Let $\M \gets \M \union \{\hash_2(X)\}$.
\end{enumerate}
%
Let $\U \gets \U \union \{I\}$.
%
Finally, return~$A^*$ to the user.

\heading{Making a query.}
%
User~$I$ with secret key~$A^*$ wishes to see if~$J$ is a registered user. To do
so, first compute $b \gets \hash_1(J)$, $B \gets bP$, $X \gets e(B, A^*)$, and
$Y \gets \hash_2(X)$.
Then send~$Y$ to the server.

\heading{Answering queries.}
%
If~$Y \in \M$, then return $\true$; otherwise return~$\false$. (Note that $e(B,
A^*) = e(bP, axP) = e(P, P)^{abx} = e(aP, bxP) = e(A, B^*)$. Thus, the order in
which users were registered doesn't matter.)

\heading{Use of SGX.}
%
Trusted computation is used for provisioning the secret key, as well as
registering the user, since this requires the secret key.


\subsection*{Security considerations}

Users use their secret keys in order to protect the identity of its contacts
in case they're not registered. Since secret keys are computed from the server's
long term secret~$x$, security of the SGX enclave is paramount. The secret key
can (and should) be forgotten immediately after registration.

Notice that the dictionary attack describe against the first scheme doesn't work
here.
%
The server's input is $Y=\hash_2(X)$ where \[X = e(B, A^*) = e(bP, axP) = e(P,
P)^{abx} = g^{abx}\] where $g \in H$ is a generator for~$H$. Even if the server
knows~$a$, it can't be sure that its guess of~$b$ is correct unless it
knows~$g^x$. It doesn't, and it can't compute it with computing a discrete
logarithm.  Recall that, by design, the user's secret key $A^* = axP$ doesn't
leave the SGX boundary.

Alrighty, so it passes that sniff test. But you shouldn't believe that it's
secure just yet. I'm not super familiar with bilinear maps, and I have no idea
whether this constitutes ``safe use'' of them. I'll need to do some research to
get a feel for how they're used and what assumptions are made.
%
The standard \emph{bilinear Diffie-Hellman} (\bdh) assumption is that,
informally, the following quantity is ``small'' for every ``reasonable''
adversary~$\A$:
\begin{eqnarray*}
  \Adv{\bdh}_{\langle G,H,e \rangle}(\A) = \Pr \Big[
    &&
      a, b, c \getsr [1..q-1];
      P \getsr G:\\
    && \A(P, aP, bP, cP) = e(P, P)^{abc}\,
  \Big]\,.
\end{eqnarray*}
%
This is used in proofs of security for Boneh-Franklin identity-based encryption
scheme, as well as 3-way key agreement protocols.
%
This is the \emph{computational} variant; there's also a \emph{decisional}
variant used in other contexts.
%
I'm not sure if these relate to our setting.
%
Maybe we don't need these strong assumptions; perhaps it suffices that computing
discrete logs in~$G$ and~$H$ is hard?
%
Anywho, these needs work.

\subsection*{Implementation considerations}

Registration runs in time $\Theta(n)$, where~$n = |\U|$. This can be
parallelized by partitioning~$\U$ and mapping each partition to a worker that
performs the prescribed computations (steps 1 and 2) for its subset. Each worker
needs the secret key~$A^*$ in order to do its job, so we're going need it to run
SGX. This posses an interesting engineering problem What's the best way to
integrate SGX into a distributed computation like this? Think ``SGX-enabled
MapReduce''.

The set~$\M$ has $\binom{n}{2} = O(n^2)$ elements. There are ways, perhaps, to
change the protocol that would reduce this complexity.
%
With the current design, however, we can do something to save space.  The set
$\M$ can be represented by a \emph{Bloom filter}, a data structures that admit
sa small false positive rate. A false positive will result in a non-user contact
being revealed to the server, so the chance of this needs to be minimized. A
small false positive rate of, say, $0.01\%$ should be tolerable, since most
users won't have an enormous amount of contacts.
%
A challenge is maintaining a low false positive rate as the set of users grows;
a variant called the \emph{scalable Bloom
filter}\footnote{\url{https://en.wikipedia.org/wiki/Bloom_filter\#Scalable_Bloom_filters}.
Accessed 2018-04-12.} might be more appropriate.

Bilinear maps might seem impractical for this application, since each registration
requires~$n$ computations of the map. But there are implementations of these
schemes that are likely more efficient than you think. A good starting point
would be the bn256 bilinear group, which has a nice Go
implementation.\footnote{\url{https://godoc.org/golang.org/x/crypto/bn256}.
Accessed 2018-04-12.} Cloudflare has used this group for an implementation of
IBE.


\ifnum\camready=0
 \bibliographystyle{alpha}
 \else
\bibliographystyle{build/splncs_srt}
\fi

\end{document}
