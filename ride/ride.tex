%
\def\camready{1}
\ifnum\camready=0
  \makeatletter
  \let\@twosidetrue\@twosidefalse
  \let\@mparswitchtrue\@mparswitchfalse
  \makeatother
\fi

\documentclass{build/llncs}
\usepackage{color,soul}
\usepackage[dvipsnames]{xcolor}
\usepackage{enumitem}
% header.tex
%
% Formatting and common macros for crypto papers. Include this first.
\usepackage[utf8]{inputenc}
\usepackage[margin=4cm]{geometry}
\usepackage{graphics}
\usepackage[toc,page]{appendix}
\usepackage[font={small}]{caption}
\usepackage{hyperref}
\usepackage{amsmath}
\usepackage{amsthm}
\usepackage{amsfonts}
\usepackage{parskip}
\usepackage{framed}
\usepackage{multicol}

\hypersetup{
    colorlinks,%
    citecolor=black,%
    filecolor=black,%
    linkcolor=black,%
    urlcolor=black
}

\def\dashuline{\bgroup
  \ifdim\ULdepth=\maxdimen  % Set depth based on font, if not set already
	  \settodepth\ULdepth{(j}\advance\ULdepth.4pt\fi
  \markoverwith{\kern.15em
	\vtop{\kern\ULdepth \hrule width .3em}%
	\kern.15em}\ULon}

\newcounter{foot}
\setcounter{foot}{1}
\setlength\parindent{2em}

% Editorial
\newcommand{\heading}[1]{\noindent \textsc{#1}}
\newtheorem{theorem}{Theorem}[section]

% Fonts for various types
\newcommand{\notionfont}[1]{\textup{#1}}
\newcommand{\varfont}[1]{\textit{#1}}
\newcommand{\flagfont}[1]{\mathsf{#1}}
\newcommand{\vectorfont}[1]{\vec{#1}}
\newcommand{\oraclefont}[1]{\textsc{#1}}
\newcommand{\schemefont}[1]{\mathsf{#1}}
\newcommand{\procfont}[1]{\schemefont{#1}}
\newcommand{\adversaryfont}[1]{\mathcal{#1}}
\newcommand{\cryptofont}[1]{\mathbf{#1}\hspace{0.5pt}}
\newcommand{\prinfont}[1]{\mathcal{#1}}

% Math
\DeclareMathAlphabet\mathbfcal{OMS}{cmsy}{b}{n}

% - Sets
\newcommand{\Z}{\mathbb{Z}}
\newcommand{\N}{\mathbb{N}}
\newcommand{\bits}{\{0,1\}}
\newcommand*\bigunion{\bigcup}
\newcommand*\bigintersection{\bigcap}
\newcommand*\union{\cup}
\newcommand*\intersection{\cap}
\newcommand*\cross{\times}
\newcommand*\by{\cross}
%\newcommand{\getsr}{\mathrel{\leftarrow\mkern-14mu\leftarrow}}
%\newcommand{\getsr}{\xleftarrow{\text{\tiny{\$}}}}
\newcommand{\getsr}{{\:{\leftarrow{\hspace*{-3pt}\raisebox{.75pt}{$\scriptscriptstyle\$$}}}\:}}

% - String operations
\newcommand{\emptystring}{\varepsilon}
\newcommand{\cat}{\, \| \,}
\newcommand{\str}[1]{\langle #1 \rangle}

% - Boolean operators
\newcommand*\AND{\wedge}
\newcommand*\OR{\vee}
\newcommand*\NOT{\neg}
\newcommand*\IMPLIES{\implies}
\newcommand*\XOR{\mathbin{\oplus}}
\newcommand*\xor{\XOR}

% - Crypto functions
\newcommand{\game}[1]{\cryptofont{Exp}^{\textnormal{\tiny \MakeLowercase{#1}}}}
\newcommand{\adv}[1]{\cryptofont{Adv}^{\textnormal{\tiny \MakeLowercase{#1}}}}

% - Asymptotics
\newcommand{\negl}{\procfont{negl}}
\newcommand{\poly}{\procfont{poly}}

% - Probablity
\newcommand{\E}{\mathrm{E}}
\newcommand{\Prob}[1]{\Pr\hspace{-2pt}\Big[\,#1\,\Big]}

% Games
\newcommand{\gamesfontsize}{\footnotesize}
\newcommand{\foreach}[3]{$\text{for }#1 \gets #2\text{ to }#3\text{ do}$}
\newcommand{\ind}{\hspace*{10pt}}
\newcommand{\outputs}{=}
\newcommand{\true}{1}
\newcommand{\false}{0}

% - Inline comment
\definecolor{CommentColor}{RGB}{125,175,230}
\newcommand{\comment}[1]{\textcolor{CommentColor}{\,\textbf{\#}\,#1}}

% - One game
\newcommand{\oneCol}[2]{
\begin{center}
        \framebox{
        \begin{tabular}{c@{\hspace*{.4em}}}
        \begin{minipage}[t]{#1\textwidth}\gamesfontsize #2 \end{minipage}
        \end{tabular}
        }
\end{center}
}

% - One game, two columns
\newcommand{\twoColsNoDivide}[3]{
  \providecommand{\pad}{6pt}
  \providecommand{\colwidth}{#1\textwidth}
  \makebox[\textwidth][c]{
  \begin{tabular}{|@{\hskip \pad}l@{}@{}@{\hskip \pad}l|}
    \hline
    \rule{0pt}{1\normalbaselineskip}
    \begin{minipage}[t]{\colwidth}\gamesfontsize
      #2 \vspace{\pad}
    \end{minipage} &
    \begin{minipage}[t]{\colwidth}\gamesfontsize
      #3 \vspace{\pad}
    \end{minipage} \\
    \hline
  \end{tabular}
  }
}

% - Two games, one per column
\newcommand{\twoCols}[3]{
  \providecommand{\pad}{6pt}
  \providecommand{\colwidth}{#1\textwidth}
  \makebox[\textwidth][c]{
  \begin{tabular}{|@{\hskip \pad}l@{}|@{}@{\hskip \pad}l|}
    \hline
    \rule{0pt}{1\normalbaselineskip}
    \begin{minipage}[t]{\colwidth}\gamesfontsize
      #2 \vspace{\pad}
    \end{minipage} &
    \begin{minipage}[t]{\colwidth}\gamesfontsize
      #3 \vspace{\pad}
    \end{minipage} \\
    \hline
  \end{tabular}
  }
}

% Notes
\newcounter{notectr}[section]
\newcommand{\getnotectr}{\stepcounter{notectr}\thesection.\thenotectr}
\newcommand{\basenote}[4]{{
  \textcolor{#1}{(#3#4---#2~\getnotectr)}
}}

\newcommand{\note}[3]{\basenote{#1}{#2}{}{#3}}
\newcommand{\todo}[3]{\basenote{#1}{#2}{TODO~}{#3}}

% - Chris' notes
\definecolor{darkgreen}{RGB}{50,127,0}
\newcommand{\cpnote}[1]{\note{darkgreen}{Chris}{#1}}
\newcommand{\cptodo}[1]{\todo{darkgreen}{Chris}{#1}}

% macros.tex
%
% Macros for this paper. Include headers.tex, then this file.

% Math
\newcommand{\perm}{\procfont{Perm}}

% Point function
\def\pt[#1,#2,#3]{{#1}_{#2,#3}}

% Function families
\newcommand{\fns}{\setfont{F}}
\newcommand{\dom}{\setfont{X}}
\newcommand{\rng}{\setfont{Y}}
\newcommand{\Fn}{\schemefont{Fn}}

%
\newcommand{\kk}{\vectorfont{K}}
\renewcommand{\k}{K}
\newcommand{\pub}{\procfont{pub}}

% TODO Automate \def\*x.
\newcommand{\Kg}{\schemefont{Kg}}
\newcommand{\Rep}{\schemefont{Rep}}
\newcommand{\Comb}{\schemefont{Comb}}
\newcommand{\Qry}{\schemefont{Qry}}
\def\Repx(#1,#2){\Rep_{#1}\!\left(#2\right)}
\def\Combx(#1,#2){\Comb_{#1}\!\left(#2\right)}
\def\Qryx(#1,#2){\Qry_{#1}\!\left(#2\right)}

% Adversaries
\def\advA{\adversaryfont{A}}

% Sets
%
% TODO Automate \def\set*.
\def\setS{\setfont{S}}
\def\setX{\setfont{X}}
\def\setY{\setfont{Y}}

% Vectors
\def\vv{\vectorfont{v}}

% Misc.
\def\coins{r}


\ifnum\camready=0
  %\usepackage{times}
  \usepackage[letterpaper,margin=0.75in]{geometry}
\fi

\ifnum\camready=0
  \usepackage{hyperref}
  \hypersetup{
    colorlinks=true,
    allcolors={black},
    citecolor={theblue},
    urlcolor={theblue},
    pdfborder={0 0 0},
  }
\fi


\date{\today}
\title{Private ride sharing}
\author{Chris and Joseph}
\institute{}

\setcounter{tocdepth}{2}

\pagestyle{plain}

\begin{document}

\maketitle

\begin{abstract}
  \cptodo{}
\end{abstract}

\section*{Problem statement}

The protocol involves the client~$R\in\bits^*$, the service provider, and a pool
of drivers $\{D_i\}_{i\in[p]} \subseteq \bits^*$. These strings are the
``current state'' of the parties. Some common metrics:
\begin{itemize}
   \item $\id:\bits^*\to\bits^m$ --- maps a principal to their unique identifier
   \item $\loc:\bits^*\to\bits^n$ --- maps a principal to their location
   \item $\dist : (\bits^n)^3 \to \N$ --- the quantity $\dist(p, s, t)$ is the ``cost'' of
     picking up a rider at location~$p$ and taking them to~$t$, starting at
     current location~$s$.
\end{itemize}
%
The client and server negotiate candidate drivers. The client and driver
determine if the cost of pick up is reasonable. The determination is shared with
the server, who will then decide to either set up a ride, or try more
candidates.
%
The security goal is to prevent leakage of the locations of any client or driver
to any other party, including the server.


\section*{The protocol}

To get a ride to location~$d$, the client and service provider begin by
negotiating a string $id(D_i)$, the public identifier of driver~$i\in[p]$.
(They may use any information they've learned from prior hail attempts as part
of the negotiation.) The client then chooses a ``cost threshold'' $c\in\N$.
%
The client and driver compute $y = (\dist(\loc(R), \loc(D_i), t) < c)$. The
result is given to the server, who may then set up the ride; otherwise they'll
negotiate a new $\id(D_j)$ for some $j\ne i$ and try again.
%
This process is called a \emph{hail}; details are given below.

\subsubsection{The hail.}

Suppose the client and server have negotiated a driver id~$\id(D_i)$ for
some~$i \in [p]$.
%
The client and driver engage in a secure function evaluation via garbled
circuits and oblivious transfer.
%
The client computes $(F, e, d) \getsr \Gb(1^k, f)$ where
\[
  f(\loc(D_i) \cat \loc(R)\cat t \cat \tostr(c)) =
    \delta(\loc(R), \loc(D_i), t) < c \,,
\]
sends~$d$ to the server, garbles its inputs $\loc(R)$, $t$, and $\tostr(c)$ to get~$A$, sends~$(F, A)$ to the
driver, and obliviously transfers the driver's garbled input~$B$, which encodes
$\loc(D_i)$.
%
The driver computes $Y \gets \Ev(F, A \cat B)$ and sends~$Y$ to the server. The
server computes final result $y \gets \De(d, Y)$. If $y=1$, the server may set up
the ride; otherwise it attempts another hail with the client.

\subsubsection{Use of SGX.}
\cpnote{The server gets some side information, but it knows nothing about
locations. I suppose it's the thing doing SGX?}

\cpnote{The goal here is to permit the semantics of Uber's ride set ups. We have
a few anecdotal details about this.}

\cpnote{This changes are formal model.}

\section*{Alternative Approaches}
\subsubsection{ORide [USENIX 17]}
\begin{itemize}
\item Active adversaries: riders (R) \& drivers (D).
\item Passive (possibly covert) adversary: service provider (SP).
\item Assume: Most R/D don't collude w/ SP;
D = independent contractors.
\item Goals:
\subitem SP can identify misbehaving R/D when needed, w/ support from D/R.
\subitem Support existing convenience/usability: credit card payment, reputation.
\subitem Once R matched with D, R can track D's location. R can contact D to coordinate
pickup. R can contact past D to find lost items.
\item Ride Setup Protocol (high-level):
\subitem R initiates ride request, submitting to SP: encrypted+encoded coordinates, deposit
token (prerequisite), ephemeral public key, and pickup zone.
\subitem SP checks validity (freshness) of deposit token (originally issued by SP).
\subitem SP sends to each D in pickup zone random index and R's pubkey.
\subitem D sends encrypted+encoded coordinates to SP.
\subitem SP sums drivers' ciphertexts (somewhat homomorphic encryption) and computes packed
squared values of Euclidean distances between all D and the R, sending result to R.
\subitem R decrypts ciphertexts, selects D with smallest squared dist.; send to SP.
\subitem SP notifies chosen D, If D rejects, SP asks R to select another D. Repeat until one D
accepts.
\subitem R and D establish secure channel via SP using DHKE. Over channel, reputation reveal.
D can abort if R reputation low. R can return to previous step to pick another D.
\subitem If both OK w/ reputation, R and D exchange precise location,
\subitem On approach, D performs proximity check to verify R's presence.
If successful, D releases identifying info to R.
\subitem R and D create fare report, which D will deposit to SP for payment. Upfront
fare payment: upon receiving fare report, ride begins.
\end{itemize}

\begin{table}[h]
\centering
\begin{tabular}{c|c|c|c|c}
Setting & \multicolumn{2}{c|}{Rider} & \multicolumn{2}{c}{Driver}\\ \hline
Algorithm & Upload (KB) & Download (KB) & Download (KB) & Upload (KB)\\ \hline
S1 & 372 & 761856 & 124 & 248\\ \hline
S2 & 372 & 186 & 124 & 248\\ \hline
S3 & 372 & 186 & 124 & 248\\ \hline
\end{tabular}
\caption{Per-ride bandwidth of ORide with $d=4096, log_2(q)=124$, and 4096 drivers
available for ride request. S1 na\"ive, S2 S3 optimized.}
\label{tbl:orideBW}
\end{table}
%
\begin{table*}[h]
\centering
\begin{tabular}{c|c|c|c|c|c|c|c}
Setting & \multicolumn{3}{c|}{Rider} & \multicolumn{2}{c|}{Driver} & \multicolumn{2}{c}{SP}\\ \hline
Algorithm & Gen. keys & Encrypt & Decrypt & Load key & Encrypt & Load key & Compute Dist.\\
& (ms) & (ms) & (ms) & (ms) & (ms) & (ms) & (ms)\\ \hline
S1 & 1.51 $\pm$ 0.06 & 2.6 $\pm$ 0.2 & 7823.4 $\pm$ 573.4 & 0.53 $\pm$ 0.01 & 2.6 $\pm$ 0.2 & 0.53 $\pm$ 0.01 & 113868.8 $\pm$ 6553\\ \hline
S2 & 1.51 $\pm$ 0.06 & 2.6 $\pm$ 0.2 & 2.2 $\pm$ 0.1 & 0.53 $\pm$ 0.01 & 2.6 $\pm$ 0.2 & 0.53 $\pm$ 0.01 & 208.9 $\pm$ 4\\ \hline
S3 & 1.51 $\pm$ 0.06 & 2.6 $\pm$ 0.2 & 2.2 $\pm$ 0.1 & 0.53 $\pm$ 0.01 & 2.6 $\pm$ 0.2 & 0.53 $\pm$ 0.01 & 745.5 $\pm$ 24.5\\ \hline
\end{tabular}
\caption{Per-ride computational overhead of ORide with $d=4096, log_2(q)=124$, and 4096 drivers
available for ride request. S1 na\"ive, S2 S3 optimized.}
\label{tbl:orideComput}
\end{table*}
%
\subsubsection{PrivatePool [CSF 17]}
\begin{itemize}
\item Key Idea No. 1: Feasible ridesharing, where users are willing to divert from their
original path.
\item Key Idea No. 2: ridesharing w/ no trust in third parties (decentralized,
no full disclosure of user location.
\item Tangential from ``ride-hailing" (e.g., ORide); privacy-preserving carpool.
\end{itemize}


\ifnum\camready=0
 \bibliographystyle{alpha}
 \else
\bibliographystyle{build/splncs_srt}
\fi

\end{document}
